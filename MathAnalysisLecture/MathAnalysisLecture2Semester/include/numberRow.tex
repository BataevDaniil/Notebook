\begin{title}
  Числовые ряды
\end{title}

\begin{title}[\Large]
  Сходимость числовых рядов. Необходимые и достаточные условие сходимости
  числовых рядов
\end{title}

  $a_k$ числовая последовательность.

  $\sum_{k=1}^{\infty} a_k = a_1 + a_2 + \ldots a_k + \ldots$

  $S_n = \sum_{k=1}^n a_k = a_1 + a_2 + \ldots + a_n$ частичная (частая)
сумма числового ряда $a_k$.

  Если существует конечный $\lim_{n \to \infty} S_n = S$ то его называют суммой
числового ряда. Ряд который сходится называют сходящимся рядом, иначе
раходящимcя рядом.

\begin{block}[Необходимые условия сходимости числового ряда]
  Числовой ряд $a_k$ cходится, тогда
  $$
  \lim_{k \to \infty} a_k = 0
  $$
\end{block}

\begin{proof}
  так как $a_k$ сходится, то
  $$
  \exists \lim_{n \to \infty} S_n = S  $$
  тогда $a_n = S_n - S_{n-1}$
  $$
  \lim_{n \to \infty} a_n = \lim_{n \to \infty} S_n -
  \lim_{n \to \infty} S_{n-1} = S - S = 0
  $$
\end{proof}

\begin{title}[\Large]
  Свойства сходящегося числового ряда
\end{title}

\begin{theorem}
  ряд $a_k$ сходится $\sum_{k=1}^{\infty} a_k = A$

  ряд $b_k$ сходится $\sum_{k=1}^{\infty} b_k = B$

  тогда $\alpha, \beta \in R$
  $$
  \sum_{k=1}^{\infty} (\alpha a_k + \beta b_k) = \alpha A + \beta B
  $$
\end{theorem}

\begin{proof}
  $$
  S_n = \sum_{k=1}^n (\alpha a_k + \beta b_k) =
  \alpha \sum_{k=1}^n a_k + \beta \sum_{k=1}^n b_k = \alpha A_n + \beta b_n
  $$
  $$
  S_n = \alpha A_n + \beta B_n ~~~
  \lim_{n \to \infty} S_n = \alpha \lim_{n \to \infty} A_n +
  \beta \lim_{n \to \infty} B_n = \alpha A + \beta B
  $$
\end{proof}

\begin{theorem}
  $a_k$ сходится
  $
  \sum_{k=1}^{\infty} a_k = S ~~~ \sum_{\varphi = 1}^{\infty} b_{\varphi}
  $
  $b_{\varphi}$ получили конечной заменой некоторых рядом стоящими членов $a_k$
  на одни без перестановки, тогда
  $
  \sum_{\varphi = 1}^{\infty} b_{\varphi} = S
  $

  $
  b_1 = a_1 + a_2 + \ldots + a_{k_1}
  $

  $
  b_2 = a_{k_1+1} + a_{k_1+2} + \ldots + a_{k_2}
  $

  $
  b_3 = a_{k_2+1} + a_{k_2+2} + \ldots + a_{k_3}
  $

  $~~~ \ldots ~~~~~~~~~~~ \ldots ~~~~~~~~~~~ \ldots$
\end{theorem}

\begin{proof}
  $S_n$ частная сумма $a_k$

  $G_m$ частная сумма $b_{\varphi}$

  $G_m = S_{k_m}$
  $$
  \lim_{m \to \infty} G_m = \lim_{n \to \infty} S_n = S
  $$
\end{proof}

\begin{theorem}
  Добавление конечного числа членов или отбрасывание конечного числа членов к
числовому ряду не изменяет его сходимость, но может изменить его сумму.
\end{theorem}

\begin{proof}
  $$
  b_1 + b_2 + \ldots + b_m + \sum_{k=1}^{\infty} a_k
  $$
  $$
  G_n = b_1 + b_2 + \ldots + b_m + S_{n-m}
  $$
  $$
  G_n = S_{n+m} - (a_1 + a_2 + \ldots + a_m)
  $$
\end{proof}

\begin{title}[\Large]
  Критерии сходимости числовых рядов.
\end{title}

\begin{block}[Критерий Коши сходимости числового ряда:]
  Для того чтобы $\sum_{k=1}^{\infty} a_k$ был сходящимся необходимо и
  достаточно
  $$
  \forall \varepsilon > 0 ~~~
  \exists n_{\varepsilon} \in N ~~~
  \forall n \ge n_{\varepsilon} ~~~
  \forall p \in N ~~~
  \left| \sum_{k=n+1}^{n+p} a_k \right| < \varepsilon
  $$
\end{block}

\begin{proof}
  $$
  S_n = \sum_{k=1}^n a_k ~~~ \lim_{n \to \infty} S_n = S
  $$
  $$
  \forall \varepsilon > 0 ~~~
  \exists n_{\varepsilon} \in N ~~~
  \forall \varphi \ge n_{\varepsilon} ~~~
  \forall p \in N ~~~
  \left| S_{\varphi + p} - S_{\varphi} \right| < \varepsilon
  $$
\end{proof}

\begin{block}[Критерий]
  $
  \forall k \in N ~~~
  a_k \ge 0
  $
  для сходимости $a_k$ необходимо и достаточно чтобы $S_n \le M$
\end{block}

\begin{title}[\Large]
  Интегральный признак сходимости числового ряда
\end{title}

\begin{block}[Интегральный признак сходимости числового ряда:]
  $f(x) \ge 0$ определена, непрерывна, $\searrow$ на $[1, +\infty]$ тогда
  $$
  \sum_{k=1}^{\infty} f(k) ~~~~~~ \int_1^{\infty} f(x)dx
  $$
  сходятся или расходятся одновременно.
\end{block}

\begin{proof}
  $\Delta_k = [k, k+1] ~~ k \in N$
  $$
  \forall x \in \Delta_k ~~~ f(k+1) \le f(x) \le f(k)
  $$
  $$
  f(k+1) \le \int_k^{k+1} f(x)dx \le f(k) ~~~ k = 1, \ldots, n
  $$
  $$
  \sum_{k=1}^n f(k+1) \le \sum_{k=1}^n \int_k^{k+1} f(x)dx \le \sum_{k=1}^n f(k)
  $$
  $$
  S_{n+1} - f(1) \le \int_1^{n+1} f(x)dx \le S_n
  $$
  пусть сходится
  $$
  \int_1^{+\infty} f(x)dx ~~~ \forall n \in N ~~~
  \int_1^{n+1} f(x)dx \le \int_1^{+\infty} f(x) dx = M
  $$
  $$
  \forall n \in N ~~~ S_{n+1} \le M + f(1)
  $$
  если сходится интеграл, то ряд сходится.

  Пусть ряд $\sum_{k=1}^{\infty} a_k$ сходится тогда
  $$
  \forall n \in N ~~~
  S_n \le M ~~~
  \int_1^b f(x)dx \le \int_1^{n+1} f(x)dx \le S_n \le M
  $$
  $\forall b>1 ~~~ \exists n+1 > b$
  $$
  \exists \lim_{b \to +\infty} \int_1^b f(x)dx = \int_1^{+ \infty} f(x)dx
  $$
\end{proof}

\begin{title}[\Large]
  Признак сравнения для числовых рядов
\end{title}

\begin{block}[Признак сравнения:]
  $\forall k \in N ~~~ 0 \le a_k \le b_k$ тогда

  $\sum_{k=1}^{\infty} b_k$ сходится следовательно $\sum_{k=1}^{\infty} a_k$
  сходится

  $\sum_{k=1}^{\infty} a_k$ расходится следовательно $\sum_{k=1}^{\infty} b_k$
  расходится
\end{block}

\begin{proof}
  Пусть $\sum_{k=1}^{\infty} b_k$ сходится
  $$
  \forall n \in N ~~~ B_n = \sum_{k=1}^n b_k \le M
  $$
  $$
  \forall n \in N ~~~ A_n = \sum_{k=1}^n a_k \le B_n \le M
  $$
  значит $a_k$ сходится.

  Второе утверждение получается и первого докательством от противного.
\end{proof}

\begin{block}[Признак сравнения в предельной форме]
  $\forall k \in N ~~~ a_k, b_k > 0 ~~~ a_k \sim b_k ~~ k \to \infty$ тогда

  $\sum_{k=1}^{\infty} a_k ~~~ \sum_{k=1}^{\infty} b_k$
  сходятся или расходятся одновременно.
\end{block}

\begin{proof}
  $$
  \forall \varepsilon > 0 ~~~ \exists n_{\varepsilon} \in N ~~~
  \forall k \ge n_{\varepsilon} ~~~
  \left| \frac{a_k}{b_k} -1 \right| < \varepsilon ~~~
  1 - \varepsilon < \frac{a_k}{b_k} < 1 + \varepsilon
  $$
  $$
  (1 - \varepsilon)b_k < a_k < (1 + \varepsilon)b_k
  $$
  $$
  \varepsilon = \frac{1}{2} ~~~
  \exists n_{\frac{1}{2}} \in N ~~~
  \forall k \ge n_{\frac{1}{2}} ~~~
  \frac{b_k}{2} < a_k < \frac{3}{2} b_k
  $$
  Пусть $\sum_{k=1}^{\infty} b_k$ сходится, тогда
  $\sum_{k=n_{\frac{1}{2}}}^{\infty} b_k$ сходится, тогда
  $\sum_{k=n_{\frac{1}{2}}}^{\infty} \frac{3}{2} b_k$ сходится и
  $\sum_{k=n_{\frac{1}{2}}}^{\infty} a_k$ сходится, тогда
  $\sum_{k=1}^{\infty} a_k$ сходится.\\

  Пусть $\sum_{k=1}^{\infty} a_k$ сходится, тогда
  $\sum_{k=n_{\frac{1}{2}}}^{\infty} a_k$ сходится, тогда
  $\sum_{k=n_{\frac{1}{2}}}^{\infty} b_k$ сходится, тогда
  $\sum_{k=1}^{\infty} b_k$ сходится.
\end{proof}

\begin{title}
  Признак Даламбера
\end{title}

\begin{block}[Признак Даламбера:]
  $\forall k \in N ~~~ a_k > 0$

  Если $\exists 0 < \lambda < 1 ~~~ \exists k_0 \in N ~~~ \forall k \ge k_0 ~~~
  \frac{a_{k+1}}{a_k} \le \lambda$ тогда $\sum_{k=1}^{\infty} a_k$ сходится

  Если $\exists k_0 \in N ~~~ \forall k \ge k_0 ~~~ \frac{a_{k+1}}{a_k} \ge 1$
  тогда $\sum_{k=1}^{\infty} a_k$ расходится.
\end{block}

\begin{proof}
  $$
  1. ~~ a_2 \le \lambda a_1
  $$
  $$
  a_3 \le \lambda a_2 \le \lambda^2 a_1
  $$
  $$
  a_4 \le \lambda a_3 \le \lambda^3 a_1
  $$
  $$
  \ldots \ldots \ldots \ldots \ldots
  $$
  $$
  a_k < a_1 \lambda^{k-1} ~~~ 0 \le \lambda < 1
  $$
  $$
  \sum_{k=1}^{\infty} a_1 \lambda^{k-1} = \frac{a_1}{a-\lambda}
  $$
  $\sum_{k=1}^{\infty} a_k$ сходится

  $$
  a_2 \ge a_1
  $$
  $$
  a_3 \ge a_2 \ge a_1
  $$
  $$
  \ldots \ldots \ldots \ldots \ldots
  $$
  $$
  a_k \ge a_1 > 0
  $$
  $$
  \lim_{k \to \infty} \not= 0
  $$
  расходится, тогда $\sum_{k=1}^{\infty} a_k$ расходится.
\end{proof}

\begin{block}[Признак Даламбера в предельной форме]
  $$
  \forall k \in N ~~~ a_k > 0 ~~~
  \exists \lim_{k \to \infty} \frac{a_{k+1}}{a_k} = \lambda
  $$
  1. Если $\lambda < 1 ~~~ \sum_{k=1}^{\infty} a_k$ сходится

  2. Если $\lambda > 1 ~~~ \sum_{k=1}^{\infty} a_k$ расходится

  3. Если $\lambda = 1 ~~~ \sum_{k=1}^{\infty} a_k$ может быть что угодно

  Наиболее эфективно его можно использовать, если члены ряда содержат факториал.
\end{block}

\begin{proof}
  $$
  \forall \varepsilon > 0 ~~~
  \exists n_{\varepsilon} \in N ~~~
  \forall k \ge n_{\varepsilon} ~~~
  \left| \frac{a_{k+1}}{a_k} - \lambda \right| < \varepsilon ~~~
  \lambda - \varepsilon < \frac{a_{k+1}}{a_k} < \lambda + \varepsilon
  $$
  1) Пусть $\lambda < 1 ~~~ \lambda + \varepsilon < 1$ тогда по
  $\frac{a_{k+1}}{a_k} < \lambda + \varepsilon$, ряд
  $\sum_{k = n_{\varepsilon}}^{\infty} a_k$ сходится, значит сходится
  $\sum_{k = 1}^{\infty} a_k$

  2) Пусть $\lambda > 1 ~~~ \lambda - \varepsilon > 1$ тогда
  $\sum_{k = n_{\varepsilon}}^{\infty} a_k$ расходится, значит расходится
  $\sum_{k = 1}^{\infty} a_k$
\end{proof}

\begin{title}[\Large]
  Признак Коши сходимости числового ряда
\end{title}

\begin{block}[Признак Коши сходимости числового ряда]
  $\forall k \in N ~~~ a_k \ge 0 $ тогда

  Если $\exists 0 < \lambda < 1 ~~~ \exists k_0 \in N ~~~ \forall k \ge k_0 ~~~
  \sqrt[k]{a_k} \le \lambda$ тогда $\sum_{k=1}^{\infty} a_k$ сходится.

  Если $\exists k_0 \in N ~~~ \forall k \ge k_0 ~~~ \sqrt[k]{a_k} \ge 1$ тогда
  $\sum_{k=1}^{\infty} a_k$ расходится.
\end{block}

\begin{proof}
  1) $0 \le a_k \le \lambda^k ~~~ \lambda < 1$ $\sum_{k=1}^{\infty} \lambda^k$
  сходится, то по признаку стравнения $\sum_{k=1}^{\infty} a_k$ cходится.

  2)$\sqrt[k]{a_k} \ge 1 ~~~ a_k \ge 1$ $\sum_{k=1}^{\infty} a_k$ расходится.
  Так как нарушено необходимое условие сходимости чилсового ряда.
\end{proof}

\begin{block}[Признак Коши в предельной форме]
  $$
  \forall k \in N ~~~ a_k \ge 0 ~~~ \exists \lim_{k \to \infty} \sqrt[k]{a_k} =
  \lambda
  $$

  1. Если $\lambda < 1 ~~~ \sum_{k=1}^{\infty} a_k$ сходится.

  2. Если $\lambda > 1 ~~~ \sum_{k=1}^{\infty} a_k$ расходится.

  3. Если $\lambda = 1$ может расходится или сходится.
\end{block}

\begin{title}[\Large]
  Признак Лейбница сходимости знакочередующихся рядов и его следствия
\end{title}

\begin{block}[Признак Лейбница:]
  $\forall k \in N ~~~ a_k > 0 ~~~ a_k \searrow ~~~ a_k \to 0$ тогда
  $\sum_{k=1}^{\infty} (-1)^{k+1} a_k$ сходится.
\end{block}

\begin{proof}
  $S_{2n} = a_1 - a_2 + a_3 - a_4 + \ldots + a_{2n-1} - a_{2n}$

  $S_{2n+2} - S_{2n} = -a_{2n+2} + a_{2n+1} > 0$

  $S_{2n+2} > S_{2n}$

  $S_{2n} = a_1 - (a_2 - a_3) - (a_4 - a_5) - \ldots - a_{2n} \le a_1$
  $$
  \lim_{n \to \infty} S_{2n} = S ~~~ \lim_{n \to \infty} S_{2n+1} =
  \lim_{n \to \infty} (S_{2n} + a_{2n+1}) = S + 0 = S \Rightarrow
  \lim_{n \to \infty} S_n = S
  $$
\end{proof}

\begin{block}[Следствие:]
  Выполнены все условия Лейбница

  $\forall n \in N ~~~ S_{2n} \le S \le S_{2n+1} ~~~ |S_n - S| \le a_{n+1}$
\end{block}

\begin{proof}
  $S_{2n+1} \searrow ~~~ S_{2n+1} - S_{2n-1} = a_{2n+1} - a_{2n} < 0$

  $S_{2n-1} - a_{2n} \le S \le S_{2n} + a_{2n+1}$

  $0 \le S_{2n-1} - S \le a_{2n} ~~~ $
  $0 \le S - S_{2n} \le a_{2n+1}$
\end{proof}

\begin{title}[\Large]
  Признаки Дирехле и Абеля сходимости числовых рядов
\end{title}

\begin{block}[Признак Дирехле:]
  1) $\sum_{k=1}^{\infty} a_k ~~~ \exists M > 0 ~~~ \forall n \in N ~~~
  |\sum_{k=1}^n a_k| \le M$

  2) $b_k \nearrow$ или $\searrow$

  3) $b_k \to 0$

  тогда $\sum_{k=1}^{\infty} a_k b_k$ сходится.
\end{block}

\begin{block}[Признак Абеля:]
  1) $\sum_{k=1}^{\infty} a_k$ сходится.

  2) $b_k \nearrow$ или $\searrow$

  3) $\exists M > 0 ~~~ \forall k \in N ~~~ |b_k| \le M$

  тогда $\sum_{k=1}^{\infty} a_k b_k$ сходится.
\end{block}

\begin{proof}
  $\lim_{k \to \infty} b_k = p ~~~ b_k^* = b_k - p \to 0$

  так как $a_k$ сходится то $| \sum_{k=1}^n a_k | \le \Delta$
  $$
  \sum_{k=1}^{\infty} a_k b_k^* = \sum_{k=1}^{\infty} a_k (b_k - p) =
  \sum_{k=1}^{\infty} a_k b_k - p \sum_{k=1}^{\infty} a_k
  $$
  $$
  \sum_{k=1}^{\infty} a_k b_k = \sum_{k=1}^{\infty} a_k b_k^* +
  p\sum_{k=1}^{\infty} a_k
  $$
\end{proof}

\begin{title}[\Large]
  Абсолютно сходящиеся числовые ряды. Простейшие свойства
\end{title}

\begin{defin}[абсолютно сходящегося ряда]
  Числовой ряд $\sum_{k=1}^{\infty} a_k$ называется \kv{абсолютно сходящимся},
  если $\sum_{k=1}^{\infty} |a_k|$ сходится.
\end{defin}

\begin{block}[Свойство]
  Если ряд абсолютно сходится, то он ведет себя как обычная конечная суммы.
\end{block}

\begin{theorem}
  Абсолютно сходящийся ряд является сходящимся
\end{theorem}

\begin{proof}
  $$
  \forall \varepsilon > 0 ~~~ \exists \delta_{\varepsilon} \in N ~~~
  \forall n \ge n_{\varepsilon} ~~~ \forall p \in N ~~~
  \sum_{k=n+1}^{n+p} |a_k| < \varepsilon
  $$
\end{proof}

\begin{theorem}
  $a_k$ абсолютно сходящийся, $b_k$ ограничена

  $\exists M > 0 ~~~ \forall k \in N ~~~ |b_k| \le M$ тогда
  $\sum_{k=1}^{\infty} a_k b_k$ абсолютно сходящийся.
\end{theorem}

\begin{proof}
  $$
  \forall \varepsilon > 0 ~~~ \exists n_{\varepsilon} \in N ~~~
  \forall n \ge n_{\varepsilon} ~~~ \forall p \in N ~~~
  \sum_{k=n+1}^{n+p} |a_k| < \frac{\varepsilon}{M}
  $$
  $$
  \sum_{k=n+1}^{n+p} |a_k b_k| \le
  M \sum_{k=n+1}^{n+p} |a_k| < M \frac{\varepsilon}{M} = \varepsilon
  $$
\end{proof}

\begin{theorem}
  $\sum_{k=1}^{\infty} a_k$ абсолютно сходится

  $\sum_{k=1}^{\infty} b_k$ абсолютно сходится

  $\alpha, \beta \in R$ тогда
  $$
  \sum_{k=1}^{\infty} ( \alpha a_k + \beta b_k )
  $$
  абсолютно сходится.
\end{theorem}

\begin{proof}
  По критерию Коши
  $$
  \forall \varepsilon > 0 ~~~ \exists n_{\varepsilon} \in N ~~~
  \forall n \ge n_{\varepsilon} ~~~ \forall p \in N ~~~
  \sum_{k=n+1}^{n+p} |a_k| < \frac{\varepsilon}{2|\alpha|} ~~~
  \sum_{k=n+1}^{n+p} |b_k| < \frac{\varepsilon}{2|\beta|}
  $$
  $$
  \sum_{k=n+1}^{n+p} |\alpha a_k + \beta b_k| \le
  \sum_{k=n+1}^{n+p} |\alpha a_k| + \sum_{k=n+1}^{n+p} |\beta b_k| =
  $$
  $$
  = |\alpha| \sum_{k=n+1}^{n+p} |a_k| + |\beta| \sum_{k=n+1}^{n+p} |b_k| <
  \frac{1}{2} \varepsilon + \frac{1}{2} \varepsilon = \varepsilon
  $$
\end{proof}

\begin{title}[\Large]
  Перестановка членов в абсолютно сходящихся рядах и перемножение абсолютно
  сходящихся рядов
\end{title}

\begin{theorem}
  $\sum_{k=1}^{\infty} a_k$ абсолютно сходится, то
  $\sum_{\varphi=1}^{\infty} b_{\varphi}$ полученный из ряда $a_k$ перестановкой
  членов ряда $a_k$ ($b_{\varphi} = a_{k_{\varphi}}$) тоже является абсолютно
  сходящейся и $\sum_{k=1}^{\infty} a_k = S \Rightarrow
  \sum_{\varphi=1}^{\infty} b_{\varphi} = S$
\end{theorem}

\begin{theorem}
  $\sum_{k=1}^{\infty} a_k = A$ абсолютно сходящийся

  $\sum_{k=1}^{\infty} b_k = B$ абсолютно сходящийся

  тогда $\sum_{i=1}^{\infty} a_{k_i} b_{\varphi_i} = A \cdot B$ тоже является
  абсолютно сходящимся.
\end{theorem}

\begin{proof}
  $$
  \overline{S_n} = \sum_{i=1}^n |a_{k_i} b_{\varphi_i}| \le
  \sum_{i=1}^n |a_{k_i}| \cdot \sum_{i=1}^n |b_{\varphi_i}| \le
  \overline{A} \cdot \overline{B}
  $$
  $$
  \overline{A} = \sum_{k=1}^{\infty} |a_k| ~~~
  \overline{B} = \sum_{k=1}^{\infty} |b_k|
  $$
  $\sum_{i=1}^n |a_{k_i} b_{\varphi_i}|$ сходится, тогда
  $\sum_{i=1}^{\infty} a_{k_i} b_{\varphi_i}$ сходится

  Расмотрим $S_{n^2} = A_n \cdot B_n$
  $$
  A_n = \sum_{k=1}^n a_k ~~~ B_n = \sum_{k=1}^n b_k
  $$
  $$
  \lim_{n \to \infty} S_{n^2} = \lim_{n \to \infty} A_n \cdot
  \lim_{n \to \infty} B_n = A \cdot B \Rightarrow \lim_{n \to \infty} S_n =
  A \cdot B
  $$
\end{proof}

\begin{title}[\Large]
  Условно сходящиеся ряды. Теорема Римана
\end{title}

\begin{defin}[условно сходящегося числового ряда]
  Числовой ряд называют условно сходящимся, если ряд $\sum_{k=1}^{\infty} a_k$
  сходится, а $\sum_{k=1}^{\infty} |a_k|$ расходится
\end{defin}

\begin{block}[Свойство]
  $\sum_{k=1}^{\infty} a_k$ условно сходящийся ряд

  $\sum_{k=1}^{\infty} |a_k|$ расходящийся ряд

  $a_k$ разного знака
  $$
  \alpha_k = \frac{|a_k| + a_k}{2} \ge 0 ~~~
  \beta_k = \frac{|a_k| - a_k}{2} \ge 0
  $$
  $$
  \alpha_k =
  \left\{
  \begin{array}{ll}
    a_k; & a_k > 0 \\
    0; & a_k \le 0
  \end{array}
  \right.
  ~~~~~~~
  \beta_k =
  \left\{
  \begin{array}{ll}
    0; & a_k \ge 0 \\
    -a_k; & a_k < 0
  \end{array}
  \right.
  $$
  $$
  \left\{
  \begin{array}{ll}
    2\alpha_k & = |a_k| + a_k \\
    2\beta_k & = |a_k| - a_k \\
  \end{array}
  \right.
  ~~~~~~~
  \left\{
  \begin{array}{ll}
    |a_k| & = \alpha_k + \beta_k \\
    a_k & = \alpha_k - \beta_k
  \end{array}
  \right.
  $$
\end{block}

\begin{theorem}
  $\sum_{k=1}^{\infty} \alpha_k$ условно сходящийся
  $$
  \lim_{k \to \infty} \alpha_k = \lim_{k \to \infty} \beta_k = 0
  $$
  тогда $\sum_{k=1}^{\infty} \beta_k$ расходится
\end{theorem}

\begin{proof}
  $\sum_{k=1}^{\infty} \alpha_k$ сходится, тогда
  $$
  \sum_{k=1}^{\infty} \alpha_k =
  \sum_{k=1}^{\infty} a_k + \sum_{k=1}^{\infty} \beta_k
  $$
  $\sum_{k=1}^{\infty} \frac{|a_k| + a_k}{2}$ сходится

  $\sum_{k=1}^{\infty} \frac{|a_k|}{2} + \sum_{k=1}^{\infty} \frac{a_k}{2}$
  противоречие

  Пусть $\sum_{k=1}^{\infty} \beta_k$ сходится
  $|a_k| = 2\beta_k + a_k$

  $\sum_{k=1}^{\infty} \frac{|a_k| - a_k}{2}$ сходится
  $$
  \sum_{k=1}^{\infty} |a_k| =
  2 \sum_{k=1}^{\infty} \beta_k +
  \sum_{k=1}^{\infty} a_k
  $$
\end{proof}

\begin{theorem}[Римана]
  $\sum_{k=1}^{\infty} a_k$ условно сходится, то $\forall \Delta \in R $ можно
  так представить члены ряда $a_k$ и получить ряд $b_{\varphi}$
  $$
  \sum_{\varphi=1}^{\infty} b_{\varphi} = \Delta
  $$
\end{theorem}

\begin{theorem}
  $\sum_{k=1}^{\infty} a_k$ абсалютно сходится, то
  $$
  \sum_{k=1}^{\infty} b_k ~~~~~~ \sum_{k=1}^{\infty} (a_k + b_k)
  $$
  либо оба расходятся

  либо оба абсалютно сходятся

  либо оба условно сходятся
\end{theorem}

\begin{proof}
  1) Пусть $\sum_{k=1}^{\infty} b_k$ расходится, тогда предположим, что

  $\sum_{k=1}^{\infty} (a_k + b_k)$ сходится

  $\sum_{k=1}^{\infty} (a_k + b_k) - \sum_{k=1}^{\infty} a_k$ противоречие

  2) Пусть $\sum_{k=1}^{\infty} (a_k + b_k)$ расходится, тогда докажем
  $\sum_{k=1}^{\infty} b_k$ расходится

  Предполжим, что $\sum_{k=1}^{\infty} (a_k + b_k) =
  \sum_{k=1}^{\infty} a_k + \sum_{k=1}^{\infty} b_k$

  2.1 Пусть $\sum_{k=1} b_k$ абсолютно сходится, тогда
  $\sum_{k=1}^{\infty} |b_k|$ сходится

  $\sum_{k=1}^{\infty} (a_k + b_k)$ абсалютно сходится

  $|a_k + b_k| \le |a_k| + |b_k|$

  2.2 Пусть $\sum_{k=1}^{\infty} (a_k + b_k)$ абсалютно сходится, тогда
  $\sum_{k=1}^{\infty} b_k$ абсалютно сходится

  $|b_k| = |b_k + a_k - a_k| \le | b_k + a_k| + |a_k|$
\end{proof}