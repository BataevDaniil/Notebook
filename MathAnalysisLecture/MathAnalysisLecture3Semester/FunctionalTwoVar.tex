\begin{title}
  Дифференциорование функций многих переменных
\end{title}

\begin{title}[\Large]
  Метрическое пространство. Пространства $R^n$
\end{title}

\begin{define}[метрического пространства]
  Метрическое пространство это множество $M$, на котором определена функция
  $(x, y) \to \rho(x, y) \ge 0$

  Аксиомы:

  1) $\forall x,y \in M ~~~ \rho(x,y) = 0$ тогда и только тогда когда $x = y$

  2) $\forall x,y \in M ~~~ \rho(x,y) =\rho(y,x)$

  3) $\forall x,y,z \in M ~~~ \rho(x,y) \le \rho(x,y) + \rho(z,y)$
  (неравества треугольника)

  Отображение с такими свойствами называется метрикой.

  Значение $\rho(x,y)$ это растояние между $x$ и $y$.
\end{define}

\begin{block}[Неравенство Коши]
  $$
  \left( \sum_{k=1}^n a_k b_k \right)^2 \le \sum_{k=1}^n a_k^2 \cdot
  \sum_{k=1}^n b_k^2
  $$
\end{block}

\begin{proof}
  $$
  P(t) = \sum_{k=1}^n (a_k + t b_k)^2 = \sum_{k=1}^n a_k^2 +
  2t\sum_{k=1}^n a_k b_k + t^2 \sum_{k=1}^n b_k^2
  $$
  $$
  D = 4\left( \sum_{k=1}^n a_k b_k \right)^2 - 4\sum_{k=1}^n a_k^2 \cdot
  \sum_{k=1}^n b_k^2 \le 0
  $$
  $$
  \left( \sum_{k=1}^n a_k b_k \right)^2 \le \sum_{k=1}^n a_k^2 \cdot
  \sum_{k=1}^n b_k^2
  $$
\end{proof}

\begin{block}[Неравенство Минского]
  $$
  \sqrt{\sum_{k=1}^n (a_k + b_k)^2} \le \sqrt{\sum_{k=1}^n a_k^2} +
  \sqrt{\sum_{k=1}^n b_k^2}
  $$
\end{block}

\begin{proof}
  $$
  \sum_{k=1}^n (a_k + b_k)^2 \sum_{k=1}^n a_k^2 +
  2\sum_{k=1}^n a_k b_k + \sum_{k=1}^n b_k^2 \le
  $$
  $$
  \le \sum_{k=1}^n a_k^2 + 2\sqrt{\sum_{k=1}^n a_k^2} \sqrt{\sum_{k=1}^n b_k^2}
  + \sum_{k=1}^n b_k^2 \le
  $$
  $$
  \le \left( \sqrt{\sum_{k=1}^n a_k^2} \right)^2 + 2\sqrt{\sum_{k=1}^n a_k^2}
  \sqrt{\sum_{k=1}^n b_k^2} + \left( \sqrt{\sum_{k=1}^n b_k^2} \right)^2 =
  $$
  $$
  = \left( \sqrt{\sum_{k=1}^n a_k^2} + \sqrt{\sum_{k=1}^n b_k^2} \right)^2
  $$
  $$
  \sqrt{\sum_{k=1}^n (a_k + b_k)^2} \le \sqrt{\sum_{k=1}^n a_k^2} +
  \sqrt{\sum_{k=1}^n b_k^2}
  $$
\end{proof}

\begin{title}[\Large]
  Сходимость в метрических пространствах
\end{title}

\begin{define}[предела в метрическом пространстве]
  $M$ с метрикой $\rho$. Последовательность $x_k \in M$ называется сходящейся
  к элементу $p \in M ~~~ \lim_{k \to \infty} \rho(x_k, \rho) = 0$
\end{define}

\begin{define}[окрестности в метрическом пространстве]
  Окрестностью точки $p \in M$ называют
  $$
  \{ x \in M: ~ \rho(x, p) < \varepsilon \} = O_{\varepsilon}(p)
  $$
  $$
  \{ x \in M: ~ 0 < \rho(x, p) < \varepsilon \} =
  \stackrel{\bullet}{O}_{\varepsilon}(p)
  $$
\end{define}

Если дано метрическое пространство, то сложенное подпротсранство тоже является
метрическим.

\begin{theorem}
  Если $x_k$ сходится в $\rho \in R^n ~ n \in N$ тогда $x_k \le M$
\end{theorem}

\begin{theorem}
  Если $x_k$ сходится то она имеет не более одного предела.
\end{theorem}

\begin{proof}
  Допустим предела два $x_k \to p$ $x_k \to q$
  $$
  0 < \rho(p, q) \le \rho(p, x_k) + \rho(x_k, q) = \rho(x_k, p) + \rho(x_k, q)
  $$
\end{proof}

\begin{theorem}
  Для того чтобы в метрическом прострастве $R^n$ последовательность
  $$
  x^{(k)} = (x_1^{(k)}, x_2^{(k)}, \ldots, x_n^{(k)}) \to
  p = (p_1, p_2, \ldots, p_n)
  $$
  необходимо и достаточно чтобы
  $$
  \left\{
  \begin{array}{c}
    x_1^{(k)} \to p_1 \\
    x_2^{(k)} \to p_2 \\
    \cdots \cdots \\
    x_n^{(k)} \to p_n
  \end{array}
  \right.
  $$
\end{theorem}

\begin{proof}
  I Пусть $x^{(k)} \to p$, тогда $|x_i^{(k)} - p_i| \le \rho(x^{(k)}, p) \to 0
  ~~~ i = 1, 2, \cdots, n$

  II Пусть все условия выполнены $\rho(x^{(k)}, p)$ - б/м тогда $x^{(k)} \to p$
\end{proof}

\begin{define}[фундаментальной последовательности]
  $x_k$ метрического пространства $M$ называесятся фундаментальной, если
  $$
  \forall \varepsilon > 0 ~~~ \exists n_{\varepsilon} \in N ~~~
  \forall k \ge n_{\varepsilon} ~~~ \forall m \in M ~~~ \rho(x_k - x_{k+m}) <
  \varepsilon
  $$
\end{define}

\begin{theorem}
  Если $x_k$ сходится, то она фундаментальна.
\end{theorem}

\begin{proof}
  $$
  x_k \to p ~~~ \forall \varepsilon > 0 ~~~ \exists n_{\varepsilon} \in N ~~~
  \forall k \ge n_{\varepsilon} ~~~ \rho(x_k, p) < \frac{\varepsilon}{2}
  $$
  $$
  ~~~~~~~~~~~~~~~~~~~~~~~~~~~~~~~~~~~~~~~~~~~~
  \forall m \ge n_{\varepsilon} ~~~ \rho(x_{k + m}, p) < \frac{\varepsilon}{2}
  $$
  тогда
  $$
  \rho(x_k, x_{k+m}) \le \rho(x_k, p) + \rho(x_{n+k}, p)
  < \frac{\varepsilon}{2} + \frac{\varepsilon}{2} = \varepsilon
  $$

  Замечание: обратное утверждение не верно.
\end{proof}

\begin{define}[полного метрического пространства]
  Метрическое пространства, в которых любая фундаментальная последовательность
  сходится называется полным.
\end{define}

\begin{theorem}
  $R^n$ метрическое пространство с евклидовой метрикой является полным.
\end{theorem}

\begin{proof}
  $(x_k, y_k, z_k)$ - пусть последовательность фундаменталная

  $x_k, y_k, z_k$ - также фундаменталные

  $$
  |x_{k+m} - x_k| \le \rho(x_k, y_k, z_k)
  $$
  $$
  \rho(x_{k+m}, y_{k+m}, z_{k+m}) < \varepsilon
  $$
  $$
  |y_{k+m} - y_k| < \varepsilon
  $$
  $$
  |z_{k+m} - z_k| < \varepsilon
  $$
  $x_k \to p_1 ~~~ y_k \to p_2 ~~~ z_k \to p_3$ имеют пределы по критерию Коши

  $(x_k, y_k, z_k) = (p_1, p_2, p_3)$ является сходящейся $\Rightarrow$ $R^n$
  - является полным.
\end{proof}

\begin{define}[внутреней точки метрического пространства]
  Точка $a$ называется внутренней точкой $D \subset M$ метрического
  пространства если  $\exists O_{\varepsilon}(a) \subset D$
\end{define}

\begin{define}[открытого множества]
  Множество из внутренних точек называют открытым множеством.
\end{define}

\begin{block}[Свойства]
  1) $\cup D_i ~ i \in I = D$ - объеденение любого числа открытых множеств
  является открытым множеством.

  \begin{proof}
    $a \in D ~~ \Rightarrow ~~ \exists i_0 \in I ~~ a \in D_{i_0}$

    $\exists O_{\varepsilon}(a) \subset D_{c_0} \subset \cup D_i ~ i \in I$
  \end{proof}

  2) $\cap D_i = D ~ i = 1, 2, \ldots, m$ пересечение конечного числа открытых
  множеств является открытым множеством.

  \begin{proof}
    $a \in D ~~~ \forall i = 1, 2, \ldots, m$
    что то неясное
  \end{proof}
\end{block}

\begin{define}[предельной точки]
  Точка $a$ называется предельной точкой множества $E \subset M$ если
  $\forall \stackrel{\bullet}{O}(a) E \not= \oslash$
\end{define}

\begin{define}[замкнутого множества]
  Если множество содержит все свои предельные точки, то оно называется
  замкнутым.
\end{define}

\begin{theorem}
  Для того, чтобы множество $E \subset M$ было замкнутым необходимоо и
  достаточно, чтобы $D = M\backslash E$ его дополнение было открытым.
\end{theorem}

\begin{proof}
  I Пусть $E$ - замкнутое. Предположим что $D = M \backslash E$ не является
  открытым $\exists a \in D$ $a$ - не внутренняя. В $\forall
  \stackrel{\bullet}{O}(a) \cap \not= \oslash$ тоисть $a$ - предельная, но $E$
  замкнуто $\Rightarrow a \in E$. Не может быть так как дополнение $E$\\

  II Пусть $D$ - открытое. Пусть $E$ не замкнутое, тогда есть хотя бы одна
  точка, которая $a \not\in E$ и $a$ предельная для $E$ $\Rightarrow \forall
  \stackrel{\bullet}{O}_{\varepsilon}(a) \cap E \not= \oslash$. $a$ - не может
  быть внутренней точкой.
\end{proof}

\begin{block}[Свойства]
  1) $\cup E_i = E ~ i = 1,2, \ldots, m$ - объединение конечного числа
  замкнутых множеств, есть множество замкнутое.

  2) $\cap E_i = E ~ i \in I$ - пересечение любого числа замкнутых множеств,
  есть множество замкнутое.

  Законы Деморгана

  $M \backslash (A \cup B) = (M \backslash A) \cap (M \backslash B)$

  $M \backslash (\cup A_i) = \cap (M \backslash A_i)$

  $M \backslash (\cap A_i) = \cup (M \backslash A_i)$

  \begin{proof}
    $\cup E_i ~ i = 1,2, \ldots, m$ - замкнутое

    $M \backslash (\cup E_i) ~ i = 1,2, \ldots, m$ - открытое

    $\cap (M \backslash E_i) ~ i = 1,2, \ldots, m$ открытое
  \end{proof}
\end{block}

\begin{title}[\Large]
  Компактные множества. Границы множества
\end{title}

\begin{define}[компактного множества]
  Множество $K \subset M$ называется компактным, если из $\forall x_m$ можно
  извлечь подпоследовательность $x_{m_k}$ которая сходится к элементу данного
  множества.
\end{define}

\begin{theorem}
  Компактное множество является ограниченным.
\end{theorem}

\begin{proof}
  Пусть $K$ - неограничена $x_1 \in K$

  $\exists x_2 \in K ~~~ \rho(x_2, x_1) \ge 1$

  $\exists x_3 \in K ~~~ \rho(x_3, x_1) \ge 1$ и $\rho(x_3, x_2) \ge 1$ и так
  далее.

  Расиотрим последовательность $x_k \to p$, то она была бы фундаментальной для
  членов с достаточно большими номерами растояние должно быть $< \varepsilon$
\end{proof}

\begin{theorem}
  Компактное множество в любом метрическом пространстве является замкнутым.
\end{theorem}

\begin{proof}
  Пусть $K$ - незамкунутое $a$ - предельная точка, $a \not\in K ~~~
  x_k \in \stackrel{\bullet}{O}_{\frac{1}{k}}(a)$, который $x_k \in K$

  Все элементы сходятся к $a$, а $a$ не входит в $K$. Противоречие
\end{proof}

\begin{theorem}[Больцана-Вейерштраса для $R^n$]
  Для простоты $n = 3$ тогда $(x_k, y_k, z_k) ~ k \in N$

  $(x_{\varphi_{s_i}}, y_{\varphi_{s_i}}, z_{\varphi_{s_i}}) \to
  (p_1, p_2, p_3)$
\end{theorem}

\begin{theorem}
  Для того чтобы множество $K \in R^n$ было компактным необходимо и достаточно
  чтобы оно было замкнутым и ограниченным.
\end{theorem}

\begin{proof}
  $R^n$ $x_k \in K$ ограничена. Из теоремы 3 $x_{m_i} \to a \in R^n$ $a$ -
  предельная точка из $K$. Так как $K$ - замкнуто, $a \in K$
\end{proof}

\begin{title}[\Large]
  Граница множества
\end{title}

\begin{define}[граничной точки множества]
  Точка называется граничной точкой множества $E \subset M$, если
  $\forall O_{\varepsilon}(b)\cap E \not= 0$

  $\partial E$ - множества всех граничных точек множества $E$
\end{define}

\begin{define}
  $$
  \text{Отрезком} ~~~ [a,b] = \{ x = (x_1, x_2, \ldots, x_n): ~
  x_i = a_i(1 - t) + b_it \}
  $$
  $$
  \text{Прямая} ~~~ \{ x = (x_1, x_2, \ldots, x_n): ~ x_i = a_i(1 - t) +
  tb_i ~~~ t \in R \}
  $$
  $$
  \text{Луч} ~~~ \frac{\varphi(a)}{e} = \{ x: ~ x_i = a_i + te_i ~~~
  t \in [0, + \infty\}
  $$
  Множество называется выпуклым, если любые две точки которые принадлежат этому
  множеству, можно провести отрезок так что он будет принадлежать этомуже
  множеству.
\end{define}

\begin{title}[\Large]
  Предел функции многих переменных. Различные типы пределов функций многих
  переменных
\end{title}