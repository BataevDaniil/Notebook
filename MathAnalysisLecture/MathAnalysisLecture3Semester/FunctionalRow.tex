\begin{title}
  Функциональные ряды
\end{title}

\begin{title}[\Large]
  Сходимость и равномерная сходимость функциональной последовательности и
  функционального ряда
\end{title}

\href{http://nuclphys.sinp.msu.ru/mathan/p3/m3102.html}{Информация по теме}

\begin{define}[функциональной последовательности]
  $f_n(x)$ определена на $E$ тогда если
  $$
  \forall x \in E ~~~ \lim_{n \to \infty} f_k(x) = f(x)
  $$
  то есть
  $$
  \forall \varepsilon > 0 ~~~ \forall x \in E ~~~ \exists n_{\varepsilon; x}
  \in N ~~~ \forall n \ge n_{\varepsilon; x} ~~~ \left| f_n(x) - f(x) \right|
  < \varepsilon
  $$
  то говорят что функциональная последовательность сходится на $E$ и равна
  $f(x)$ и обозначают
  $$
  f_n(x) \stackrel{E}{\to} f(x)
  $$
\end{define}

\begin{define}[равномерной сходимости функциональной последовательности]
  $f_n(x)$ определена на $E$ тогда если
  $$
  \forall \varepsilon > 0 ~~~ \exists n_{\varepsilon} \in N ~~~ \forall x \in E
  ~~~ \forall n \ge n_{\varepsilon} ~~~ |f_n (x) - f(x)| < \varepsilon
  $$
  то говорят что функциональная последовательность сходится на $E$ равномерно и
  обозначают
  $$
  f_n(x) \stackrel{E}{\rightrightarrows} f(x)
  $$
  Кудрявцев 3 том 71 страница
\end{define}

\begin{define}[сходимости функционального ряда]
  $f_n(x)$ определена на $E$ тогда если
  $$
  \forall x \in E ~~~ \lim_{n \to \infty} \sum_{k=1}^n f_k(x) =
  \lim_{n \to \infty} S_n(x) = S(x)
  $$
  то есть $\forall x \in E$ числово ряд $f_n(x)$
  сходится

  то есть
  $$
  \forall \varepsilon > 0 ~~~ \forall x \in E ~~~ \exists n_{\varepsilon; x}
  \in N ~~~ \forall n \ge n_{\varepsilon; x} ~~~ \left| S_n(x) - S(x) \right|
  < \varepsilon
  $$
  то говорят что функциональный ряд сходится на $E$ и его суммa равна $S(x)$ и
  обозначают
  $$
  S_n(x) \stackrel{E}{\to} S(x)
  $$
\end{define}

\begin{define}[равномерной сходимости функционального ряда]
  $f_n(x)$ определена на $E$ тогда если
  $$
  \forall \varepsilon > 0 ~~~ \exists n_{\varepsilon} \in N ~~~ \forall x \in E
  ~~~ \forall n \ge n_{\varepsilon} ~~~ |S_n (x) - S(x)| < \varepsilon
  $$
  то говорят что функциональный ряд сходится на $E$ равномерно и обозначают
  $$
  S_n(x) \stackrel{E}{\rightrightarrows} S(x) ~~~
  S(x) - S_n(x) \stackrel{E}{\rightrightarrows} 0 ~~~
  r_n(x) \stackrel{E}{\rightrightarrows} 0
  $$
  $S_n(x)$ называется неравномерной сходимостью на множестве $E$ если $S_n(x)$
  сходится $\forall x \in E$ но условие равномерной сходимости нарушено.
\end{define}

\begin{title}[\Large]
  Критерии равномерной сходимости функциональной последовательности и
  функционального ряда
\end{title}

\begin{block}[Критерий Коши равномерной сходимости функциональной
              последовательности]
  $f_n(x)$ определена на $E$ тогда если
  $$
  \forall \varepsilon > 0 ~~~ \exists n_{\varepsilon} \in N ~~~
  \forall x \in E ~~~ \forall n \ge n_{\varepsilon} ~~~ \forall p \in N ~~~
  |f_{n+p}(x) - f_n(x)| < \varepsilon
  $$
  тогда $f_n(x) \stackrel{E}{\rightrightarrows}$

  Кудрявцев 3 том 77 страница
\end{block}

\begin{proof}
  $$
  \forall \varepsilon > 0 ~~~ \exists n_{\varepsilon} \in N ~~~
  \forall x \in E ~~~ \forall n \ge n_{\varepsilon} ~~~
  |f_n(x) - f(x)| < \varepsilon
  $$
  $$
  |f_{n+p}(x) - f_n(x)| = |f_{n+p}(x) - f(x) + f(x) - f_n(x)| \le
  $$
  $$
  \le |f_{n+p}(x) - f(x)| + |f(x) - f_n(x)| < \frac{\varepsilon}{2} +
  \frac{\varepsilon}{2} = \varepsilon
  $$
\end{proof}

\begin{block}[Критерий равномерной сходимости функциональной последовательности]
  $f_n(x)$ определена на $E$ тогда если
  $$
  \lim_{n \to \infty} \sup\limits_{x \in E} |f_n(x) - f(x)| = 0 ~~ \text{тогда}
  ~~~ f_n(x) \stackrel{E}{\rightrightarrows} f(x)
  $$
\end{block}

\begin{block}[Критерий Коши равномерной сходимости функционального ряда]
  $f_n(x)$ определена на $E$ тогда если
  $$
  \forall \varepsilon > 0 ~~~ \exists n_{\varepsilon} \in N ~~~
  \forall x \in E ~~~ \forall n \ge n_{\varepsilon} ~~~ \forall p \in N ~~~
  |S_{n+p}(x) - S_n(x)| = \left| \sum_{k = n + 1}^{n+p} f_k(x) \right|
  < \varepsilon
  $$
  тогда $S_n(x) \stackrel{E}{\rightrightarrows}$
\end{block}

\begin{block}[Критерий равномерной сходимости функционального ряда]
  $f_n(x)$ определена на $E$ тогда если
  $$
  \lim_{n \to \infty} \sup\limits_{x \in E} |S_n(x) - S(x)| = 0 ~~ \text{тогда}
  ~~~ S_n(x) \stackrel{E}{\rightrightarrows} S(x)
  $$
\end{block}

\begin{title}[\Large]
  Признак Вейерштрасcа равномерной сходимости функционального ряда
\end{title}

\begin{block}[Признак Вейерштраcса]
  $f_n(x)$ определена на $E$ $a_k \ge 0 ~~ \sum_{k=1}^{\infty}a_k$ cходится
  тогда если
  $$
  \forall x \in E ~~~ \forall k \in N~~~ |f_k(x)| \le a_k ~ \text{тогда} ~
  S_n (x) \stackrel{E}{\rightrightarrows} ~ \text{и абсолютно}
  $$
  Кудрявцев 3 том 77 страница
\end{block}

\begin{proof}
  Из сходимости ряда $a_n$ по критерию Коши следует
  $$
  \forall \varepsilon > 0 ~~~ \exists n_{\varepsilon} \in N ~~~ \forall n \ge
  n_{\varepsilon} ~~~ \forall p \in N ~~~ \left| \sum_{k=n+1}^{n+p} a_k \right|
  < \varepsilon
  $$
  тогда из условия теоремы
  $$
  \left| \sum_{k = n + 1}^{n+p} f_k(x) \right| \le
  \sum_{k = n + 1}^{n+p} |f_k(x)| \le \sum_{k = n+1}^{n+p} a_k < \varepsilon ~
  \Rightarrow ~ S_n(x)\stackrel{E}{\rightrightarrows}
  $$
\end{proof}

\begin{title}[\Large]
  Признаки Дирехле и Абеля равномерной сходимости функционального ряда
\end{title}

\begin{block}[Признак Дирихле]
  1) $\exists M > 0 ~~ \forall x \in E ~~ \forall n \in N ~~~ |F_n(x)| \le M$

  2) $g_k(x) ~~~ \searrow$ или $\nearrow$

  3) $G_k(x) \stackrel{E}{\rightrightarrows} 0$

  тогда $\sum_{k=1}^{\infty} f_k(x) \cdot g_k(x)\stackrel{E}{\rightrightarrows}$

  Кудрявцев 3 том 87 страница
\end{block}

\begin{block}[Признак Абеля]
  1) $\exists M > 0 ~~~ \forall x \in E ~~~ \forall k \in N ~~~ |g_k(x)| \le M$

  2) $g_k(x) ~~~ \searrow$ или $\nearrow$

  3) $F_k(x) \stackrel{E}{\rightrightarrows}$

  тогда $\sum_{k=1}^{\infty} f_k(x) \cdot g_k(x)\stackrel{E}{\rightrightarrows}$

  Кудрявцев 3 том 89 страница
\end{block}

\begin{title}[\Large]
  Непрерывность суммы функционального ряда
\end{title}

\begin{theorem}
  1) $\forall k \in N ~~~ f_k(x)$ непрерывна на $[a,b]$

  2) $S_k (x) \stackrel{[a,b]}{\rightrightarrows} S(x)$

  тогда $S(x)$ непрерывна на $[a,b]$ тоже что и
  $$
  \lim_{x \to x_0} \sum_{n=1}^{\infty} f_n (x) = \sum_{n=1}^{\infty}
  \lim_{x \to x_0} f_n(x)
  $$
  Кудрявцев 3 том 90 страница
\end{theorem}

\begin{proof}
  $$
  x_0 \in [a,b] ~~~ \forall \varepsilon > 0 ~~~ \exists \delta_{\varepsilon} > 0
   ~~~ \forall x \in [a,b] ~~~ |x - x_0| < \delta_{\varepsilon} ~~~
  |S(x) - S(x_0)| < \frac{\varepsilon}{3} ~ \text{этого хочется}
  $$
  $$
  \forall \varepsilon > 0 ~~~ \exists n_{\varepsilon} \in N ~~~ \forall n \ge
  n_{\varepsilon} ~~~ \forall x \in [a,b] ~~~ |S_n(x) - S(x)| <
  \frac{\varepsilon}{3} ~~~ |S_n(x_0) - S(x_0)| < \frac{\varepsilon}{3}
  $$
  $$
  n_0 \ge n ~~~ S_{n_0}(x) = \sum_{k=1}^{n_0} f_k(x) ~~~ \forall \varepsilon > 0
  ~~~ \exists \delta_{\varepsilon} > 0 ~~~ \forall x \in [a,b] ~~~ |x - x_0| <
  \delta_{\varepsilon} ~~~ |S_{n_0}(x) - S_{n_0}(x_0)| < \frac{\varepsilon}{3}
  $$
  $$
  |S(x) - S(x_0)| =  |(S(x) - S_{n_0}(x)) + (S_{n_0}(x) - S_{n_0}(x_0))
  + (S_{n_0}(x_0) - S(x_0))| \le
  $$
  $$
   \le |S_{n_0}(x) - S(x)| +
  |S_{n_0}(x) - S_{n_0}(x_0)| + |S_{n_0}(x) - S(x_0)| <
  $$
  $$
  < \frac{\varepsilon}{3} +
  \frac{\varepsilon}{3} + \frac{\varepsilon}{3} = \varepsilon
  $$
\end{proof}

\begin{title}[\Large]
  Почленное интегрирование функционального ряда
\end{title}

\begin{theorem}
  1) $\forall k \in N ~~ f_k(x)$ непрерывна на $[a,b]$

  2)$S_k(x) \stackrel{[a,b]}{\rightrightarrows} S(x)$

  тогда
  $$
  \forall c \in [a,b] ~~~ \int_a^c S(x)dx =
  \sum_{k=1}^{\infty} \int_a^c f_k(x)dx
  $$
  Кудрявцев 3 том 94 страница
\end{theorem}

\begin{proof}
  $$
  \forall \varepsilon  > 0 ~~~ \exists n_{\varepsilon} \in N ~~~
  \forall n \ge n_{\varepsilon} ~~~ \forall x \in[a,b] ~~~
  |S_n(x) - S(x)| < \frac{\varepsilon}{(b-a)}
  $$
  $$
  \left| \int_a^c S(x)dx - \sum_{k=1}^n \int_a^c f_k(x)dx \right| =
  \left| \int_a^c \left( S(x) - \sum_{k=1}^n f_k(x) \right)dx \right| \le
  $$
  $$
  \le \int_a^c |S(x) - S_n(x)|dx < \frac{\varepsilon}{b-a} \int_a^c dx =
  \frac{\varepsilon}{b-a}(c-a) \le \varepsilon
  $$
\end{proof}

\begin{theorem}
  1) $\forall n \in N ~~~ S_n(x)$ непрерывна на $[a,b]$

  2) $S_n(x) \stackrel{[a, b]}{\rightrightarrows} S(x)$

  тогда
  $$
  \forall c \in [a,b] ~~~ \int_a^c S(x)dx = \lim_{n \to \infty}
  \int_a^c S_n(x)dc
  $$
\end{theorem}

\begin{title}[\Large]
  Почленное дифференциирование функционального ряда
\end{title}

\begin{theorem}
  1) $\forall k \in N ~~~ f_k(x)$ непрерывна дифференцируема на $[a,b]$

  2) $\sum_{k=1}^{\infty} f_k'(x) \stackrel{[a, b]}{\rightrightarrows}$

  3) $\exists! c \in [a,b] ~~ S_k(c) \stackrel{[a,b]}{\rightarrow} S(c)$

  тогда
  $$
  S_k(x) \stackrel{[a, b]}{\rightrightarrows} S(x) ~
  \text{непрерывна дифференцируема и} ~ \sum_{k=1}^{\infty} f_k'(x) = S'(x)
  $$
  Кудрявцев 3 том 97 страница
\end{theorem}

\begin{proof}
  $$
  \sum_{k=1}^{\infty} f_n'(x) \stackrel{[a,b]}{\rightrightarrows} G(x)
  $$
  $$
  \int_a^x G(t)dt = \sum_{k=1}^{\infty} \int_a^x f_k'(t) dt = S(x) - S(a)
  $$
  $$
  S(x) = \int_a^x G(t)dt + S(a)
  $$
  $$
  S'(x) = G(x) + 0
  $$
\end{proof}

\begin{theorem}
  1) $\forall n \in N ~~~ S_n(x)$ непрерывна дефференцируема на $[a,b]$

  2) $S_n'(x) \stackrel{[a,b]}{\rightrightarrows} G(x)$

  3) $S_n(a) \to S(a)$

  тогда $S_n(x) \stackrel{[a,b]}{\rightrightarrows} S(x)$ и $S'(x) = G(x)$
\end{theorem}

\begin{title}[\Large]
  Степенные ряды. Теорема Абеля
\end{title}

\begin{define}
  $$
  \sum_{k=0}^{\infty} a_k (x-x_0)^k ~ \text{- степенной ряд}
  $$
  $$
  \sum_{k=0}^{\infty} a_k t^k ~~~ t = x - x_0
  $$
  $a_k$ - последовательность коэффицентов степенного ряда
\end{define}

\begin{theorem}[Абеля]
  Если $\sum_{k=0}^{\infty} a_k x^k$ сходится в точке $x_1 \not= 0$ тогда
  $\forall x ~~ |x| < |x_1|$ сходится абсолютно.

  Если $\sum_{k=0}^{\infty} a_k x^k$ расходится в точке $x_2 \not= 0$ тогда
  $\forall x ~~ |x| > |x_2|$ расходится

  Кудрявцев 3 том 104 страница
\end{theorem}

\begin{proof}
  1) Пусть $\sum_{k = 0}^{\infty}a_k x_1^k$ cходится тогда на основании
  необходимости условия сходимости числовго ряда
  $$
  \lim_{k \to \infty} a_k x_1^k = 0 ~ \Rightarrow ~ \exists M > 0 ~~
  \forall k \in N \cup \{0\} ~~ |a_k x_1^k| \le M
  $$
  $$
  |a_k x^k| = |a_k x_1^k| \left| \frac{x}{x_1} \right|^k \le
  M \left| \frac{x}{x_1} \right|^k
  $$
  $\left| \frac{x}{x_1} \right| < 1$ то этот ряд будет суммой геометрической
  прогрессии $\Rightarrow \sum_{k=0}^{\infty} M \left| \frac{x}{x_1} \right|^k$
  сходится

  2) Предположим, что $\exists x_1 ~~ |x_1| > |x_2| ~~~
  \sum_{k=0}^{\infty} a_k x_1^k$ сходится тогда ряд должен сходится в $x_2$
  что протеворечит данному условию.
\end{proof}

\begin{block}[Следствие из теоремы Абеля]
  Степенной ряд

  1) сходится только в одной точке

  2) сходится на всей числовой прямой

  3) сходится во всех точках некоторого интеравала, симметричен относительно
  точки $0$
\end{block}

\begin{title}[\Large]
  Радиус сходимости степенного ряда. Формула Коши-Адамара
\end{title}

\begin{define}[радиуса сходимости функционального ряда]
  $\rho = \sup\{ |x| : x \in D \}$ $D \not= 0$ - радиус сходимости

  Свойства

  1) $D = \{0\} ~ \Rightarrow ~ \rho = 0$

  2) $D = R ~ \Rightarrow ~ \rho = +\infty$

  3) $0 < \rho < +\infty$
\end{define}

\begin{theorem}
  $$
  \exists \lim_{k \to \infty} \left| \frac{a_k}{a_{k+1}} \right| = \rho
  $$
  Кудрявцев 3 том 106 страница
\end{theorem}

\begin{proof}
  $$
  \lim_{k \to \infty} \left| \frac{a_{k+1} x^{k+1}}{a_k x^k} \right| =
  |x| \lim_{k \to \infty} \left| \frac{a_{k+1}}{a_k} \right| < 1
  $$
  $$
  |x| < \lim_{k \to \infty} \left| \frac{a_k}{a_{k+1}} \right| ~~~
  \text{- сходится}
  $$
  $$
  |x| > \lim_{k \to \infty} \left| \frac{a_k}{a_{k+1}} \right| ~~~
  \text{- расходится}
  $$
\end{proof}

\begin{theorem}
  $$
  \exists \lim_{k \to \infty} \sqrt[k]{|a_k|} = \frac{1}{\rho}
  $$
  Кудрявцев 3 том 106 страница
\end{theorem}

\begin{proof}
  $$
  \lim_{k \to \infty} \sqrt[k]{|a_k x^k|} = |x| \lim_{k \to \infty}
  \sqrt[k]{|a_k|} < 1
  $$
  $$
  |x| < \lim_{k \to \infty} \frac{1}{\sqrt[k]{a_k}} ~~~ \text{- сходится}
  $$
  $$
  |x| > \lim_{k \to \infty} \frac{1}{\sqrt[k]{a_k}} ~~~ \text{- расходится}
  $$
\end{proof}

\begin{block}[Формула Коши-Адамара]
  $$
  \overline{\lim_{k \to \infty}} \sqrt[k]{|a_k|} =
  \lim_{n \to \infty} \sup_{k > n}\sqrt[k]{|a_k|} = \frac{1}{\rho}
  $$
  Кудрявцев 3 том 108 страница
\end{block}

\begin{title}[\Large]
  Свойства степенных рядов
\end{title}

\begin{theorem}
  Если степенной ряд имеет $\rho > 0$, то он сходится равномерно
  на любом отрезке внутри интеравала сходимости.
\end{theorem}

\begin{block}[Следствие 1]
  $\sum_{k=1} a_k x^k$ сходится на $\rho > 0$ тогда непрерывна на
  $(-\rho, \rho)$
\end{block}

\begin{block}[Следствие 2]
  $\rho > 0 ~~~ x \in (-\rho, \rho)$ степенных рядов можно
  $$
  \int_0^x S(t)dt = \sum_{k=0}^{\infty} a_k \frac{x^{k+1}}{k+1}
  $$
  при этом $\rho_k = \rho$
\end{block}

\begin{proof}
  $$
  \rho_k = \overline{\lim_{k \to \infty}} \sqrt[k]{\frac{|a_{k-1}|}{k}} =
  \overline{\lim_{k \to \infty}} \frac{\sqrt[k]{|a_{k-1}|}}{
  \cancelto{1}{\sqrt[k]{k}}} =
  \overline{\lim_{k \to \infty}} \sqrt[k]{|a_{k-1}|} =
  $$
  $$
  = \overline{\lim_{k \to \infty}} |a_{k-1}|^{\frac{1}{k-1}\frac{k-1}{k}} =
  \overline{\lim_{k \to \infty}} \rho^{\frac{k-1}{k}} = \rho
  $$
\end{proof}

\begin{block}[Следствие 3]
  $S'(x) = \sum_{k=1}^{\infty}k a_k x^{k-1}$ при $\rho_k = \rho$
\end{block}

\begin{proof}
  $$
  \rho_k = \lim_{k \to \infty} \left| \frac{k a_k}{(k+1)a_{k+1}} \right| =
  \lim_{k \to \infty} \left| \frac{k a_k}{(k+1)a_{k+1}} \right| =
  \lim_{k \to \infty} \left| \frac{a_k}{a_{k+1}} \right| = \rho
  $$
\end{proof}

\begin{title}[\Large]
  Ряд Тейлора
\end{title}

\begin{define}
  $f(x)$ имеет $f^{(k)}(a) ~~~ \forall k \in N$ тогда ряд вида
  $$
  \sum_{k=0}^{\infty} \frac{f^{(k)}(a)}{k!} (x-a)^k
  $$
  называют рядом Тейлора в точке $a$

  Если $a = 0$ тогда это ряд Маклорена

  $R_n(x) = f(x) - S_n(x)$ называется формулой Тейлора

  Если $x_0 \in R ~~~ \lim_{n \to \infty} R_n(x_0) = 0$ то ряд Тейлора сходится
  к $f(x_0)$
\end{define}

\begin{theorem}
  $f(x) = \sum_{k=0}^{\infty} c_k(x-a)^k$ тогда $x \in O_{\delta}(a)$
  представимо в виде ряда Тейлора
  $c_k = \frac{f^{(k)}(a)}{k!}$ где $k = 0,1,2, \ldots$

  Замечание: Обратное утверждение не верно.
\end{theorem}

\begin{proof}
  $$
  x = a ~~~ f(x) = c_0
  $$
  $$
  f'(x) = \sum_{k=1}^{\infty} k c_k(x - a)^{k - 1} ~~~ f'(a) = 1 \cdot c_1 ~~~
  c_1 = \frac{f'(a)}{1!}
  $$
  $$
  f''(x) = \sum_{k=2}^{\infty} k(k-1) c_k (x - a)^{k-2} ~~~
  f''(a) = 1 \cdot 2 \cdot c_2 ~~~ c_2 = \frac{f''(a)}{2!}
  $$
  $$
  f''(x) = \sum_{k=3}^{\infty} k(k-1)(k-2) c_k (x - a)^{k-3} ~~~
  f''(a) = 1 \cdot 2 \cdot 3 \cdot c_3 ~~~ c_3 = \frac{f''(a)}{3!}
  $$
  $$
  \cdots ~~~ \cdots ~~~ \cdots ~~~ \cdots ~~~ \cdots ~~~ \cdots
  ~~~ \cdots ~~~ \cdots ~~~ \cdots ~~~ \cdots ~~~ \cdots ~~~ \cdots
  $$
\end{proof}

\begin{define}
  $$
  R_n(x) = \frac{f^{n+1}(c)}{(n+1)!} (x - a)^{n+1} ~~~
  c = a + \theta(x-a) ~~~ 0 < \theta < 1
  $$
  остаточный член в форме Лагранжа

  Кудрявцев 3 том 118 страница
\end{define}

\begin{theorem}
  $\forall k \in N ~~~ \forall x \in O_{\delta}(a) ~~~ |f^{(k)}(x)| \le M L^k$
  тогда $\forall x \in O_{\delta}(a)$ представим в виде ряда Тейлора

  Кудрявцев 3 том 120 страница
\end{theorem}

\begin{proof}
  $$
  |R_n(x)| = \frac{| f^{(n+1)}(c) |}{(n+1)!} (x-a)^{n+1} \le
  \frac{ML^{n+1}}{(n+1)!} \delta^{n+1} =
  \frac{ M (L\delta)^{n+1} }{(n+1)!} \to 0 ~~ n \to +\infty
  $$
\end{proof}