\begin{title}
	Гомоморфизм колец и идеалов.
\end{title}

Кольцои является группа по сложению и полугруппа по умножению.\\
Гомоморфизм сохраняет обе операции.\\
Базовое в кольце это операция сложения и сначала надо изучить гомоморфизм
коммутативной группы по сложению.\\


\bd{Структура гомоморфизма групп.}\\

\begin{defin}
$\varphi : G \to H$ ядром гоморфизма $\varphi$ называется такое подмножество
группы $G$ так что $Ker \varphi = \{g \in G| \varphi (g) = e_H\}$\\
\end{defin}
\bd{Что это значит?}\\
Ядро гомоморфизма - это совокупность элементов(элементы) группы $G$, которые 
при отображении $\varphi(g)$, переходят в $e = 1$ (в группе $H$).\\
Ядро обладает свойствами подгруппы:\\
Свойство 1:\\
Докажим это по критерию подгруппы.\\
Пусть $g_1,g_1 \in Ker~\varphi$\\
$g_1 \cdot g_2 \in Ker~\varphi$\\
$g_1^{-1} \in Ker \varphi \Rightarrow 1 \in Ker \varphi$\\
$\varphi(g_1 \cdot g_2) = \varphi (g_1) \cdot \varphi (g_2) = e_H \cdot e_H = e_H$ - определение гомоморфизма\\
По определению обратого элемента. $g\cdot g^{-1} = 1 = e_G$\\
$\varphi(e_G) = \varphi(e_G e_G) = \varphi(e_G) \cdot \varphi(e_G) \in H$\\
Так как $H$ группа то у элемента $\varphi(e_G)$ есть обратный, умножая на него
обе части равенства $\varphi(e_G) = e_H$ таким образом едbничный переходит в
едbничный.\\
$e_H = \varphi(e_G)$\\
$e_H = \varphi(g_1)\cdot \varphi(g^{-1})$
$\Rightarrow \varphi(g^{-1}) = e_H$\\
$\Rightarrow g^{-1} \in Ker~\varphi$\\

\begin{defin}
	Пусть $L$ подгруппа группы $G$ нормальная, если  $G \ge L$\\
	$\ge$ - означает число подмножеств имеют согласованную структуру с объедененим.\\
	Подгруппа называется нормальной если для $\forall g \in G ~ g^{-1} Lg \le L$\\
	$g_{-1} hg$ - это сопряжение элемента $h$ с элементом $g$
           $\Rightarrow h^g$.\\
	Понятия ненормальной подгруппы нет.\\
	Нормальная подгруппа выдерживает все сопряжения.\\
          
           Пусть $L$ нормальна в $G$, то есть $L = Ker \varphi$\\
           $\varphi(a^{-1}ha) = \varphi(a^{-1})\varphi(h)\varphi(a) = \varphi(a^{-1})\varphi(a) = \varphi(e) = e$\\
\end{defin}

\begin{defin}
	Ядро является нормальной подгруппой.
\end{defin}

\begin{proof}
	Пусть $f \in Ker~\varphi, g\in G$ проверим что $\varphi(g^{-1}fg) = ei_H$\\
	По определению гомоморфизма $\varphi(g^{-1} fg) = \varphi(g^{-1}) \cdot
	\varphi(f) \cdot \varphi(g) = \varphi(g^{-1}) \cdot \varphi_H \cdot
	\varphi(g) = \varphi(g^{-1}) \cdot \varphi(g) = e_H$
\end{proof}

\begin{defin}
Группа называется \kv{простой} если в ней нет неправильных нормальных подгрупп.
\end{defin}
Правильная это единичная или вся группа.\\

Основная задача группы описывать все простые группы, а остальные группы получить
из расширения.\\

Бесконечные группы не описаны.\\

В теории конечных групп вроде решена. Было найденоо 17 бесконенчо серий простых
групп. 26 не серийных изолированных групп.\\\\

Проблема 4 красок:\\
Ломанными линиями разделим плоскость на облости. Области
называеются соседними если у них есть общий кусок границы.\\
Задача: хватит ли 4 красок чтобы раскрасить всю карту чтобы соседние не были
одного цвета.\\
Эта задача с перебором 1500 графов была решена на компьютере.\\

\bd{Факторгруппы, факторкольца.}\\
треугольник - нормальная группа.\\
Пусть $G$ - группа, $H$ - нормальная подгруппа\\
$gH = \{g\cdot h | h \in H\}$ называется смежными классами.\\
Не сложно проверить, что разные смежные классы не пересекаются
и все имеют одинаковую мощность.\\
$h \to gh$ - биекция.\\
На множество смежных классов мы введем уномжение $g_1H\cdot g_2H = (g_1g_2)H$\\

Смежный класс - это произвольное множество элементов группы, которое
удовлетворяет условиям, описанным выше.\\
\begin{defin}
	В смежном классе $gH$ - элемент, $g$ - это представитель.\\
           Не сложно проверить что в качестве представителя можно взять 
           любой элемент этого смежного класса.\\
	Необходимо проверить коректность задания операцией, так как она определена через
	представителей. Убедимся в этом,что заменив представителей, мы получим тот же
	результат.\\
	$g_1' = g_1 h_1 \in H$\\
	$g_2 H = g_2 H ~~~~ g_2' = g_2' h_2 \in H$\\
	$
		(g_1' H)\cdot(g_2' H) = (g_1'\cdot g_2') H = (g_1 h_1)(g_2 h_2)H =
		(g_1 \cdot g_2)\cdot(g_{1}^{-1} \cdot h_1 \cdot g_2)h_2 H = g_1 g_2H
	$\\
	Так как подгруппа $H$ нормальная то $g_2 h_1 g_{2}^{-1} \in H$\\
	Не трудно проверить $\forall h\in H,  h\cdot H = H$\\
\end{defin}

\begin{defin}
	Группа, элементами которой являются смежные классы, а операция задана
	как указано выше называется \kv{факторгруппой} и обозначается $G/H$.\\
	Нейтральным элементом в этой группе $e\cdot H = H$\\
	Обратный $(gH)^{-1} = g^{-1}H$\\
	Ассоциативность следует из ассоциативности умножения из $G$\\

	Пусть $L$ группа, $H$ нормальная подгруппа, $L(H)$ - факторгруппа, говорят что
	группа $G$ является расширением группы $H$ при помощи группы $L$.\\
           В конце концов мы можем свести к простым группам, где строить факторгруппы 
           не получится.\\
	Даже зная все простые группы, построить все группы крайне затруднительно,
	потому что способов расширения очень много.
\end{defin}

\bd{Идеалы и факторкольца.}\\

Пусть $K$ кольцо, а $L$ его подкольцо и одновременно подгруппа по сложению.\\
Подкольцо называется идеалом,\\
если для $\forall k \in K$,  $k\cdot L=\{k\cdot e |e\in L\} = Lk = L$\\
$L$ идеал - обобщение нуля.\\
Смежные классы по идеалу $K + l$ имеют вид $k\in K$. На множестве смежных классов введем
операцию сложения\\
$(k_1 + L) + (k_2 + L) = (k_1 + k_2)+ L$\\
$(k_1+L)\cdot(k_2+L) = k_1\cdot k_2 + L$\\
Так как здесь операции задаются при помощи представителей, то нужно проверить
коректность.\\
$k_1' + L = k_1 + L$\\
$k_2' + L = k_2 + L$\\

Пример:\\
Кольцо целых чисел, четные числа - это подкольцо целых чисел
И как раз это подкольцо является идеалом кольца целых чисел.