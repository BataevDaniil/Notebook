%================================Заголовок=================
\begin{title}
	Поля разложения.
\end{title}

\kv{Многочлен называется неприводимым} над полем P если он
не раскладывается на произведение многочленов меньших
степеней.\\

\kv{Неприводимый многочлен} это простой элемент кольца
многочлена.\\

Простые чила неприводимые многочлены.\\

Нужно построить такое поле F в котором многочлен
разлагается на линейные множетели и называется F полем
разложения.

Что такое поле разложения?\\
\begin{defin}
\kv{Полем разложения многочлена} $f$ над полем $F$ называют 
минимальное расширение $P$ поля $F$ в котором многочлен $f$
разлагается в произведение линейных множителей над полем $P$.\\
\end{defin}

$p(x) = a(x - x_1)(x - x_2)...(x - x_n)$\\
при этом $P = F(x_1, x_2,...,x_n)$\\

\begin{theorem}
Пусть $f$ - многочлен над полем $F$. Тогда
1. Существует поле разложения многочлена $f$ нд полем $F$.\\
2. Любые поля разложения многочлена $f$ над полем $F$ изоморфны.
То есть взаимооднозначное соответствие, сохраняющее
линейные операции. Умножение(сложение) чисел в этих
полях дают одинаковый результат.\\
\end{theorem}

$Q_f = \{g(x) | g(x)_{\mt{ост}}\subset f(x)_{\mt{ост}}\}$
$Q_{f}$ - это множество остатков.

$Z_n$ идейно делени на многочлен.\\

На множестве $Q_f$ заданы операции сложения и умножения.
Сложение это обычное сложение многочленов. Оперция
умножения обычное умножениу, а потом деление с остатком
на $f(x)$  так же как и а кольце $Z_n$  проверяется $Q_f$
это комутативное кольцо $1\in P ~  0\in P$\\

Проверим что $Q_f$ поле:\\
Как и в случае $Z_p$ произвольный многочлен
НОД$(g(x),f(x)) = 1$ По следствию из алгоритма Евклида
$\exists u(x),v(x) ~ u(x)g(x) + v(x)f(x) = 1 ~
u(x)g(x) = 1 - v(x)f(x)$ значит остаток от деления на
$f(x)$ произведение $u(x)g(x)=1 u(x)=g(x)_1 ~ Q_f\supset P$
корень $f(x) = 0$.\\

1 шаг. Различие для множества $f(x)\in f(x)P[x]$ x-это
просто символ для записи многочлена, а в $f(x) = 0$ х
это отсаток принадлежащий $x\in Q_f$. Из чисто методических
соображений это вновь приобретенный корень смыслах
$x\in Q_t$ мы обозначим $\alpha$, тогда  $f(\alpha) = 0
Q_f = \{g(\alpha) |
g(\alpha)_{\mt{ост}}\subset f(\alpha)_{\mt{ост}}\}
$\\

2 шаг. $Q_f f(x) = Q_f[x] ~
\exists (x-\alpha)_{\alpha}\cdot g(x)$ $g(x)$ разлогаем
на неприводимые множетели и к этим множетелям применяем
шаг 1.\\
Мы добавим корни $\beta \gamma$ и так далее. В конце концов
$f(x)$ разложится на множетели.\\

Пример:\\
\bd{Поле комплексных чисел $\mathbb{C}$ служит полем разложения многочлена $x^2 + 1$ над полем $P$}\\
$P\in R f(x) = x^2 + 1$\\
Не имеет корней значит неприводима.\\
$Q_{x^2+1} = \{a, ax + b | a,b\in R\}$\\
Чтобы не менять традицию поле остатков будем 
обозначать вместо х символ i.\\
$Q_{x^2+1} = \{a, ax + b | a,b\in\mathbb{C}\}$\\
$(a + ib)(a_1 + ib_1) =
a{a_1} + i(a{b_1} + b{a_1}) + ib{b_1}$\\
для поле разложения $x^2 + 1$ есть специальное обозначение
$\mathbb{C} = \{a + bi | i^2 = -1 a,b\in R\}$\\
$\mathbb{C}$ расматривается как вектороное пространство над полем $P$,
имеет размерность $dim = 2$ и обозначают\\
$Z = ai + b ~ a\in R$\\
$Z_1 = ai + b Z + Z_1 = (a + a_1) + i(b_1 + b)$\\
Если мнимой части меняет знак то получится так называемое
сопряженное число.
$Z + Z' = 2a\in R$\\
$Z \cdot Z' = (a + bi)(a - bi) = a^2 - {i^2}{b^2} =
a^2 + b^2 ~ \in P$\\
$|Z| = \sqrt{Z\cdot Z'} = \sqrt{a^2 + b^2}$\\
$(Z\cdot Z)' = Z' \cdot Z'$\\
$(Z + Z)' = Z' + Z'$\\

Нахождение обратного элемента\\
НОД $(a+ib ~~ i^2 + 1) = 1$\\

$Z \cdot Z' = a^2 + b^2$ тогда
$\frac{Z \cdot Z'}{a^2 + b^2} = 1 \Rightarrow
Z^{-1} = \frac{Z'}{Z\cdot Z'}$\\

