\begin{title}
  Детерминант как отображение
\end{title}

\[det: M_n(K) \to K\]
- Отображение над кольцом $K$
\[det(A\cdot B) = det~A \cdot det~B\]
Данное отображение является гомоморфизмом мультипликативных групп.\\

\kv{Отображение кососиметрично}\\

{\bf Абстрактное описание детерминанта.}\\
Если матрицу рассматривать как совокупность строк, то детерминант - это
полилинейное кососиметричное нормарованное отображение кольца $K$, являющееся
гомоморфизмом мультипликативных групп матрицы и кольцы $K$\\

\begin{title}
  Линейные отображения
\end{title}

$V$ и $W$ - векторы пространства надполем $P$. $\varphi: V \to W ~~~ \forall
\alpha, \beta \forall u, v \in V$\\
$\varphi(\alpha u + \beta v) = \alpha\varphi(u) + \beta\varphi(v)$\\
Пусть $e_1 e_2 ... e_n$ - базис $V$ ~~~ $f_1 f_2 ... f_n$ - базис $W$\\
$V = \alpha_1 \varphi(e_1) + ... \alpha_n \varphi(e_n)$\\
Таким образом линейное отображение однозначно определяется своим действием на
базисных векторах. Так как просторанство $W$ имеет размерность $W$, то каждый
образ $\varphi(e_1)$ можно рассмотреть как $[\varphi] = (\varphi(e_1) ...
\varphi(e_n)) \Rightarrow [\varphi(v)] = [\varphi][v]$. Так как $[\varphi]
зависит от базисов, то ее можно записать в виде матрицы$\\
Ненулевой вектор $x$ называется собственным отображением вектора $x$\\
Если существует базис из собственных векторов, отобразим $\varphi ~и~ \lambda_1
... \lambda_n$ соответствующим собственным значениям.
\begin{displaymath}
[\varphi] = \left(\begin{array}{lccr}
\lambda_1 & 0 & \cdots & 0\\
0 & \lambda_2 & \cdots & 0\\
0 & 0 & \ddots & 0\\
0 & 0 & \cdots & \lambda_n
\end{array}\right)
\end{displaymath}

Свойства линейного отображения:\\
{\bf I} При линейном отображении нулевой вектор всегда переходит в нулевой вектор
$\varphi(0) = \varphi(0+0) = \varphi(0) + \varphi(0)$ Так как в $W$ есть обратна
по сложению, то $\varphi(0) = 0$\\
{\bf II} Обратный переходит в обратный.\\
{\bf III} Ядро линейного отображения $\varphi$ обозначается $Ker \varphi =
\{v \in V | \varphi(v) = 0\} ~~~ Ker \varphi \in V$\\

$Im$ - образ линейного отображения\\
$Im \varphi = \{w \in W | \exists v \in V \varphi(v) = w\}$\\

Когда $Ker$ состоит их 1-го нулевого вектора - разные вектора переходят в разные
 и если при этом $Ker = {0}$, то $Im(\varphi) = w$\\

