\documentclass{article}

\usepackage[utf8]{inputenc}
\usepackage{amsmath,amssymb}
\usepackage[russian]{babel}
\usepackage{pgfplots}
\pgfplotsset{compat=1.9}
%\usepackage[pdftex]{color} пакет pdfplots уже подключил пакет color
\usepackage{cancel}

\parindent=0pt
\newcommand{\bd}[1]{{\bfseries #1}} %текст жирным (bold)
\newcommand{\kv}{\emph}  %текст курсивом
\newcommand{\mt}{\mathit} %математический шрифт в формулах (mathText)
\newcommand{\bk}[1]{\bd{\kv{#1}}} %текст жирный и курсивным шрифтом

\newenvironment{theorem}{\colorbox[rgb]{1, 0.4, 0.6}{\bd{Теорема:}}\\}%
{\colorbox[rgb]{0, 0, 0}{1}\\} %Теорема

\newenvironment{proof}{\colorbox[rgb]{0, 0.8, 0.5}{\bd{Доказательство:}}\\}%
{\colorbox[rgb]{0, 0, 0}{1}\\} %Доказательство

\newenvironment{defin}{\colorbox[rgb]{1, 0.4, 1}{\bd{Определение:}}\\}%
{\colorbox[rgb]{0, 0, 0}{1}\\} %definition определение

\renewenvironment{title}{\begin{center} \bfseries \LARGE}{\end{center}}
%заголовок

\begin{document}

Лектор Виктория Юрьевна Барсукова

ДУ - дефференциальное уравнение.

OДУ - обыкновенное дефференциальное уравнение.

\begin{title}[\Large]
  Дифференциальное уравнение первого порядка. Основные понятия. Геометрический
  смысл уравнения первого порядка
\end{title}

\begin{define}[обыкновенного дифференциального уравнения]
  ОДУ называется уравнение содержащее неизвестную функцию одной переменной и ее
  производные или дифференциалы

  $F(x, y(x), y'(x), \ldots, y^{(m)} (x)) = 0$ или


  $F(x, y(x), dx, dy, d^2 x, d^2 y, \ldots d^n x, d^n y) = 0$

  $$
  y'(x) = \frac{dy}{dx}
  $$
\end{define}

\begin{define}[порядка уравнения]
  Порядком уравнения называется максимальный порядок производной или
  дифференциала входящего в уравнение.
\end{define}

\begin{define}[решения ДУ]
  Решение ДУ называется функция определенная на $<a,b>$ дифференциируема столько
  раз каков порядок уравнения и которая при подстановке в уравнение обращает его
  в тождество.
\end{define}

\begin{block}[ДУ 1 порядка]
  $F(x, y(x), y'(x)) = 0$ или $F(x, y(x), dy, dx) = 0$

  $y'(x) = f(x, y(x))$ - уравнение разрешенное относительно производной. Тоже
  что $\frac{dy(x)}{dx} = f(x, y(x))$

  $f(x, y)$ - заданая функция с двумя независимыми переменными. Будем считать
  что эта функция определена и непрерывна в некоторой односвязной области $D$.
\end{block}

\begin{define}[решения ДУ 1 порядка]
  $y = \varphi (x)$ определена на $<a,b>$ то $\varphi (x)$
  называется решением уравнения $y'(x) = f(x, y(x))$ если

  1) $\varphi (x)$ диференциируема на $<a,b>$

  2) $\forall x \in <a,b> ~~ (x, \varphi(x)) \in D$

  3) $\forall x \in <a,b> ~~ \frac{d\varphi (x)}{dx} = f(x, \varphi(x))$

  График решения называется интегральной кривой
\end{define}

\begin{define}[задачи Коши]
  Задачей Коши для уравнения $y'(x) = f(x, y(x))$ называется следущая задача
  найти такое решение уравнения которое при заданом $x$ принимает
  заданное значение $y_0$, тоесть удовлитворяет условию $y(x_0) = y_0$
  - условие Коши или начальное условие
  $$
  \left\{
  \begin{array}{l}
    y'(x) = f(x, y(x)) \\
    y(x_0) = y_0
  \end{array}
  \right.
  $$
\end{define}

\begin{define}[общего решния ДУ]
  Общим решением уравнения $y'(x) = f(x, y(x))$ называется функция $y = y(x,c)$
  такая что $\forall C$ функция является решением уравнения и любое
  решение уравнение входит в это семейство при некоторых $C$.

  Решение при конкретном $C$ называется частным решением.
\end{define}

\begin{define}[изоклина]
  $y' = f(x,y) ~~~ y'(x) = \tg \alpha ~~~ \tg \alpha = f(x,y) ~~~
  \tg \alpha = A ~ \Rightarrow ~ f(x,y) = A$

  $f(x, y) = A$ - изоклин
\end{define}

\begin{define}[интегральных типов уравнений]
  $$
  y'(x) = f(x, y(x))
  $$
  $$
  y'(x) = f(x) ~~~ y(x) = \int f(x)dx + C
  $$
  $$
  \left( \int_{x_0}^x f(t) dt \right)' = f(x) ~~~ \int f(x)dx = \int_{x_0}^x
  f(t)dt + C
  $$
  $$
  y(x) = \int_{x_0}^x f(t)dt + C ~ \text{- общее решение}
  $$
  $$
  y(x) = \int_{x_0}^x f(t)dt + y_0 ~ \text{- решение задачи Коши}
  $$
\end{define}

\begin{title}[\Large]
  Уравнения с разделяющимися переменными
\end{title}

\begin{define}[уравнения с разделяющимися переменными]
  Уравнения с разделяющимися переменными (УРП). Тоесть уравнения которые
  могут быть переведены к такому виду
  $$
  y'(x) = f(x) \cdot g(y(x))
  $$
  $$
  \frac{dy(x)}{dx} = f(x) \cdot g(y(x))
  $$
\end{define}

\begin{block}[Общий вид решения УРП]
  $$
  \frac{dy(x)}{g(y(x))} = \int f(x)dx ~~~ g(y(x)) \not= 0
  $$
  $$
  \int \frac{dy(x)}{g(y(x))} = \int f(x)dx
  $$
\end{block}

\begin{title}[\Large]
  Теорема существования и единственности решения задачи Коши для уравнения с
  разделяющимися переменными
\end{title}

\begin{theorem}[о $\exists !$ решения УРП]
  Пусть $f(x), g(u)$ определены и непрерывны на $x \in <a, b> ~ u \in <c, d>$
  $g(u) \not= 0 ~~~ \forall u \in <c, d>$ тогда
  $\forall x_0 \in <a, b> ~~~ \forall y_0 \in <c, d>$
  $$
  \left\{
  \begin{array}{l}
    y'(x) = f(x, y(x)) \\
    y(x_0) = y_0
  \end{array}
  \right.
  $$
  имеет единственное решение

  Замечание: $\exists m \in <c, d> ~ g(m) = 0$ тогда $y(x) \equiv m$
\end{theorem}

\begin{proof}
  Предположим что $\varphi(x)$ - решение уравнения $y'(x) = f(x, y(x))$
  удовлетворяющее $y(x_0) = y_0$ тоесть $\varphi(x_0) = y_0$
  $$
  \frac{d\varphi(x)}{dx} \equiv f(x)g(\varphi(x)) ~~~ x \in <a, b>
  $$
  $$
  \frac{d(\varphi(x))}{g(\varphi(x))} \equiv f(x)dx
  $$
  $$
  \int_{x_0}^x \frac{d(\varphi(t))}{g(\varphi(t))} = \int_{x_0}^x f(t)dt
  $$
  $$
  u = \varphi(t) ~~~ \int_{\varphi(x_0)}^{\varphi(x)} \frac{du}{g(u)} =
  \int_{x_0}^x f(t)dt
  $$
  $$
  \int_{y_0}^{\varphi(x)} \frac{du}{g(u)} = \int_{x_0}^x f(t)dt
  $$
  $$
  G(u) |_{y_0}^{\varphi(x)} = F(x) - F(x_0)
  $$
  $$
  G(\varphi(x)) = G(y_0) + F(x) - F(x_0)
  $$
  $g(u) \not= 0$ $g(u)$ - сохраняет знак $G'(u) = \frac{1}{g(u)}$ - сохраняет
  знак $G'(u) \not= 0$ $G(u)$ - строго монотонна и непрерывна $\Rightarrow$
  существает обратная ей функция
  $$
  \varphi(x) = G^{-1} \left( G(y_0) + \int_{x_0}^x f(t)dt \right)
  $$
  из едеинственности обратной функции $\Rightarrow$ единственность решения \\

  Докажем существование
  $$
  \varphi(x_0) = G^{-1}(G(y_0)) = y_0
  $$
  $$
  (G^{-1}(y))' = \frac{1}{G'(G^{-1}(y))}
  $$
  $$
  \varphi'(x) = \frac{1}{G' \left( G^{-1} \left( G(y_0)
  + \int_{x_0}^x f(t)dt \right) \right)}
  \left( G(y_0) + \int_{x_0}^x f(t)dt \right)' =
  $$
  $$
  = f(x) \cdot g \left( G^{-1} \left( G(y_0) +
  \int_{x_0}^x f(t)dt \right) \right) = f(x) \cdot g(\varphi(x))
  $$
  $\Rightarrow ~ \varphi(x)$ - решение задачи
\end{proof}

\begin{theorem}
  Если $\int_m \frac{du}{g(u)}$ - расходится, то через каждую точку области
  проходит единственное решение

  Если $\int_m \frac{du}{g(u)}$ - сходится, то в точка $y = m$ единтвсенность
  нарушена
\end{theorem}

\begin{title}[\Large]
  Уравнения приводящие к УРП
\end{title}

\begin{block}[Уравнения сводящиеся к линейной заменой]
  $$
  y'(x) = f(\alpha x + \beta y(x) + \gamma) ~~~ \alpha, \beta, \gamma \in R
  $$
  $$
  z(x) = \alpha x + \beta y(x) + \gamma ~ \Rightarrow z'(x) = \alpha +
  \beta y'(x) ~~~ \beta \not= 0
  $$
  $$
  \frac{z'(x) - \alpha}{\beta} = f(z(x))
  $$
  $$
  z'(x) = \beta f(z(x)) + \alpha ~~ \text{тоже что и} ~~
  z'(x) = 1 \cdot g(z(x))
  $$
\end{block}

\begin{block}[Однородные уравнения]
  $$
  y'(x) = \Phi \left( \frac{y(x)}{x} \right) ~ \text{замена} ~ z(x) =
  \frac{y(x)}{x}
  $$
  $$
  y(x) = z(x) \cdot x
  $$
  $$
  y'(x) = z'(x) x + z(x)
  $$
  $$
  x z'(x) + z(x) = \Phi(z(x))
  $$
  $$
  z'(x) = \frac{\Phi(z(x)) - z(x)}{x}
  $$
\end{block}

\begin{title}[\Large]
  Линейное уравнение 1-ого порядка
\end{title}

\begin{define}[линейного однородного и неоднородного уравнения]
  $y'(x) = a(x)y(x) + b(x)$ - линейное неоднородное уравнение

  $y'(x) = a(x)y(x)$ - линейное однородное уравнение
\end{define}

\begin{block}[Общее решение линейного однородного уравнения]
  $$
  y'(x) = a(x)y(x)
  $$
  $$
  \frac{dy(x)}{dx} = a(x)y(x)
  $$
  $$
  \int \frac{dy(x)}{y} = \int a(x)dx ~~~ y \not= 0
  $$
  $$
  \ln y(x) = \in a(x)dx
  $$
  $$
  |y(x)| = e^{\int_{x_0}^x a(t) dt + C}
  $$
  $$
  y(x) = C e^{\int_{x_0}^x a(t) dt}
  $$
\end{block}

\begin{block}[Метод вариации произвольной постоянной]
  $$
  y(x) = C(x) e^{\int_{x_0}^x a(t) dt}
  $$
  найдем $C(x)$ там чтобы $y(x)$ стал решением
  $$
  y'(x) = C'(x) \cdot e^{\int_{x_0}^x a(t) dt} +
  C(x) \cdot \left( e^{\int_{x_0}^x a(t) dt} \right)' =
  $$
  $$
  = C'(x) \cdot e^{\int_{x_0}^x a(t) dt} +
  C(x) \cdot e^{\int_{x_0}^x a(t) dt} a(x)
  $$
  $$
  C'(x) \cdot e^{\int_{x_0}^x a(t) dt} +
  C(x) \cdot e^{\int_{x_0}^x a(t) dt} a(x) =
  a(x) C(x) \cdot e^{\int_{x_0}^x a(t) dt} + b(x)
  $$
  $$
  C'(x) = \frac{b(x)}{e^{\int_{x_0}^x a(t) dt}} =
  b(x) e^{-\int_{x_0}^x a(t) dt}
  $$
  $$
  C(x) = \int_{x_0}^x b(s) e^{\int_{x_0}^s a(t) dt} ds + D
  $$
  $$
  y(x) = \left( \int_{x_0}^{\alpha} e^{-\int_{x_0}^t a(\tau) d\tau}
  b(t)dt + D \right) \cdot e^{\int_{x_0}^x a(\tau) d\tau} =
  $$
  $$
  = D e^{\int_{x_0}^x a(\tau) d\tau} + \int_{x_0}^x
  e^{\int_{x_0}^x a(\tau) d\tau - \int_{x_0}^x a(\tau) b\tau} d(t)dt
  $$
  $$
  y(x) = D e^{\int_{x_0}^x a(\tau) d\tau} + \int_{x_0}^x
  e^{\int_{t}^x a(\tau) d\tau} b(t)dt
  $$
  общее решение неоднородного уравнения
\end{block}

\begin{block}[Формула Коши]
  $$
  \left\{
  \begin{array}{l}
    y'(x) = a(x)y(x) + b(x) \\
    y(x_0) = y_0
  \end{array}
  \right. ~~~ \text{задача Коши}
  $$
  $$
  y(x) = y_0 e^{\int_{x_0}^x a(\tau) d\tau} + \int_{x_0}^x
  e^{\int_{t}^x a(\tau) d\tau} b(t)dt
  $$

  Введем обозначения

  $$
  K(x, t) = e^{\int_t^x a(\tau) d\tau}
  $$
  $$
  y(x) = y_0 K(x, x_0) + \int_{x_0}^x K(x, t) b(t) dt
  $$
  $K(x, t)$ - функция Коши
\end{block}

\begin{block}[Свойства]
  1) $K(x, x) = 1$

  2) при каждом фиксираванном $t \in <\alpha, \beta> ~ K(x, t)$ есть решение
  $y'(x) = a(x)y(x)$
\end{block}

\begin{theorem}
  Пусть $a(x), b(x)$ определена и непрерывна $\forall \in <\alpha, \beta>$
  тогда задача Коши $\forall x_0 \in <\alpha, \beta> ~~~ \forall y_0 \in R$
  $\exists !$ решение которое выражается
  $$
  y(x) = y_0 e^{\int_{x_0}^x a(\tau) d\tau} + \int_{x_0}^x
  e^{\int_{t}^x a(\tau) d\tau} b(t)dt
  $$
\end{theorem}

\begin{proof}
  Предположим что есть другое решение задачи Коши тогда
  $$
  \varphi(x) = u(x) e^{\int_{x_0}^x a(\tau)d\tau}
  $$
  проделов те же действия что и в методе вариаций получим что
  $\varphi(x) = y(x)$ $\Rightarrow$ единственность есть.

  Проверим что формула Коши действительно определяет решение задачи Коши
  $$
  y'(x) = y_0 e^{\int_{x_0}^x a(\tau) d\tau} a(x) + \left( \int_{x_0}^x
  e^{\int_{t}^x a(\tau) d\tau} b(t)dt \cdot
  e^{\int_{x_0}^x a(\tau)d\tau} \right)' =
  $$
  $$
  \left( \int_{\alpha(x)}^{\beta(x)} F(x, t) dt \right)' =
  F(x, \beta(x)) \beta'(x) - F(x, \alpha(x)) \alpha'(x) +
  \int_{\alpha(x)}^{\beta(x)} F_x(x, t)dt
  $$
  $$
  = y_0 e^{\int_{x_0} a(\tau)d\tau} a(x) + b(x) e^{-\int_{x_0}^x a(\tau) d\tau}
  e^{\int_{x_0}^x a(\tau) d\tau} +
  $$
  $$
  + \int_{x_0}^x b(t)
  e^{-\int_{x_0}^t a(\tau) d\tau} e^{\int_{x_0}^x a(\tau) d\tau} a(x) dt =
  $$
  $$
  a(x) \left( y_0 e^{\int_{x_0}^x a(\tau) d\tau} + \int_{x_0}^x b(t)
  e^{\int_t^x a(\tau) d\tau dt} \right) + b(x) =
  $$
  $$
  = a(x)y(x) + b(x)
  $$
\end{proof}

\begin{block}[Уравнение Бернули]
  $$
  y'(x) = a(x)y(x) + b(x)y^m(x) ~~~ m \not= 0 ~~~ m \not= 1
  $$
  $$
  \frac{y'(x)}{y^m(x)} = a(x) y^{1 - m}(x) + b(x)
  $$
  $$
  z(x) = y^{1 - m}(x)
  $$
  $$
  z'(x) = (1 - m) y^{-m}(x) y'(x) = \frac{(1-m)y'(x)}{y^m(x)} ~ \Rightarrow
  $$
  $$
  \frac{y'(x)}{y^m(x)} = \frac{z'(x)}{1 - m}
  $$
  $$
  z'(x) = (1 - m) a(x)z(x) + (1 - m)b(x) ~
  \text{- линейное уравнение}
  $$
\end{block}

\begin{title}[\Large]
  Уравнение в полных дифференциалах
\end{title}

Если частные производные 2-ого порядка существуют и непрерывны, то она равны
$$
\left( \frac{\partial^2 f}{\partial x \partial y},
\frac{\partial^2 f}{\partial y \partial x} \right)
$$

\begin{define}[уравнения в полных дифференциалах]
  Уравнение вида $P(x, y)dx + Q(x, y)dy = 0$ называется, уравнением в полных
  дифференциалах, если левая часть уравнения есть дифференциал некоторой
  функции тоесть выполнено условие
  $$
  \frac{\partial P(x, y)}{\partial y} = \frac{\partial Q(x, y)}{\partial x}
  $$
  $\Rightarrow ~ \exists F(x, y)$ чей дифференциал стоит слева
  $$
  \left\{
  \begin{array}{l}
    \frac{\partial F(x, y)}{\partial x} = P(x, y) \\
    \frac{\partial F(x, y)}{\partial y} = Q(x, y)
  \end{array}
  \right.
  $$
  $dF(x, y) = 0 ~ \Rightarrow ~ F(x, y) \equiv C$ 1-ый интеграл уравнения
\end{define}

\begin{block}[Решение уравнений в полых дифференциалах в общем виде]
  $$
  F(x, y) = \int_{x_0}^x P(t, y)dt + C(y)
  $$
  $$
  \frac{\partial F(x, y)}{\partial y} =
  \left( \int_{x_0}^x P(t, y)dt + C(y) \right)'_y =
  \int_{x_0}^x \frac{\partial P(t, y)}{\partial y} dt + C'(y) =
  $$
  $$
  = \int_{x_0}^x \frac{\partial Q(t, y)}{\partial t} dt + C'(y) =
  Q(t, y)|_{x_0}^x + C'(y) = Q(x, y) - Q(x_0, y) + C'(y) = Q(x, y)
  $$
  $$
  C'(x) = Q(x_0, y)
  $$
  $$
  C(y) = \int_{y_0}^y Q(x_0, s) ds + D
  $$
  $$
  F(x, y) = \int_{x_0}^x P(t, y) \int_{y_0}^y Q(x_0, s) ds + D
  $$
  $$
  \int_{x_0}^x P(t, y)dt + \int_{y_0}^y Q(x_0, s) ds = C ~ \text{- решение}
  $$
\end{block}

\begin{theorem}
  Пусть $P(x, y) ~ Q(x, y)$ непрерывны в некоторой односвязной области $G$
  $Q(x, y) \not= 0 ~~~ (x,y) \in G$ и существуют непрерывные частные
  производные, тогда $\forall (x_0, y_0) \in G_0$ задача Коши
  $\exists !$ решение
\end{theorem}

\begin{title}
  Система дифференциальных уравнений в нормальной форме
\end{title}

\begin{define}
  $$
  \left\{
  \begin{array}{l}
    x_1'(t) = f_1(t, x_1(t), x_2(t), \ldots, x_n(t)) \\
    x_2'(t) = f_2(t, x_1(t), x_2(t), \ldots, x_n(t)) \\
    \ldots ~~~ \ldots ~~~ \ldots ~~~ \ldots ~~~ \ldots ~~~ \ldots \\
    x_n'(t) = f_n(t, x_1(t), x_2(t), \ldots, x_n(t))
  \end{array}
  \right.
  $$
  $$
  x(t) =
  \left(
  \begin{array}{l}
    x_1(t) \\
    x_2(t) \\
    \ldots \\
    x_n(t)
  \end{array}
  \right) ~~~
  f(t) =
  \left(
  \begin{array}{l}
   f_1(t, x_1(t), x_2(t), \ldots, x_n(t)) \\
   f_2(t, x_1(t), x_2(t), \ldots, x_n(t)) \\
    \ldots ~~~ \ldots ~~~ \ldots ~~~ \ldots ~~~ \ldots \\
   f_n(t, x_1(t), x_2(t), \ldots, x_n(t))
  \end{array}
  \right)
  $$
  $x'(t) = f(t, x(t))$ $x(t)$ - неизвестный вектор функции
  $$
  <\alpha, \beta> \in R^n ~~~ x(t_0) = x_0 ~~~ t_0 \in <\alpha, \beta> ~~~
  x_0 \in R^n
  \left\{
  \begin{array}{l}
    x_1(t_0) = x_{0_1} \\
    x_2(t_0) = x_{0_2} \\
    \ldots ~~~ \ldots \\
    x_n(t_0) = x_{0_n}
  \end{array}
  \right.
  $$
\end{define}

\begin{block}[Решением системы ДУ называется]
  $\varphi(t): ~ <\varphi, \beta> \to R^n$ которая

  1) непрерывна дифференциируема на $<\alpha, \beta>$

  2) $\forall t \in <\alpha, \beta> ~~~ (t, \varphi(t)) \in D$

  3) $\frac{d\varphi(t)}{dt} \equiv f(t, \varphi(t)) ~~~
  t \in <\alpha, \beta>$
\end{block}

\begin{title}[\Large]
  Линейные системы дифференциальных уравнений (ЛСДУ)
\end{title}

$$
\left\{
\begin{array}{l}
  x'_1(t) = a_{11}(t)x_1(t) + a_{12}(t)x_2(t) + \ldots
  + a_{1n}x_n(t) + g_1(t) \\
  x'_2(t) = a_{21}(t)x_1(t) + a_{22}(t)x_2(t) + \ldots
  + a_{2n}x_n(t) + g_2(t) \\
  \ldots ~~~ \ldots ~~~ \ldots ~~~ \ldots ~~~ \ldots ~~~ \ldots ~~~
  \ldots ~~~ \ldots ~~~ \ldots ~~~ \ldots\\
  x'_n(t) = a_{n1}(t)x_1(t) + a_{n2}(t)x_2(t) + \ldots
  + a_{nn}x_n(t) + g_n(t) \\
\end{array}
\right.
$$

$$
x(t) =
\left(
\begin{array}{l}
  x_1(t) \\
  x_2(t) \\
  \ldots \\
  x_n(t)
\end{array}
\right) ~~~
A(t) =
\left(
\begin{array}{cccc}
  a_{11} & a_{12} & \ldots & a_{1n} \\
  a_{21} & a_{22} & \ldots & a_{2n} \\
  \ldots & \ldots & \ldots & \ldots \\
  a_{n1} & a_{n2} & \ldots & a_{nn}
\end{array}
\right) ~~~
g(t) =
\left(
\begin{array}{l}
  g_1(t) \\
  g_2(t) \\
  \ldots \\
  g_n(t) \\
\end{array}
\right)
$$

$x'(t) = A(t)x(t) + g(t)$ - неоднородная система

$x'(t) = A(t)x(t)$ - однородная система

\begin{title}[\Large]
  Теорема о существовании единственности для линейных систем. Рещение задачи
  Коши
\end{title}

\begin{theorem}
  Пусть $A(t) g(t)$ непрерывны на $<\alpha, \beta>$ тогда
  $\forall t_0 \in <\alpha, \beta> ~~~ \forall x_0 \in R^n$ задачи Коши
  $\exists !$ решение на $<\alpha, \beta>$
\end{theorem}

\begin{block}[Лемма]
  Задача Коши эквивалентна
  $$
  x(t) = x_0 + \int_{t_0}^t A(s)x(s)dx +
  \int_{t_0}^t g(s)ds
  $$
  тоесть любое решение задачи Коши явялется решением этого уравнения и наоборот
\end{block}

\begin{proof}
  Доказательство в одну сторону
  Пусть $\varphi$ - решение задачи Коши тогда
  $$
  \varphi'(t) = A(t)\varphi(t) + g(t)
  $$
  $$
  \int_{t_0}^t \varphi'(s)ds = \int_{t_0}^t A(s) \varphi(s)ds +
  \int_{t_0}^t g(s) ds
  $$
  $$
  \int_{t_0}^t \varphi'(s)ds = \varphi(s)|_{t_0}^t = \varphi(t) - x_0
  $$
  $$
  \varphi(t) = x_0 + \int_{t_0}^t A(s) \varphi(s) ds  + \int_{t_0}^t g(s)ds
  $$
  Доказательство в другую сторону

  Пусть $\varphi(t)$ - решение
  $$
  x(t) = x_0 + \int_{t_0}^t A(s)x(s)dx +
  \int_{t_0}^t g(s)ds
  $$
  тогда $\varphi(t)$ - непрерывная $\Rightarrow$ дифференцируема

  $\varphi'(t) = A(t)\varphi(t) + g(t)$

  $\varphi(t_0) = x_0$
\end{proof}

\begin{title}[\Large]
  Множество решений линейной системы
\end{title}

\begin{theorem}[прицапа суперпозии решений]
  $\varphi(t)$ решение $x'(t) = A(t)x(t) + g_1(t)$

  $\psi(t)$ решение $x'(t) = A(t)x(t) + g_2(t)$

  тогда $C\varphi(t) + B\psi(t)$ решение $x'(t) = A(t)x(t) + Cg_1(t) + Bg_2(t)$
  $C,B$ - числа
\end{theorem}

\begin{proof}
  $$
  C\varphi'(t) + B\varphi'(t) = A(t)(C\varphi(t) + B\psi(t)) + Cg_1(t) +
  Bg_2(t)
  $$
  $$
  C\varphi'(t) + B\varphi'(t) = C(A(t)\varphi(t) + g_1(t))+
  B(A(t)\psi(t) + g_2(t))
  $$
  $$
  \Rightarrow ~ A(t)\varphi(t) + g_1(t) = \varphi'(t) ~~~
  A(t)\psi(t) + g_2(t) = \psi'(t)
  $$
\end{proof}

\begin{block}[Следствие 1]
  Разность двух решений неоднородной системы $x'(t) = A(t)x(t) + g(t)$
  является решением неоднородной системы $x'(t) = A(t)x(t)$
\end{block}

\begin{proof}
  $x'(t) = A(t)x(t) + g(t) ~~ C = 1 ~~ B = 1$

  $\varphi(t) - \psi(t)$

  $g(t) - g(t) = 0$
\end{proof}

\begin{block}[Слeдствие 2]
  Если $\varphi_0(t)$ некоторое решение сисетмы $x'(t) = A(t)x(t) + g(t)$ то
  множество всех решений этой системы совпадает с множеством следующего вида
  $\{\varphi_0(t) + \varphi(t)\}$ где $\varphi(t)$ пробегает множество решений
  однородной системы
\end{block}

\begin{proof}
  $\varphi_0(t)$ решение $x'(t) = A(t)x(t) + g(t)$

  $\varphi(t)$ решение $x'(t) = A(t)x(t)$

  $\varphi_0(t) + \varphi(t)$ решение $x'(t) = A(t)x(t) + g(t)$

  Пусть $\varphi(t)$ решение $x'(t) = A(t)x(t) + g(t)$

  $\psi(t) - \varphi_0(t)$ решение $x'(t) = A(t)x(t)$

  $\Rightarrow ~ \varphi(t) = \varphi_0(t) - \varphi(t)$
\end{proof}

\begin{define}
  $$
  \varphi_1 (t) =
  \left(
  \begin{array}{c}
    \varphi_{11}(t) \\
    \cdots \\
    \varphi_{1n}(t) \\
  \end{array}
  \right),~~~
  \varphi_2 (t) =
  \left(
  \begin{array}{c}
    \varphi_{21}(t) \\
    \cdots \\
    \varphi_{2n}(t) \\
  \end{array}
  \right), \cdots,
  \varphi_k (t) =
  \left(
  \begin{array}{c}
    \varphi_{k1}(t) \\
    \cdots \\
    \varphi_{kn}(t) \\
  \end{array}
  \right) ~~~ t \in <\alpha, \beta>
  $$
  $\varphi, \cdots, \varphi_k(t)$ называется ЛЗ если $\exists C_1, C_2, \ldots,
  C_k(\sum_{j=1}^k cj^2 \not= 0)$ то комбинация $C_1\varphi_1(t) +
  C_2\varphi_2(t) + \ldots + C_k\varphi_k(t) = 0 ~ t \in <\alpha, \beta>$. В
  противном случае система функций ЛНЗ
\end{define}

\begin{theorem}
  Множество всех решений однородной системы $x'(t) = A(t)x(t)$ образует
  линейное пространство размерность которого совпадает с размерностью системы
\end{theorem}

\begin{proof}
  если $\varphi_1(t), \varphi_2(t)$ - решение $x'(t) = A(t)x(t) ~ \Rightarrow ~
  C_1\varphi_1(t) + C_2\varphi_2(t)$ решение $x'(t) = A(t)x(t)$

  $\varphi'_j(t) = A(t)\varphi_j(t)$

  $C_1 \varphi'_1(t) + C_2 \varphi'_2(t) = C_1(A(t) \varphi_1(t)) +
  C_2A^{(t)}\varphi_2(t) = A(t)(C_1\varphi_1(t) + C_2\varphi_2(t)) ~
  \Rightarrow ~ C_1 \varphi_1(t) + C_2 \varphi_2(t)$ решение
  $x'(t) = A(t)x(t)$\\

  Построим $n$ ЛНЗ решений $t \in <\alpha, \beta>$ возьмем $t_0 \in
  <\alpha, \beta>$
  $$
  e_1 =
  \left(
  \begin{array}{c}
    1 \\
    0 \\
    \cdots \\
    0
  \end{array}
  \right), ~~~
  e_2 =
  \left(
  \begin{array}{c}
    0 \\
    1 \\
    \cdots \\
    0
  \end{array}
  \right), \cdots,
  e_k =
  \left(
  \begin{array}{c}
    0 \\
    0 \\
    \cdots \\
    1
  \end{array}
  \right) ~~ \text{ЛНЗ}
  $$
  $$
  \left\{
  \begin{array}{c}
    x'(t) = A(t)x(t) \\
    x(t_0) = e_k
  \end{array}
  \right. ~~~ k = 1, 2, \ldots, n
  $$
  $\varphi_k(t)$ - единственое решение $\varphi_1(t), \ldots, \varphi_n(t) ~~~
  t \in <\alpha, \beta$

  покажем ЛНЗ $C_1 \varphi_1(t) + \ldots + C_n \varphi_n(t) \equiv 0 ~~~
  t = t_0$

  $C_1 \varphi_1(t_0) + C_2 \varphi_2(t_0) + \ldots + C_n \varphi_n(t_0)
  \equiv 0$
  $$
  C_1
  \left(
  \begin{array}{c}
    1 \\
    0 \\
    \cdots \\
    0
  \end{array}
  \right) +
  C_2
  \left(
  \begin{array}{c}
    0 \\
    1 \\
    \cdots \\
    0
  \end{array}
  \right) + \cdots +
  C_k
  \left(
  \begin{array}{c}
    0 \\
    0 \\
    \cdots \\
    1
  \end{array}
  \right) =
  \left(
  \begin{array}{c}
    0 \\
    0 \\
    \cdots \\
    0
  \end{array}
  \right)
  $$
  $$
  \left(
  \begin{array}{c}
    C_1 \\
    C_2 \\
    \cdots \\
    C_n
  \end{array}
  \right) =
  \left(
  \begin{array}{c}
    0 \\
    0 \\
    \cdots \\
    0
  \end{array}
  \right) ~ \Rightarrow ~ C_1 = C_2 = \ldots = C_n = 0 ~ \Rightarrow ~
  \varphi_1(t), \ldots, \varphi_n(t) ~~~ \text{- ЛНЗ}
  $$\\

  $\psi(t)$ произвольное решение
  $$
  \psi(t_0) =
  \left(
  \begin{array}{c}
    a_1 \\
    a_2 \\
    \cdots \\
    a_n
  \end{array}
  \right)
  $$
  $\varphi = a_1 \varphi_1(t) + \ldots + a_n \varphi_n(t)$

  $\psi(t) \equiv \varphi(t)$

  Заменим:
  $$
  \psi(t) =
  \left\{
  \begin{array}{c}
  x'(t) = A(t)x(t)\\
  x(t_0) =
  \left(
  \begin{array}{c}
    a_1 \\
    a_2 \\
    \cdots \\
    a_n
  \end{array}
  \right)
  \end{array}
  \right. ~~ \text{- решение задачи Коши}
  \varphi(t) =
  \left\{
  \begin{array}{c}
  x'(t) = A(t)x(t)\\
  x(t_0) =
  \left(
  \begin{array}{c}
    a_1 \\
    a_2 \\
    \cdots \\
    a_n
  \end{array}
  \right)
  \end{array}
  \right.
  $$
  $$
  \varphi(t_0) = a_1 \varphi_1(t_0) + \ldots + a_n \varphi_n(t_0) =
  a_1
  \left(
  \begin{array}{c}
    1 \\
    0 \\
    \cdots \\
    0
  \end{array}
  \right) +
  a_2
  \left(
  \begin{array}{c}
    0 \\
    1 \\
    \cdots \\
    0
  \end{array}
  \right) + \ldots +
  a_n
  \left(
  \begin{array}{c}
    0 \\
    0 \\
    \cdots \\
    1
  \end{array}
  \right) =
  \left(
  \begin{array}{c}
    a_1 \\
    a_2 \\
    \cdots \\
    a_n
  \end{array}
  \right)
  $$
  так как они обе удовлетворяеют задачи Коши то $\varphi(t) \equiv \psi(t) ~
  \Rightarrow ~ \psi(t) = \sum_{j=1}^n a_j \varphi_j(t)$
\end{proof}

\begin{define}[ФСР]
  ФСР - называется базис пространства решений тоесть $n$ - ЛНЗ решений данной
  системы
\end{define}

\begin{define}
  $\varphi_1(t), \ldots, \varphi_n(t)$ - вектор функции
  $$
  \varphi_j(t) =
  \left(
  \begin{array}{c}
    \varphi_{j1}(t) \\
    \varphi_{j1}(t) \\
    \cdots \\
    \varphi_{jn}(t)
  \end{array}
  \right)
  $$
  $$
  W[\varphi_1, \ldots, \varphi_n](t) =
  \left|
  \begin{array}{cccc}
    \varphi_{11}(t) & \varphi_{21}(t) & \ldots & \varphi_{n1}(t) \\
    \varphi_{12}(t) & \varphi_{22}(t) & \ldots & \varphi_{n2}(t) \\
    \ldots & \ldots & \ldots & \ldots \\
    \varphi_{1n}(t) & \varphi_{2n}(t) & \ldots & \varphi_{nn}(t)
  \end{array}
  \right|
  $$
\end{define}

\begin{block}[Утверждения]
  1) Если $\varphi_1(t), \ldots, \varphi_n(t)$ ЛЗ $\Rightarrow W(t) \equiv 0
  ~~~ t \in <\alpha, \beta>$

  2) Если $W(t) \not\equiv 0 ~ \Rightarrow ~ \varphi_1(t), \ldots,
  \varphi_n(t)$ ЛНЗ
\end{block}

\begin{block}[Критерий ЛНЗ решений системы]
  Пусть $\varphi_1, \ldots, \varphi_n(t)$
  решение задачи Коши тогда

  1) $W(t) = 0$ хотябы в одной точке
  $t \in <\alpha, \beta>$ то решение $\varphi_1(t), \ldots, \varphi_n(t)$ ЛЗ

  2) $\varphi_1, \ldots, \varphi_n(t)$ ЛНЗ то $W(t) \not0$ ни в одной точке
  $t \in <\alpha, \beta>$
\end{block}

\begin{proof}
  Пусть $W(t_0) = 0 ~~~ t \in <\alpha, \beta> ~ \Rightarrow ~ \varphi_1(t_1),
  \ldots, \varphi_n(t_1)$ ЛЗ $\exists C_1, \ldots, C_n ~~~
  C_1 \varphi_1(t_1) + \ldots + C_n \varphi_n (t_1) = 0$

  $\varphi(t) = C_1\varphi_1(t) + \ldots + C_n \varphi_n(t)$
  $$
  \varphi(t) =
  \left(
  \begin{array}{c}
    x'(t) = A(t)x(t) \\
    x(t_0) = 0
  \end{array}
  \right)
  $$
  $\psi \equiv 0$ другое решение $\Rightarrow ~ \varphi(t) = C_1 \varphi_1(t) +
  \ldots + C_n \varphi_n(t) \equiv 0 ~ \Rightarrow ~ \varphi_1, \ldots,
  \varphi_n$ ЛЗ \\
\end{proof}

\begin{theorem}
  $$
  W(t) =
  \left|
  \begin{array}{cccc}
    \varphi_{11}(t) & \varphi_{21}(t) & \ldots & \varphi_{n1}(t) \\
    \varphi_{12}(t) & \varphi_{22}(t) & \ldots & \varphi_{n2}(t) \\
    \ldots & \ldots & \ldots & \ldots \\
    \varphi_{1n}(t) & \varphi_{2n}(t) & \ldots & \varphi_{nn}(t)
  \end{array}
  \right|
  $$
  пусть $W(t)$ - определитель Вронского решение $\varphi_1(t), \ldots,
  \varphi_n(t)$ системы $x'(t) = A(t)x(t)$ то он является
  $\frac{W(t)}{dt} = (t_2A(t))W(t) ~~~ t \in <\alpha, \beta>$

  $t_2A(t) = a_{11}(t) + a_{22}(t) + \ldots + a_{nn}(t)$
\end{theorem}

\begin{block}[Формула Леовиля]
  $$
  W(t) = W(t_0)e^{\int_{t_0}^t t_2 A(s)ds}
  $$
  если определитель Вронского в некоторой точке $= 0$ то $W(t) \equiv 0$
\end{block}

\begin{title}[\Large]
  Фундаментальная матрица
\end{title}

\begin{define}
  Матрица $\Phi(t) ~ t \in <\alpha, \beta>$ называется решением системы
  $x'(t) = A(t)x(t)$ если столбцы матрицы являются решением системы
\end{define}

\begin{block}[Утверждение]
  $\Phi(t)$ является матрицей решений если удовлетворяет матричной системе
  $X'_{n \times n}(t) = A_{n \times n}(t) X_{n \times n}(t)$ $n$ - уравней
\end{block}

\begin{define}
  Невырожденная матрица решений называется фундаментальной матрицей
  $|\Phi(t)| = W(t)$ столбца ЛНЗ. Фкндаментальная матрица это матрица столбцы
  которой образуют ФСР
\end{define}

\begin{block}[Предложение]
  Если $\Phi(t)$ фундаментальная матрица $x'(t) = A(t)x(t)$
  тогда множество решений системы совпадает со множеством функций вида
  $\varphi(t) = \Phi(t)C$ где $C$ произвольный постоянный вектор из $e^n$
\end{block}

\begin{proof}
  $\Phi(t)$ фундаментальная матрица и $\varphi_1(t), \ldots, \varphi_n(t)$
  это ФСР

  $\Phi(t)C = (\varphi_1(t), \ldots, \varphi_n(t))C = C_1 \varphi_1(t) +
  C_2 \varphi_2(t) + \ldots + C_n \varphi_n(t)$
\end{proof}

\begin{block}[Утверждение]
  $\Phi(t)$ фундаментальная матрица $x'(t) = A(t)x(t)$ тогда множество всех
  фундаментальных матриц системы совпадает с множеством матриц следующего вида
  $\Psi(t) = \Phi(t)B ~~ detB \not= 0$
\end{block}

\begin{proof}
  так как $\Phi(t)$ фундаментальная матрица $det\Phi(t) \not= 0 ~~
  \forall t \in <\alpha, \beta>$ удовлетворяет $\Phi(t) = A(t)\Phi(t)$
  $det\Psi(t) \not= 0$ невырожденная

  $\Phi'(t)B = A(t)(\Phi(t)B)$

  $(\Phi(t)B)' = A(t)(\Phi(t)B)$

  $\Psi'(t) = A(t)\Psi(t)$ $\Psi$ - фундаментальная матрица \\

  Пусть $\Psi(t)$ фиксированная фундаментальная матрица

  $B = \Phi^{-1}(t_0) \Psi(t_0) ~~~ t_0 \in <\alpha, \beta>$

  $\Psi(t) = \Phi(t) \Phi_{-1}(t_0) \Psi(t_0)$

  $t = t_0 ~~~ \Phi(t_0) \Phi^{-1}(t_0) \Psi(t_0)$
  $$
  \left\{
  \begin{array}{c}
    X(t_0) = \Psi(t_0) \\
    X'(t) = A(t)X(t)
  \end{array}
  \right. ~ \Rightarrow ~
  \Psi(t) = \Phi(t) \Phi^{-1}(t_0)\Phi(t_0)
  $$
\end{proof}

Вывод: $x'(t) = A(t)x(t)$

1) Множество ее решений образует линейное пространство степени $n$

2) Для ее решения достаточно найти ФСР
$$
\varphi(t) = \sum_{j=1}^n C_j \varphi_j(t)
$$

\begin{title}[\Large]
  Линейная неоднородная системы
\end{title}

$x'(t) = A(t) x(t) + g(t) ~~~ t \in <\alpha, \beta>$ $A(t), g(t)$ -
непрерывные функции. Метод вариаций производных постоянных.

Расмотрим $x'(t) = A(t)x(t)$ $\Phi(t)$ - фундаментальная матрица
$x(t) = \Phi(t) C$ общее решение $x'(t) = A(t)x(t)$
$$
C =
\left\{
\begin{array}{c}
  C_1 \\
  C_2 \\
  \cdots \\
  C_n
\end{array}
\right) ~~~
C(t) =
\left\{
\begin{array}{c}
  C_1(t) \\
  C_2(t) \\
  \cdots \\
  C_n(t)
\end{array}
\right)
$$
$x(t) = \Phi(t)C(t) ~~~ x'(t) = \Phi'(t)C(t) + \Phi(t)C'(t)$

$\Phi'(t)C(t) + \Phi(t)C'(t) = A(t)\Phi(t)C(t) + g(t)$

$\Phi'(t) = A(t) \Phi(t)$

$\Phi(t)C'(t) = g(t)$

$C'(t) = \Phi^{-1}(t)g(t)$

$C(t) = \int_{t_0}^2 \Phi^{-1}(s)g(s)ds + D$

$x(t) = \Phi(t)D + \int_{t_0}^t\Phi(t) \Phi^{-1}(s)g(s)ds$ формула общего
решения. $x(t_0) = x_0$ задача Коши

$x(t_0) = \Phi(t_0)D ~~~ D = \Phi^{-1}(t_0)x(t_0)$

$x(t) = \Phi(t) \Phi^{-1}(t_0)x_0 + \int_{t_0}^t \Phi(t)\Phi^{-1}(s)g(s)ds$
формула Коши

$$
\left\{
\begin{array}{c}
  x'(t) = A(t)x(t) + g(t) \\
  x(t_0) = x_0
\end{array}
\right. ~~~ \text{задача Коши}
$$

\begin{theorem}
  Пусть $A(t), g(t)$ непрерывная матрица и вектор $t \in <\alpha, \beta>$ тогда
  решение задачи Коши определяется формулой Коши
\end{theorem}

$C(t,s) = \Phi(t)\Phi^{-1}(s)$ матрица Коши

$x(t) = C(t, t_0)x_0 + \int_{t_0}^t C(t,s)g(s)ds$

$C(t,s)$ зависит от выбора фундаментальной матрицы $\Phi(t)$. Пусть $\Psi(t)$
другая фундаментальная матрица $D(t,s) = \Psi(t)\Psi^{-1}(s)$ $\exists B$
невырожденаая матрица которая обеспечивает

$\Psi(t) = \Phi(t)B$

$\Psi^{-1}(s) = B^{-1} \Phi^{-1}(s)$

$\Psi(t) \Psi^{-1}(s) = \Phi(t) B B^{-1} \Phi^{-1}(s)$

$D(t, s) = C(t, s) ~ \Rightarrow$ функция Коши не зависит от выбра
фундаментальной матрица $\Phi(t)$

\begin{block}[Свойства]
  1) При каждом фиксиравоном $s \in <\alpha, \beta>$ $C(t, s)$ решение системы
  $X'(t) = A(t)X(t)$

  2) $C(t,t) = E$ еденичная матрица
\end{block}
\begin{title}
  Иследование выпоклости функций с помощью производной.
\end{title}

$f(x)$ непрерывна на $<a,b>$ \bd{называется выпуклой вниз} $\forall x_1, x_2
\in <a,b> f(\frac{x_1 + x_2}{2}) \le \frac{f(x_1) + f(x_2)}{2}$.\\
И \bd{называется строго выпуклой вниз} если  $f(x)$ непрерывна на $<a,b>$
$\forall x_1 \not= x_2 \in <a,b> ~~ f(\frac{x_1 + x_2}{2}) <
\frac{f(x_1) + f(x_2)}{2}$\\

$f(x)$ непрерывна на $<a,b>$ \bd{называется выпуклой вниз} $\forall x_1, x_2
\in <a,b> f(\frac{x_1 + x_2}{2}) \ge \frac{f(x_1) + f(x_2)}{2}$.\\
И \bd{называется строго выпуклой вниз} если  $f(x)$ непрерывна на $<a,b>$
$\forall x_1 \not= x_2 \in <a,b> ~~ f(\frac{x_1 + x_2}{2}) >
\frac{f(x_1) + f(x_2)}{2}$\\

\begin{theorem}
  Пусть $f(x)$ непрерывна на $(a,b)$ и $f'(x) ~~ x \in (a,b) ~~ \forall
  x \in (a,b) ~~ f''(x) \ge 0$ функция выпукла вниз на $(a,b)$, если
  $f''(x) \le 0$ то выпукла вверх.
\end{theorem}

\begin{proof}
  Для определенности $x_1 < x_2$ и $2h = x_2 - x_1$\\
  $
  x_0 = \frac{x_2 + x_1}{2}\\
  x_2 = x_0 + h\\
  x_1 = x_0 - h\\
  f(x_2) = f(x_0 + h) = f(x_0) + \frac{f'(x_0)}{1!}h +
         \frac{f''(c_2)}{2!}h^2 ~ x_0 < c_2 < x_2\\
  f(x_1) = f(x_0 - h) = f(x_0) + \frac{f'(x_0)}{1!}h +
         \frac{f''(c_1)}{2!}h^2 ~ x_1 < c_1 < x_0\\
  f(x_1) + f(x_2) = 2f(x_0) +
                  \frac{h^2}{2}(f''(c_2) + f''(c_1))\\
  $\\
\end{proof}
Теория будет справидливой и для строгих выпуклостей.

\begin{title}
  Нахождение точек перегиба функции с помощью производной.
\end{title}

\bd{Точка $a$ называется точкой перегиба} $y = f(x)$ если в точке
$a ~ \exists f'(x)$ конечная или бесконечная определенного знака и при переходе
через точку $a$ меняется направление выпуклости.\\
$(a, f(x))$ точка пергиба графика функции.\\

\bd{Необходимыйе условия точки перегиба:}\\
$f(x)$ имеет $f''(x)$ в некоторой окрестности точки $a$, $a$ точка перегиба
$f(x)$, тоисть $f''(a) = 0$.

\bd{Докозательство:} Предположим что $f''(a)\not= 0$, то в силу непрерывности
1-ой производной в точке $a$, она имеет тот же знак что и знак в некоторой
окрестности $a$ $f''(x)\ge 0 ~ \forall x\in O(a)$ то $f(x)$ выпукла вверх.
$a$ не может быть точкой перегиба так как направление выпуклости не меняется,
противоречие.\\

\bd{Достаточные условия:}\\
Пусть в точке $a ~ f(x)$ имеет конечную или бесконечную определенного занка
первую производную в проколотой открестности точки $a$ $\exists f''(x)$.
Если $f''$ меняет знак при переходе через точку $a$, то точка $a$ является
точкой перегиба этой функции.\\

\begin{theorem}
  Если $f''(a) = 0$ значит $f'(x)$ - конечная и $f''(a) \not= 0$ то
  $a$ точка перегиба.\\
\end{theorem}

\bd{План иследования функции}\\
1. D(y), четность, переодичность.\\
2. Найти точки пересечения графика функции с Ox и Oy, и
промежутки возрастания и убывания.\\
3. Найти точки разрыва и их классификация. Вычисление
односторонних приделов в точке разрыва и в ограниченных
точках области определния.\\
4. Нахождение всех ассимтот графика.\\
5. Иследование функция на монотоность и экстремумы с
помощью производной.\\
6. Иследование на выпуклости и перегибы.\\
7. Построение таблицы значений.\\
8. Построение графика.\\
\begin{center}
  \bd{Интегрирование функции одной переменной.}
\end{center}

\bd{Первообразная и неопределенный интеграл. Свойства:}\\
Пусть $F(x); f(x)$ пределена на $<a;b>$ F - непрерывна на $<a,b>$,
дифференцируема на $(a;b)$ и $\forall x \in (a; b) ~~ F'(x) = f(x)$, то $F(x)$
называют первообразной для $f(x)$ на $(a;b)$.\\

\bd{Свойства:}\\
1) Если F(x) - первообразная для $f(x)$ определена на $\Delta$ то $F(x) + C$
также первообразная к $f(x)$.
Доказательство: $(F(x) + C)' = F'(x) + 0 = f(x)$.\\

2) Если $F_{1}(x), F_{2}(x)$ обе первообразнозные для $f(x)$ на промежутке
$\Delta$, то $F_{1}(x) - F_{2}(x) = C$.\\
Доказательство: $(F_{1}(x) - F_{2}(x))' = F'_{1}(x) - F'_{2}(x) =
f(x) - f(x) \equiv 0$\\
$F_{1}(x) - F_{2}(x) = const$.\\

\bd{Определение:}\\
Совокупность всех первообразных для $f(x)$ на промежутке $\Delta$ называют
неопределенным интегралом от $f(x)$ и обозначают
\[\int f(x)dx\]
$f(x)$ - называют подынтервальной функцией.\\
$f(x)dx$ - Подинтегральным выражением.
\[\int f(x)dx = \not\{ F(x) + C \not\} \]
Где $F(x) + C$ - семейство функций.\\

1) Диффиренциал от неопределенного интеграла равен подинтергальному выражению:
\[d \left ( \int f(x)dx \right ) = f(x)dx \]
Доказательство:\\
\[d \left ( \int f(x)dx \right ) = d(F(x) + C) = dF(x) + 0 = F'(x)dx = f(x)dx\]

2)\[\int dF(x) = F(x) + C\]
Доказательство:\\
$\int dF(x) = \int F'(x)dx = F(x) + C$ \\

3) $a, b \in R$\\
\[\int (af(x) + bf(x)) = a \cdot \int f(x)dx + b \cdot \int g(x)dx\]
Доказательство в одну сторону:\\
$aF(x) + bG(x) \in \int (af(x) + bg(x))dx\\
aF'(x) + G'(x) = af(x) + bg(x)$ \\

\begin{center}
  \bd{Замена переменных (подстановка) в неопределенных интегралах.}\\
\end{center}
\bk{Теорема:}\\
Пусть $F(t)$ - первообразная для $f(x)$ на промежутке $T$ \\
$t = \varphi (x)$ непрерывна и дифференциируема на $\varphi(\delta) \subset T$ то\\
\[\int f(\varphi (x)) \varphi'(x)dx = F(\varphi (x)) + C\]
Доказательство:
\[(F(\varphi(x)) + C)' = F'(\varphi (x)) \cdot \varphi'(x) + 0 = f(\varphi(x)) \varphi'(x)\]\\
\begin{title}[\Large]
  Интегрирование по частям в неопределенном интеграле.
\end{title}
$u = u(x)$\\
$v = v(x)$\\
$(u(x)\cdot v(x))' = u'(x)\cdot v(x) + u(x)\cdot v'(x)$\\
$\int(u(x)\cdot v(x))'dx = \int u'(x)\cdot v(x)dx + \int u(x)\cdot v'(x)dx$\\
$u(x)\cdot v(x) = \int v(x)du(x) = \int u(x)dv(x)$\\
\[\int udv = uv - \int vdu\]

\begin{title}[\Large]
  Интегрирование простых рациональных дробей.
\end{title}
Рациональная дробь $\frac{P_n(x)}{Q_m(x)}$ где $P_n(x), Q_m(x)$ многочлены.\\
Если $P_n(x) \ge Q_m(x)$ неправильная дробь.\\
Если $P_n(x) < Q_m(x)$ правильная дробь.\\

Существует 4 типа рациональных дробей.\\
1. $\frac{A}{x-a}$\\
2. $\frac{A}{(x-a)^n} ~~~ n > 1$\\
3. $\frac{Ax + B}{x^2 + px +q} ~~~ D = p^2 - 4q < 0$\\
4. $\frac{Ax + B}{(x^2 + px +q)^n} ~~~ D < 0 ~~ n > 1$\\

1. $\int \frac{A}{x-a}dx = A\int \frac{d(x-a)}{x-a} = A\ln|x-a| + C$\\
2. $\int \frac{Adx}{(x-a)^n} = A\int (x-a)^{-n}d(x-a)
  = \frac{A(x-a)^{1-n}}{1-n} + C$\\
3. Приводим числетель в равенство с продиффиринциоравным знаменателем плюс
  константа. (путем вынесени из числителя числа за интеграл).
  Раскладываем интеграл на суммы (разности) интегралов. Интеграл с константой
  в числителе решаем с помощью выражения в знаменателе полного кдвадрата
  (приведение к табличному интегралу). Интеграл с линейным выражением
  в числителе решаем внеся знаменатель под дефференциал.\\
4. То же что в 3 пункте, только надо брать не весь знаменатель, а то что под
  степенью знаменателя. После применить 2 метод.\\

\begin{title}[\Large]
  Интегрирование рациональных дробей.
\end{title}
Любуюю неправильнульную рациональную дробь можно представить в виде:\\
$\frac{P_n(x)}{Q_m(x)} = S_k(x) + \frac{H_c(x)}{Q_n(x)} ~~ m>n$ можно разложить
на сумму простых дробей.\\
\begin{title}
	Гомоморфизм колец и Идеалов
\end{title}

Кольцо является группой по сложению и полугруппой по умножению и гомоморфизм
созраняет обе операции. \\
Базовая в кольце операция - это \kv{сложение}. И прежде всего надо изучить
гомоморфизм комутативной группы по сложению.\\

\bd{Структуры гомоморфизма групп}\\
$\varphi:G \to H$\\
Ядром гомоморфизма $\varphi$ -называют такое подмножество группы G, где
$Ker \varphi = {g \in G|\varphi (g) = e_H}$ \kv{Ker - ядро}\\

\begin{defin}[Свойство 1]
	Ядро - подгруппа.
\end{defin}

\begin{proof}
	Докажем по критерию подгруппы\\
	$g_1, g_2, \in Ker\varphi, g^{-1}_1 \in Ker, ~~~ Ker \in 1$\\
	$\varphi(g_1, g_2) = \varphi(g_1) \varphi(g_2) = e_H \cdot e_H = e_H$\\
	По определению обратного элемента $g_1 \cdot g^{-1}_1 = e_G ~~~
	\varphi(e_G) = \varphi(e_G e_G) = \varphi(e_G) \cdot \varphi(e_G)$\\

	Так как \kv{H} - группа, то у элемена $\varphi(e_G)$ - есть обратный. Умножим
	на него обе части равенства $\varphi(e_G) = e_H$\\
	Таким образом единичный элемент переходит в единичный\\
	$e_H = g(e_G) = \varphi(g_1 g^{-1}_1) = \varphi(g_1)\varphi(g^{-1}_1)
	\Rightarrow \varphi(g^{-1}_1) = e_H$
\end{proof}

\begin{defin}
	$G \ge L$ ~~~ $\ge$ - означает, что подмножество имеет согласованную
	структуру\\
	Подгруппа нормальная, если $\forall g \in G ~~~ g^-1 Lg \subseteq L$\\
	$G^{-1}hg - сопряжается элемент h при помощи элемента q$\\
	\kv{Понятие не нормальной подгруппы нет. ее название подгруппа, не являющаяся
	нормальной}
\end{defin}

\begin{defin}[Свойство 2]
	Ядро - нормальная подгруппа.
\end{defin}

\begin{proof}
	Пусть $f \in Ker\varphi ~~~ \varphi(g^{-1} fg) = e_H ~~~ g\in G$\\
	По определению гомоморфизма
	$\varphi(g^{-1} fg) = \varphi(g^{-1}) \varphi(f) \varphi(g) = \varphi(g)^{-1}
	\varphi(f) \varphi(g) = \varphi(g)^{-1} e_H \varphi(g) = \varphi(g)^{-1}
	\varphi(g) = e_H$\\
	Группа называется простой, если нет нетривиальной нормальной подгруппы.
	(Тривиальная - это группа, состоящая из одного элемента)
\end{proof}

\kv{Основной задачей теории групп является описание всех простых групп, а
остальные группы получаются из простых при помощи расширений.}\\
\kv{Бесконечные группы - не описаны.}\\
\kv{В теории конечных групп задача вроде решена (были найдены 17 бесконечных
серий групп и 26 не серийных изолированных групп)}\\

\begin{title}
	Фактор группы Фактор кольца
\end{title}

$G \ge H$ - H нормальная подгруппа. Подмножество вида $gH = {gh | h \in H}$
называется смежными классами.\\
\kv{Не сложно проверить, что разные смежные классы не пересекаются}\\
$h \mapsto gh$ - биекция.\\
$g_1H /cdot g_2H = (g_1 g_2)H$\\

\begin{defin}
	В смежных классах $gH$, элемент п - представитель. Не сложно проверить, что в
	качестве представителя можно взять любой элемент этого смежного класса.
	Необходимо проверить корректность задания операции, так как она определяется
	через представителей убедимся, что заменив представителей, мы получим тот же
	результат.\\
	$g_1H = g'_1H\\
	g'_1 = g_1 h_1 \in H\\
	g_2H = g'_2H\\
	g'_2 h_2 \in H\\
	h \mapsto gh ~~~ (g'_1 H)(g'_2 H) = (g'_1 g'_2)H = (g_1 h_1)(g_2 h_2)H =
	(g_1 g_2)(g'_2 h_1 g_2)h_2 H = g_1 g_2 H$\\
	Так как подгруппа H - нормальная, то $g^{-1}_2 h_1 g_2 \in H = h_3 ~~~
	\forall h \in H ~~~ h \cdot H = H$\\
\end{defin}

\begin{defin}
	Группа элементы, которой являются смежными классами, определенно задана как
	указано выше и называется фактор группы - $G/H$\\
	Нейтральный элемент в этой группе $e \cdot H = H$\\
	Обратный смежный класс $(gH)^{-1} = g^{-1}H$ ассоциотивность следует из
	ассоциотивности умножения.\\
	Когда целове множество воспринимаем как единый объект - оно называется
	фактором множества - стандартная идея при обобщении.\\
	$G/H = L$ Пусть $G$ - группа, $H$ - нормальная подгруппа, $G/H = L$ - фактор
	группы. Говорят что группа $b$ расширение группы $H$ при помощи группы С.\\
	В конце концов можно свести к простым группам, где строить факторы группы не
	получится. Даже зная все простые группы - все группы построить затруднительно
	так как способов расширений - бесконечное число.
\end{defin}

\begin{title}
	Идеалы и фактор кольцa
\end{title}

Пусть $K$ - кольцо, $L$ - подкольцо и подгруппа по сложению. Подкольцо
называется идеалом, если $\forall k \in K ~~~ k \cdot L = {kl | l \in L} =
Lk = L$\\
\bk{Идеал} - обобщение понятия нуля.\\
Смежные класс по идеалу имеют вид $k + L, k \in K$. На множестве смежных классов
введем операцию сложения и умножения\\
$(k_1 + L) + (k_2 + L) = (k_1 + k_2) + L\\
(k_1 + L) \cdot (k_2 + L) = (k_1 \ cdot k_2) + L$\\
Операции задаются при помощи представителей - проверим их корректность.\\
$k'_1 + L = K_1 + L\\
k'_2 + L = K_2 + L ~~~ I \triangle K ~~~ IK \le I ~~~ I + I = I$\\
Если $K$ - комутативное кольцо, то существуют левые идеалы и правые, а также
двусторонние.\\

\begin{defin}
	Идеал называется максимальным, если он не содержится не в каком больше.
	В дальнейшем будем будем считать, что кольцо комутативное $K + I = {K + i
	| i \in I}$\\
	А множество всех счетных классов заданы сложением и умножением и называются
	фактор-кольцом - $K/I$\\

	\kv{Самые полезные идеалы - главые. Главные идеалы - пораждаются одним
	элементом который является главным}
\end{defin}

\begin{theorem}
	Любое Евклидово кольцо - кольцо главных идеалов.
\end{theorem}

\begin{proof}
	Пусть есть произвольный идеал $I = {i_1, i_2, ... i_n}$ Надо выбрать такой
	элемент $I$, через который можно выразить все остальные $i$. Так как кольцо
	Евклидово, то в нем любые 2 элемента имеют Наибольций Общий Делитель.\\
	$d = НОД(i_1; i_2) ~~~ i_1 d \cdot i'_1 ~~~ i_2 = d \cdot i'_2$\\
	Таким образом и $i_1$ и $i_2$ попадают в идеал, пораждая элемент $d$.
	Пораждающий элемент $I$ - НОД всех этих элементов.
\end{proof}

\begin{theorem}[Основная теорема Алгебры]
	Поле комплексных чисел является алгебраически замкнутым, то есть в нем любой
	многочлен с комплексными коэффициентами раскладывается на минимальный
	множитель
\end{theorem}

\begin{title}
	Симметрические многочлены. Теорема Виета
\end{title}

Пусть P - поле. Рассмотрим кольцо многочленов от n элементов\\
$F(x_1, x_4) = \sum \alpha x^{i_1}_1, x^{i_2}_2, x^{i_3}_3, x^{i_4}_4$\\
Каждое слогаемое называется мономом $x^{5}_1, x^{7}_2, x^{16}_3$
Общая степень = 28\\
Когда у нас одна переменная $x$, то мы можем мономы легко сравнивать. Но когда
переменных $x0$ - несколько, то упорядочить их можно многими способами.\\
Способ упорядочения, который принят в словарях - называется
лексико-графическим.\\
Допустим на множестве мономов мы ввели линейное упорядочение:\\
Первое условие - самый большой моном в смысле этого упорядочения - старший
моном.\\
Второе условие - упорядочение должно быть таким, чтобы количество меньших или
старших было конечно.\\
Третье условие - Сначала мономы сравниваются по общей степени, а когда она
совпадает, то сравниваем по лексико-графически.\\

\begin{defin}
	Многочлен от n-переменных называется симметричным, если он не изменяется при
          любой перестановке входящих в него символов.\\

          Например:\\
          $x + y = y + x$\\
          $xy = yx$\\
          $x^2y + xy^2 = yx^2 + y^2x$\\

         Многочлен от $x$ и $y$ называют \kv{симметрическим}, если он не изменяется
         при замене $x$ на $y$,а $y$ на $x$.
\end{defin}

\begin{theorem}
	Любой симметрический многочлен может быть выражен через элементы симметрии.

           Любой симметрический многочлен от $x$ и $y$ можно представить в виде 
           многочлена от $s_1 = x + y$ и $s_2 = x \cdot y$
\end{theorem}

\begin{theorem}[ Виета]
	Пусть $x_1, x_2, ... x_n$ - корни $n$ - ой степени многочлена $f(x)$\\
	$f(x) = (x - x_1)(x - x_2) ... (x - x_n) = \\
           = S_0 \cdot x^n + S_1 \cdot  x^{n - 1} + S_2 \cdot x^{n - 2}+ S_{n-1}\cdot x+ S_n = 0$\\
           $a_0, a_1, ... , a_{n-1}, a_n$\\
	Коэффициент при степени многочлена с точностью до знака - является элементом
	симметрии многочлена от его корней.\\
          Предположим, что $k_1, k_2, ... , k_{n-1}, k_n$ - корни уравнения\\
          Нагляднее видно формулы Виета не в общем виде,\\
           а на примере, допустим $n = 4$\\
          $a_0\cdot x^4 + a_1 \cdot x^3 + a_2\cdot x^2 + a_3\cdot x + a_4 = 0$\\
          \begin{equation*}
          \begin{cases}
            k_1 + k_2 + k_3 + k_4 = - \frac{a_1}{a_0}\\
            k_1\cdot k_2 + k_1\cdot k_3 + k_1\cdot k_4 + k_2\cdot k_3 + k_2\cdot k_4 + k_3\cdot k_4 = \frac{a_2}{a_0}\\
            k_1 \cdot k_2\cdot k_3 + k_1\cdot k_2\cdot k_4 + k_2\cdot k_3\cdot k_4 = - \frac{a_3}{a_0}\\
            k_1\cdot k_2\cdot k_3\cdot k_4 = \frac{a_4}{a_0}\\
          \end{cases}
          \end{equation*}
	Комбинируя теорему Виета и основную теорему о симметричности многочленов, мы
	можем не находить корни многочлена (вычислять некоторые функции от неизвестных
	нам корней).\\
	Так как нам требуются изучать перестановки переменных - надо ввести обозначать
	и терминалогию из теории групп и перестановок.\\
	$S_n = n!$ S - множество всех перестановок.
\end{theorem}

\begin{displaymath}
\left( \begin{array}{lccr}
1 & 2 & ... & n \\
i_1 & i_2 & ... & i_n
\end{array}\right)
\end{displaymath}

\begin{title}
	Математические основы Криптографии
\end{title}

4 математические идеи\\

\bd{Первая идея} - ключ/шифр должен быть выбран случайным образом, но все
имеющиеся алгебраическое основание.\\
\bd{Вторая идея} - однонаправленная функция $f$ - однонаправлена, если
$f(n) = m$ - вычислить легко, но язная $m$ - найти $n$ - трудно.\\
\bd{Третья идея} - Хэш функция - отображает переводящее сообщение произвольной
длины в сообщении фиксированной длины. Разных хэшей - $2^256$. Но один хэш имеет
бесконечно много сообщений. Это свойство не для одного хэша  строго не доказано.
Хэш криптографический, если в процессе его вычисления используется шифрование.\\
\bd{Четвертая идея} - анализ протоколов. Интерактивный алгоритм в котором
примимают участие 2 или больше ПРОПУСТИЛ В ЛЕКЦИИ - называется протоколом.
Если используется шифрование, то протокол - криптографический. Все компьютерные
процессы - осуществляется посредством протокола.

\end{document}
