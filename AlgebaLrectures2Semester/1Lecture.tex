
\begin{title}
  Фундаментальная и компьютерная \\
  Алгебра\\
  (Второй семестр) \\
\end{title}

\begin{title}[\Large]
  Группа.
\end{title}

  \kv{Группой} называется множество с одной бинарной алгебраической операцией и
должна присутcтвовать ассоциативность, нейтральный эелемент и обратная операция.
Если есть коммутативность, то такую группу называют \kv{абелевой группой}.\\

  Каждой паре элементов $a,b$ отвечает элемент $a*b$ называемый умножением $a,b$
причем:

1) ассоциативно: $(a*b)*c = a*(b*c)$

2) обратная операция: $\forall a ~~ \exists ! a^{-1} ~~ a * (a^{-1}) = 0$

3) нейтральный элемент: $a*e = a$

4) коммутативность (если абелева группа): $a*b = b*a$

\begin{title}[\Large]
  Кольцо.
\end{title}

  \kv{Кольцом} называется множество в двумя бинарными алгебраическими операциями
сложеннием и уномжножением связаных дистрибутивностью. Сложения должно быть
коммутативно, ассоциативно и иметь обратную операцию. Умножение должно быть
accоциативно. Если умножение коммутативно, то кольцо называют \kv{коммутативным
кольцом}. Если кольцо имеет нейтральный элемент относительно умножения то
кольцо называют \kv{кольцом с еденицей}.\\

  Каждой паре элементов $a,b$ отвечает элемент $a+b$ называемый сложением $a,b$
причем:

1) ассоциативно: $\forall a,b,c ~~ (a+b)+c = a+(b+c)$

2) обратная операция: $\forall a ~~ \exists ! a^{-1} ~~ a + (a^{-1}) = 0$

3) коммутативность: $\forall a,b ~~ a+b = b+a$

4) нейтральный элемент: $\forall a ~~ \exists e ~~ a+e = a$\\

  Каждой паре элементов $a,b$ отвечает элемент $a*b$ называемый умножением $a,b$
причем:

1) ассоциативно: $\forall a,b,c ~~ (a*b)*c = a*(b*c)$

2) коммутативность (если коммутативной кольцо): $\forall a,b ~~ a*b = b*a$\\

3) нейтральный элемент (если кольцо с еденицей):
$\forall a ~~ \exists e a*e = a$

  Операция сложения и умножения связаный дистрибутивностью относительно сложения

Дистрибутивность: $\forall a,b ~~ (a + b)*c = a*c + b*c$

\begin{title}[\Large]
  Поле.
\end{title}

  Каждой паре элементов $a,b$ отвечает элемент $a+b$ называемый сложением $a,b$
причем:

1) ассоциативно: $\forall a,b,c ~~ (a+b)+c = a+(b+c)$

2) обратная операция: $\forall a ~~ \exists ! a^{-1} ~~ a + (a^{-1}) = 0$

3) коммутативность: $\forall a,b ~~ a+b = b+a$

4) нейтральный элемент: $\forall a ~~ \exists e ~~ a+e = a$\\

  Каждой паре элементов $a,b$ отвечает элемент $a*b$ называемый умножением $a,b$
причем:

1) ассоциативно: $\forall a,b,c ~~ (a*b)*c = a*(b*c)$

2) коммутативность $\forall a,b ~~ a*b = b*a$

3) нейтральный элемент: $\forall a ~~ \exists e a*e = a$

4) обратная операция: $\forall a ~~ \exists ! a^{-1} ~~ a * (a^{-1}) = 1$

  Операция сложения и умножения связаный дистрибутивностью относительно сложения

Дистрибутивность: $\forall a,b ~~ (a + b)*c = a*c + b*c$

\begin{title}[\Large]
  Линейное пространство.
\end{title}

Вектор.

  Каждой паре векторов $a,b$ отвечает вектор $a+b$ называемый сложением $a,b$
причем:

1) ассоциативно: $\forall a,b,c ~~ (a+b)+c = a+(b+c)$

2) обратная операция: $\forall a ~~ \exists ! a^{-1} ~~ a + (a^{-1}) = 0$

3) коммутативность: $\forall a,b ~~ a+b = b+a$

4) нейтральный элемент: $\forall a ~~ \exists e ~~ a+e = a$\\

  Каждой паре $a,b$ где $a$ число, $b$ вектор отвечает вектор $ab$ называемый
умножением $a,b$ причем:

1) ассоциативно: $(ab)c = a(bc)$

2) нейтральный элемент: $a*e = a$

  Операция сложения и умножения связаный дистрибутивностью:

1) умножение на число дистрибутивно относительно сложения:$(a + b)c = a*c + b*c$

2) умножение на вектор дистрибутивно относительно сложения чисел
$(a + b)c = a*c + b*c$\\

  Множество $K$ называют \kv{линейным пространством} над полеь $P$ если для всех
элементов из $K$ определены операции сложения и умножения на число из $P$,
причем выполнены аксиомы выше.

  Элементы любого линейного пространства называют \kv{векторами}.

  В линейном пространстве $K$ заданом надо полем $P$ выбрано конечное число
произвольных не обязательно различных векторов $e_1, e_2 \ldots e_n$ и называют
эти векторы \kv{система векторов}.

  Одну систему векторов называют \kv{подсистемой} второй системы, если первая
система содержит лишь какие-то векторы второй системы и не содержит никаких
других векторов.

\[
  x = \alpha_1 e_1 + \alpha_2 e_2 + \ldots + \alpha_n e_n
\]

  $\alpha_1, \alpha_2 \ldots \alpha_n$ числа и поля. $e_1, e_2 \ldots e_n$
векторы заданной системы.  В отношении вектора $x$ говорят что он
\kv{линейно выражается} через векторы $e_1, e_2 \ldots e_n$. Правую часть
называют \kv{линейной комбинацией} этих векторов.
$\alpha_1, \alpha_2 \ldots \alpha_n$ называют \kv{коэффицентами линейной
комбинации}.

  Зафиксируем систему векторов $\alpha_1, \alpha_2 \ldots \alpha_n$ и позволим
коэффицентам линейных комбинаций принимать любые значения из поля $P$. Тогда
будет опредленено некоторое множество векторов из $K$. Это множество называется
\kv{линейной оболочкой} векторов $\alpha_1, \alpha_2 \ldots \alpha_n$ и
обозначают $L(\alpha_1, \alpha_2 \ldots \alpha_n)$

\begin{title}
  Теория полей.
\end{title}

\bd{Поле} - множество с двумя алгебраическими операциями (сложение и умножение),
удовлетворяет 10 аксиомам. Фактически выполнимы все 4 арифметические операции.\\

По сравнению с остальными алгебраическими объектами - полей мало. \\

\bd{Поля:}\\
1) {\emph {Конечные}} - Поля Галуа ( построена вся современная криптография)\\
2) {\emph {Бесконечные}} \\

\bd{Характеристика Поля}\\
Определенное минимальное натуральное число \bd{K} - называется {\emph
{характеристикой поля}}, если $\forall \mathit{a} \in \mathit {P,a} \neq 0,
\exists \mathit {k}, \mathit{ka} = 0 $. \\
А если такого {\emph {k}} не существует, то говорят, что характеристика равна
нулю.\\

$\mathbb{Q, R, C}$ -имеют нулевую характеристику.\\

\kv {Как связаны конечные и бесконечные характеристики?} \\

\begin{theorem}
  Если поле имеет нулевую характеристику, то оно бесконечно и содержит в
  качестве под-поля, поле рациональных чисел.
\end{theorem}

\begin{proof}
  Пусть \bk{e} - нейтральный элемент по умножению (мультипликативная еденица).
  Так как поле имеет нулевую характеристику, то $\mathit {e, e+e, ...}$ или
  $\mathit {(e, 2e, ... ne)}$ - все это не нулевые элементы, более того попарно
  различные.\\
  $\mathit {ne = me, n < m}$ Так как это поле, то у элемента \bk{e} есть
  обратный по сложению $\mathit {(m-n)e=0}$ \\
  Если это равенство умножить на любой , то получим, что $\mt {m-n}$ делится на
  характеристику, а значит характеристика не нулевая.\\
  Таким образом последовательность бесконечна, элементы разные, а значит и поле
  бесконечно.\\
  Сконструируем внутри поля \bd{K} поле \bd{Q} \\
  $\mt {e} \mapsto 1 \\
  \mt {ne} \mapsto \mt {n} \\
  \frac{ne}{me} \mapsto \frac{n}{m}$ \\

  Легко проверить, что для элементов вида \bk{ne и me} выполняются все 10 аксиом
  поля и это поле совпадает с полем \bd{Q}
  \end{proof}

\begin{theorem}
  Если поле конечно, то его характеристика не нулевая.
\end{theorem}

\begin{proof}
  Пусть \bk{e} - нейтральный элемент по умножению. Расмотрим последовательность
  $\mt {\{e,2e...ne\}}$. Так как поле конечно, то в последовательности
  встречаются одинаковые элементы.\\
  $\mt{ne=me}$ \\
  Рассуждая как и выше \\
  $\mathit {(m-n)e=0}$ и характеристика не нулевая.
\end{proof}

\kv {Замечание} \\
Обратное утверждение: из конечной характеристики следует конечное поле -
\bd{неверно} \\

\kv {Конечных полей в характеристике всего 2??????????}\\
Пример: $Z_{2} = \{0,1\}$ \\
Построим кольцо многочленов. \\
$Z_{2}[x]$ оно бесконечно, но элементов 2 и от них берем поле дробей:
$Z_{2}(x) \frac{f(x)}{g(x)}$ \\

\begin{theorem}[Теорема о характеристике поля]
  \kv{Если характеристика не нулевая, то она простое число}
\end{theorem}

\begin{proof}
  Пусть \bd{Р} - поле, \bd{k} - его характеристика. $\mt{ke}=0$ \bd {e} -
  нейтральный элемент по умножению. Пусть \bd {k} не простое то есть $k = sr$,
  тогда $(s \cdot r)e = (e+e+e...) \cdot (e+e+e...)$ \\
  Так как в поле нет делителя нуля, то \kv {se = 0 или re = 0} \\
  Допустим \kv {se = 0}. Если $s$ не простое, то продолжим эту же процедуру.\\
  По основной теореме - мы доберемся до простого числа.
\end{proof}

\kv {Пусть \bd{p} - простое число. Существует ли поле с такой
характеристикой?}\\

\begin{theorem}[Первая теорема Галуа]
  Для любого простого числа \bd {p} существует поле с характеристикой \bd {p} \\
  $Z_{p} = \{0,1, p-1\}$
\end{theorem}

\begin{proof}
  Так как оно не комутативное кольцо с едицей, то для того, чтобы стало полем -
  нужно проверить наличие обратных по умножению. \\
  $0<a<p$ \\
  НОД (a,p) = 1 \\
  и из алгоритма Евклида: \\
  $
  \exists \mt {u,v} \in \mathbb {R} \\
  ua + vp = 1 \\
  ua = 1 - pv
  $ \\
  Остаток деления \kv {ua на p = 1} \\
  $\mt {u \cdot a = 1 ~in~} Z_{p}$ \\
  \kv {u} - обратный по умножению к \kv {a} \\
  Таким образом $Z_p$ - поле или GF(p) - поле Галуа
\end{proof}

\begin{title}
  Расширение полей
\end{title}

Если $P \supset F$, то F - \kv {подполе} поля Р, а Р - \kv {расширение} поля F\\

\begin{defin}
  Поле Р - является векторным пространством над полем F
\end{defin}

\begin{proof}
  Пусть $\mt{a,b \in P \quad \alpha,\beta\in F}$ \\
  Нужно 4 аксиомы коммутативности группы, и 4 аксиомы действия, но так как Р -
  поле, то они выполнены. \\
  \kv{Аксиомы действия:} \\
  $\mt{\alpha(a+b)=\alpha a + \alpha b}$ - Дистрибутивность \\
  $\mt{(\alpha \beta)a=\alpha a + \beta a}$ - Дистрибутивность \\
  $\mt{(\alpha \beta)a=\alpha(\beta a)}$ - Ассоциотивность умножения \\
  $1\alpha=a$ - Нейтральный элемент
\end{proof}

\begin{defin}
  Размерность пространства $\mt{dim_{F}P = |P:F|}$ называется степенью
  расширения поля. Если степень \kv{бесконечна}, то размерность
  \kv{бесконечномерна} \\
  $\mt {K \supset P \supset F}$ Если есть цепочка(последовательность)
  расширений, то она называется \kv{башней расширений}
\end{defin}

\begin{theorem}[Теорема о башне конечных расширений]
  Пусть $\mt {K \supset P \supset F}$ - башня конечных расширений, тогда
  размерность $\mt {|K:F|=|K:P|\cdot |P:F|}$
\end{theorem}

\begin{proof}
  $\mt {|K:P|=n \quad (a_{1}, a_{2}...a_{n}}$ \\
  $\mt {|P:F|=n \quad (b_{1}, b_{2}...b_{m}}$ \\
  по определению размерности, поле \bd{K} над полем \bd{P} имеет базис из
  $\mt {a_{1}, a_{2}...a_{n}}$ (из n элементов), а поле \bd{P} над полем \bd{F}
  базис $\mt {b_{1}, b_{2}...b_{m}}$ (из m элементов). \\
  Чтобы доказать теорему, необходимо проверить: $\mt{a_i b_j \quad i = 1 \ldots
  n \quad j = 1 \ldots m}$ \\
  являются ли базисом \bd{K} под \bd{F} и проверить: \\
  1) \kv{Они пораждают множество} \\
  2) \kv{Они линейно не зависимы} \\
  по определению базиса: \\
  $\mt {C \in K}$ \\
  $\mt {\alpha_{1} a_{1}+...\alpha_{n}a_{n}}$\\
  $\mt{\alpha_{i} \in P}$ \\
  А элемент из \kv {P} можно выразить через $\mt {b_{1}, b_{2}...b_{m}}$ c
  коэффициентом из \kv {F} \\
  В итоге \kv {C} выразится через $\mt{a_{i}b_{j}}$ с коэффициентом из \kv{F} \\

  \kv{Докажем линейную независимость}\\
  Пусть напротив они линейно зависимы: \\
  \[ \sum_{\substack {ij}} \gamma_{ij} \quad ij \in F \quad a_{i}b_{j} = 0\] \\
  Запишем эту сумму как линейную комбинацию элементов $a_i$ с коэффициентами из
  \kv {b и $\gamma$} \\
  \[ \sum a_{i} \left ( \sum \gamma_{ij} b_{j} \right) = 0 \] \\
  Так как $a_{i}$ - линейно не зависима, то \\
  \[\sum \gamma_{ij} b_{j} = 0\] \\
  Но так как $b_{i}$ тоже линейно не зависима, то $\gamma_{ij}=0$
\end{proof}

\begin{defin}
  Пусть \kv{P} - некоторое конечное расширение $\mathbb Q$\\
  \[P \supset Q \]
  Число $\alpha$ - называется \kv{алгебраическим}, если оно является корнем
  многочлена с целыми каэффициентами. \\
  Если число не является \kv {алгебраическими}, то оно \bd {трансцендентно}
\end{defin}

\kv {Алгебраических чисел - счетное число}, так как разных многочленов с целыми
коэффициентами - \kv { счетное число}. \\
\kv{Действительных чисел - континуум} \\
Число Эллера {\bk e} = 2.718281828459045 \\
Число ПИ $\pi $ \\
Они трансцендентны.\\

\begin{theorem}
  Если \kv{P} - конечное расширение поля $\mathbb {Q}$, то все его элементы -
  \kv{алгебраические числа}
\end{theorem}

\begin{proof}
  $\mt {a \in P}$ и возводим $\mt {1, a, a^2, a^3...a^n}$ \\
  $\mt {|P:Q| = n}$ \\
  Так как элементов \kv{n+1}, то они \kv{линейно зависимы} \\
  $\alpha_{i} + \alpha_{i}a + ... + \alpha_{i}a$ %в этой формуле я не уверен,
  %да и в следующей тоже. сверь со своим конспектом.
  Находим НОК знаменателя $\alpha_{0} ... \alpha_{n}$ \\
  Умножение на него обе части и выходит многочлен с целыми коэффициентами, а
  элемент \kv{а} - его корень.
\end{proof}