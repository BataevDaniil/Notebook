  В множестве $A$ которое содержит хотябы один элемент определена
\kv{Алгебраическай операция} если любой паре элементов $a,b$ из множества $A$
взятых в определенном порядке соотвецтвует единственный $c$ из тогоже множества.

\begin{title}[\Large]
  Группа.
\end{title}

  \kv{Группой} называется множество с одной бинарной алгебраической операцией и
должна присутcтвовать ассоциативность, нейтральный эелемент и обратная операция.
Если есть коммутативность, то такую группу называют \kv{абелевой группой}.\\

  Каждой паре элементов $a,b$ отвечает элемент $a*b$ называемый умножением $a,b$
причем:

1) ассоциативно: $(a*b)*c = a*(b*c)$

2) обратная операция: $\forall a ~~ \exists ! a^{-1} ~~ a * (a^{-1}) = 0$

3) нейтральный элемент: $a*e = a$

4) коммутативность (если абелева группа): $a*b = b*a$

\begin{title}[\Large]
  Кольцо.
\end{title}

  \kv{Кольцом} называется множество в двумя бинарными алгебраическими операциями
сложеннием и уномжножением связаных дистрибутивностью. Сложения должно быть
коммутативно, ассоциативно и иметь обратную операцию. Умножение должно быть
accоциативно. Если умножение коммутативно, то кольцо называют \kv{коммутативным
кольцом}. Если кольцо имеет нейтральный элемент относительно умножения то
кольцо называют \kv{кольцом с еденицей}.\\

  Каждой паре элементов $a,b$ отвечает элемент $a+b$ называемый сложением $a,b$
причем:

1) ассоциативно: $\forall a,b,c ~~ (a+b)+c = a+(b+c)$

2) обратная операция: $\forall a ~~ \exists ! a^{-1} ~~ a + (a^{-1}) = 0$

3) коммутативность: $\forall a,b ~~ a+b = b+a$

4) нейтральный элемент: $\forall a ~~ \exists e ~~ a+e = a$\\

  Каждой паре элементов $a,b$ отвечает элемент $a*b$ называемый умножением $a,b$
причем:

1) ассоциативно: $\forall a,b,c ~~ (a*b)*c = a*(b*c)$

2) коммутативность (если коммутативной кольцо): $\forall a,b ~~ a*b = b*a$\\

3) нейтральный элемент (если кольцо с еденицей):
$\forall a ~~ \exists e a*e = a$

  Операция сложения и умножения связаный дистрибутивностью относительно сложения

Дистрибутивность: $\forall a,b ~~ (a + b)*c = a*c + b*c$

\begin{title}[\Large]
  Поле.
\end{title}

  Каждой паре элементов $a,b$ отвечает элемент $a+b$ называемый сложением $a,b$
причем:

1) ассоциативно: $\forall a,b,c ~~ (a+b)+c = a+(b+c)$

2) обратная операция: $\forall a ~~ \exists ! a^{-1} ~~ a + (a^{-1}) = 0$

3) коммутативность: $\forall a,b ~~ a+b = b+a$

4) нейтральный элемент: $\forall a ~~ \exists e ~~ a+e = a$\\

  Каждой паре элементов $a,b$ отвечает элемент $a*b$ называемый умножением $a,b$
причем:

1) ассоциативно: $\forall a,b,c ~~ (a*b)*c = a*(b*c)$

2) коммутативность $\forall a,b ~~ a*b = b*a$

3) нейтральный элемент: $\forall a ~~ \exists e a*e = a$

4) обратная операция: $\forall a ~~ \exists ! a^{-1} ~~ a * (a^{-1}) = 1$

  Операция сложения и умножения связаный дистрибутивностью относительно сложения

Дистрибутивность: $\forall a,b ~~ (a + b)*c = a*c + b*c$

\begin{title}[\Large]
  Линейное пространство.
\end{title}

Вектор.

  Каждой паре векторов $a,b$ отвечает вектор $a+b$ называемый сложением $a,b$
причем:

1) ассоциативно: $\forall a,b,c ~~ (a+b)+c = a+(b+c)$

2) обратная операция: $\forall a ~~ \exists ! a^{-1} ~~ a + (a^{-1}) = 0$

3) коммутативность: $\forall a,b ~~ a+b = b+a$

4) нейтральный элемент: $\forall a ~~ \exists e ~~ a+e = a$\\

  Каждой паре $a,b$ где $a$ число, $b$ вектор отвечает вектор $ab$ называемый
умножением $a,b$ причем:

1) ассоциативно: $(ab)c = a(bc)$

2) нейтральный элемент: $a*e = a$

  Операция сложения и умножения связаный дистрибутивностью:

1) умножение на число дистрибутивно относительно сложения:$(a + b)c = a*c + b*c$

2) умножение на вектор дистрибутивно относительно сложения чисел
$(a + b)c = a*c + b*c$\\

\begin{defin}[линейного пространства]
  Множество $K$ называют \kv{линейным пространством} над полем $P$ если для всех
элементов из $K$ определены операции сложения и умножения на число из $P$,
причем выполнены аксиомы выше.
\end{defin}

  Элементы любого линейного пространства называют \kv{векторами}.

\begin{defin}[системы векторов]
  В линейном пространстве $K$ заданом надо полем $P$ выбрано конечное число
произвольных не обязательно различных векторов $e_1, e_2 \ldots e_n$ и называют
эти векторы \kv{система векторов}.
\end{defin}

\begin{defin}[подсистемы системы векторов]
  Одну систему векторов называют \kv{подсистемой} второй системы, если первая
система содержит лишь какие-то векторы второй системы и не содержит никаких
других векторов.
\end{defin}

\[
  x = \alpha_1 e_1 + \alpha_2 e_2 + \ldots + \alpha_n e_n
\]

  $\alpha_1, \alpha_2 \ldots \alpha_n$ числа из поля. $e_1, e_2 \ldots e_n$
векторы заданной системы.  В отношении вектора $x$ говорят что он
\kv{линейно выражается} через векторы $e_1, e_2 \ldots e_n$. Правую часть
называют \kv{линейной комбинацией} этих векторов.
$\alpha_1, \alpha_2 \ldots \alpha_n$ называют
\kv{коэффицентами линейной комбинации}.

\begin{defin}[линейной оболочки]
  Зафиксируем систему векторов $\alpha_1, \alpha_2 \ldots \alpha_n$ и позволим
коэффицентам линейных комбинаций принимать любые значения из поля $P$. Тогда
будет опредленено некоторое множество векторов из $K$. Это множество называется
\kv{линейной оболочкой} векторов $\alpha_1, \alpha_2 \ldots \alpha_n$ и
обозначают $L(\alpha_1, \alpha_2 \ldots \alpha_n)$
\end{defin}

\begin{defin}[базиса]
  Линейно независимая система векторов, через которые линейно выражается каждый
вектор пространства, называется \kv{базисом пространства}.
\end{defin}

  \kv{Размерность} линейного пространства $K$ это число линейно независимых
векторов этого линейного пространства и обозначается $\dim K$. Если $\dim K = n$
то пространство $K$ называется n-мерным. Собственно это тоже что и \kv{ранг}
линейного пространство, но ранг это характеристика эквивалентных линейных
пространств.(информация связанная с рангом это такая себе проверенная
информация)

\begin{defin}[подпространства]
  Пусть в линейном пространстве $K$ задано множество векторов $L$. Если при тех
же операциях что и в пространстве $K$, множество $L$ само является линейным
пространством, то мы будем называть его линейым \kv{подпространством},
подчеркнув тот факт что подпространство состоит из векторов некоторого
пространства.
\end{defin}

\begin{theorem}
  Если в каком-либо подпространстве $L$ размерности $s$ выбран произвольный
базис $t_1, t_2, \ldots, t_s$ то всегда можно выбрать векторы
$t_{s+1}, \ldots, t_n$ в пространстве $K$ размерности $n$, что система
$t_1, \ldots, t_s, t_{s+1}, \ldots, t_n$ будет базисом во всем $K$.
\end{theorem}

  \kv{Суммой} $L_1 + L_2$ линейных пространсв $L_1, L_2$ называется множество
всех векторов вида $z = x + y$ где $x \in L_1, y \in L_2$

  \kv{Пересечением} $L_1 \cap L_2$ линейных подпространств $L_1, L_2$ называется
множество всех векторов, одновременно принадлежащих как $L_1$ так и $L_2$.\\

\begin{theorem}
  Для любых двух коненчомерных подпространств $L_1, L_2$ имеет место равенство
  $$
  \dim(L_1 \cap L_2) + \dim(L_1 + L_2) = \dim L_1 + \dim L_2
  $$
\end{theorem}

\begin{defin}[прямой суммы конечного числа подространств]
  Говорят, что линейное пространство $K$ есть прямая сумма своих подпространств
$L_1, L_2, \ldots, L_n$
  $$
  K = L_1 + L_2 + \ldots + L_n
  $$
  если каждый вектор $x \in K$ предствляется в виде суммы
  $$
  x = x_1, x_2, \ldots, x_n
  $$
  притом единственным образом.
\end{defin}

\begin{theorem}
  Любые два конечномерные линейные пространства, имеющие одинаковую размерность
и заданные над одним и тем же полем, изоморфны.
\end{theorem}

\begin{theorem}
  Для того чтобы пространство $K$ было прямо суммой своих подпространств
$L_1, \ldots, L_m$ необходимо и достаточно чтобы объединение базисов этих
подпространств составляло базис всего пространства.
\end{theorem}

\begin{defin}[изоморфизма]
  Два линейных пространства, заданных над одним и тем же полем, называются
\kv{изоморфным}, если между их векторами можно установить такое взаимно
однозначное соответствие, при котором сумме любых двух векторов первого
пространства будет отвечать сумма соответвсвующих векторов второго пространства,
а произведению какоголибо числа на вектор первого пространства будет отвечать
произведение того же числа на соответствующий вектор второго пространства.
\end{defin}
  Любой вектор $x$ из линейного пространства $K$ может быть представлен в виде
линейной комбинации
\[
  x = \alpha_1 e_1 + \alpha_2 e_2 + \ldots + \alpha_n e_n
\]
где $\alpha_1, \alpha_2 \ldots \alpha_n$ числа из $P$, а $e_1, e_2 \ldots, e_n$
базис $K$. Линейная комбинация называется
\kv{разложением вектора $x$ по базису}, а числа
$\alpha_1, \alpha_2 \ldots \alpha_n$ называются
\kv{координатами вектора $x$ относительно этого базиса}. Записывается это
$x = (\alpha_1, \alpha_2 \ldots \alpha_n)$

  \kv{Величенной} $\{\overrightarrow{AB}\}$ напрвленного отрезка
$\overrightarrow{AB}$ называют число равное длине $\overrightarrow{AB}$, взятое
со знаком плюси если направление $\overrightarrow{AB}$ совподает с направлением
оси, и с минусом если противороложено направлению оси.

\begin{title}[\Large]
  Линейная независимость.
\end{title}

\begin{defin}[линейной независимости]
  \kv{Линейной независимостью} называется система векторов состоящая из одного
ненулевого векторова или из ненулевых векторов не выражаемые линейно через
остальные.
\end{defin}

\begin{theorem}
  Система векторов $e_1, e_2 \ldots e_n$ линейно независима тогда
и только тогда когда из равенства

\[
  \alpha_1 e_1 + \alpha_2 e_2 + \ldots + \alpha_n e_n = 0
\]
следует равенство нулю всех коэффицентов линейной комбинации.
\end{theorem}

\begin{theorem}
  Если не все из векторов $e_1, e_2 \ldots e_n$ нулевые и эта система линейно
зависима, то в ней можно найти линейно независимую подсистему векторов через
которые линейно выражется любой из векторов $e_1, e_2 \ldots e_n$
\end{theorem}

\begin{theorem}
  Если некоторые из векторов $e_1, e_2 \ldots e_n$ линейно зависимы, то и вся
система $e_1, e_2 \ldots e_n$ линейно зависима.
\end{theorem}

\begin{theorem}
  Если среди векторов $e_1, e_2 \ldots e_n$ есть хотябы один нулевой вектор,
то и вся система $e_1, e_2 \ldots e_n$ линейно зависима.
\end{theorem}

\begin{theorem}
  Векторы $e_1, e_2 \ldots e_n$ линейно зависимы тогда и только тогда, когда
либо $e_1 = 0$ либо некоторый вектор $e_k, 2 \le k \le n$ является линейной
комбинацией предшествующих векторов.
\end{theorem}

\begin{title}[\Large]
  Эквивалентность систем векторов.
\end{title}

\begin{defin}[отношения эквивалентности]
  1) Рефликсивность: $\forall a \in A ~~ a \sim a$\\
  2) Симетричность: если $a \sim b$ то $b \sim a$\\
  3) Транзитивность: если $a \sim b ~~ b \sim c$ то $a \sim c$\\
Признак улидовлетворяющий этим условием, называется
\kv{отношением эквивалентности}.
\end{defin}

\begin{defin}[эквивалентность систем векторов]
  Две системы векторов обладают тем свойством, что любой вектор каждой системы
линейно выражается через векторы другой системы, то такие системы назвают
\kv{эквивалентными}.
\end{defin}

\begin{theorem}
  Для того чтобы линейные оболочки двух систем векторов совпадали, необходимо и
достаточно чтобы эти системы были эквивалентными.
\end{theorem}

\begin{theorem}
  Если каждый из векторов линейно независимой системы $e_1, e_2 \ldots e_n$
линейно выражается через векторы $y_1, y_2 \ldots y_m$ то $n \le m$
\end{theorem}

\begin{theorem}
  Процесc последовательной замены двух независимых систем векторов можно
осущиствить так что все промежуточные системы будут линейно зависимы.
\end{theorem}