\begin{title}[\Large]
  Внутренность и замыкания множеств топологического пространства
\end{title}

\begin{define}[внутренней точки]
  $(X, T)$ то и $A \subseteq X ~~ a \in Q$ называется внутренней точкой
  множества $Q$ если $\exists u(a)$ окретность точки $u(a) \subseteq A$
\end{define}

\begin{define}[точки прикосновения]
  $x \in X$ называется точкой прикосновения множества $A$ если всякая его
  окрестность пересекается $A$
\end{define}

\begin{theorem}
  1) $A$ является замкнутым $\Leftrightarrow$ $x \in A$ точка является
  внутренней

  2) $A$ является замкнутой $\Leftrightarrow$ содержит все свои точки
  прикосновения
\end{theorem}

\begin{proof}
  1) $\Rightarrow$ Предположим что $A$ открытой множество тогда $\forall a \in A
  ~~ u(a) = A \subseteq$

  $\Leftarrow$ Предположим что все точкм $A$ является внутренней то есть
  $\forall a \in A ~~ \exists u(a)$ точка $u(a) \subseteq A$ тогда
  $A \cup_{a \in A} u(a)$ открыто

  2) $\Rightarrow$ Предположим что $A$ замкнуто покажем что если $x \in A$
  то $x$ не является точка прикосновения $\Rightarrow$ $u(a) = A$

  $\Leftarrow$ Предположим что $A$ содержит все точкм прикосновения что $A$
  замкнуто то $CA$ открыто, покажем что $\forall x \in CA$ внутренней так как
  $x \in CA$ и $CA$ не является прикосновением $A$ $\Rightarrow$ $\exists u(x)$
  не пересекается с $A$ $\Rightarrow$ $u(x) \subseteq CA$
\end{proof}

\begin{block}[Следствие]
  Внутренностью $A$ называется множество всех его внутренних точек
  $\stackrel{\bullet}{A} \subseteq A$
\end{block}

\begin{define}[Замыкания]
  Замыканием множества $A$ называется множество всех его прикосновений
\end{define}

\begin{theorem}
  1) $A$ открыто когда все его точки внутренни $\Rightarrow$ $A \subseteq A
  ~ \Rightarrow ~ A \subseteq \stackrel{\bullet}{A}$

  2) $A$ является замкнутым $\Leftrightarrow ~ \overline{A} \subseteq A ~
  \Leftrightarrow ~ A = \overline{A}$
\end{theorem}

\begin{theorem}
  1) Внутренность $\stackrel{\bullet}{A}$ является наибольшим открытым множество
  содержащиеся в $A$

  2) Замыкание множество $A$ является наименьшим замкнутым множеством
  содержащим $A$
\end{theorem}
