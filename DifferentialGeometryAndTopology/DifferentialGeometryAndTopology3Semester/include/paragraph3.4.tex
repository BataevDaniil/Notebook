\begin{title}[\Large]
  Параграф 3.4
\end{title}

\begin{title}[\Large]
  Непрерывные отображения топологических пространств
\end{title}

\begin{define}[непрерывности]
  $(X, \mathcal{T}), (Y, \mathcal{T}')$ топологическое пространство
  $f: X \to Y$ называются непрерывными в точке $x_0 \in X$ если
  $$
  \forall O(f(x_0)) ~~ \exists O(x_0) ~~ f(O(x_0)) \subseteq O(f(x_0))
  $$
  Все непрерывные функции в математическом анализе непрерывны в топологии.

  Если отображение непрерывно в каждой точке пространства $X$ то оно
  называется непрерывным
\end{define}

\begin{theorem}
  $f:(X, \mathcal{T}) \to (Y, \mathcal{T}')$ непрерыва
  $\Leftrightarrow$ $\forall V \in \mathcal{T}'$ его полный прообраз
  $$
  f^{-1}(V) = \{ x \in X ~ | ~ f(x) \in V \} \in \mathcal{T}
  $$
  открыт.

  В учебнике Александра Мерзаханяна 59
\end{theorem}

\begin{proof}
  $\Rightarrow$ Пусть $f: X \to Y$ непрерывно и $V \subset Y$ открыто. Докажем,
  что множество $U = f^{-1}(V)$ открыто в $X$.
  Пусть $x_0 \in U ~~ y_0 = f(x_0)$ тогда поскольку $V$ является
  открытой окрестностью точки $y_0 \in Y$, то в силу непрерывности
  отображения $f(x_0)$ существует открытая $O(x_0) \subset X ~~
  f(O(x_0))\subset V ~ \Rightarrow O(x_0) \subset U$ $\Rightarrow$ $U$ открыто в $X$.

  $\Leftarrow$ Пусть $V \subset Y$ открыто $U = f^{-1}(V)$ где $U \subset X$ открыто
  и $x_0 \in X$. Докажем, $f(x_0)$ непрерывно. Пусть
  $y_0 = f(x_0)$, $V$ произвольная открытая окрестность $y_0$ тогда
  $U = f^{-1}(V)$ будет, по условию, открытой окрестностью точки $x_0$, образ
  которой, очевидно, целиком содержится в $V$. Таким образом, отображение $f$
  непрерывно в каждом точке $x_0 \in X$, т. е. является непрерывным отображением.
\end{proof}

\begin{title}[\Large]
  Свойства непрерывных отображений
\end{title}

\begin{theorem}
  Суперпозиция непрерывных отображений является непрерывным отображением.
\end{theorem}

\begin{proof}
  $$
  (X, \mathcal{T}) \stackrel{f}{\to} (Y, \mathcal{T}') \stackrel{g}{\to}
  (Z, \mathcal{T}'') ~~~ (gf)(x) = g(f(x))
  $$
  $$
  V \in \mathcal{T}'' ~~~ (gf)^{-1}(V) =
  f^{-1}(g^{-1}(V)) \in \mathcal{T}
  $$
\end{proof}

\begin{theorem}
  $f: (X, \mathcal{T}) \to (Y, \mathcal{T}')$ непрерывна тогда его
  подпространство $(S, \mathcal{T}_s) \subseteq (X, \mathcal{T})$
  $f|S : (S, \mathcal{T}_s) \to (Y, \mathcal{T}')$ непрерывно.
\end{theorem}

\begin{proof}
  $$
  V \in \mathcal{T}' ~~~ (f | S)^{-1}(V) = \{ x \in S ~ | ~ f(x) \in V \} =
  $$
  $$
  = \{x \in X ~ | ~ f(x) \in V\} \cap S = S \cap f^{-1}(V) \in \mathcal{T}_s
  $$
\end{proof}

\begin{title}[\Large]
  Гомеоморфизмы топологических пространств
\end{title}

\begin{define}[гомеоморфизма]
  $(X, \mathcal{T}), (Y, \mathcal{T}')$ топологические пространства
  $f: X \to Y$ называется гомеоморфизмом если

  1) $f$ биективное отображение

  2) $f$ непрерывное отображений

  3) $f^{-1}$ непрерывное отображений

  $X \approx Y$
\end{define}

\begin{theorem}
  Гомеоморфизм открытых множеств $\Leftrightarrow$ биективность открытых
  множеств.
\end{theorem}

\begin{proof}
  $\Rightarrow$ $f:(X, \mathcal{T}) \stackrel{\approx}{\to} (Y, \mathcal{T}')$
  $$
  \alpha : U \in \mathcal{T} \mapsto f(U) = (f^{-1})^{-1}(U) \in \mathcal{T}'
  $$
  $$
  \beta : V \in \mathcal{T}' \mapsto f^{-1}(V) \in \mathcal{T}
  $$
  $$
  \alpha \cdot \beta (V) = \alpha(\beta(V)) = f(f^{-1}(V)) = V
  $$
  $$
  \beta \cdot \alpha (U) = \beta(\alpha(U)) = f^{-1}(f(U)) = U
  $$
  $\Leftarrow$ так как биктивны открытые множества
  $\Rightarrow$ непрерывны $f, f^{-1}$
\end{proof}