\begin{title}[\Large]
  Параграф 3.4
\end{title}

\begin{title}[\Large]
  Непрерывные отображения топологических пространств
\end{title}

\begin{define}[непрерывности]
  $(X, \mathcal{T}), (Y, \mathcal{T}')$ топологическое пространство
  $f: X \to Y$ называются непрерывными в точке $x_0 \in X$ если
  $$
  \forall O(f(x_0)) ~~ \exists O(x_0) ~~ f(O(x_0)) \subseteq O(f(x_0))
  $$
  Все непрерывные функции в математическом анализе непрерывны в топологию.

  Если отображение непрерывно в каждой точке пространства $X$ то оно
  называется непрерывным
\end{define}

\begin{theorem}
  $f:(X, \mathcal{T}) \to (Y, \mathcal{T}')$ является непрерывным когда
  $\forall V \in \mathcal{T}'$ его полный праобраз
  $$
  f^{-1}(V) = \{ x \in X ~ | ~ f(x) \in V \} \in \mathcal{T}
  $$
  является открытым
\end{theorem}

\begin{proof}
  В учебнике Александра Мерзаханяна
\end{proof}

\begin{title}[\Large]
  Свойства непрерывных отображений
\end{title}

\begin{theorem}
  Суперпозиция непрерывных отображений является непрерывным отображением.
\end{theorem}

\begin{proof}
  $$
  (X, \mathcal{T}) \stackrel{f}{\to} (Y, \mathcal{T}') \stackrel{g}{\to}
  (Z, \mathcal{T}'') ~~~ (gf)(x) = g(f(x))
  $$
  $$
  V \in \mathcal{T}'' ~~~ (gf)^{-1}(V) =
  f^{-1}(\overbrace{g^{-1}(V)}^{\mathcal{T}'}) \in \mathcal{T}
  $$
\end{proof}

\begin{theorem}
  Ограничения непрерывного отображения на подпространства является непрерывным.
\end{theorem}

\begin{proof}
  $f: (X, \mathcal{T}) \to (Y, \mathcal{T}')$ непрерывна и $(S, \mathcal{T}_s)
  \subseteq (X, \mathcal{T})$ топологическое подпространство

  нужно доказать что $f|S : (S, \mathcal{T}_s) \to (Y, \mathcal{T}')$
  $$
  V \in \mathcal{T}' ~~~ (f | S)^{-1}(V) = \{ x \in S ~ | ~ f(x) \in V \} =
  $$
  $$
  = \{x \in X ~ | ~ f(x) \in V\} \cap S = S \cap f^{-1}(V) \in \mathcal{T}_s
  $$
  $f^{-1}(V)$ открыто так как $f$ непрерывно
\end{proof}

\begin{title}[\Large]
  Гомеоморфизмы топологических пространств
\end{title}

\begin{define}[гомеоморфизма]
  $(X, \mathcal{T}), (Y, \mathcal{T}')$ топологические пространства тогда
  $f: X \to Y$ называется гомеоморфизмом если

  1) $f$ биективное отображение

  2) $f$ непрерывное отображений

  3) $f^{-1}$ непрерывное отображений

  $X \approx Y$
\end{define}

\begin{theorem}
  Всякий гомеоморфизм устанавливает взаимооднозначное соответствие между
  открытыми множествами. И обратно, если отображение устанавливает
  взаимооднозначное соответствие мужду открытыми множествами то оно является
  гомеоморфизмом.
\end{theorem}

\begin{proof}
  $\Rightarrow$ $f:(X, \mathcal{T}) \stackrel{\approx}{\to} (Y, \mathcal{T}')$
  $$
  \alpha : U \in \mathcal{T} \mapsto f(U) = (f^{-1})^{-1}(U) \in \mathcal{T}
  $$
  так как $f^{-1}$ непрерывно (всякое открытое множество отображается в открытое
  при гомеоморфизме)
  $$
  \beta : V \in \mathcal{T} \mapsto f^{-1}(V) \in \mathcal{T}
  $$
  $$
  \alpha \cdot \beta (V) = \alpha(\beta(V)) = f(f^{-1}(V)) = V
  $$
  $$
  \beta \cdot \alpha (V) = \beta(\alpha(V)) = f(f^{-1}(V)) = V
  $$
  $\Leftarrow$
\end{proof}
