\begin{title}[\Large]
  Параграф 3.5
\end{title}

\begin{title}[\Large]
  Компактные пространства и множества: определение, примеры, основные свойства
\end{title}

\begin{define}[покрытия и подпокрытия]
  $(X, \mathcal{T})$ топологическое пространство $A \subseteq X$ открыто тогда
  покрытием множества $A$ называется $\{ U_{\alpha} \in \mathcal{T} ~ | ~
  \alpha \in I \} ~~ A \subseteq \cup_{\alpha \in I} U_{\alpha}$

  Если $|I| < \infty$ тогда называется конечным.

  Если $J \subset I ~~ A \subseteq \cup_{\alpha \in J} U_{\alpha}$ тогда
  $\{U_{\alpha} \in \mathcal{T} ~ | ~ \alpha \in J \}$ называется подпокрытием.
\end{define}

\begin{define}[компактного множества]
  Множество $A$ называется компактным если $\forall$ покрытия $\exists$
  конечное подпокрытие.
\end{define}

\begin{block}[Примеры]
  1)  $(0,1)$ в топологии не компактен так как $\{(\frac{1}{n}, 1),
  n = 2,3, \ldots\} ~~~ (0,1) = \cup (\frac{1}{n}, 1)$ это покрытие не имеет
  конечного подпокрытия

  2) $[0,1]$ компактен
\end{block}

\begin{theorem}
  1) $A$ компактно

  2) $f(A)$ непрерывыно

  тогда $f(A)$ компактно.
\end{theorem}

\begin{proof}
  Нужно доказать что $f(A)$ также компактно

  Пусть $f: X \to Y$ непрерывное отображение и $A \subseteq X$ компактное
  множество и $\{ V_{\alpha} ~ | ~ \alpha \in I \}$ покрытие для $f(A) \subseteq
  \cup_{\alpha \in I} V_{\alpha}$ тогда
  $$
  A \subseteq f^{-1} (\cup_{\alpha \in I} V_{\alpha}) =
  \cup_{\alpha \in I} f^{-1}(V_{\alpha}) ~ \text{открыто так как $f$ непрерывно}
  $$
  $\Rightarrow$ $\{f^{-1}(V_{\alpha}) ~ | ~ \alpha \in I\}$ покрытие для $A$
  так как $A$ компактное множество то
  $$
  A \subseteq f^{-1} (V_{\alpha_1}) \cup \ldots \cup f^{-1} (V_{\alpha_k})
  ~ \Rightarrow
  $$
  $$
  f(A) \subseteq V_{\alpha_1} \cup \ldots \cup V_{\alpha_k} ~ \Rightarrow
  $$
  $\{V_{\alpha_1}, \ldots, V_{\alpha_k}\}$ конечное подпокрытие для $f(A)$
\end{proof}

\begin{block}[Следствие]
  $X \approx Y$ тогда если $X$ компактное то $Y$ компактное.
\end{block}

\begin{proof}
  $f:X \stackrel{\approx}{\to} Y$ тогда $Y = f(X)$ компактно
\end{proof}

\begin{title}[\Large]
  Теоремы о связи компактности и замкнутости множеств
\end{title}

\begin{theorem}
  Компактное множество в Хаусдорфовом пространстве замкнуто.
\end{theorem}

\begin{proof}
  Пусть $A \subseteq (X, \mathcal{T})$ компактно и $ca \in CA$ тогда
  $\forall a \in A ~~ \exists O(a) \cap O(ca) = \oslash$

  $\{O(a), a \in A\}$ покрытие для $A$

  так как $A$ компактно то $A \subseteq O(a_1) \cup \ldots \cup O(a_k)$

  $O(ca) = O_{a_1}(ca) \cap \ldots \cap O_{a_k}(ca)$ не пересекаются ни с одной
  окрестностью $O(a_1), \ldots, O(a_k)$

  $\Rightarrow$ $O(ca) \cap A = \oslash$

  $\Rightarrow$ $O(ca) \subseteq CA$

  $\Rightarrow$ все точки $CA$ являются внутренними

  $\Rightarrow$ $CA$ открытое множество

  $\Rightarrow$ $A$ замкнутое
\end{proof}

\begin{theorem}
  Замкнутое подмножество компактного пространства компактно.
\end{theorem}

\begin{proof}
  Учебник Александра Мерзкаханян
\end{proof}

\begin{title}[\Large]
  Критерий компактности в $R^n$
\end{title}

\begin{theorem}
  Подмножество в $R^n$ компактно $\Leftrightarrow$ ограничена и
  замкнута.
\end{theorem}

\begin{proof}
  $\Rightarrow$ пусть $A \subseteq R^n$ компактно. Так как $R^n$ Хаусдорфово
  то по теореме выше $A$ замкнуто. Расмотрим покрытие $\{B_n(0) ~ | ~
  n = 1,2,3, \ldots \}$ Так как $A$ компактно то $A \subseteq B_{n_1}(0) \cup
  \ldots \cup B_{n_k}(0) = B_N(0)$ шар самого большого радиуса
  $\Rightarrow$ $A$ ограничена

  $\Leftarrow$ пусть $A$ замкнуто и ограничено тогда $A \subseteq [\alpha_1,
  \beta_1] \times \ldots \times [\alpha_n, \beta_n]$ $\Rightarrow$ $A$ компактно
  по теоереме выше
\end{proof}

\begin{title}[\Large]
  Теоремы о непрерывных отображениях компактных пространств
\end{title}

\begin{theorem}[Вейерштрасса]
  1) $X$ компактное топологическое пространство

  2) $f: X \to R$ непрерывна

  тогда $\exists x_1, x_2 \in X ~~ \forall x \in X ~~
  f(x_1) \le f(x) \le f(x_2)$
\end{theorem}

\begin{proof}
  $f(x) \subseteq R$ компактно по теоереме выше

  По теореме выше $f(x)$ ограничена и замкнуто $\Rightarrow$ $\inf f(x),
  \sup f(x) \in R$

  $\inf f(x), \sup f(x) \in \overline{f(x)} = f(x)$ где $\overline{f(x)}$
  точки соприкосновения и замыкания $\Rightarrow$ $\inf f(x) = f(x_1) ~~
  \sup f(x) = f(x_2)$ для непрерывных $x_1, x_2 \in X$
\end{proof}

\begin{block}[Следствие]
  1) $X$ компактноe топологическое пространство

  2) $f: X \to R$ непрерывно

  3) $\forall x \in X ~~ f(x) > 0$

  тогда $\exists c > 0 ~~~ f(x) \ge c$
\end{block}

\begin{proof}
   $$
   f(x) \ge \overbrace{\inf f(x) = f(x_1)}^c > 0
   $$
\end{proof}