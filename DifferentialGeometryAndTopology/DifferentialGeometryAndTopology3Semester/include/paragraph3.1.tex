\begin{title}
  Элементы топологии
\end{title}

\begin{title}[\Large]
  Параграф 3.1
\end{title}

\begin{title}[\Large]
  Топологические пространства, открытые и замкнутые множества. Топология
  метрических пространств
\end{title}

\begin{define}[топологической структуры]
  На множестве $X$ задана топология или топологическая структура, если
  выбрана некоторая система подмножеств $\mathcal{T}$ удовлитворяющих условиям

  1) $\oslash, X \in \mathcal{T}$

  2) $U_{\alpha} \in \mathcal{T} ~~~ \forall \alpha ~~~ \cup U_{\alpha} \in
  \mathcal{T}$ (замкнутость $\mathcal{T}$ относительно объедениненя)

  3) $u_1, \ldots, u_k \in \mathcal{T} ~~~ u_1 \cap \ldots \cap u_k \in
  \mathcal{T}$ (замкнутость $\mathcal{T}$ относительно конечных пересечений)
  или тоже что и $u, \upsilon \in \mathcal{T} ~~~ u \cap \upsilon \in
  \mathcal{T}$
\end{define}

\begin{define}[топологического пространства]
  Множество $(X, \mathcal{T})$ на котором задана топологическая структура это
  топологическое пространство.

  Замечание: топология на $X$ может быть задана выделением системы замкнутых
  подмножеств.
\end{define}

\begin{define}[открытого множества]
  $(X, \mathcal{T})$ топологическое пространство тогда множество
  $\mathcal{T}$ будем называть открытым множеством.
\end{define}

\begin{define}[замкнутого множества]
  Множество $A \subseteq X$ называется замкнутым, если его дополнение
  $CA = X \backslash A$ является открытым.
\end{define}

\begin{block}[Свойства замкнутых множеств]
  1) $\oslash, X$ тревиальные подмножества являются замкнутыми

  2) $F_{\alpha}$ замкнуто тогда $\forall \alpha ~~
  \cap F_{\alpha}$ является замкнутым

  3) $F_1, \ldots, F_k$ замкнуто тогда $F_1 \cup \ldots \cup F_k$ замкнуто
\end{block}

\begin{proof}
  1) $\oslash = CX$ замкнуто

  $X = C \oslash$ замкнуто

  2) $F_{\alpha}$ замкнуто тогда $F_{\alpha} = CU_{\alpha}$ для
  $U_{\alpha} \in \mathcal{T}$ тогда
  $$
  \forall \alpha ~ \cap F_{\alpha} = \cap CU_{\alpha} = C(\cup U_{\alpha})
  $$
  замкнуто так как $\forall \alpha ~ \cup U_{\alpha} \in \mathcal{T}$

  2) $i = 1,2, \ldots k$ $F_i$ замкнуто тогда $F_i = CU_i$ для $U_i \in
  \mathcal{T}$ тогда
  $$
  \cup F_i = \cup CU_i = C(\cap U_i)
  $$
  замкнуто так как $\cap U_i \in \mathcal{T}$
\end{proof}

\begin{define}
  Топология на множестве $X$ называется метризуемой если она может быть
  определена с помощью некоторой метрики $\mathcal{T_{\rho}}$
\end{define}

\begin{proof}[топологии метрических пространств]
  $$
  B_r(a) = \{x \in X : ~ \rho(x,a) < r\} ~~ \text{шар}
  $$
  $(X,\rho)$ множество $X$ с метрикой $\rho$

  $U \subseteq X$ называется открытым если $\forall a \in U ~~ \exists r > 0 ~~
  B_r(a) \subseteq U$

  $\mathcal{T}_{\rho}$ семейство открытых множеств в $X$

  1) $\oslash, X \in \mathcal{T}_{\rho}$

  2) $U_{\alpha} \in \mathcal{T}_{\rho}$ тогда $\forall \alpha ~ \cup U_{\alpha}
   \in \mathcal{T}_{\rho}$

  \begin{proof}
    $U = \cup_{\alpha} U_{\alpha}$ по определению открыто

    пусть $a \in U$ тогда $\exists \alpha_0 ~~
    a \in U_{\alpha_0}$ так как $U_{\alpha_0}$ открыто то $\exists B_r(a)$
    лежащий в $U_{\alpha_0} ~ \Rightarrow ~ B_r (a) \subseteq U ~ \Rightarrow ~
    U \in \mathcal{T}_{\rho}$
  \end{proof}
  3)
\end{proof}

\begin{theorem}
  $\forall B_r(a)$ в метрическом пространстве является открытым множеством.
\end{theorem}

\begin{proof}
  Пусть $x \in B_r(a) ~~ d = \rho(x, a)$ тогда $B_{r-d}(x) \subseteq B_r(a)$
\end{proof}

\begin{theorem}
  Подмножество метрического пространства является открытым $\Leftrightarrow$
  его можно представить в виде объеденение шаров.
\end{theorem}

\begin{proof}
  $\Rightarrow$ пусть $U$ открытое множество тогда $\forall a \in U ~~~
  \exists r_a > 0 ~~ B_{r_a}(a) \subseteq U$ тогда
  $$
  U = \bigcup_{a \in U} B_{r_a}(a)
  $$
  $\Leftarrow$ пусть есть объеденение шаров тогда $U$ открыто по теореме выше.
\end{proof}

\begin{define}[Хауздорфовой топологии]
  Топологическое пространство $(X, \mathcal{T})$ называется Хауздорфовой если
  $\forall a,b \in X ~~ a \not= b ~~ \exists U, V \in \mathcal{T} ~~
  a \in U ~~ b \in V ~~ U \cap V = \oslash$
\end{define}

\begin{theorem}
  Всякая метрическая топология является Хауздорфовой.
\end{theorem}

\begin{proof}
  В метрическом прстранстве достаточно
  $$
  U = B_{\frac{d}{2}}(a) ~~~ V = B_{\frac{d}{2}}(b) ~ \text{где} ~ d = \rho(a,b)
  $$
\end{proof}
