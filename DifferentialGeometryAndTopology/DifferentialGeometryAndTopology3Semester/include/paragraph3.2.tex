\begin{title}[\Large]
  Параграф 3.2
\end{title}

\begin{title}[\Large]
  Внутренние точки и точки прикосновения. Замыкание и внутренность множества
\end{title}

\begin{define}[внутренней точки]
  $(X, \mathcal{T})$ и $A \subseteq X ~~ a \in A$ называется внутренней точкой
  множества $A$ если $\exists O(a) \subseteq A$
\end{define}

\begin{define}[точки прикосновения]
  $(X, \mathcal{T})$  и $A \subseteq X$ и $x \in X$ называется точкой
  прикосновения множества $A$ если $\forall O(x) \cap A \not= \oslash$
\end{define}

\begin{theorem}
  1) $A$ открыто $\Leftrightarrow$ содержит все свои внутренние точки

  2) $A$ замкнуто $\Leftrightarrow$ содержит все свои точки прикосновения
\end{theorem}

\begin{proof}
  1) $\Rightarrow$ $A$ открытое множество тогда $\forall a \in A
  ~~ \exists O(a) \subseteq A$ $\Rightarrow$ $a$ внутренняя точка

  $\Leftarrow$ все точки $A$ внутренние то есть
  $\forall a \in A ~~ \exists O(a) \subseteq A$ тогда
  $$
  A = \bigcup_{a \in A} O(a) ~ \text{открыто}
  $$
  2) $\Rightarrow$ $A$ замкнуто тогда если $x \in CA$
  то $x$ не точка прикосновения $\Rightarrow$ $\exists O(x) \cap A = \oslash$
  $\Rightarrow$ $x$ не точка прикосновения

  $\Leftarrow$ $A$ содержит все свои точки прикосновения $\forall x \in CA$ не
  точка прикосновения $A$ $\Rightarrow$ $\exists O(x) \cap A = \oslash ~
  \Rightarrow ~ O(x) \subseteq CA$
\end{proof}

\begin{define}[внутренности множества]
  Внутренностью множества $A$ называется множество всех его внутренних точек
  $\stackrel{\bullet}{A} \subseteq A$
\end{define}

\begin{define}[замыкания]
  Замыканием множества $A$ называется множество всех его точек прикосновений
  $A \subseteq \overline{A}$
\end{define}

\begin{block}[Следствие]
  1) $A$ открыто $\Leftrightarrow$ множество $A$ внутренненость

  2) $A$ замкнуто $\Leftrightarrow$ замыкание множества $A$
\end{block}

\begin{define}[границы множества]
  Граница множества $\partial A = \overline{A} \backslash \stackrel{\bullet}{A}$
\end{define}

\begin{theorem}
  1) Внутренность $A$ является наибольшим открытым множеством
  содержащиеся в $A$

  2) Замыкание множество $A$ является наименьшим замкнутым множеством
  содержащим $A$
\end{theorem}

\begin{proof}
  Учебник Александра Мерзаханяна
\end{proof}
