\begin{title}[\Large]
  Параграф 3.2
\end{title}

\begin{title}[\Large]
  Внутренние точки и точки прикосновения. Замыкание и внутренность множества
\end{title}

\begin{define}[внутренней точки]
  $(X, \mathcal{T})$ открыто и $A \subseteq X ~~ a \in A$ называется внутренней точкой
  множества $A$ если $\exists O(a) \subseteq A$
\end{define}

\begin{define}[точки прикосновения]
  $(X, \mathcal{T})$ открыто и $x \in X$ называется точкой прикосновения множества $A$
  если всякая его окрестность пересекается с $A$
\end{define}

\begin{theorem}
  1) $A$ является открытым $\Leftrightarrow$ $\forall x \in A$ точка является
  внутренней

  2) $A$ является замкнутой $\Leftrightarrow$ содержит все свои точки
  прикосновения
\end{theorem}

\begin{proof}
  1) $\Rightarrow$ Предположим что $A$ открытое множество тогда $\forall a \in A
  ~~ O(a) = A \subseteq A$

  $\Leftarrow$ Предположим что все точки $A$ являются внутренними то есть
  $\forall a \in A ~~ \exists O(a) \subseteq A$ тогда
  $$
  A = \bigcup_{a \in A} O(a) ~ \text{открыто}
  $$
  2) $\Rightarrow$ Предположим что $A$ замкнуто тогда если $x \in CA$
  то $x$ не является точка прикосновения $\Rightarrow$ $O(a) = A$ не пересекаеся
  c $A$  $\Rightarrow$ $x$ не является точкой прикосновения

  $\Leftarrow$ Предположим что $A$ содержит все свои точки прикосновения нужно
  доказать что $A$ замкнуто или $CA$ открыто. Покажем что $\forall x \in CA$
  внутренняя так как
  $x \in CA$ то $x$ не является точкой прикосновением $A$ $\Rightarrow$
  $\exists O(x)$ не пересекающаяся с $A$ $\Rightarrow$ $O(x) \subseteq CA$
\end{proof}

\begin{define}[внутренности множества]
  Внутренностью множества $A$ называется множество всех его внутренних точек
  $\stackrel{\bullet}{A} \subseteq A$
\end{define}

\begin{define}[Замыкания]
  Замыканием множества $A$ называется множество всех его прикосновений
  $A \subseteq \overline{A}$
\end{define}

\begin{block}[Следствие]
  1) $A$ открыто когда все его точки внутренни то есть
  $A \subseteq \stackrel{\bullet}{A} ~ \Leftrightarrow ~
  A = \stackrel{\bullet}{A}$

  2) $A$ является замкнутым $\Leftrightarrow ~ \overline{A} \subseteq A ~
  \Leftrightarrow ~ A = \overline{A}$
\end{block}

\begin{define}[границы множества]
  Граница множества $\partial A = \overline{A} \backslash \stackrel{\bullet}{A}$
\end{define}

\begin{theorem}
  1) Внутренность $A$ является наибольшим открытым множество
  содержащиеся в $A$

  2) Замыкание множество $A$ является наименьшим замкнутым множеством
  содержащим $A$
\end{theorem}

\begin{proof}
  Учебник Александра Мерзаханяна
\end{proof}
