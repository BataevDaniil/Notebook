\begin{title}
  Первая квадратичная форма поверхности
\end{title}

Пусть $\vec \varphi = \vec \varphi(u, \upsilon)$ регулярная поверхность
$$
\left\{
  \begin{array}{l}
    u = u(t) \\
    \upsilon = \upsilon(t)
  \end{array}
\right. ~ \text{- кривая на поверхности, тогда} ~~
S = \int_{t_1}^{t_0} |\vec \varphi'(t)| dt ~ \text{- длина}
$$
$\vec \varphi' = \vec \varphi_u u' + \vec \varphi_{\upsilon} \upsilon'$

$
|\varphi'|^2 = (\varphi', \varphi') =
(\vec \varphi_u u' + \vec \varphi_{\upsilon} \upsilon',
\vec \varphi_u u' + \vec \varphi_{\upsilon} \upsilon') =
$

$
= (\varphi_u, \varphi_u)u'^2 + 2(\varphi_u, \varphi_{\varepsilon}')
u' \upsilon' + (\varphi_u, \varphi_{\upsilon} \upsilon')
$

$E = (\varphi_u, \varphi_u), F = (\varphi_u, \varphi_{\upsilon}),
G = (\varphi_{\upsilon}, \varphi_{\upsilon})$ - зависит только от поверхности
но не от кривой

$|\vec \varphi'|^2 = E u'^2 + 2Fu'\upsilon' + G\upsilon'^2$ кривая задана в
локальной системе координат

$$
S = \int_{t_0}^{t_1} \sqrt{E u'^2 + 2Fu'\upsilon' + G\upsilon'^2}dt
$$

$ds^2 Edu'^2 + 2Fdu'd\upsilon' + Gd\upsilon'^2dt$ первая квадратичная форма
поверхности в дифференциале

Пусть $\vec a = \xi_1 \vec \varphi_u + \xi_2 \vec \varphi_{\upsilon} ~~~
\vec b = \zeta_1 \vec \varphi_u + \zeta_2 \vec \varphi_{\upsilon}$

$(\vec a, \vec b) = (\xi_1 \vec \varphi_u + \xi_2 \vec \varphi_{\upsilon},
\zeta_1 \vec \varphi_u + \zeta_2 \vec \varphi_{\upsilon}) =
E \xi_1 \zeta_1 + F(\xi_1 \zeta_2 + \xi_2 \zeta_1) + G\xi_2 \zeta_2$

1) $|\vec a| = \sqrt{(\vec a, \vec a)} = \sqrt{E\xi_1^2 + 2F(\xi_1, \xi_2) +
G\xi_2^2}$

2) $\cos \measuredangle (\vec a, \vec b) =
\frac{(\vec a, \vec b)}{|\vec a||\vec b|} =
\frac{E \xi_1 \zeta_1 + F(\xi_1 \zeta_2 + \xi_2 \zeta_1) + G\xi_2 \zeta_1}
{\sqrt{E\xi_1^2 + 2F\xi_1\xi_2 + G\xi_2^2}
\sqrt{E\zeta_1^2 + 2F\zeta_1\zeta_2 + G\zeta_2^2}}$

Расмотрим следующие кривые
$$
\left\{
  \begin{array}{l}
    u = u_1(t) \\
    \upsilon = \upsilon_1(t)
  \end{array}
\right. ~~~
\left\{
  \begin{array}{l}
    u = u_2(t) \\
    \upsilon = \upsilon_2(t)
  \end{array}
\right.
$$

Угол между кривыми это угол между их касательными в точке пересечения

$\cos \alpha = |\cos \measuredangle (\vec \varphi_1', \vec \varphi_2')|$

$\vec \varphi_1' = \vec \varphi_u u_1' + \vec \varphi_{\varepsilon}'
\varepsilon_1' = (u_1', \upsilon_1')$

$\vec \varphi_2' = \vec \varphi_u u_2' + \vec \varphi_{\varepsilon}'
\varepsilon_2' = (u_2', \upsilon_2')$
$$
\cos \alpha = \frac{Eu'_1 u'_2 + F(u'_1\upsilon_2' + u'_2 u'_1) + Gu_1'
\upsilon'_1}
{\sqrt{Eu_1'^2 + 2Fu_1'\upsilon_1' + G\upsilon_1'^2}
\sqrt{Eu_1'^2 + 2Fu_1'\upsilon_1' + G\upsilon_1'^2}}
$$

\begin{block}[Свойства]
  1. $E,G > 0$
  \begin{proof}
    $\vec \varphi = \vec \varphi (u, \upsilon)$ регулярное поверхность
    $ds^2 = Edu^2 + 2Fdud\upsilon + Gd\upsilon^2$

    $E = (\vec \varphi_u, \vec \varphi_u) = |\vec \varphi_u|^2 > 0$
    $G = (\vec \varphi_{\upsilon}, \vec \varphi_{\upsilon}) =
    |\vec \varphi_{\upsilon}|^2 > 0$
  \end{proof}

  2. первая квадратичная формула положительна определена
  $$
  A =
  \left|
  \begin{array}{cc}
    E & F \\
    F & G
  \end{array}
  \right|
  $$
  $$
  \left|
  \begin{array}{cc}
    E & F \\
    F & G
  \end{array}
  \right|
  = EG - F^2 = |\vec \varphi_u|^2|\vec \varphi_u|^2 - |\vec \varphi_u|^2
  |\vec \varphi_{\upsilon}|^2 \cos^2 \alpha = |\vec \varphi_u|^2
  |\vec \varphi_{\upsilon}|^2 \sin^2 \alpha = |[\vec \varphi_u, \vec
  \varphi_{\upsilon}]|^2 > 0
  $$
  так как $\vec \varphi_u, \vec \varphi_{\upsilon}$ лнз

  $d\upsilon = |[\vec \varphi_u du, \vec \varphi_{\upsilon} d\upsilon]| =
  |[\vec \varphi_u, \vec \varphi_{\upsilon}]| du d\upsilon$

  $$
  G = \iint \limits_{\Omega} |[\vec \varphi_u, \vec \varphi_{\upsilon}]|
  du d\upsilon = \iint \limits_{\Omega} \sqrt{EL - F^2}dud\upsilon
  $$
\end{block}

\begin{define}[изометрического отображения поверхности]
  Отображение поверхноси называется изометрическим если оно сохраняется длины
  кривых и углы между кривыми.
\end{define}

\begin{theorem}
  Поверхность является локальным изометрическим тогда когда выбрать
  параметризацию так что соответствующим квадтртичной формы будут одинаковы
\end{theorem}

\begin{define}[конформного отображения поверхности]
  Отображения поверхности сохраняющая углы между кривыми называется
  конформным.
\end{define}

\begin{theorem}
  Две поверхности локально комфорны тогда и только тогда когда на них можно
  выбрать параметризацию так что соответсвующие квадратичные формы будут
  параметризованными.
\end{theorem}

\begin{title}[\Large]
  Нормальная кривизна кривой на поверхности второй квадратичной формы
  поверхности
\end{title}

Пусть $\vec \varphi = \vec \varphi(u, \upsilon)$ регулярная поверхность

$\vec m$ еденичная нормаль к поверхности

$\vec m = \frac{[\vec \varphi_u, \vec \varphi_{\upsilon}]}{[\vec \varphi_u,
\vec \varphi_{\upsilon}]} = \frac{[\vec \varphi_u, \vec \varphi_{\upsilon}]}
{EG - F^2}$

пусть
$$
\left\{
\begin{array}{c}
  u = u(s) \\
  \upsilon = \upsilon(s)
\end{array}
\right. ~ \text{кривая на поверхности с натуральным параметром $s$}
$$
$\vec \varphi = \vec \varphi(u(s), \upsilon(s))$ уравнение кривой в декартовой
системе координат

$\cos \alpha = K \cos \alpha$

$\stackrel{\bullet}{\vec \varphi} = \vec \varphi_u \stackrel{\bullet}{u} +
\vec \varphi_{\upsilon} \stackrel{\bullet}{\upsilon}$

$$
\stackrel{\bullet \bullet}{\vec \varphi} = \vec \varphi{uu}
\stackrel{\bullet}{u^2} + 2\vec \varphi{u\upsilon}\stackrel{\bullet}{u}
\stackrel{\bullet}{\upsilon} + \vec \varphi_{\upsilon \upsilon} \upsilon^2
+ 2\stackrel{\bullet}{\upsilon}\stackrel{\bullet \bullet}{\upsilon} +
\vec \varphi_u \stackrel{\bullet \bullet}{u} =
$$
$$
= \frac{Ldu^2 + 2Mdud\upsilon + Nd\upsilon^2}{ds^2} =
= \frac{Ldu^2 + 2Mdud\upsilon + Nd\upsilon^2}{Edu^2 + 2Fdu d\upsilon +
Gd\upsilon^2}
$$
$II Ldu^2 + 2Mdud\upsilon + Nd\upsilon^2$ вторая квадратичная форма поверхности