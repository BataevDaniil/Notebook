\begin{title}
  Элементы топологии
\end{title}

\begin{title}[\Large]
  Топологическое пространство
\end{title}

\begin{define}[топологической структуры]
  На множестве $X$ задана топология или топологическая структура, если в нем
  выбрана некоторая система подмножеств $T$ удовлитворяющих условиям

  1. $\oslash ~~~ x \in T$

  2. если $u_{\alpha}, \alpha \in T$ принадлежит семейству $T$ тогда
  $\cup u_{\alpha} \in T$ (замкнутость $T$ относительно объедениненя)

  3. если $u_1, \ldots, u_k \in T$ тогда $u_1 \cap \ldots \cap u_k \in T$
  (замкнутость $T$ относительно конечных пересичений)

  $\sim$ 3 если $u, \upsilon \in T$ тогда $u \cap \upsilon \in T$
\end{define}

\begin{define}[топологического пространства]
  Множество $(X, T)$ это топологическое пространство
\end{define}

если $(X, T)$ топологическое пространство тогда множество $T$ будем называть
открытым множеством

\begin{define}
  Множество $A \subseteq X$ называется замкнутым, если его дополнение
  $CA = X \backslash A$ является открытым
\end{define}

\begin{block}[Свойства]
  1) $\oslash ~ X$ тревиальные подмножества являются замкнутыми

  2) Если $F_{\alpha}, \alpha \in I$ замкнутым тогда $\cap_{\alpha}F_{\alpha}$
  является замкнутым

  3) Если $F_1, \ldots, F_k$ замкнуто тогда $F \cup \ldots \cup F_k$ замкнуто
\end{block}

\begin{proof}
  1) $\oslash = CX$ - замкнуто

  $X = C \oslash$ - замкнуто

  2) Так как $F_{\alpha}$ замкнуто тогда $F_{\alpha} = CU_{\alpha}$ для
  $U_{\alpha} \in T$ тогда $\cap_{\alpha}F_{\alpha} = \cap_{\alpha}CU_{\alpha}
  = C(\cup_{\alpha}U_{\alpha})$ замкнуто так как $\cup_{\alpha} U_{\alpha}
  \in T$

  3) Аналогично
\end{proof}

\begin{block}[Замечание]
  Топология на $X$ может быть задана выделением системы замкнутых подмножеств
\end{block}

\begin{title}[\Large]
  Топология метрических пространства
\end{title}

$$
B_r(a) = \{x \in X : ~ \rho(x,a) < r\} ~~ \text{шар}
$$

Открытое множество $(x,\rho)$ $U \supseteq X$ называется открытым если
$\forall a \in U ~~ \exists r > 0$

$r_{\rho}$ семейство открытых множеств $x$ докажем что задает топологическую
структуру

1) $\oslash, X \in \tau_{\rho}$

2) $U_{\alpha} \in \tau_{\rho} (\alpha \in I)$

тогда $\cup_{\varphi \in I} U_{\varphi} \tau_{\rho}$

\begin{proof}
  $U = \cup_{\alpha} u_{\alpha}$ пусть $a \in U$ тогда $a \in U_{\alpha}$ для
  некоторого $\alpha_0$ так как $U_{\alpha_0}$ открыто то $\exists B_r(a)$
  лежащий в $u_{\alpha_0} ~~ \Rightarrow ~ B_r (a) \subseteq U ~~ \Rightarrow ~
  U \in \tau_{\rho}$
\end{proof}

\begin{theorem}
  $\forall B_r(a) \in$ метрическому пространству является открытым множеством
\end{theorem}

\begin{theorem}
  Подмножество метрического пространства является открытым тогда и только
  тогда когда его можно представить в виде объеденение шароф
\end{theorem}

\begin{proof}
  $\Rightarrow$ пусть $u$-открытое множество тогда $\forall a \in K ~~~
  \exists r_a > 0$ точка $B_{r_a}(a) \subset U$ тогда $u = \cup_{a \in u}
  B_{r_a}(a)$

  $\Leftarrow$ пусть есть объеденение шаров тогда $U$ открыто так как T2
\end{proof}

\begin{theorem}
  Всякая метрическая топология является Хауздорфовой.

  Топологическое пространство называется Хауздорфовой если $\forall a,b \in X ~
  x \not= 0$ открыто $\exists U, V \in T$ тогда $a \in U, ~ b \in V ~~~
  U \cap V = \oslash$
\end{theorem}


