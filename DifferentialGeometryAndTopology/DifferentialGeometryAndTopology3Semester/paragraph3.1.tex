\begin{title}
  Элементы топологии
\end{title}

\begin{title}[\Large]
  Топологическое пространство
\end{title}

\begin{define}[топологической структуры]
  На множестве $X$ задана топология или топологическая структура, если в нем
  выбрана некоторая система подмножеств $T$ удовлитворяющих условиям

  1. $\oslash ~~~ x \in T$

  2. если $u_{\alpha}, \alpha \in T$ принадлежит семейству $T$ тогда
  $\cup u_{\alpha} \in T$ (замкнутость $T$ относительно объедениненя)

  3. если $u_1, \ldots, u_k \in T$ тогда $u_1 \cap \ldots \cap u_k \in T$
  (замкнутость $T$ относительно конечных пересичений)

  $\sim$ 3 если $u, \upsilon \in T$ тогда $u \cap \upsilon \in T$
\end{define}

\begin{define}[топологического пространства]
  Множество $(X, T)$ это топологическое пространство
\end{define}

если $(X, T)$ топологическое пространство тогда множество $T$ будем называть
открытым множеством

\begin{define}
  Множество $A \subseteq X$ называется замкнутым, если его дополнение
  $CA = X \backslash A$ является открытым
\end{define}

\begin{block}[Свойства]
  1) $\oslash ~ X$ тревиальные подмножества являются замкнутыми

  2) Если $F_{\alpha}, \alpha \in I$ замкнутым тогда $\cap_{\alpha}F_{\alpha}$
  является замкнутым

  3) Если $F_1, \ldots, F_k$ замкнуто тогда $F \cup \ldots \cup F_k$ замкнуто
\end{block}

\begin{proof}
  1) $\oslash = CX$ - замкнуто

  $X = C \oslash$ - замкнуто

  2) Так как $F_{\alpha}$ замкнуто тогда $F_{\alpha} = CU_{\alpha}$ для
  $U_{\alpha} \in T$ тогда $\cap_{\alpha}F_{\alpha} = \cap_{\alpha}CU_{\alpha}
  = C(\cup_{\alpha}U_{\alpha})$ замкнуто так как $\cup_{\alpha} U_{\alpha}
  \in T$

  3) Аналогично
\end{proof}

\begin{block}[Замечание]
  Топология на $X$ может быть задана выделением системы замкнутых подмножеств
\end{block}
