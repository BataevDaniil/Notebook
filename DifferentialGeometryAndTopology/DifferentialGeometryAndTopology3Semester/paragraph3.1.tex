\begin{title}
  Элементы топологии
\end{title}

\begin{title}[\Large]
  Параграф 3.1
\end{title}

\begin{title}[\Large]
  Топологические пространства, открытые и замкнутые множества. Топология
  метрических пространств
\end{title}

\begin{define}[топологической структуры]
  На множестве $X$ задана топология или топологическая структура, если
  выбрана некоторая система подмножеств $\mathcal{T}$ удовлитворяющих условиям

  1) $\oslash, X \in \mathcal{T}$

  2) $U_{\alpha} \in \mathcal{T} ~~~ \forall \alpha ~~~ \cup U_{\alpha} \in
  \mathcal{T}$ (замкнутость $\mathcal{T}$ относительно объедениненя)

  3) $u_1, \ldots, u_k \in \mathcal{T} ~~~ u_1 \cap \ldots \cap u_k \in
  \mathcal{T}$ (замкнутость $\mathcal{T}$ относительно конечных пересечений)
  или тоже что и $u, \upsilon \in \mathcal{T} ~~~ u \cap \upsilon \in
  \mathcal{T}$
\end{define}

\begin{define}[топологического пространства]
  Множество $(X, \mathcal{T})$ на котором задана топологическая структура это
  топологическое пространство.

  Множества, принадлежащие $\mathcal{T}$ называются открытыми множествами.

  Замечание: топология на $X$ может быть задана выделением системы замкнутых
  подмножеств.
\end{define}

\begin{define}[открытого множества]
  Если $(X, \mathcal{T})$ топологическое пространство тогда множество
  $\mathcal{T}$ будем называть открытым множеством.
\end{define}

\begin{define}[замкнутого множества]
  Множество $A \subseteq X$ называется замкнутым, если его дополнение
  $CA = X \backslash A$ является открытым.
\end{define}

\begin{block}[Свойства замкнутых множеств]
  1) $\oslash, X$ тревиальные подмножества являются замкнутыми

  2) Если $F_{\alpha} \in \mathcal{T}$ замкнуто тогда $\forall \alpha ~~~
  \cap F_{\alpha}$ является замкнутым

  3) Если $F_1, \ldots, F_k$ замкнуто тогда $F \cup \ldots \cup F_k$ замкнуто
\end{block}

\begin{proof}
  1) $\oslash = CX$ замкнуто

  $X = C \oslash$ замкнуто

  2) $F_{\alpha}$ замкнуто тогда $F_{\alpha} = CU_{\alpha}$ для
  $U_{\alpha} \in \mathcal{T}$ тогда
  $$
  \forall \alpha ~ \cap F_{\alpha} = \cap CU_{\alpha} = C(\cup U_{\alpha})
  $$
  замкнуто так как $\forall \alpha ~ \cup U_{\alpha} \in \mathcal{T}$

  2) $F_{\alpha}$ замкнуто тогда $F_{\alpha} = CU_{\alpha}$ для
  $U_{\alpha} \in \mathcal{T}$ тогда
  $$
  \forall \alpha ~ \cup F_{\alpha} = \cup CU_{\alpha} = C(\cap U_{\alpha})
  $$
  замкнуто так как $\forall \alpha ~ \cap U_{\alpha} \in \mathcal{T}$
\end{proof}

Топология метрических пространства
$$
B_r(a) = \{x \in X : ~ \rho(x,a) < r\} ~~ \text{шар}
$$
Открытое множество $(x,\rho)$ $U \supseteq X$ называется открытым если
$\forall a \in U ~~ \exists r > 0$

$r_{\rho}$ семейство открытых множеств $x$ докажем что задает топологическую
структуру

1) $\oslash, X \in \tau_{\rho}$

2) $U_{\alpha} \in \tau_{\rho} (\alpha \in I)$

тогда $\cup_{\varphi \in I} U_{\varphi} \tau_{\rho}$

\begin{proof}
  $U = \cup_{\alpha} u_{\alpha}$ пусть $a \in U$ тогда $a \in U_{\alpha}$ для
  некоторого $\alpha_0$ так как $U_{\alpha_0}$ открыто то $\exists B_r(a)$
  лежащий в $u_{\alpha_0} ~~ \Rightarrow ~ B_r (a) \subseteq U ~~ \Rightarrow ~
  U \in \tau_{\rho}$
\end{proof}

\begin{theorem}
  $\forall B_r(a) \in$ метрическому пространству является открытым множеством
\end{theorem}

\begin{theorem}
  Подмножество метрического пространства является открытым тогда и только
  тогда когда его можно представить в виде объеденение шароф
\end{theorem}

\begin{proof}
  $\Rightarrow$ пусть $u$-открытое множество тогда $\forall a \in K ~~~
  \exists r_a > 0$ точка $B_{r_a}(a) \subset U$ тогда $u = \cup_{a \in u}
  B_{r_a}(a)$

  $\Leftarrow$ пусть есть объеденение шаров тогда $U$ открыто так как T2
\end{proof}

\begin{theorem}
  Всякая метрическая топология является Хауздорфовой.

  Топологическое пространство называется Хауздорфовой если $\forall a,b \in X ~
  x \not= 0$ открыто $\exists U, V \in T$ тогда $a \in U, ~ b \in V ~~~
  U \cap V = \oslash$
\end{theorem}
