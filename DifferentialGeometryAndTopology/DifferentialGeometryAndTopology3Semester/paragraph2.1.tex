\begin{title}
  Глава 2
\end{title}

\begin{title}
  Теория поверхностей
\end{title}

Формула для нахождения кривизны плоской кривой
$$
k = \frac{
  + \left|
  \begin{array}{cc}
    x' & y' \\
    x'' & y''
  \end{array}
  \right|
}{(x'^2 + y'^2)^{\frac{3}{2}}}
$$
$y = f(x)$
$$
  \left\{
  \begin{array}{c}
    x = t \\
    y = f(t)
  \end{array}
  \right. ~~~
k = \frac{
  + \left|
  \begin{array}{cc}
    1 & f' \\
    0 & f''
  \end{array}
  \right|
}{(x'^2 + f'^2)^{\frac{3}{2}}}
=
\frac{|f''|}{(1 + f'^2)^{\frac{3}{2}}}
$$

\begin{title}
  Параграф 1
\end{title}

\begin{title}[\Large]
  Определение и способы задания поверхностей
\end{title}

\begin{define}
  Поверхность - это отображение непрерывной поверхностью в $R^3$, называется
  непрерывное отображение $\varphi: ~ U \subseteq R^2 \to R^3$, где
  $U \subseteq R^2$ - открытое множество на плоскости

  $(u, \upsilon) \in U \to \varphi(u, \upsilon) = (x(u, \upsilon),
  y(u, \upsilon), z(u, \upsilon))$

  $\vec \varphi (U) \subseteq R^2$ - образ поверхности

  $\vec z = (x(u, \upsilon), y(u, \upsilon), z(u, \upsilon))$ - вектор функция

  $$
  \left\{
  \begin{array}{c}
    x = x(u, \upsilon) \\
    y = y(u, \upsilon) \\
    z = z(u, \upsilon)
  \end{array}
  \right. ~~~ \text{параметрическое уравнение поверхности}
  $$

  $\vec \varphi = \vec \varphi(u, \upsilon)$ - называется гладкая если
  задающие ее координаты функции являются гладкими
\end{define}

\begin{define}
  Гладка поверхность $\vec \varphi = (x(u, \upsilon), y(u, \upsilon),
  z(u, \upsilon))$ называется регулярной если
  $$
  \forall (u, \upsilon) ~~~
  \left(
  \begin{array}{ccc}
    \frac{\partial x}{\partial u} & \frac{\partial y}{\partial u} &
    \frac{\partial z}{\partial u} \\
    \frac{\partial x}{\partial \upsilon} &
    \frac{\partial y}{\partial \upsilon} &
    \frac{\partial z}{\partial \upsilon}
  \end{array}
  \right) = 2 ~ \Rightarrow
  $$
  $\vec \varphi_u = (x_u, y_u, z_u) ~~~ \vec \varphi_{\upsilon} ~~~
  (x_{\upsilon}, z_{\upsilon}, z_{\upsilon})$ - линейно независимы
\end{define}

\begin{theorem}
  Пусть $\varphi = \varphi(u, \upsilon)$ - гладкая поверхность и
  $(u_0, \upsilon_0)$ такова что $\vec \varphi_{u} (u_0, \upsilon_0),
  \vec \varepsilon_{\upsilon} (u_0, \upsilon_0)$ линейно независисмы, тогда
  $\exists O(U(u_0, \upsilon_0))$ такова что $\vec \varphi :~ U(u_0,
  \upsilon_0) \to R^3$ - инективно. (разные точки поверхности переходят в
  разные точки пространства)
\end{theorem}

\begin{block}[Следствие]
  Локально каждая точка образа поверхности однозначно определяется своими
  параметрами $(u, \upsilon)$ (невозможно что две точки проходят в одну точку
  образа)

  Будем говорит что $(u, \upsilon)$ задаются на поверхности локальную систему
  координат
\end{block}

\begin{theorem}
  Пусть $f: ~ U \subseteq R^n \to R^m ~~ m \le n$ гладкое отображение и точка
  $x_0 \in U$ такова что $rang ~ kdf(x_0) = u$ тогда $\exists O(U(x_0))$
  такая что $f:~ U(x_0) \to R^m$ инективно
\end{theorem}

\begin{proof}
  $f: U \subseteq R^2 \to R^3$

  $f = (x(u, \upsilon)), y(u, \upsilon), z(u, \upsilon))$ в каждой точке
  дифференциала это линейное отображение
  $$
  df =
  \left(
  \begin{array}{ccc}
    x_u & y_u & z_u \\
    x_{\upsilon} & y_{\upsilon} & z_{\upsilon} \\
  \end{array}
  \right)^T
  $$
  по условию $rang ~ kdf(u_0, \upsilon_0) = 2$ тогда по следствию из теоремы о
  неявной функции следует что найдется окрестность в точке $(u_0, \upsilon_0)$
  такая что $\exists O(U(u_0, \upsilon_0))$ такое что $f: U(u_0,
  \upsilon_0) \to R^3$ инективн
\end{proof}

\begin{theorem}
  Пусть $F(x,y,z)$ - гладкая функция, $(x_0, y_0, z_0) ~~ F(x_0, y_0, z_0) = 0
  ~~ \grad F(x_0, y_0, z_0) = \left(
  \frac{\partial F}{\partial x}(x_0, y_0, z_0),
  \frac{\partial F}{\partial y}(x_0, y_0, z_0),
  \frac{\partial F}{\partial z}(x_0, y_0, z_0)
  \right) \not= 0$ тогда $\exists O(U(x_0, y_0, z_0))$ в которой множество
  точек улидотворяет $F(x, y, z) = 0$ является образом регулярной поверхности.

  В этом случае мы будем говорит ьчто эта регулярная поверхность задана
  уравнением $F(x, y, z) = 0$
\end{theorem}