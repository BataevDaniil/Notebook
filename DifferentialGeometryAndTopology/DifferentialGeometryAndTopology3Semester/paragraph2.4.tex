\begin{title}[\Large]
  Нормальная кривизна кривой на поверхности второй квадратичной формы
  поверхности
\end{title}

Пусть $\vec \varphi = \vec \varphi(u, \upsilon)$ регулярная поверхность

$\vec m$ еденичная нормаль к поверхности

$\vec m = \frac{[\vec \varphi_u, \vec \varphi_{\upsilon}]}{[\vec \varphi_u,
\vec \varphi_{\upsilon}]} = \frac{[\vec \varphi_u, \vec \varphi_{\upsilon}]}
{EG - F^2}$

пусть
$$
\left\{
\begin{array}{c}
  u = u(s) \\
  \upsilon = \upsilon(s)
\end{array}
\right. ~ \text{кривая на поверхности с натуральным параметром $s$}
$$
$\vec \varphi = \vec \varphi(u(s), \upsilon(s))$ уравнение кривой в декартовой
системе координат

$K_n = \stackrel{\bullet \bullet}{\vec \varphi} \cos \alpha = K \cos \alpha$

$\stackrel{\bullet}{\vec \varphi} = \vec \varphi_u \stackrel{\bullet}{u} +
\vec \varphi_{\upsilon} \stackrel{\bullet}{\upsilon}$
$$
\stackrel{\bullet \bullet}{\vec \varphi} = \vec \varphi_{uu}
\stackrel{\bullet}{u^2} + 2\vec \varphi_{u\upsilon}\stackrel{\bullet}{u}
\stackrel{\bullet}{\upsilon} + \vec \varphi_{\upsilon \upsilon}
\stackrel{\bullet}{\upsilon^2}
+ \vec \varphi_u \stackrel{\bullet \bullet}{\upsilon} +
\vec \varphi_{\upsilon} \stackrel{\bullet \bullet}{\upsilon}
$$
$$
K_n = (\vec m, \stackrel{\bullet \bullet}{\vec \varphi}) = (\vec \varphi_{uu}
\stackrel{\bullet}{u^2} + 2\vec \varphi_{u\upsilon}\stackrel{\bullet}{u}
\stackrel{\bullet}{\upsilon} + \vec \varphi_{\upsilon \upsilon}
\stackrel{\bullet}{\upsilon^2}
+ \vec \varphi_u \stackrel{\bullet \bullet}{u} +
\vec \varphi_{\upsilon} \stackrel{\bullet \bullet}{\upsilon}, \vec m) =
$$
$$
= \overbrace{(\vec m, \vec \varphi_{uu})}^L \stackrel{\bullet}{u^2} +
2\overbrace{(\vec m, \vec \varphi_{u\upsilon})}^M \stackrel{\bullet}{u}
\stackrel{\bullet}{\upsilon} + \overbrace{(\vec m, \vec \varphi_{\upsilon})}^N
\stackrel{\bullet}{\upsilon^2} =
$$
$$
= \frac{Ldu^2 + 2Mdud\upsilon + Nd\upsilon^2}{ds^2}
= \frac{Ldu^2 + 2Mdud\upsilon + Nd\upsilon^2}{Edu^2 + 2Fdu d\upsilon +
Gd\upsilon^2}
$$
$Ldu^2 + 2Mdud\upsilon + Nd\upsilon^2$ вторая квадратичная форма поверхности

\begin{theorem}
  Нормальная кривизна кривой
  $$
  \left\{
  \begin{array}c
    u = u(t) \\
    \upsilon = \upsilon(t)
  \end{array}
  \right. ~~ \text{на регулрной плоскости} ~~ \vec \varphi = \vec \varphi(u,
  \upsilon)
  $$
  вычисляется по формуле
  $$
  K_n = \frac{Lu'^2 + 2Mu'\upsilon' + N\upsilon'^2}{Eu'^2 + 2Fu'\upsilon' +
  G\upsilon'^2}
  $$
  вчастности из этой формулы следует что нормальная кривизна кривой зависит
  только от напрвления касательных векторов
\end{theorem}

\begin{theorem}
  Нормальная кривизна кривой на поверхности равна по абсолютной велечине
  кривезне соответствующего нормального сечения
\end{theorem}

\begin{proof}
  Достаточно показать что нормальная кривизна нормального сечени по
  абсолютной величине равна его кривизне
\end{proof}

\begin{define}[индекрисы Дюпена]
  Индекриса Дюпена в точке $M$ поверхности называется кривая в касательной
  плоскости которую описывает конец отрезка длины $\sqrt{R_n}$ начало
  которого находится в точке $M$. $R = \frac{1}{K_n}$ радиус нормальной
  кривизны
\end{define}