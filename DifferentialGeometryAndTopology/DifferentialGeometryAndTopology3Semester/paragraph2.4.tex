\begin{title}[\Large]
  Нормальная кривизна кривой. Вторая квадратичная форма поверхности
\end{title}

Пусть $\vec \varphi = \vec \varphi(u, \upsilon)$ регулярная поверхность

$\vec m$ еденичная нормаль к поверхности

$\vec m = \frac{[\vec \varphi_u, \vec \varphi_{\upsilon}]}{[\vec \varphi_u,
\vec \varphi_{\upsilon}]} = \frac{[\vec \varphi_u, \vec \varphi_{\upsilon}]}
{EG - F^2}$

пусть
$$
\left\{
\begin{array}{c}
  u = u(s) \\
  \upsilon = \upsilon(s)
\end{array}
\right. ~ \text{кривая на поверхности с натуральным параметром $s$}
$$
$\vec \varphi = \vec \varphi(u(s), \upsilon(s))$ уравнение кривой в декартовой
системе координат

$K_n = \stackrel{\bullet \bullet}{\vec \varphi} \cos \alpha = K \cos \alpha$

$\stackrel{\bullet}{\vec \varphi} = \vec \varphi_u \stackrel{\bullet}{u} +
\vec \varphi_{\upsilon} \stackrel{\bullet}{\upsilon}$
$$
\stackrel{\bullet \bullet}{\vec \varphi} = \vec \varphi_{uu}
\stackrel{\bullet}{u^2} + 2\vec \varphi_{u\upsilon}\stackrel{\bullet}{u}
\stackrel{\bullet}{\upsilon} + \vec \varphi_{\upsilon \upsilon}
\stackrel{\bullet}{\upsilon^2}
+ \vec \varphi_u \stackrel{\bullet \bullet}{\upsilon} +
\vec \varphi_{\upsilon} \stackrel{\bullet \bullet}{\upsilon}
$$
$$
K_n = (\vec m, \stackrel{\bullet \bullet}{\vec \varphi}) = (\vec \varphi_{uu}
\stackrel{\bullet}{u^2} + 2\vec \varphi_{u\upsilon}\stackrel{\bullet}{u}
\stackrel{\bullet}{\upsilon} + \vec \varphi_{\upsilon \upsilon}
\stackrel{\bullet}{\upsilon^2}
+ \vec \varphi_u \stackrel{\bullet \bullet}{u} +
\vec \varphi_{\upsilon} \stackrel{\bullet \bullet}{\upsilon}, \vec m) =
$$
$$
= \overbrace{(\vec m, \vec \varphi_{uu})}^L \stackrel{\bullet}{u^2} +
2\overbrace{(\vec m, \vec \varphi_{u\upsilon})}^M \stackrel{\bullet}{u}
\stackrel{\bullet}{\upsilon} + \overbrace{(\vec m, \vec \varphi_{\upsilon})}^N
\stackrel{\bullet}{\upsilon^2} =
$$
$$
= \frac{Ldu^2 + 2Mdud\upsilon + Nd\upsilon^2}{ds^2}
= \frac{Ldu^2 + 2Mdud\upsilon + Nd\upsilon^2}{Edu^2 + 2Fdu d\upsilon +
Gd\upsilon^2}
$$
$Ldu^2 + 2Mdud\upsilon + Nd\upsilon^2$ вторая квадратичная форма поверхности

\begin{theorem}
  Нормальная кривизна кривой
  $$
  \left\{
  \begin{array}{c}
    u = u(t) \\
    \upsilon = \upsilon(t)
  \end{array}
  \right. ~~ \text{на регулрной плоскости} ~~ \vec \varphi = \vec \varphi(u,
  \upsilon)
  $$
  вычисляется по формуле
  $$
  K_n = \frac{Lu'^2 + 2Mu'\upsilon' + N\upsilon'^2}{Eu'^2 + 2Fu'\upsilon' +
  G\upsilon'^2}
  $$
  вчастности из этой формулы следует что нормальная кривизна кривой зависит
  только от напрвления касательных векторов
\end{theorem}

\begin{theorem}
  Нормальная кривизна кривой на поверхности равна по абсолютной велечине
  кривезне соответствующего нормального сечения
\end{theorem}

\begin{proof}
  Достаточно показать что нормальная кривизна нормального сечени по
  абсолютной величине равна его кривизне
\end{proof}

\begin{define}[индекрисы Дюпена]
  Индекриса Дюпена в точке $M$ поверхности называется кривая в касательной
  плоскости которую описывает конец отрезка длины $\sqrt{R_n}$ начало
  которого находится в точке $M$. $R = \frac{1}{K_n}$ радиус нормальной
  кривизны
\end{define}

Выведем уравнение индикатрисы Дюпена. В касательной плоскости выберем аффинную
систему координа $M \vec \varphi_u \vec \varphi_{\upsilon}$ $P$ лежит на
индактрисе Дюпена когда $|\overrightarrow MP| = \sqrt{R_n}$

$|\overrightarrow MP| = \sqrt{Ex^2 + 2Fxy + Gy^2}$

$\sqrt{R_n} = \frac{1}{\sqrt{|K_n|}} = \sqrt{\frac{Ex^2 + 2Fxy + Gy^2}{
|Lx^2 + 2Mxy + Ny^2|}}$

$\sqrt{I} = \sqrt{\frac{I}{II}} ~ \Rightarrow ~ |Lx^2 + 2Mxy + Ny^2| = 1 ~~~
Lx^2 + 2Mxy + Ny^2 = \pm 1$

$$
1)
\left|
\begin{array}{cc}
  L & M \\
  M & N
\end{array}
\right| = LN - M^2 > 0 ~ \text{эллипс}
$$

2) $LN - M^2 < 0$ мнимый эллипс

3) $LN - M^2 = 0$ пара паралельных прямых

\begin{define}[главного напрвления и главной кривизны]
  Напрвление в касательной плоскости соответствующее направлениям осей симетрии
  индектрисы Дюпена называется главным направлением соответствующие им кривизны
  нормальных сечений называется главными кривизнами.
\end{define}

\begin{theorem}
  В окрестности каждой точке регулярной кривой поверхности можно выбрать
  параметризацию $(u, \upsilon)$ так что в $M$

  $I(M) = du^2 + d\upsilon^2$

  $II(M) = Ldu^2 + Nd\upsilon^2$

  пояснениe

  $E = |\varphi_u|^2 = 1$

  $F = (\varphi_u, \varphi_{\upsilon}) = 0$

  $F = |\vec \varphi_{\upsilon}|^2 = 1$ в точке $M$

  $(\varphi_u, \varphi_{\upsilon})$ - ортонормированный базис в касательной
  плоскости в точке $M$

  $Lx^2 + Ny^2 = \pm 1$ - уравнение индектрисы Дюпена

  оси координат являются осями симметрии

  $\vec \varphi_u, \vec \varphi_{\upsilon}$ - главные направления

  $K_n = \frac{II}{I} = \frac{Ldu^2 + Ndu^2}{du^2 + d\upsilon^2}$

  $\varphi_u = (1,0) ~~~ (du, d\upsilon) = (1,0)$

  $K_1 = \frac{L1^2 + N0^2}{1^2 + 0^2} = L$

  $\vec \varphi_{\upsilon} = (0,1) ~~~ K_2 = N ~ \Rightarrow ~ II(M) =
  K_1 du^2 + K_2d\upsilon^2$
\end{theorem}

\begin{theorem}[Эйлера]
  Нормальная кривизна кривой выражается следующей формулой
  $K_n = K_1 \cos^2 \theta +K_2 \sin^2 \theta$ $\theta$ - угол между первыми
  главными направлением и направлением  нормального сечения
\end{theorem}

\begin{proof}
  $(du, d\upsilon) = (\cos \theta, \sin \theta)$

  $K_n = \frac{K_1du^2 + K_2d\upsilon^2}{du^2 + d\upsilon^2} = K_1\cos^2 \theta
  + K_2 \sin^2 \theta$
\end{proof}

\begin{block}[Следствие]
  Если $K_1 \le K_2$ тогда $K_1 \le K_n \le K_2$
\end{block}

\begin{proof}
  $K_n = K_1 \cos^2 \theta K_2 \sin^2 \theta = K_1 \cos^2 \theta +
  K_1 \sin^2 \theta - K_1 \sin^2 \theta + K_2 \sin^2 \theta =$

  $= K_1 + (K_2 - K_1)\sin^2 \theta ~ \Rightarrow ~ K_1 \le K_n \le K_2$
  так как $0 \le \sin^2 \theta \le 1$
\end{proof}

\begin{theorem}
  Координаты $(du, d\upsilon)$ главных напрвлений удовлитворяющий следующему
  уравнению
  $$
  \left|
  \begin{array}{ccc}
    d\upsilon^2 & -dud\upsilon & d\upsilon^2 \\
    E & F & G \\
    L & M & N
  \end{array}
  \right| = 0
  $$
\end{theorem}

\begin{proof}
  $I = Edu^2 + 2Fdud\upsilon + Gd\upsilon^2$

  $II = Ldu^2 + 2Mdud\upsilon + Ndu^2$

  пусть главные напрвления задаются касательным вектором $a = (\xi_1,\xi_2)$
  $b = (\eta_1, \eta_2)$

  1) главные напрвления ортогональны $(a,b) = 0$

  $E\xi_1 \eta_1 + F(\xi_1 \eta_2 + \xi_2 \eta_1) + G\xi_2 \eta_2 = 0$

  $\xi_1(E\eta_1 + F\eta_2) + \xi_2(F\eta_1 + G\eta_2) = 0$

  2) Функция напрвленна сопряженно относительно индикатрисы Дюпена

  $$
  \left\{
  \begin{array}{c}
    \xi_1 = -(M\eta_1 + N \eta_2) \\
    \xi_2 = L\eta_1 + M \eta_2
  \end{array}
  \right.
  $$
  $-(M\eta_1 + N\eta_2)(E\eta_1 + E\eta_2) + (L\eta_1 + M\eta_2)(F\eta_1 +
  G\eta_2) = 0$
  $$
  \left|
  \begin{array}{cc}
    E\eta_1 + F\eta_2 & F\eta_1 + G\eta_2 \\
    L\eta_1 + M\eta_2 & M\eta_1 + N\eta_2
  \end{array}
  \right| = 0
  $$
  $$
  \left|
  \begin{array}{cc}
    Edu + Fd\upsilon & Fdu + Gd\upsilon \\
    Ldu + Md\upsilon & Mdu + Md\upsilon
  \end{array}
  \right| = 0
  $$
  $$
  \left|
  \begin{array}{ccc}
    d\upsilon^2 & -dud\upsilon & d\upsilon^2 \\
    E & F & G \\
    L & M & N
  \end{array}
  \right| = 0
  $$
\end{proof}

\begin{theorem}
  Главные кривизны поверхности удовлитворяют характеристическому уравнению
  поверхности
  $$
  \left|
  \begin{array}{cc}
    L - KE & M - KF \\
    M - KF & N - KG
  \end{array}
  \right| = 0 ~~~ |II - kI| = 0
  $$
  $(EG - F^2)K^2 - (LG - 2MF + NE)K + (LN - M^2) = 0$
  $$
  K_1 + K_2 = \frac{LG - 2MF + NE}{EG - F^2}
  $$
  $$
  K_1 K_2 = \frac{LN - M^2}{EG - F^2}
  $$
  $H = \frac{K_1 + K_2}{2}$ - среднаяя кривизна поверхности
  $K = K_1 K_2$ полная Гаусова кривизна

  $LN - M^2 > 0$ $K > 0$ эллипс $K < 0$ гиперболическая $K = 0$ параболическая
\end{theorem}
