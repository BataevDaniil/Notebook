\begin{title}
  Параграф 2
\end{title}

\begin{title}[\Large]
  Натуральная параметризация кривой, кривизна и кручение кривой
\end{title}

Пусть $\vec \varphi = \vec \varphi(t)$ - регулярная кривая, тогда длина дуги
кривой
$$
S = \int_{t_0}^{t_1} | \vec \varphi'(t)| dt
$$
если форма записи
$$
\left\{
\begin{array}{c}
  x = x(t) \\
  y = y(t) \\
  z = z(t)
\end{array}
\right.
S = \int_{t_0}^{t_1} \sqrt{ x'^2(t) + y'^2(t) + z'^2(t)}
$$

\begin{define}[натуральной параметризации кривой]
  Параметризация при которой каждой точке кривой сопаставляется длина дуги
  кривой от точки до некоторой фиксированной точки называется натуральной или
  естественной.
\end{define}

\begin{block}[Свойства]
  1) Натуральная параметризация эквивалентна исходной параметризации.

  \begin{proof}
    $\vec \varphi = \vec \varphi(t)$ - уравнение кривой

    $s = s(t)$ - натуральная параметризация
    $$
    \int_{t_0}^{t} |\varphi'(t)| dt ~~~
    \frac{ds}{dt} = |\varphi'(t)| > 0 ~ \Rightarrow ~ s = s(t) ~ \nearrow ~
    $$
    $\Rightarrow ~ s = s(t)$ является биективной
  \end{proof}

  2) $\left| \frac{d \vec \varphi}{ds} \right| = 1$ для любой точки касательный
  вектор имеет длину $1$

  \begin{proof}
    $\varphi(s) = \varphi (t(s))$

    $\frac{d\vec \varphi}{ds} = \frac{d\varphi}{dt} ~~~ \frac{dt}{ds} =
    \vec \varphi' \cdot \frac{1}{|\varphi'|}$ вектор длины 1
  \end{proof}

  3)
  $$
  \frac{d^2 \vec \varphi}{ds^2} \perp \frac{d\vec \varphi}{ds}
  $$
\end{block}

\begin{block}[Лемма]
  Если $|\varphi(t)| = const$, тогда $\vec \varphi'(t) \perp
  \vec \varphi(t)$
\end{block}

\begin{proof}
  $(\vec \varphi, \vec \varphi) = |\vec \varphi|^2 = const$
  $$
  \frac{d}{dt}(\varphi, \varphi) = (\varphi', \varphi) +
  (\varphi, \varphi') = 2(\varphi', \varphi) = 0
  $$
  $$
  \Rightarrow (\varphi', \varphi) = 0 ~~~ \Rightarrow \varphi' \perp \varphi
  $$
\end{proof}

\begin{define}[кривизны кривой]
  $$
  k =
  \left|
    \frac{d^2 \varphi}{d \xi^2}
  \right| ~~~
  \text{называется кривизной кривой}
  $$
  $$
    \frac{d^2 \varphi}{d \xi^2} ~~~ \text{вектор кривизны}
  $$
\end{define}

\begin{theorem}
  Кривизна кривой $= 0$ тогда когда ее образ является прямой
\end{theorem}

\begin{proof}
  Доказательство в одну сторону пусть $k = 0$ тогда
  $$
  \left| \frac{d^2 \varphi}{ds^2} \right| = 0 ~ \Rightarrow ~
  \frac{d^2 \varphi}{ds^2} = \vec 0 ~ \Rightarrow ~
  \frac{d\varphi}{d s} = \vec a (=const) ~ \Rightarrow ~
  \varphi = \vec a S + \vec b
  $$
  прямая это $\varphi = \vec \varphi_0 + \vec g t$

  Доказательство в другую сторону что $k = 0$
  $$
  \varphi = \vec \varphi_0 + \vec g t ~ \Rightarrow ~
  \frac{d^2 \varphi}{ds^2} = \vec 0 ~ \Rightarrow ~ k = 0
  $$
\end{proof}

$\vec \varphi = \vec \varphi (s)$ - регулярная кривая, $s$ - натуральный
параметр
%img
$\vec t = \frac{d \vec \varphi}{ds}$

$\vec n = \frac{1}{k} ~~~ \frac{d^2 \varphi}{ds^2}$ - вектор главной нормали

$\vec b = [\vec t, \vec n]$ - вектор бинормали

$\vec t, \vec n, \vec b$ - репер Френе (в каждой точке получается
ортонормированный базис)
%img

\begin{block}[Лемма 2]
  $$
  \frac{d\vec b}{ds}a~ || ~ \vec n ~ \text{- колинеарны}
  $$
\end{block}

\begin{proof}
  $\vec b = \vec b(s)$ докажем что $\frac{db}{ds} \perp \vec t, \vec b$

  $|b| = 1 = const ~ \Rightarrow ~ \frac{db}{ds} \perp \vec b$ (по лемме 1)

  $
  \vec b = [\vec t, \vec n] ~ \Rightarrow ~ \frac{d\vec b}{ds} =
  [\frac{dt}{ds}, \vec b] + [\vec t, \frac{d\vec n}{ds}] =
  [\vec t, \frac{d\vec n}{ds}] \perp \vec t
  $
\end{proof}

\begin{block}[Следствие]
  Из леммы следует $\frac{\vec b}{ds} = - \kappa$ $\kappa$ - коэффицент
  пропорциональности каппа называется кручениче кривой в точке.
\end{block}

\begin{theorem}
  Кругление кривой равно 0 $\Rightarrow$ когда кривая явялется плоскостью
  (ее образ лежит в некоторой плоскости)
\end{theorem}

\begin{proof}
  Пусть $\kappa = 0$, тогда $\frac{d\vec b}{ds} = \vec 0 \Rightarrow \vec b =
  const = (\vec \alpha, \vec \beta, \vec \gamma)$ расмотрим
  $\frac{d}{ds}(\vec \varphi, \vec b) = (\frac{d \vec \varphi}{ds}, b) +
  (\vec \varphi, \frac{d\vec b}{ds}) = 0 ~ \Rightarrow ~
  (\vec \varphi, \vec b) = const$

  $\alpha x(s) + \beta y(s) + \gamma z(s) = \delta ~ \Rightarrow$ кривая
  $\vec \varphi = \vec \varphi (s)$ лужит в плоскости $\alpha x + \beta y +
  \gamma z = \delta$

  Доказательство в обратную сторону. Пусть кривая лежит в некоторой плоскости
  %img
  $\vec b = const ~ \Rightarrow ~ \frac{db}{ds} = 0 ~ \Rightarrow ~ \kappa = 0$
\end{proof}

\begin{theorem}[(гометрический смысл кривизны кручения)]
  1) Для плоской кривой $k = |\frac{d \vec t}{ds}|$ равна скорости вращения
  касательного вектора.

  2) $|\kappa| = |\frac{d\vec b}{ds}|$ равна скорости вращения соприкасающейся
  плоскости относительно касательной прямой.
\end{theorem}

\begin{theorem}
  Для всяких гладких функций $k = k(s) > 0$ и $\kappa = \kappa(s)$ существует
  единственный с точностью до пложения в пространстве кривая
  $\varphi = \varphi(s)$ кривизна которой равна кручению
\end{theorem}