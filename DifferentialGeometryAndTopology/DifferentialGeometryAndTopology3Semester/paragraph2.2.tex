\begin{title}
  Касательные векторы поверхности. Касательная плоскости поверхности
\end{title}

\begin{define}
  Каcательный вектор поверхности в точке $m_0$ называется вектор в точке $m_0$
  являющийся касательнам вектором некоторой кривой поверxности проходящей
  через точку $m_0$

  Кривая на поверхности может быть задана в локальной системой координат
  $(u, \upsilon)$
  $$
  \left\{
    \begin{array}{l}
      u = u(t) \\
      \upsilon = \upsilon(t)
    \end{array}
  \right.
  $$

  $\varphi = \varphi(u(t), \upsilon(t))$ уравнение кривой в декартовой
  системе координат

  $\vec \varphi' = \vec \varphi_u u' + \vec \varphi_{\upsilon}
  \upsilon'$ является линейной комьинацией векторов $\vec \varphi_u
  \vec \varphi_{\upsilon}$
\end{define}

\begin{theorem}
  Множество касательных векторов поверхности точки $m_0$ является двумерным
  линейным пространством, в качестве базиса которого можно взять
  $\vec \varphi_u \vec \varphi_{\upsilon}$
\end{theorem}

\begin{proof}
  1) $\vec \varphi_u \vec \varphi_{\upsilon}$ - л.н.з

  2) Всякий касательный вектор поверхости точки $m_0$ линейно выражается через
  $\vec \varphi_u \vec \varphi_{\upsilon}$

  $\alpha \vec \varphi_u +\beta \vec \varphi_{\upsilon} ~~~
  \alpha, \beta \in R$ - нужно доказать

  $$
  \text{расмотрим кривую} ~~~
  \left\{
    \begin{array}{l}
      u = \alpha t + u_0 \\
      \upsilon = \beta t + \upsilon_0
    \end{array}
  \right. ~~~ m_0 (u_0, \upsilon_0)
  $$
  $$
  \text{при} ~ t = 0 ~
  \left\{
    \begin{array}{l}
      u = u_0 \\
      \upsilon = \upsilon_0
    \end{array}
  \right. ~ \Rightarrow ~ \text{кривая проходит через точку} ~
  m_0 (u_0, \upsilon_0)
  $$
  $\vec \varphi' = \vec \varphi_u u' + \vec \varphi_{\upsilon}
  \upsilon' = \alpha \vec \varphi_u + \beta \vec \varphi_{\upsilon}$

  $\vec \varphi_u, \vec \varphi_{\upsilon}$ касательный векторы, которые
  образуют базис для всех касательных векторов.

  Расмотрим координатные линии $\upsilon = \const ~~~ u = \const$

  $$
  u = \const
  \left\{
    \begin{array}{l}
      u = \const \\
      \upsilon = t
    \end{array}
  \right.
  $$
  $\vec \varphi' = \vec \varphi_u u' + \vec \varphi_{\upsilon}
  \upsilon' = \vec \varphi_u 0 + \vec \varphi_{\upsilon} 1 =
  \vec \varphi_{\upsilon}$
\end{proof}

\begin{define}[координат касательного вектора]
  Каждый касательный вектор $\vec a$ имеет вид $\vec a =
  \alpha \vec \varphi_u + \beta \vec \varphi_{\upsilon}$ тогда
  $(\alpha, \beta)$ - называются
  координатами касательного вектора $\vec a$ относительно локальной системы
  координат $(u, \upsilon)$
\end{define}

\begin{define}[касательной плоскости поверхности в точке]
  Касательная плоскость поверхности в точке $m_0$ называется плоскость
  проходящая через точку $m_0$ и вектор $\vec \varphi_0 (m_0)
  \vec \varphi_{\upsilon}(m_0)$
\end{define}
