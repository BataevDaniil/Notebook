\begin{title}[\Large]
  Параграф 2.2
\end{title}

\begin{title}[\Large]
  Касательный вектор поверхности и касательная плоскость в точке поверхности.
  Координаты касательного вектора относительно локальной системы координат
\end{title}

\begin{define}[касательного вектора поверхности]
  Вектор в точке $m_0$
  являющийся касательнам вектором некоторой кривой поверxности проходящей
  через точку $m_0$ называют каcательным вектором поверхности в точке $m_0$.

  Кривая на поверхности может быть задана в локальной системой координат
  $(u, \upsilon)$
  $$
  \left\{
    \begin{array}{l}
      u = u(t) \\
      \upsilon = \upsilon(t)
    \end{array}
  \right.
  $$

  $\vec \varphi = \vec \varphi(u(t), \upsilon(t))$ уравнение кривой в декартовой
  системе координат

  $\vec \varphi' = \vec \varphi_u u' + \vec \varphi_{\upsilon}
  \upsilon'$ является линейной комбинацией векторов $\vec \varphi_u
  \vec \varphi_{\upsilon}$
\end{define}

\begin{theorem}
  Множество касательных векторов поверхности точки $m_0$ является двумерным
  линейным пространством, в качестве базиса которого можно взять
  $\vec \varphi_u \vec \varphi_{\upsilon}$
\end{theorem}

\begin{proof}
  1) $\vec \varphi_u \vec \varphi_{\upsilon}$ - л.н.з

  2) Всякий касательный вектор поверхости точки $m_0$ линейно выражается через
  $\vec \varphi_u \vec \varphi_{\upsilon} ~~~ \alpha \vec \varphi_u +\beta
  \vec \varphi_{\upsilon} ~~~ \alpha, \beta \in R$ нужно доказать
  $$
  \text{расмотрим кривую} ~~~
  \left\{
    \begin{array}{l}
      u = \alpha t + u_0 \\
      \upsilon = \beta t + \upsilon_0
    \end{array}
  \right. ~~~ m_0 (u_0, \upsilon_0)
  $$
  $$
  \text{при} ~ t = 0 ~
  \left\{
    \begin{array}{l}
      u = u_0 \\
      \upsilon = \upsilon_0
    \end{array}
  \right. ~ \Rightarrow ~ \text{кривая проходит через точку} ~
  m_0 (u_0, \upsilon_0)
  $$
  $\vec \varphi' = \vec \varphi_u u' + \vec \varphi_{\upsilon}
  \upsilon' = \alpha \vec \varphi_u + \beta \vec \varphi_{\upsilon}$

  $\vec \varphi_u, \vec \varphi_{\upsilon}$ касательные векторы, которые
  образуют базис для всех касательных векторов.

  Расмотрим координатные линии $\upsilon = \const ~~~ u = \const$
  $$
  u = \const ~~~
  \left\{
    \begin{array}{l}
      u = \const \\
      \upsilon = t
    \end{array}
  \right.
  $$
  $\vec \varphi' = \vec \varphi_u u' + \vec \varphi_{\upsilon}
  \upsilon' = \vec \varphi_u 0 + \vec \varphi_{\upsilon} 1 =
  \vec \varphi_{\upsilon}$
  $$
  \upsilon = \const ~~~
  \left\{
    \begin{array}{l}
      u = t \\
      \upsilon = \const
    \end{array}
  \right.
  $$
  $\vec \varphi' = \vec \varphi_u u' + \vec \varphi_{\upsilon}
  \upsilon' = \vec \varphi_u 1 + \vec \varphi_{\upsilon} 0 = \vec \varphi_u$
\end{proof}

\begin{define}[локальной системы касательных векторов в точке поверхности]
  Каждый касательный вектор $\vec a$ в точке $m_0$ имеет вид $\vec a =
  \alpha \vec \varphi_u(m_0) + \beta \vec \varphi_{\upsilon}(m_0)$ тогда
  $(\alpha, \beta)$ называются
  координатами касательного вектора $\vec a$ относительно локальной системы
  координат.
\end{define}

\begin{define}[касательной плоскости поверхности в точке]
  Плоскость проходящая через точку поверхности $m_0$ и векторы
  $\vec \varphi_u (m_0), \vec \varphi_{\upsilon}(m_0)$ называют касательной
  плоскостью поверхности.
\end{define}

\begin{theorem}
  Поверхности $F(x,y,z) = 0$
  $$
  \grad F = \left( \frac{\partial F}{\partial x},
  \frac{\partial F}{\partial y},
  \frac{\partial F}{\partial z}
   \right)
  $$
  является нормалью к касательной плоскости
\end{theorem}

\begin{proof}
  $$
  F(x(t), y(t), z(t)) = 0
  $$
  $$
  \frac{dF}{dt} =
  \frac{\partial F}{\partial x} \frac{dx}{dt} +
  \frac{\partial F}{\partial y}\frac{dy}{dt} +
  \frac{\partial F}{\partial z}\frac{dz}{dt} =
  (\grad F, \vec \varphi') = 0 ~ \Rightarrow ~ \grad F \perp \vec \varphi'
  $$
\end{proof}
