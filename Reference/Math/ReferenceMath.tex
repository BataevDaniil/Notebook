\documentclass{article}

\usepackage[utf8]{inputenc} 
\usepackage{amsmath,amssymb}
\usepackage[russian]{babel}
\usepackage{rotating}
\usepackage{layout}
\renewcommand{\arraystretch}{1.7}

\newcommand{\arcsh}{\mathrm{arcsh}}
\newcommand{\arcch}{\mathrm{arcch}}
\newcommand{\arcth}{\mathrm{arcth}}
\newcommand{\arccth}{\mathrm{arccth}}

\parindent=0pt

\begin{document}

%таблица производных
\begin{displaymath}
\begin{array}{|c|c|c|}

  \hline
  1 &		(x^{a})'		    &	ax^{a-1}\\
  \hline
  2 &		(a^{x})'		    &	a^{x}\ln a\\
  \hline
  3 &		(\log_{a}x)'	  &	\frac{1}{x\ln a}\\
  \hline
  4 &		(\sin x)'			  &	\cos x\\
  \hline
  5 &		(\cos x)'		    &	-\sin x\\
  \hline
  6 &		(\tg x)'			  &	\frac{1}{\cos^{2} x}\\
  \hline
  7 & 	(\ctg x)'		    &	-\frac{1}{\sin^{2} x}\\
  \hline
  8 &		(\arcsin x)'		&	\frac{1}{\sqrt{1-x^{2}}}\\
  \hline
  9 &		(\arccos x)'		&	-\frac{1}{\sqrt{1-x^{2}}}\\
  \hline
  10&	(\arctg x)'		    &	\frac{1}{1-x^{2}}\\
  \hline
  11&	(\arcctg x)'		  &	-\frac{1}{1-x^{2}}\\
  \hline
  12&	(\sh x)'			    &	\ch x\\
  \hline
  13&	(\ch x)'			    &	\sh x\\
  \hline
  14&	(\th x)' 			    &	\frac{1}{\ch^{2} x}\\
  \hline
  15&	(\cth x)'		      &	-\frac{1}{\sh^{2} x}\\
  \hline
  16&	(\arcsh x)'		    &	\frac{1}{\sqrt{x^{2}+1}}\\
  \hline
  17&	(\arcch x)'		    &	\frac{1}{\sqrt{x^{2}-1}}\\
  \hline
  18&	(\arcth x)'		    &	\frac{1}{1-x^{2}}\\
  \hline
  19&	(\arccth x)'		  &	\frac{1}{1-x^{2}}\\
  \hline

\end{array}
\end{displaymath}

%производная сложной функции 
\[   \Big(f(g(x))\Big) ' = f'(g(x)) \cdot g'(x)  \]

%правила дефференцирования
\begin{displaymath}
\begin{array}{cl}

  1.&  (C\cdot u)'                         =   C\cdot u'\\
  2.&  \left(u + \upsilon       \right)'   =   u' + \upsilon'\\
  3.&  \left(u\cdot \upsilon    \right)'   =   u'\cdot\upsilon + u\cdot\upsilon' \\
  4.&  \left(\frac{u}{\upsilon} \right)'   =   \frac{u'\cdot\upsilon + u\cdot\upsilon'}{\upsilon^2}\\

\end{array}
\end{displaymath}

%производная параметрически заданой функции
\begin{displaymath}
\left\{ \begin{array}{l}
          x = g(t)\\
          y = f(t)\\
\end{array} \right.
\end{displaymath}
\[ y_x = \frac{f'(t)}{g'(t)}\]

%первый замечательный предел и его следствия
\begin{displaymath}
\begin{array}{|c|l|}

  \hline
  1 &\lim_{x \to 0} \frac{\sin f(x)}{f(x)} = 1\\
  \hline
  2 &\lim_{x \to 0} \frac{\tg f(x)}{f(x)} = 1\\
  \hline
  3 &\lim_{x \to 0} \frac{\arcsin f(x)}{f(x)} = 1\\
  \hline
  4 &\lim_{x \to 0} \frac{\arctg f(x)}{f(x)} = 1\\
  \hline
  5 &\lim_{x \to 0} \frac{1 - \cos x}{\frac{x^2}{2}} = 1\\
  \hline

\end{array}
\end{displaymath}

%второй замечательный предел и его следствия
\begin{displaymath}
\begin{array}{|c|l|}

  \hline
  1 &\lim_{x \to 0} (1+x)^{\frac{1}{x}} = e\\
  \hline
  2 &\lim_{x \to \infty} ( 1 + \frac{k}{x} )^x  = e^k \\
  \hline
  3 &\lim_{x \to 0} \frac{\ln (1 + x)}{x} = 1 \\
  \hline
  4 &\lim_{x \to 0} \frac{e^x - 1}{x} = 1\\
  \hline
  5 &\lim_{x \to 0} \frac{a^x - 1}{x\ln a} = 1 ~\text{для}~ a>0, a \not = 1\\
  \hline
  6 &\lim_{x \to 0} \frac{(1 + x)^a - 1}{ax} = 1\\
  \hline

\end{array}
\end{displaymath}

%таблица интегрирования
\begin{displaymath}
\begin{array}{|c|c|c|}
  \hline
  1 &\int dx & x + C\\
  \hline
  2 &\int x^{n}dx &\frac{x^{n+1}}{n+1} + C, ~ x \not -1\\
  \hline
  3 &\int x^{-1}dx &\ln |x| + C\\
  \hline
  4 &\int a^{x}dx &\frac{a^x}{\ln a} + C\\
  \hline 
  5 &\int e^{x}dx &e^x + C\\
  \hline
  6 &\int \cos x dx &\sin x + C\\
  \hline
  7 &\int \sin x dx &-\cos x + C\\
  \hline
  8 &\int \frac{dx}{\cos^2 x} &\tg x + C\\
  \hline
  9 &\int \frac{dx}{\sqrt{\sin^2 x}} &-\ctg x + C\\
  \hline
  10&\int \frac{dx}{\sqrt{1-x^2}} &\arcsin x + C\\
  \hline 
  11&\int \frac{dx}{1+x^2} &\arctg x + C\\
  \hline
  12&\int \frac{dx}{a^2+x^2} &\frac{1}{a}\arctg \frac{x}{a} + C\\
  \hline
  13&\int \frac{dx}{\sqrt{a^2-x^2}}& \arcsin \frac{x}{a} + C\\
  \hline 
  14&\int \frac{dx}{x^2-a^2} &\frac{1}{2a}\ln |\frac{x-1}{x+a}| + C\\
  \hline 
  15&\int \frac{dx}{\sqrt{x^2 \pm a}} &\ln |x + \sqrt{x^2 \pm a}| + C\\
  \hline
  16&\int \frac{dx}{\sin x} &\ln |\frac{1-\cos x}{\sin x}| + C\\
  \hline
  17&\int \frac{dx}{\cos x} &\ln |\frac{1+\sin x}{\cos x}| + C\\
  \hline
  18&\int \sh dx &\ch x\\
  \hline
  19&\int \ch dx &\sh x\\
  \hline
  20&\int \frac{dx}{\ch^2 x} & {\th} x\\
  \hline
  21&\int \frac{dx}{\sh^2 x} & -\cth x\\
  \hline
\end{array}
\end{displaymath}

%формулы тригонометрии
\begin{displaymath}
\begin{array}{|c|c|}

  \hline
  1 &\cos^2 x + \sin^2 x = 1\\
  \hline
  2 &\tg = \frac{\sin x}{\cos x}\\
  \hline
  3 &\ctg = \frac{\cos x}{\sin x}\\
  \hline
  4 &\tg x \cdot \ctg x = 1\\
  \hline
  5 &\tg x = \frac{1}{\ctg x}\\
  \hline
  6 &\ctg x = \frac{1}{\tg x}\\
  \hline
  7 &1 + \tg^2 x = \frac{1}{\cos^2 x}\\
  \hline
  8 &1 + \ctg^2 x = \frac{1}{\tg x}\\
  \hline
  9 &\sin (x\pm y) = \sin x \cdot \cos y \pm \cos x \cdot \sin y\\
  \hline
  10&\cos (x\pm y) = \cos x \cdot \cos y \mp \sin x \cdot \sin y\\
  \hline
  11&\tg (x\pm y) = \frac{\tg x \pm \tg y}{1 \mp \tg x \cdot \tg y}\\
  \hline
  12&\sin 2x = 2\sin x \cdot \cos x\\
  \hline
  13&\cos 2x = \cos^2 x - \sin^2 x\\
  \hline
  14&\cos 2x = 2\cos^2 x - 1\\
  \hline
  15&1 - 2\sin^2 x\\
  \hline
  16&\tg 2x = \frac{2\tg x}{1 - \tg^2 x}\\
  \hline 
  17&\ctg 2x = \frac{\ctg^2 x - 1}{2\ctg x}\\
  \hline
  18&\sin x \pm \sin y = 2\sin \frac{x\pm y}{2} \cdot \cos \frac{x\mp y}{2}\\
  \hline
  19&\cos x + \cos y = 2\cos\frac{x + y}{2} \cdot \cos\frac{x - y}{2}\\
  \hline
  20&\cos x - \cos y = -2\sin\frac{x + y}{2} \cdot \sin\frac{x - y}{2}\\
  \hline

\end{array} 
\end{displaymath}

\begin{displaymath}
\begin{array}{|c|c|}

  \hline
  21&\tg x \pm \tg y = \frac{\sin (x \pm y)}{\cos x \cdot \cos y}\\
  \hline
  22&\sin^2 x = \frac{1 - \cos 2x}{2}\\
  \hline
  23&\cos^2 x = \frac{1 + \cos 2x}{2}\\
  \hline
  24&\sin x \cdot \cos y = \frac{\sin (x+y) + \sin (x-y)}{2}\\
  \hline
  25&\cos x \cdot \cos y = \frac{\cos (x+y) + \cos (x-y)}{2}\\
  \hline
  26&\sin x \cdot \sin y = \frac{\cos (x-y) - \cos (x+y)}{2}\\
  \hline
  27&|\sin \frac{x}{2}| = \sqrt{\frac{1 - \cos x}{2}}\\
  \hline
  28&|\cos \frac{x}{2}| = \sqrt{\frac{1 + \cos x}{2}}\\
  \hline
  29&\tg \frac{x}{2} = \frac{\sin x}{1 + \cos x}\\
  \hline
  30&\ctg \frac{x}{2} = \frac{\sin x}{1 - \cos x}\\
  \hline
\end{array} 
\end{displaymath}

%свойства логарифмов
\begin{displaymath}
\begin{array}{|c|c|}
  \hline
  1&\log_{a}b = c, ~ a^c = b\\
  \hline
  2&\log_{a^k}b^m = \frac{m}{k}\log_{a}b\\
  \hline
  3&\log_{c}(ab) = \log_{c}a + \log_{c}b\\
  \hline
  4&\log_{c}(\frac{a}{b}) = \log_{c}a - \log_{c}b\\ 
  \hline
  5&\log_{a}b = \frac{\log_{c}b}{\log_{c}a}\\
  \hline
  6&\log_{a}b = \frac{1}{\log_{b}a}\\
  \hline  
  7&a^{\log_{c}b} = b^{\log_{c}a}\\
  \hline
  8&\log_{a}b \cdot \log_{c}d = \log_{c}b \cdot \log_{a}d\\
  \hline    
\end{array} 
\end{displaymath}

\end{document}
