\documentclass{article}

\usepackage{../style/LectureCommand}

\begin{document}

%справочник по математике в таблицах
\begin{turn}{90}
$
\begin{array}{llll}
  %Первая таблица производных
  \begin{array}{|c|c|c|}
  \hline
  1 &   (x^{a})'        & ax^{a-1}\\
  \hline
  2 &   (a^{x})'        & a^{x}\ln a\\
  \hline
  3 &   (\log_{a}x)'    & \frac{1}{x\ln a}\\
  \hline
  4 &   (\sin x)'       & \cos x\\
  \hline
  5 &   (\cos x)'       & -\sin x\\
  \hline
  6 &   (\tg x)'        & \frac{1}{\cos^{2} x}\\
  \hline
  7 &   (\ctg x)'       & -\frac{1}{\sin^{2} x}\\
  \hline
  8 &   (\arcsin x)'    & \frac{1}{\sqrt{1-x^{2}}}\\
  \hline
  9 &   (\arccos x)'    & -\frac{1}{\sqrt{1-x^{2}}}\\
  \hline
  10& (\arctg x)'       & \frac{1}{1+x^{2}}\\
  \hline
  11& (\arcctg x)'      & -\frac{1}{1+x^{2}}\\
  \hline
  12& (\sh x)'          & \ch x\\
  \hline
  13& (\ch x)'          & \sh x\\
  \hline
  14& (\th x)'          & \frac{1}{\ch^{2} x}\\
  \hline
  15& (\cth x)'         & -\frac{1}{\sh^{2} x}\\
  \hline
  16& (\arcsh x)'       & \frac{1}{\sqrt{x^{2}+1}}\\
  \hline
  17& (\arcch x)'       & \frac{1}{\sqrt{x^{2}-1}}\\
  \hline
  18& (\arcth x)'       & \frac{1}{1-x^{2}}\\
  \hline
  19& (\arccth x)'      & \frac{1}{1-x^{2}}\\
  \hline
  \end{array} &

%вторая таблица
%таблица интегрирования
  \begin{array}{|c|c|c|}
  \hline
  1 &\int dx & x + C\\
  \hline
  2 &\int x^{n}dx &\frac{x^{n+1}}{n+1} + C, ~ n \not= -1\\
  \hline
  3 &\int \frac{1}{x}dx &\ln |x| + C\\
  \hline
  4 &\int a^{x}dx &\frac{a^x}{\ln a} + C\\
  \hline
  5 &\int e^{x}dx &e^x + C\\
  \hline
  6 &\int \cos x dx &\sin x + C\\
  \hline
  7 &\int \sin x dx &-\cos x + C\\
  \hline
  8 &\int \frac{dx}{\cos^2 x} &\tg x + C\\
  \hline
  9 &\int \frac{dx}{\sin^2 x} &-\ctg x + C\\
  \hline
  10&\int \frac{dx}{\sqrt{1-x^2}} &\arcsin x + C = -\arccos x + C\\
  \hline
  11&\int \frac{dx}{1+x^2} &\arctg x + C = -\arcctg x + C\\
  \hline
  12&\int \frac{dx}{a^2+x^2} &\frac{1}{a}\arctg \frac{x}{a} + C\\
  \hline
  13&\int \frac{dx}{\sqrt{a^2-x^2}}& \arcsin \frac{x}{a} + C\\
  \hline
  14&\int \frac{dx}{x^2-a^2} &\frac{1}{2a}\ln |\frac{x-a}{x+a}| + C\\
  \hline
  15&\int \frac{dx}{\sqrt{x^2 \pm a}} &\ln |x + \sqrt{x^2 \pm a}| + C\\
  \hline
  16&\int \frac{dx}{\sin x} &\ln |\frac{1-\cos x}{\sin x}| + C\\
  \hline
  17&\int \frac{dx}{\cos x} &\ln |\frac{1+\sin x}{\cos x}| + C\\
  \hline
  18&\int \sh dx &\ch x + C\\
  \hline
  19&\int \ch dx &\sh x + C\\
  \hline
  20&\int \frac{dx}{\ch^2 x} & {\th} x + C\\
  \hline
  21&\int \frac{dx}{\sh^2 x} & -\cth x + C\\
  \hline
  \end{array}&

  %третья таблица тригенометрии
  \begin{array}{|c|c|}

  \hline
  1 &\cos^2 x + \sin^2 x = 1\\
  \hline
  2 &\tg x = \frac{\sin x}{\cos x}\\
  \hline
  3 &\ctg x = \frac{\cos x}{\sin x}\\
  \hline
  4 &\tg x \cdot \ctg x = 1\\
  \hline
  5 &\tg x = \frac{1}{\ctg x}\\
  \hline
  6 &\ctg x = \frac{1}{\tg x}\\
  \hline
  7 &1 + \tg^2 x = \frac{1}{\cos^2 x}\\
  \hline
  8 &1 + \ctg^2 x = \frac{1}{\sin^2 x}\\
  \hline
  9 &\sin (x\pm y) = \sin x \cdot \cos y \pm \cos x \cdot \sin y\\
  \hline
  10&\cos (x\pm y) = \cos x \cdot \cos y \mp \sin x \cdot \sin y\\
  \hline
  11&\tg (x\pm y) = \frac{\tg x \pm \tg y}{1 \mp \tg x \cdot \tg y}\\
  \hline
  12&\sin 2x = 2\sin x \cdot \cos x\\
  \hline
  13&\cos 2x = \cos^2 x - \sin^2 x\\
  \hline
  14&\cos 2x = 2\cos^2 x - 1\\
  \hline
  15&\cos 2x = 1 - 2\sin^2 x\\
  \hline
  16&\tg 2x = \frac{2\tg x}{1 - \tg^2 x}\\
  \hline
  17&\ctg 2x = \frac{\ctg^2 x - 1}{2\ctg x}\\
  \hline
  18&\sin x \pm \sin y = 2\sin \frac{x\pm y}{2} \cdot \cos \frac{x\mp y}{2}\\
  \hline
  19&\cos x + \cos y = 2\cos\frac{x + y}{2} \cdot \cos\frac{x - y}{2}\\
  \hline
  20&\cos x - \cos y = -2\sin\frac{x + y}{2} \cdot \sin\frac{x - y}{2}\\
  \hline
  21&\tg x \pm \tg y = \frac{\sin (x \pm y)}{\cos x \cdot \cos y}\\
  \hline
  22&\sin^2 x = \frac{1 - \cos 2x}{2}\\
  \hline
  23&\cos^2 x = \frac{1 + \cos 2x}{2}\\
  \hline
  24&\sin x \cdot \cos y = \frac{1}{2}(\sin (x+y) + \sin (x-y))\\
  \hline
  25&\cos x \cdot \cos y = \frac{1}{2}(\cos (x+y) + \cos (x-y))\\
  \hline
  26&\sin x \cdot \sin y = \frac{1}{2}(\cos (x-y) - \cos (x+y))\\
  \hline

  \end{array}&

  %Четвертая таблица: дописана тригонометрия, логорифмы, первый замечательный
  %предел и второй замечательный предел.
  \begin{array}{|c|c|}

  \hline
  27&|\sin \frac{x}{2}| = \sqrt{\frac{1 - \cos x}{2}}\\
  \hline
  28&|\cos \frac{x}{2}| = \sqrt{\frac{1 + \cos x}{2}}\\
  \hline
  29&\tg \frac{x}{2} = \frac{\sin x}{1 + \cos x}\\
  \hline
  30&\ctg \frac{x}{2} = \frac{\sin x}{1 - \cos x}\\
  \hline

  %Первый замечательный предел
  \hline
  1 &\lim_{x \to 0} \frac{\sin f(x)}{f(x)} = 1\\
  \hline
  2 &\lim_{x \to 0} \frac{\tg f(x)}{f(x)} = 1\\
  \hline
  3 &\lim_{x \to 0} \frac{\arcsin f(x)}{f(x)} = 1\\
  \hline
  4 &\lim_{x \to 0} \frac{\arctg f(x)}{f(x)} = 1\\
  \hline
  5 &\lim_{x \to 0} \frac{1 - \cos x}{\frac{f(x)^2}{2}} = 1\\
  \hline

  %Второй замечательный предел
  \hline
  1 &\lim_{x \to 0} (1+f(x))^{\frac{1}{f(x)}} = e\\
  \hline
  2 &\lim_{x \to \infty} ( 1 + \frac{k}{f(x)} )^{f(x)}  = e^k \\
  \hline
  3 &\lim_{x \to 0} \frac{\ln (1 + f(x))}{f(x)} = 1 \\
  \hline
  4 &\lim_{x \to 0} \frac{e^{f(x)} - 1}{f(x)} = 1\\
  \hline
  5 &\lim_{x \to 0} \frac{a^{f(x)} - 1}{f(x)\ln a} = 1 ~\text{для}~ a>0, a \not = 1\\
  \hline
  6 &\lim_{x \to 0} \frac{(1 + f(x))^a - 1}{af(x)} = 1\\
  \hline

  %Логорифмы
  \hline
  1&\log_{a}b = c, ~ a^c = b\\
  \hline
  2&\log_{a^k}b^m = \frac{m}{k}\log_{a}b\\
  \hline
  3&\log_{c}(ab) = \log_{c}a + \log_{c}b\\
  \hline
  4&\log_{c}(\frac{a}{b}) = \log_{c}a - \log_{c}b\\
  \hline
  5&\log_{a}b = \frac{\log_{c}b}{\log_{c}a}\\
  \hline
  6&\log_{a}b = \frac{1}{\log_{b}a}\\
  \hline
  7&a^{\log_{c}b} = b^{\log_{c}a}\\
  \hline
  8&\log_{a}b \cdot \log_{c}d = \log_{c}b \cdot \log_{a}d\\
  \hline

  \hline
  1&(a\pm b)^3 = a^3 \pm 3a^{2}b + 3ab^2 \pm b^3 \\
  \hline

  \end{array}
\end{array}
$
\end{turn}

\begin{turn}{90}
$
\begin{array}{|c|c|}

  \hline
  1&e^x = \sum_{k=0}^{\infty} \frac{x^k}{k!}, ~ x \in (-\infty, +\infty) \\
  \hline
  2&\ch(x) = \sum_{k=0}^{\infty} \frac{x^{2k}}{(2k)!},
  ~ x \in (-\infty, +\infty)\\
  \hline
  3&\sh(x) = \sum_{k=0}^{\infty} \frac{x^{2k + 1}}{(2k +1)!},
  ~ x \in (-\infty, +\infty)\\
  \hline
  4&\sin(x) = \sum_{k=1}^{\infty} \frac{(-1)^{k+1} x^{2k-1}}{(2k-1)!},
  ~ x \in (-\infty, +\infty)\\
  \hline
  5&\cos(x) = \sum_{k=0}^{\infty} \frac{(-1)^k x^{2k}}{(2k)!},
  ~ x \in (-\infty, +\infty)\\
  \hline
  6&\ln(1 + x) = \sum_{k=1}^{\infty} \frac{(-1)^{k+1} x^k}{k}, ~ x \in (-1, 1]\\
  \hline
  7&(1 + x)^p = \sum_{k=0}^{\infty} \frac{p(p-1)(p-2)\ldots(p-k+1) x^k}{k!},
  ~ x \in (-1,1)\\
  \hline
  8&\arctg(x) = \sum_{k=0}^{\infty} \frac{(-1)^k x^{2k+1}}{2k + 1},
  ~ x \in [-1,1]\\
  \hline
  9&\arcsin(x) = x + \sum_{k=1}^{\infty}
  \frac{(2k - 1)!! x^{2k+1}}{(2k)!! (2k+1)}, ~ x \in [-1,1]\\
  \hline

\end{array}
$
\end{turn}

\begin{block}[Необходимые условия сходимости числового ряда]
  $$
  \lim_{k \to \infty} a_k = 0
  $$
\end{block}

\begin{block}[Критерий Коши сходимости числового ряда:]
  Для того чтобы $\sum_{k=1}^{\infty} a_k$ был сходящимся необходимо и
  достаточно
  $$
  \forall \varepsilon > 0 ~~~
  \exists n_{\varepsilon} \in N ~~~
  \forall n \ge n_{\varepsilon} ~~~
  \forall p \in N ~~~
  \left| \sum_{k=n+1}^{n+p} a_k \right| < \varepsilon
  $$
\end{block}

\begin{block}[Критерий]
  $
  \forall k \in N ~~~
  a_k \ge 0
  $
  для сходимости $a_k$ необходимо и достаточно чтобы $S_n \le M$
\end{block}

\begin{block}[Интегральный признак сходимости числового ряда:]
  $f(x) \ge 0$ определена, непрерывна, $\searrow$ на $[1, +\infty]$ тогда
  $$
  \sum_{k=1}^{\infty} f(k) ~~~~~~ \int_1^{\infty} f(x)dx
  $$
  сходятся или расходятся одновременно.
\end{block}

\begin{block}[Признак сравнения:]
  $\forall k \in N ~~~ 0 \le a_k \le b_k$ тогда

  $\sum_{k=1}^{\infty} b_k$ сходится следовательно $\sum_{k=1}^{\infty} a_k$
  сходится

  $\sum_{k=1}^{\infty} a_k$ расходится следовательно $\sum_{k=1}^{\infty} b_k$
  расходится
\end{block}

\begin{block}[Признак сравнения в предельной форме]
  $\forall k \in N ~~~ a_k, b_k > 0 ~~~ a_k \sim b_k ~~ k \to \infty$ тогда

  $\sum_{k=1}^{\infty} a_k ~~~ \sum_{k=1}^{\infty} b_k$
  сходятся или расходятся одновременно.
\end{block}

\begin{block}[Признак Даламбера:]
  $\forall k \in N ~~~ a_k > 0$

  Если $\exists 0 < \lambda < 1 ~~~ \exists k_0 \in N ~~~ \forall k \ge k_0 ~~~
  \frac{a_{k+1}}{a_k} \le \lambda$ тогда $\sum_{k=1}^{\infty} a_k$ сходится

  Если $\exists k_0 \in N ~~~ \forall k \ge k_0 ~~~ \frac{a_{k+1}}{a_k} \ge 1$
  тогда $\sum_{k=1}^{\infty} a_k$ расходится.
\end{block}

\begin{block}[Признак Даламбера в предельной форме]
  $$
  \forall k \in N ~~~ a_k > 0 ~~~
  \exists \lim_{k \to \infty} \frac{a_{k+1}}{a_k} = \lambda
  $$
  1. Если $\lambda < 1 ~~~ \sum_{k=1}^{\infty} a_k$ сходится

  2. Если $\lambda > 1 ~~~ \sum_{k=1}^{\infty} a_k$ расходится

  3. Если $\lambda = 1 ~~~ \sum_{k=1}^{\infty} a_k$ может быть что угодно

  Наиболее эфективно его можно использовать, если члены ряда содержат факториал.
\end{block}

\begin{block}[Признак Коши сходимости числового ряда]
  $\forall k \in N ~~~ a_k \ge 0 $ тогда

  Если $\exists 0 < \lambda < 1 ~~~ \exists k_0 \in N ~~~ \forall k \ge k_0 ~~~
  \sqrt[k]{a_k} \le \lambda$ тогда $\sum_{k=1}^{\infty} a_k$ сходится.

  Если $\exists k_0 \in N ~~~ \forall k \ge k_0 ~~~ \sqrt[k]{a_k} \ge 1$ тогда
  $\sum_{k=1}^{\infty} a_k$ расходится.
\end{block}

\begin{block}[Признак Коши в предельной форме]
  $$
  \forall k \in N ~~~ a_k \ge 0 ~~~ \exists \lim_{k \to \infty} \sqrt[k]{a_k} =
  \lambda
  $$

  1. Если $\lambda < 1 ~~~ \sum_{k=1}^{\infty} a_k$ сходится.

  2. Если $\lambda > 1 ~~~ \sum_{k=1}^{\infty} a_k$ расходится.

  3. Если $\lambda = 1$ может расходится или сходится.
\end{block}

\begin{block}[Признак Лейбница:]
  $\forall k \in N ~~~ a_k > 0 ~~~ a_k \searrow ~~~ a_k \to 0$ тогда
  $\sum_{k=1}^{\infty} (-1)^{k+1} a_k$ сходится.
\end{block}

\begin{block}[Признак Дирехле:]
  1) $\sum_{k=1}^{\infty} a_k ~~~ \exists M > 0 ~~~ \forall n \in N ~~~
  |\sum_{k=1}^n a_k| \le M$

  2) $b_k \nearrow$ или $\searrow$

  3) $b_k \to 0$

  тогда $\sum_{k=1}^{\infty} a_k b_k$ сходится.
\end{block}

\begin{block}[Признак Абеля:]
  1) $\sum_{k=1}^{\infty} a_k$ сходится.

  2) $b_k \nearrow$ или $\searrow$

  3) $\exists M > 0 ~~~ \forall k \in N ~~~ |b_k| \le M$

  тогда $\sum_{k=1}^{\infty} a_k b_k$ сходится.
\end{block}

\begin{define}[равномерной сходимости]
  $$
  \forall \varepsilon > 0 ~~~ \exists n_{\varepsilon} \in N ~~~ \forall x \in E
  ~~~ \forall n \ge n_{\varepsilon} ~~~ |S_n (x) - f(x)| < \varepsilon
  $$
\end{define}

\begin{block}[Критерий Коши равномерной сходимости]
  $$
  \sum_{k=1}^{\infty} f_k(x) ~ \text{на} ~ E
  $$
  $$
  \forall \varepsilon > 0 ~~~ \exists n_{\varepsilon} \in N ~~~
  \forall x \in E ~~~ \forall n \ge n_{\varepsilon} ~~~ \forall p \in N ~~~
  |S_{n+p}(x) - S_n(x)| < \varepsilon ~ \text{или} ~
  \left| \sum_{k = n + 1}^{n+p} f_k(x) \right| < \varepsilon
  $$
  тогда $S_n(x) \stackrel{E}{\rightrightarrows} f(x)$
\end{block}

\begin{block}[Критерий равномерной сходимости]
  $$
  \lim_{n \to \infty} \sup\limits_{x \in E} |S_n(x) - f(x)| = 0 ~~ \text{тогда} ~~
  S_n(x) \stackrel{E}{\rightrightarrows} f(x)
  $$
\end{block}

\begin{block}[Признак Вейштрасса]
  $$
  \forall x \in E ~~~ \forall k \in N ~~~ |f_k(x)| \le a_k ~~~
  \sum_{k=1}^{\infty}a_k ~ \text{сходится, тогда} ~~~
  \sum_{k=1}^{\infty} f_k (x) ~ \text{равномерно сходится}
  $$
\end{block}

\begin{block}[Признак Дирихле]
  1)
  $
  M > 0 ~~ \forall x \in E ~~ \forall n \in N ~~~
  \left| \sum_{k=1}^n f_k(x) \right| \le M
  $

  2) $g_n(x) ~~~ \searrow$ или $\nearrow$

  3) $g_n(x) \rightrightarrows 0$

  тогда $\sum_{k=1}^{\infty} f_n(x) \cdot g_n(x)$ равномерно сходится
\end{block}


\begin{block}[Признак Абеля]
  1) $\sum_{k=1}^{\infty} f_k(x) \rightrightarrows$

  2) $g_n(x) ~~~ \searrow$ или $\nearrow$

  3) $|g_n(x)| \le M$

  тогда $\sum_{k=1}^{\infty} f_n(x) \cdot g_n(x)$ равномерно сходится
\end{block}

\end{document}