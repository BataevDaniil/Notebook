\documentclass{article}

\usepackage[utf8]{inputenc}
\usepackage{amsmath,amssymb}
\usepackage[russian]{babel}
\usepackage{cancel}
\usepackage{pgfplots}
\pgfplotsset{compat=1.9}
%\usepackage[pdftex]{color} пакет pdfplots уже подключил пакет color
\usepackage{ulem}

\parindent = 0pt

\newcommand{\bs}{$\backslash$}

\newcommand{\bd}[1]{{\bfseries #1}} %жирный текст

\newcommand{\bb}[1]{\bd{\bs #1}} %текст становится жирным
%и перед ним ставит \


\begin{document}

Команды \TeX чувствительны к регистру.\\

Символы  \{ \} \bs \% \_ \^{}  \~{} \& \$ \# просто так
неупотребляются в тексте их можно употреблять
\verb|\{ \}  $\backslash$ \% \_  \^{}  \~{}  \&  \$  \#|.\\
|

Окружение \bb{documentclass}\{параметр\} определяет какого типаа будет текст
и как он будет оформлен.\\

Класс \bb{begin}\{document\} \bb{end}\{document\} между ними надо набирать текст
если вводить текст после то он будет игнорироватсья.\\

\bb{usepackage}\{имя\_пакета\} подлкючает пакеты.\\
\bb{parindent=ед.изм} абзац становится ед.изм.\\

\bd{МАТЕМАТИЧЕСКИЕ ФОРМУЛЫ.}\\
Матматические формулы в тексте записываются между \$формула\$ или
\bs (формула\bs ) или
\bs begin\{math\}формула\bs end\{math\}. Отдельно от текста
\bs [формула\bs ] или
\bs begin\{displaymath\}формула\bs end\{displaymath\}.
Для того чтобы формулы нумеровались
\bs begin\{equation\}формула\bs end\{equation\}
и можно отметить фомулу \bs label\{метка\}
и сослаться на нее в тексте с помощью \bs ref\{метка\}
или \bs eqref\{метка\}.\\

\bb{not} перечеркивает символ стоящий после него.\\
\bb{ge} $\ge$ больше или равно.\\
\bb{le} $\le$ меньше или равно.\\
\bb{subset} $\subset$ содержится.\\
\bb{supset} $\supset$ содержится перевернутое.\\
\bb{subseteq} $\subseteq$ содержится или равно.\\
\bb{supseteq} $\supseteq$ содержится или равно перевернутой.\\
\bb{sqcup} $\sqcup$ попарно непересекающиеся множества.\\
\bb{vee} $\vee$ .\\
\bb{wedge} $\wedge$.\\
\bb{cup} $\cup$.\\
\bb{cap} $\cap$.\\
\bb{bigcup} $\bigcup_{i \in I}$.\\
\bb{bigcap} $\bigcap_{i \in I}$.\\
\bb{mid} $\mid$ вертикальная палка.\\
\bb{perp} $\perp$ перпендикулярно.\\
\bb{approx} $\approx$ приближенное значение.\\
\bb{ast} $\ast$ умножение звездочкой.\\
\bb{frac}\{числитель\}\{знаменатель\} $\frac{1*2}{1-3}$\\
\bb{left}(тип скобки) открывает скобку \bb{right}(тип скобки) закрывает скобку\\
\bb{sqrt}[степень корня]\{подкореное значение\}\\
\bb{surd} $\surd$ знак корня.\\
\bb{pm} $\pm$ плюс минус.\\
\bb{mp} $\mp$ минус плюс.\\
\bb{div} $\div$ делить.\\
\bb{setminus} $\setminus$ черта с лева на право.\\
\bb{times} $\times$ умножение крестом.\\
\bb{oplus} $\oplus$ плюс в круге.\\
\bb{odot} $\odot$ точка в круге.\\
\bb{dot}\{символы\} $\dot{f}(x)$ точка над символом.\\
\bb{sim} $\sim$ эквивалентность.\\
\bb{otimes} $\otimes$ умноожение крестом в круге.\\
\bb{triangle} $\triangle$ треугольник.\\
\bb{diamondsuit} $\diamondsuit$ ромб.\\
\bb{Box} $\Box$ квадрат.\\
\bb{neg или lnot} $\lnot$ знак отрицатия.\\
\bb{ominus} $\ominus$ минус в круге.\\
\bb{oslash} $\oslash$ слэш и круге.\\
\bb{circ} $\circ$ маленький круглешек.\\
\bb{bullet} $\bullet$ закрашенный круглешок.\\
\bb{cdots} $\cdots$ многоточия по центру.\\
\bb{ldots} \ldots многоточия с низу.\\
\bb{vdots} $\vdots$ вертикальные 3 точки.\\
\bb{ddots} $\ddots$ диагональный 3 точки.\\
\bb{sin \bs cos} и тп элементарные функции записываются так.\\
\bb{measuredangle} $\measuredangle$ угол.\\
\bb{lim} $\lim$ предел.\\
\bb{sum} $\sum$ знак суммы.\\
\bb{infty} $\infty$ бесконечность.\\
\bb{to} $\to$ знак стремления.\\
\bb{forall} $\forall$ любой.\\
\bb{exists} $\exists$ существует.\\
\bb{mathbf}\{символы\} $\mathbf{ABCdef}$ делает симолы жирными.\\
\bb{mathrm} $\mathrm{ABCdef}$.\\
\bb{mathit} $\mathit{ABCdef}$.\\
\bb{mathnormal} $\mathnormal{ABCdef}$.\\
\bb{mathsf} $\mathsf{ABCdef}$.\\
\bb{mathtt} $\mathtt{ABCdef}$.\\
\bb{mathcal} $\mathcal{ABCdef}$.\\
\bb{mathfrac} $\mathfrak{ABCdef}$.\\
\bb{mathbb}\{символы\} $\mathbb{ABCdef}$ делает символы полужирными.\\
\bb{in} $\in$ пренадлежит.\\
\bb{ni} $\ni$ пренадлежит перевернутое.\\
\bb{notin} $\notin$ не пренадлежит.\\
\bb{neq} $\neq$ не равно.\\
\bb{overline}\{выражение\} $\overline{a+b=c}$ рисует линию над выражением.\\
\bb{underline}\{выражение\} $\underline{a+b=c}$ рисует линию под выражением.\\
\bb{overbrace}\{выражение\}\^ \{$a+c$\}$\overbrace{a+b=c}^{a-c=-b}$ фигурная
скобка над выражением.\\
\bb{underbrace}\{выражение\}\_\{$a+c$\}$\underbrace{a=c}_{c=a}$фигурная скобка
под выражением.\\
\bb{!} $a\! b$ отрицательный пробел.\\
\bb{,} $a\, b$ пробел.\\
\bb{:} $a\: b$ пробел.\\
\bb{;} $a\; b$ пробел.\\
\bb{} $a\ b$ пробел.\\
\bb{quad} $a\quad b$ пробел.\\
\bb{qquad} $a\qquad b$ пробел.\\
\bb{cdot} $\cdot$ знак точки.\\
\bb{vec} $\vec a$ знак вектора над одним сомволом.\\
\bb{overrightarrow}\{символы\}$\overrightarrow{AB}$знак вектора над символами.\\
\bb{overleftarrow}\{символы\} $\overleftarrow{AB}$ знак вектора над символами
со стрелкой в обратную сторону.\\
\bb{equiv} $a\equiv b$ эквивалентно.\\
\bb{bmod} $a \bmod b$ деление с остатком.\\
\bb{pmod}\{символы\} $x \equiv a \pmod{b}$ деление с остатком.\\
\bb{rightarrow или \bs to} $\rightarrow$ стрелка напрвленая права.\\
\bb{leftarrow или \bs gets} $\leftarrow$ стелка напрвленая в лево.\\
\bb{leftrightarrow} $\leftrightarrow$ стрелка напрвленная в лево и в вправо.\\
\bb{uparrow} $\uparrow$ стрелка направленная в верх.\\
\bb{downarrow} $\downarrow$ стрелка направленная вниз.\\
\bb{updownarrow} $\updownarrow$ стрелка напрвленная в верх и вниз.\\
\bb{nearrow} $\nearrow$ стрелка  по диагонали верх в прво.\\
\bb{searrow} $\searrow$ стрелка по диагонали вниз в прво.\\
\bb{swarrow} $\swarrow$ стрелка по диагонали вниз в лево.\\
\bb{nwarrow} $\nwarrow$ стрелка по диагонали вверз в лево.\\
\bb{leftleftarrows} $\leftleftarrows$ две стрелки наровлены в лево.\\
\bb{rightrightarrows} $\rightrightarrows$ две стрелки напрвлены в право.\\
\bb{leftrightarrows}$\leftrightarrows$ стрелка смотрит в лево, другая в право.\\
\bb{rightleftarrows}$\rightleftarrows$ стрелка смотрит в право, другая в лево.\\
\bb{upuparrows} $\upuparrows$ две стрелки смотрят вврех.\\
\bb{downdownarrows} $\downdownarrows$ две стрелки смотрят вниз.\\
\bb{Leftarrow} $\Leftarrow$ стедовательно в левую сорону.\\
\bb{Rightarrow} $\Rightarrow$ следовательно в правую сторону.\\
\bb{Leftrightarrow} $\Leftrightarrow$ слеодовательно в леов и в право.\\
\bb{Uparrow} $\uparrow$ слодовательно в верх.\\
\bb{Downarrow} $\Downarrow$следовательно вниз.\\
\bb{Updownarrow} $\Updownarrow$ следовательно вниз и вверх.\\
\bb{mapsto} $\mapsto$ отображение.\\
\bb{rangle} $\rangle$.\\
\bb{langle} $\langle$.\\
\bb{binom}\{символы\}\{символы\} $\binom{a}{b}$ биномиальные коэффиценты.\\
\bb{int} $\int$ интеграл.\\
\bb{iint} $\iint$ 2 интеграла.\\
\bb{iiint} $\iiint$ 3 интеграла.\\
\bb{alpha} $\alpha$ греческая буква.\\
\bb{beta} $\beta$ греческая буква.\\
\bb{pi} $\pi$ греческая буква.\\
\bb{gamma} $\gamma$ греческая буква.\\
\bb{delta} $\delta$ греческая буква.\\
\bb{Delta} $\Delta$ греческая буква.\\
\bb{theta} $\theta$ греческая буква.\\
\bb{lambda} $\lambda$ греческая буква.\\
\bb{mu} $\mu$ греческая буква.\\
\bb{rho} $\rho$ греческая буква.\\
\bb{sigma} $\sigma$ греческая буква.\\
\bb{tau} $\tau$ греческая d
буква.\\
\bb{partial} $\partial$ знак частной произовдной.\\
\bb{upsilon} $\upsilon$ греческая буква.\\
\bb{phi} $\phi$ греческая буква.\\
\bb{varphi} $\varphi$ греческая буква.\\
\bb{chi} $\chi$ греческая буква.\\
\bb{psi} $\psi$ греческая буква.\\
\bb{omega} $\omega$ греческая буква.\\
\bb{stackrel}\{символы с верху\}\{символы снизу\} $\stackrel{!}{=}$\\
\bb{prod} $\prod$ оператор умножения.\\
символы \bb{limits}\_\{символы\} $\sup \limits_{1<j<i<n}$ символы под
командой.\\
символы\_\{\bb{substack}\{символы\}\} \[\sum_{\substack{a<b \\ a+b}}\] или
можно использовать окружение \bb{begin}\{subarray\}\{спецификатор\}
\bb{end}\{subarray\}\\ \\

\bb{begin}\{array\}\{позиция в столбцах\} \bb{end}\{array\} используется
для построения матрицы примерно так же как и для таблиц. Можно использовать
для верстки выражений таких как система неравенств. Так же как и в таблицах
можно рисовать линии. Если формула вылазиет с одной страницы на другую
то вместо equation надо использовать окружение
eqnarray(у каждой строки свой номер) и eqnarray*(номера не ставятся)\\
\bb{setlength}\bb{arraycolsep}\{рстояние\} задает растояние между столбцами в
матрице, вызывается перед окружением.\\
\bb{nonumber} не пишит номер для строки в которой вызван. \\
Пример:
\begin{verbatim}
  \begin{displaymath}
    \left\( \begin{array}{lcr}
      a_{11} & a_{12} & a_{13} \\
      a_{21} & a_{22} & a_{23} \\
      a_{31} & a_{32} & a_{33}
  \end{array}\right\)
\end{displaymath}
\end{verbatim}
\begin{displaymath}
  \left( \begin{array}{lcr}
    a_{11} & a_{12} & a_{13} \\
    a_{21} & a_{22} & a_{23} \\
    a_{31} & a_{32} & a_{33}
  \end{array}\right)
\end{displaymath}
\\\\

\bb{phantom}\{символы\} сомволы становятся невидимыми.\\
\bb{newtheorem}\{название\}[счетчик]\{текст\}[раздел] Аргумент название это
краткое ключевое слово, используемоу для индинтификации теорему.
Аргумент текст определяет настоящее название теоремы. Необязательные аргументы
применяемы для немерации теоремы.\\

\bb{input}\{имя\_файла\}  вставляет исходник имя\_файла
\bb{endinput} если в вызавать после \bs input \{ \} то читаться файл будет до
вызова \bs endinput в этом файле.\\
\bb{include}\{имя\_файла\} вставляет начиная с новой страницы исходник
имя\_файла (\bs include нельзя употреблять в файле,
который сам включается в текст с помощью \bs include)\\
\bb{incudeonly}\{имена\_файлов\} обявляется в преамбуле и \bs include \{\ldots\}
открывает файлы объявленные в \bs incudeonly \{\ldots\}\\

\bd{ЕДЕНИЦЫ ИЗМЕРЕНИЯ В \TeX}\\
pt = 0.35\\
pc = 12pt\\
mm = милиметр\\
cm = 10mm\\
in = 25.4mm\\
dd = пукт Дидо 1.07pt\\
cc = 12dd\\

\bd{itemize} подходить для простых списков.\\
\bd{enumerate} для нумерованных списков.\\
\bd{descriprtion} для описания.\\
\bd{flushleft} выравнивает текст по левому краю.\\
\bd{flushright} выравнивает текст по левому краю.\\
\bd{center} выравнивает текс по центру.\\
\bd{verbatim} тест будет написан так как он есть без какого либо вмешательсва
\LaTeX~или можно написать в тексте \verb|\verb+текст+| где вместо знак <<+>>
может быть любой знак в виде окраничителя кроме <<*>>, пробела и букв.\\

\bd{tabular} окружение таблицы.\\
\bb{begin}\{tabular\}[\emph{позиция}]\{\emph{спецификация}\}\\

\bd{Спецификация} определяет формат таблицы:\\
\bd{l} для столбца текста выровненого слева.\\
\bd{r} для стоблца текста выровненого справа.\\
\bd{c} для централизованного текста.\\
\bd{p}\{\emph{ширина}\} ширина для столбца содержащего выровненый текст
с переносом строк.\\
\bd{|} для вертикальной линии.\\
\bd{@}\{\ldots\} удаляет пробелы между столбцами
и заменяет их тем что между собок.\\

\bd{Позиция} определяет вертикальное положение всего табличного окружения:\\
\bd{t} выравнивание во верхнему краю.\\
\bd{b} выравнивание по нижнему краю.\\
\bd{c} выравнивание по центру окржения.\\

Команды используемые \bd{внутри откружения tabular:}\\
\bd{\&} переходит к следующему столбцу.\\
\verb|\\| начинает новую строку.\\
\bb{hline} всавляет горизонтальную линию.\\
\bb{cline}\{j-i\} добовляет в таблицу неполные линии,
где j и i номера столбцов над которыми должна проходить линия.\\
\bb{multicolumn}\{i\}\{спецификация\}\{текст\} объеденяет ячейки,
где i кол-во обедененых столбцов, спецификация спецификация,
текст это выведеный текст в обедененые столбцы.\\

Если надо набирать таблцы которые будут переходить на другую страницу,
то надо испольховать окружения longtable supertabular.\\ \\

\bd{Плавающие обекты.}\\
\bs begin\{figure\}[спецификации размещения]\\
\bs begin\{table\}[спецификации размещения]\\
По умолчанию спецификация [tbr].\\
\bd{h} ставит объект здесь же.\\
\bd{t} ставит объект на верху страницы.\\
\bd{b} ставит объекст внизу страницы.\\
\bd{p} ставит объкт на специальной страницы для плавающих обектов.\\
\bd{!} нерасматривает большинство внутренних параметров,
которые могут предотвратить размещение этого обекта.\\
\bb{caption}[краткий вариант]\{текст заголовка таблицы или рисунка\}\\
\bb{listoffigures} печатает список рисунков.\\
\bb{listoftables} печатет список таблиц.\\
При помощи \bs label \bs ref можно делать в тексте сылки на плавающие объекты.\\
\bb{clearpage} немедленно размещает все плавающие обекты в очереди по месту
и переходит на новую страницу.\\
\bb{cleardoublepage} начинает новую правостороннею страницу.\\

\bb{@} означает конец предложения, ставится после точки для того чтобы
\LaTeX~не делал лишнего растояния.\\
\verb|\\| переход на новую строку.\\
\bb{newline} переход на новую строку.\\
\bb{newpage} переход на новую страницу.\\
\~{} делает ровно один пробел.\\
- одна черточка (дефис)\\
-- две черточки (короткое тире)\\
--- три черточки (длинное тирe)\\
$-$ \$черточка\$ (знак миуса)\\
\bb{mbox}\{текст\} текст не будет переносится.\\
\bb{fbox}\{текст\} текст не будет переносится и вокруг появляется рамка.\\
\bb{today} сегодняшняя дата (\today).\\
\bb{underline}\{текст\} текст будет \underline{подчернкнут}.\\
\bb{emph}\{текст\} текст будет выделен \emph{курсивом}.\\
\bb{sout}\{text\} \sout{текст зачеркнут.}
\bb{usepackage}\{ulem\} пакет для зачеркивания.\\
ЗАМЕЧАНИЕ: если вы используетет выделение в выделом тексте
то будет обычный текст.\\
\verb|<<текст>>| обозначают кавычки <<текст>>.\\

%шрифт
\bb{textrm}\{текст\} \textrm{прямой шрифт}.\\
\bb{texttt}\{текст\} \texttt{пишущая машинка.}\\
\bb{textmd}\{текст\} \textmd{нормальный.}\\
\bb{textup}\{текст\} \textup{прямой шрифт.}\\
\bb{textsl}\{текст\} \textsl{наклонный шрифт.}\\
\bb{textsf}\{текст\} \textsf{без зачесек.}\\
\bb{textbf}\{текст\} \textbf{без зачечек.}\\
\bb{textit}\{текст\} \textit{полужирный.}\\
\bb{textsc}\{текст\} \textsc{капитель.}\\
\bb{textnormal}\{текст\} \textnormal{обычный.}\\
\bb{bfseries} {\bfseries жирный} шрифт.\\
\bb{mdseries} отменяет жирный шрифт.\\
\bb{slshape} {\slshape наклонный} шриф.\\
\bb{upshape} отменяет наклонный шрифт.\\

%размер шрифта
\bb{tiny} {\tiny крошечный}.\\
\bb{scriptsize} {\scriptsize очень маленький}.\\
\bb{footnotesize} {\footnotesize довольно маленький}.\\
\bb{small} {\small маленький}.\\
\bb{normalsize} {\normalsize нормальный}.\\
\bb{large} {\large большой}.\\
\bb{Large} {\Large еще больше}.\\
\bb{LARGE} {\LARGE очень большой}.\\
\bb{huge} {\huge огромный}.\\
\bb{Huge} {\Huge Громадный}.\\

\bb{label}\{метка\} оставляет метку.\\
\bb{ref}\{метка\} заменяет вызов команды на номер раздела, подраздела,
абзаца, уравнения, илюстрации.\\
\bb{pageref}\{метка\} заменяет вызов команды на страницу метки.\\
\bb{protect} делает из хрупких команд нормальные.
Относится только к команде следущей сразуе за ней.\\
\bb{footnote}\{текст\_сноски\} образет сноску на экране.\\

\bb{newcommand}\{cmd\}[integer][default]\{definition\} объявляет новую команду
cmd. Замещающим её текстом является definition. По умолчанию команда не имеет
аргументов. Первая опция — целое число integer от 1 до 9 — указывает количество
аргументов у команды. При наличии второй опции первый аргумент новой команды
становится необязательным и по умолчанию принимает значение default. Аргументы
команды входят в definition в виде \#n, где n — порядковый номер аргумента.\\
\bb{renewcommand}\{cmd\}[integer][default]\{definition\} создает команду если
команда с таким именем уже существует.\\
\bb{providecommand}\{cmd\}[integer][default]\{definition\} если команда уже
существует то ничгео не делает.\\

\bb{newenvironment}\{name\}[integer][default]\{begdef\}\{enddef\} определяет
новое окружение name. \bb{begin}\{name\} замещается на begdef,
а \bb{end}\{name\} — на enddef. По умолчанию окружение не имеет аргументов.
Первая опция — целое число integer от 1 до 9 — указывает количество аргументов
у окружения. При наличии второй опции первый аргумент нового окружения
становится необязательным и по умолчанию принимает значение default.
Аргументы входят в begdef и enddef в виде \#n, где n — порядковый номер.\\
аргумента.\\
\bb{renewenvironment}\{name\}[integer][default]\{begdef\}\{enddef\} создает\\
новое окружение даже если окржуение с таким названием уже существует.\\

\bb{ProvidesPackage}\{название пакета\} надо вызвать в отедльном файле
с расширением .sty и после определить новые команды.
И можно будет подключать как обычный пакет в файле.\\

\bb{par} эквивалент пустой строки.\\

\bb{linespread}\{коээфицент\} изменяет интревал между строк.
По умолчанию коэффицент 1.\\
\bb{hspace}\{длина\} добовляет горизонта\\льный пробел.\\
\bb{stretch}\{n\} резиновый пробел.\\
\bb{vspace}\{длина\} длина между абзацами.\\

%цвета
Цвета. RGB принимает значения 0-1.\\
\bb{usepackage}[pdftex]\{color\} пакет использования цветов.\\
\bb{definecolor}\{name\}\{model\}\{spec\} определение нового имени цвета.\\
\bb{textcolor}[model]\{spec\}\{text\}
\textcolor[rgb]{1,0,0}{задает цвет тексту, модель надо выбрать rgb.}\\
{\bb{color}[model]\{spec\}text\} {\color[rgb]{0,1,0}задает цвет тексту.}\\
\bb{colorbox}[model]\{spec\}\{lr-text\}
\colorbox[rgb]{0,0,1}{помещает текст в цветную коробку.}\\
\bb{fcolorbox}[model]\{fr-spec\}\{spec\}\{lr-text\}
\fcolorbox[rgb]{1,0,0}{0,1,1}{помещает текст в цветовую коробку%
и обводит бокс рамкой.}\\
\bb{pagecolor}[model]\{spec\} окрашеивает до тех пор этим цветом
пока не будет встречена команда с другим цветом.\\

%зачеркивания
\bb{usepackage}\{cancel\} зачеркивания для илюстрации сокращения\\
\bb{cancel} \cancel{$x+334+434$}\\
\bb{bcancel} \bcancel{$x-33+343$}\\
\bb{xcancel} \xcancel{$x-3*3$}\\
\bb{cancelto}\{на что указывает стрелка\}\{зачеркнутое\}
$\cancelto{x+8}{x+3+5}$ ~~~~ работает только в мат формулах.\\

%графики
\bb{usepackage}\{pgfplots\} пакет для рисования графиков.\\
\bb{pgfplotsset}\{compat = 1.9\} версия пакета\\
Код должен быть написан в окружении \bd{tikzpicture}\\
Окружения рисуят систему координат.\\
\bd{axis} Стандартные оси с линейным масштабированием\\
\bd{semilogxaxis} Логарифмическое масштабирование оси x
и стандартное масштабирование оси y\\
\bd{semilogyaxis} Логарифмическое масштабирование оси y
и стандартное масштабирование оси x\\
\bd{loglogaxis} Логарифмическое масштабирование обеих осей\\

У этих окружений можно задать необязтельный параметрый:\\
\bd{width, height} Устанавливают ширину и высоту графика соответственно.\\
\bd{domain = min:max} Устанавливает область значений для функции
в диапазоне от min до max.\\
\bd{ymin, ymax} Устанавливают минимальное и максимальное значение
на оси ординат соответственно.\\
\bd{xlabel, ylabel} Устанавливают подпись к оси абсцисс
и оси ординат соответственно.\\
\bd{view} Устанавливает поворот камеры, при этом свойство указывается
следующим образом view = {азимут}{угол возвышения}; при этом
азимут – это угол между положением камеры и осью z,
а угол возвышения – это угол между положением камеры и осью x.\\
\bd{grid} Указывает тип сетки major – линии сетки проходят только
через основные деления, minor – линии сетки проходят через
дополнительные деления (между основными), both – линии сетки проходят
через оба вида делений, none – сетка отсутствует [по умолчанию].\\
\bd{colormap} Устанавливает используемую цветовую схему hot, hot2, jet,
blackwhite, bluered, cool, greenyellow, redyellow, violet
и другие, созданные пользователем.\\
\bd{pgfplotsset}\{параметрый\} параметры к коружениям можно задать не только
локально но и глобально.\\

\bb{addplot}[options]\{функция\} рисует функцию 2D, addplot3 рисует 3D функцию.
Опции <options> являются необязательным параметром, в которых указываются:
тип графика, его цвет, стиль, тип маркеров и т.п.\\
\bb{addplot}[options] coordinates \{coordinate list\} соеденяет указаные
точки линиями как двумерного случая так и трехмерного.\\

\bd{mod} Оператор взятия остатка.\\
! Постфиксный оператор вычисления факториала.\\
<, >, ==, <=, >=\\
\bd{abs} Функция взятия модуля.\\
\bd{sin cos tan asin acos atan} Основные тригонометрические функции
и обратные им.\\
\bd{ln log2 log10} Натуральный, двоичный и десятичный логарифмы.\\
\bd{deg} Функция преобразования радиан в градуса (она особенно полезна, если
учесть,что по умолчанию рассматриваемый парсер работает именно с градусами).\\

\bd{legend\{ \ldots \}} внутри которой через запятую перечисляются описания
графиков (PGFPlots определяет соответствие между описанием и графиком по порядку
следования описаний и порядку добавления самих графиков). Свойство legend pos,
которое позволяет указать положение легенды на графике
(south west, south east, north west, north east, outer north east )\\

\bd{pgfplotsset}\{style\_name/.style\=\{key-value-list\}\} задает стиль.\\
Примеры:\\

График построенный по точкам которые соеделенили линиями.\\
\begin{tikzpicture}
  \begin{axis}
   \addplot coordinates {
      (0,1) (2,4) (7,8) (10,15)
    };
  \end{axis}
\end{tikzpicture}

\begin{verbatim}
\begin{tikzpicture}
  \begin{axis}
   \addplot coordinates {
      (0,1) (2,4) (7,8) (10,15)
    };
  \end{axis}
\end{tikzpicture}
\end{verbatim}

График посмтроенный таблично.\\
\begin{tikzpicture}
  \begin{axis}
    \addplot3 table [x = b, y = a, z = c] {
             a      b      c
             1      1      1
             2      3      4
             3      5      5
             4      8      6
             5      2      7
   };
  \end{axis}
\end{tikzpicture}

\begin{verbatim}
\begin{tikzpicture}
  \begin{axis}
    \addplot3 table [x = b, y = a, z = c] {
             a      b      c
             1      1      1
             2      3      4
             3      5      5
             4      8      6
             5      2      7
    };
  \end{axis}
\end{tikzpicture}
\end{verbatim}

График построенный с помощью заданной функции.\\
\begin{tikzpicture}
  \begin{axis}[
              grid = both,
              width = 200,
              height = 200,
              xlabel = $x$,
              ylabel = $y$,
              ymin = -100,
              ymax = 100,
              domain = -10:4,
              ]
    \addplot[blue]{e^x};
  \end{axis}
\end{tikzpicture}

\begin{verbatim}
\begin{tikzpicture}
  \begin{axis}[
              grid = both,
              width = 200,
              height = 200,
              xlabel = $x$,
              ylabel = $y$,
              ymin = -100,
              ymax = 100,
              domain = -10:4,
              ]
    \addplot[blue]{e^x};
  \end{axis}
\end{tikzpicture}
\end{verbatim}

\end{document}
