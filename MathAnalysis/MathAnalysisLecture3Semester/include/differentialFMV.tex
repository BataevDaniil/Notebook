\begin{title}
  Дифференциорование функций многих переменных
\end{title}

\begin{title}[\Large]
  Метрическое пространство. Пространства $R^n$
\end{title}

\begin{define}[метрического пространства]
  Метрическое пространство это множество $M$, на котором определена функция
  $(x, y) \to \rho(x, y) \ge 0$

  Аксиомы:

  1) $\forall x,y \in M ~~~ \rho(x,y) \ge 0$ причем $\rho(x,y) = 0 ~
  \Leftrightarrow ~ x = y$

  2) $\forall x,y \in M ~~~ \rho(x,y) =\rho(y,x)$

  3) $\forall x,y,z \in M ~~~ \rho(x,z) \le \rho(x,y) + \rho(y,z)$
  (неравества треугольника)

  Отображение с такими свойствами называется метрикой.

  Значение $\rho(x,y)$ растояние между $x$ и $y$.
\end{define}

\begin{theorem}
  $C[a,b]$ множество всех непрерывных функций на $[a,b]$ тогда
  $$
  \rho(f(x), g(x)) = \sup_{x \in [a,b]} |f(x) - g(x)|
  $$
\end{theorem}

\begin{block}[Неравенство Коши]
  $$
  \left( \sum_{k=1}^n a_k b_k \right)^2 \le \sum_{k=1}^n a_k^2 \cdot
  \sum_{k=1}^n b_k^2
  $$
  Кудрявцев 2 том 167 страница
\end{block}

\begin{proof}
  $$
  P(t) = \sum_{k=1}^n (a_k + t b_k)^2 = \sum_{k=1}^n a_k^2 +
  2t\sum_{k=1}^n a_k b_k + t^2 \sum_{k=1}^n b_k^2
  $$
  $$
  D = 4\left( \sum_{k=1}^n a_k b_k \right)^2 - 4\sum_{k=1}^n a_k^2 \cdot
  \sum_{k=1}^n b_k^2 \le 0
  $$
  $$
  \left( \sum_{k=1}^n a_k b_k \right)^2 \le \sum_{k=1}^n a_k^2 \cdot
  \sum_{k=1}^n b_k^2
  $$
\end{proof}

\begin{block}[Неравенство Минского]
  $$
  \sqrt{\sum_{k=1}^n (a_k + b_k)^2} \le \sqrt{\sum_{k=1}^n a_k^2} +
  \sqrt{\sum_{k=1}^n b_k^2}
  $$
  Кудрявцев 2 том 167 страница
\end{block}

\begin{proof}
  $$
  \sum_{k=1}^n (a_k + b_k)^2 = \sum_{k=1}^n a_k^2 +
  2\sum_{k=1}^n a_k b_k + \sum_{k=1}^n b_k^2 \le
  $$
  $$
  \le \sum_{k=1}^n a_k^2 + 2\sqrt{\sum_{k=1}^n a_k^2} \sqrt{\sum_{k=1}^n b_k^2}
  + \sum_{k=1}^n b_k^2 \le
  $$
  $$
  \le \left( \sqrt{\sum_{k=1}^n a_k^2} \right)^2 + 2\sqrt{\sum_{k=1}^n a_k^2}
  \sqrt{\sum_{k=1}^n b_k^2} + \left( \sqrt{\sum_{k=1}^n b_k^2} \right)^2 =
  $$
  $$
  = \left( \sqrt{\sum_{k=1}^n a_k^2} + \sqrt{\sum_{k=1}^n b_k^2} \right)^2
  $$
  $$
  \sqrt{\sum_{k=1}^n (a_k + b_k)^2} \le \sqrt{\sum_{k=1}^n a_k^2} +
  \sqrt{\sum_{k=1}^n b_k^2}
  $$
\end{proof}

\begin{title}[\Large]
  Сходимость в метрических пространствах
\end{title}

\begin{define}[предела в метрическом пространстве]
  Множество $M$ с метрикой $\rho$ последовательность $x_k \in M$ называется
  сходящейся к $p \in M$ если $\exists \lim_{k \to \infty} \rho(x_k, p)
  = 0$
\end{define}

\begin{define}[окрестности в метрическом пространстве]
  Окрестностью точки $p \in M$ называют
  $$
  \{ x \in M: ~ \rho(x, p) < \varepsilon \} = O_{\varepsilon}(p)
  $$
  $$
  \{ x \in M: ~ 0 < \rho(x, p) < \varepsilon \} = ~
  \stackrel{\bullet}{O}_{\varepsilon}(p)
  $$
\end{define}

\begin{block}[Свойство]
  В метрическом пространстве подпространство тоже метрическое.
\end{block}

\begin{theorem}
  $x_k$ в метрическом пространсве сходится тогда $\exists M ~~ |x_k| \le M$
\end{theorem}

\begin{theorem}
  $x_k$ в метрическом пространстве сходится тогда $\exists ! \lim x_k$
\end{theorem}

\begin{proof}
  Допустим предела два $x_k \to p$ $x_k \to q$
  $$
  0 < \rho(p, q) \le \rho(p, x_k) + \rho(x_k, q) = \rho(x_k, p) + \rho(x_k, q)
  \to 0 ~~ k \to \infty ~ \Rightarrow ~ p = q
  $$
\end{proof}

\begin{block}[Критерий сходимости]
  Для того чтобы в метрическом прострастве $R^n$ последовательность
  $$
  x^{(k)} = (x_1^{(k)}, x_2^{(k)}, \ldots, x_n^{(k)}) \to
  p = (p_1, p_2, \ldots, p_n)
  $$
  необходимо и достаточно чтобы
  $$
  \left\{
  \begin{array}{c}
    x_1^{(k)} \to p_1 \\
    x_2^{(k)} \to p_2 \\
    \cdots \cdots \\
    x_n^{(k)} \to p_n
  \end{array}
  \right.
  $$
  Кудрявцев 2 том 175 страница
\end{block}

\begin{proof}
  1) $\Rightarrow$ $x^{(k)} \to p$ тогда $|x_i^{(k)} - p_i| \le
  \rho(x^{(k)}, p) \to 0 ~~~ i = 1, 2, \ldots, n$

  2) $\Leftarrow$ все условия выполнены $\rho(x^{(k)}, p) \to 0$ тогда
  $x^{(k)} \to p$
\end{proof}

\begin{define}[фундаментальной последовательности]
  $x_k$ метрического пространства $M$ называется фундаментальной
  последовательностью если
  $$
  \forall \varepsilon > 0 ~~~ \exists n_{\varepsilon} \in N ~~~
  \forall k \ge n_{\varepsilon} ~~~ \forall m \in N ~~~ \rho(x_k, x_{k+m}) <
  \varepsilon
  $$
\end{define}

\begin{theorem}
  $x_k$ сходится тогда $x_k$ фундаментальна.

  Замечание: обратное утверждение не верно.
\end{theorem}

\begin{proof}
  $$
  x_k \to p ~~~ \forall \varepsilon > 0 ~~~ \exists n_{\varepsilon} \in N ~~~
  \forall k \ge n_{\varepsilon} ~~~ \rho(x_k, p) < \frac{\varepsilon}{2}
  $$
  $$
  ~~~~~~~~~~~~~~~~~~~~~~~~~~~~~~~~~~~~~~~~~~~~
  \forall m \ge n_{\varepsilon} ~~~ \rho(x_{k + m}, p) < \frac{\varepsilon}{2}
  $$
  $$
  \rho(x_k, x_{k+m}) \le \rho(x_k, p) + \rho(x_{k+m}, p)
  < \frac{\varepsilon}{2} + \frac{\varepsilon}{2} = \varepsilon
  $$
\end{proof}

\begin{define}[полного метрического пространства]
  Метрическое пространство в котором любая фундаментальная последовательность
  сходится к элементу того же множества называется полным.
\end{define}

\begin{theorem}
  $R^n$ метрическое пространство с евклидовой метрикой является полным.
\end{theorem}

\begin{proof}
  $(x_k, y_k, z_k)$ пусть последовательность фундаменталная

  $x_k, y_k, z_k$ также фундаменталные
  $$
  |x_{k+m} - x_k| \le \rho(x_k, y_k, z_k)
  $$
  $$
  \rho(x_{k+m}, y_{k+m}, z_{k+m}) < \varepsilon
  $$
  $$
  |y_{k+m} - y_k| < \varepsilon
  $$
  $$
  |z_{k+m} - z_k| < \varepsilon
  $$
  $x_k \to p_1 ~~~ y_k \to p_2 ~~~ z_k \to p_3$ имеют пределы по критерию Коши

  $(x_k, y_k, z_k) = (p_1, p_2, p_3)$ является сходящейся $\Rightarrow$ $R^n$
  является полным.
\end{proof}

\begin{title}[\Large]
  Открытые множества в метрических пространствах
\end{title}

\begin{define}[внутреней точки метрического пространства]
  Точка $a$ называется внутренней точкой $D \subset M$ метрического
  пространства если  $\exists O_{\varepsilon}(a) \subset D$
\end{define}

\begin{define}[открытого множества]
  Множество из внутренних точек называют открытым множеством.
\end{define}

\begin{block}[Свойства]
  1) $\bigcup_{i \in I} D_i = D$ объеденение любого числа открытых множеств
  является открытым множеством.

  \begin{proof}
    $$
    a \in D ~~ \exists i_0 \in I ~~ a \in D_{i_0} ~ \Rightarrow ~
    \exists O_{\varepsilon}(a) \subset D_{i_0} \subset \bigcup_{i \in I} D_i
    $$
  \end{proof}

  2) $\bigcap_{i=1}^m D_i = D$ пересечение конечного числа открытых
  множеств является открытым множеством.

  \begin{proof}
    $a \in D ~~ \forall i = 1, 2, \ldots, m ~~~ a \in D_i ~~~
    \exists O_{\varepsilon_i}(a) \subset D_i$

    $\varepsilon = \min \{\varepsilon_1, \varepsilon_2, \ldots, \varepsilon_m\}$

    $\forall i = 1, 2, \ldots, m ~~~
    O_{\varepsilon}(a) \subset O_{\varepsilon_i}(a) \subset D$
  \end{proof}
\end{block}

\begin{title}[\Large]
  Замкнутые множества в метрических пространствах
\end{title}

\begin{define}[предельной точки]
  Точка $a$ называется предельной точкой множества $E \subset M$ если

  $\forall \stackrel{\bullet}{O}(a) \cap E \not= \oslash$
\end{define}

\begin{define}[замкнутого множества]
  Если множество содержит все свои предельные точки тогда оно называется
  замкнутым.
\end{define}

\begin{block}[Критерий]
  Чтобы $E \subset M$ было замкнутым необходимо и
  достаточно чтобы $D = M\backslash E$ его дополнение было открытым.
\end{block}

\begin{proof}
  1) $\Rightarrow$ $E$ замкнутое. Предположим что $D = M \backslash E$ не
  является открытым $\exists a \in D$ $a$ не внутренняя. В $\forall
  \stackrel{\bullet}{O}(a) \cap E \not= \oslash$ то есть $a$ предельная, но $E$
  замкнуто $\Rightarrow a \in E$. Не может быть так как $D$ дополнение $E$

  2) $\Leftarrow$ Пусть $D$ открытое. Пусть $E$ не замкнутое, тогда есть хотя
  бы одна
  точка, которая $a \not\in E$ и $a$ предельная для $E$ $\Rightarrow \forall
  \stackrel{\bullet}{O}_{\varepsilon}(a) \cap E \not= \oslash$. $a$ не может
  быть внутренней точкой.
\end{proof}

\begin{block}[Законы Деморгана]
  $$
  M \backslash (A \cup B) = (M \backslash A) \cap (M \backslash B)
  $$
  $$
  M \backslash (\cup A_i) = \cap (M \backslash A_i)
  $$
  $$
  M \backslash (\cap A_i) = \cup (M \backslash A_i)
  $$
\end{block}

\begin{block}[Свойства]
  1) $\cup E_i = E ~ i = 1,2, \ldots, m$ объединение конечного числа
  замкнутых множеств, есть множество замкнутое.

  \begin{proof}
    $\cup E_i ~ i = 1,2, \ldots, m$ замкнутое

    $M \backslash (\cup E_i) ~ i = 1,2, \ldots, m$ открытое

    $\cap (M \backslash E_i) ~ i = 1,2, \ldots, m$ открытое
  \end{proof}

  2) $\cap E_i = E ~ i \in I$ пересечение любого числа замкнутых множеств,
  есть замкнутое множество.

  \begin{proof}
    $\cap E_i ~ i \in I$ замкнутое

    $M \backslash (\cap E_i) ~ i \in I$ открытое

    $\cup (M \backslash E_i) ~ i \in I$ открытое
  \end{proof}
\end{block}

\begin{title}[\Large]
  Компактные множества в метрических пространствах
\end{title}

\begin{define}[компактного множества]
  Множество $K \subset M$ называется компактным, если из $\forall x_m$ можно
  извлечь подпоследовательность $x_{m_k}$ которая сходится к элементу данного
  множества.
\end{define}

\begin{theorem}
  Компактное множество ограниченно.
\end{theorem}

\begin{proof}
  Пусть $K$ неограничена $x_1 \in K$

  $\exists x_2 \in K ~~~ \rho(x_2, x_1) \ge 1$

  $\exists x_3 \in K ~~~ \rho(x_3, x_1) \ge 1$ и $\rho(x_3, x_2) \ge 1$ и так
  далее.

  Расмотрим последовательность $x_k \to p$ то она была бы фундаментальной для
  членов с достаточно большими номерами растояние должно быть $< \varepsilon$
\end{proof}

\begin{theorem}
  Компактное множество в любом метрическом пространстве замкнуто.
\end{theorem}

\begin{proof}
  Пусть $K$ незамкунутое $a$ предельная точка $a \not\in K ~~~
  x_k \in \stackrel{\bullet}{O}_{\frac{1}{k}}(a)$, который $x_k \in K$

  Все элементы сходятся к $a$ а $a$ не входит в $K$. Противоречие
\end{proof}

\begin{theorem}[Больцана-Вейерштраса для $R^n$]
  Для простоты $n = 3$
  $$
  (x_k, y_k, z_k) ~~ x_{k_l}\to p_1
  $$
  $$
  (x_{k_l}, y_{k_l}, z_{k_l}) ~~ y_{k_{l_s}} \to p_2
  $$
  $$
  (x_{k_{l_s}}, y_{k_{l_s}}, z_{k_{l_s}}) ~~ z_{k_{l_{s_i}}} \to p_3
  $$
  $$
  (x_{k_{l_{s_i}}}, y_{k_{l_{s_i}}}, z_{k_{l_{s_i}}}) \to (p_1, p_2, p_3)
  $$
\end{theorem}

\begin{block}[Критерий]
  $K \in R^n$ компактно $\Leftrightarrow$ замкнуто и ограничено.
\end{block}

\begin{proof}
  $x_k \in K$ ограничена. Из теоремы выше $x_{m_i} \to a \in R^n$ $a$
  предельная точка из $K$. Так как $K$ замкнуто то $a \in K$
\end{proof}

\begin{define}[граничной точки множества]
  Точка $b$ называется граничной точкой $E \subset M$ если
  $\forall O_{\varepsilon}(b)\cap E \not= \oslash$

  $\partial E$ множества всех граничных точек множества $E$
\end{define}

\begin{define}
  $$
  \text{Отрезком} ~~~ [a,b] = \{ x = (x_1, x_2, \ldots, x_n): ~
  x_i = a_i + t(b_i - a_k) ~~ t \in [0,1]\}
  $$
  $$
  \text{Прямая} ~~~ \{ x = (x_1, x_2, \ldots, x_n): ~ x_i = a_i(1 - t) +
  tb_i ~~~ t \in R \}
  $$
  $$
  \text{Луч} ~~~ \frac{\varphi(a)}{e} = \{ x: ~ x_i = a_i + te_i ~~~
  t \in [0, + \infty\}
  $$
\end{define}

\begin{define}[выпуклого множества]
  Множество называется выпуклым, если любые две точки которые принадлежат этому
  множеству, можно провести отрезок так что он будет принадлежать томуже
  множеству.
\end{define}

\begin{title}[\Large]
  Предел функции многих переменных. Различные типы пределов функций многих
  переменных (предел по множеству, предел по направлению, бесконечные приделы,
  повторные пределы)
\end{title}

\begin{define}[предела функции многих переменных]
  $f(x)$ определена на $\stackrel{\bullet}{O}(a) \subset R^n$ число
  $p = \lim_{x \to a} f(x)$ если
  $$
  \forall \varepsilon > 0 ~~~ \exists \delta_{\varepsilon} > 0 ~~~
  \forall x \in \stackrel{\bullet}{O}(a) ~~~ \rho(x, a) < \delta_{\varepsilon}
  ~~~ |f(x) - p| < \varepsilon
  $$
  определение по Коши
  $$
  \forall x_k \in \stackrel{\bullet}{O}(a) ~~~ x_k \to a \Rightarrow
  f(x_k) \to p
  $$
  определение по Гейне
\end{define}

\begin{define}[предела по множеству функции многих переменных]
  Число $p$ называется пределом $f(x)$ в точке $a$ по множеству $E$ если
  $p = \lim \limits_{x \to a ~ x \in E} f(x)$ где $a$ предельная точка
  множества $E$
  $$
  \forall \varepsilon > 0 ~~~ \exists \delta_{\varepsilon} > 0 ~~~
  \forall x \in E ~~~ \rho(x, a) < \delta_{\varepsilon} ~~~
  |f(x) - p| < \varepsilon
  $$
\end{define}

\begin{define}[предела по направлению функции многих переменных]
  $a = (a_1, a_2, \ldots, a_n)$ предел по
  напрвалению луча $e = (e_1, e_2, \ldots, e_n)$ (направляющие косинусы) тогда
  $$
  \lim_{t \to 0 +0} f(a_1 + te_1, a_2 + te_2, \ldots, a_n + te_n)
  $$
  называеют предел функции $f(x)$ по направлению $e$ в точке $a$.

  Предел по направлению, есть частный случай предела по множеству, когда луч
  выходящий из точки $a$ выходит по направлению к вектору $e$
  $$
  D = \{ (x,y): ~ 0 < |x - a| < \alpha ~~ 0 < |y - b| < \beta \}
  $$
  $\vec e_2 = (\cos \alpha, \sin \alpha)$
\end{define}

\begin{define}[повторного предела]
  $$
  \forall y \in \stackrel{\bullet}{O}(b) ~~~
  \lim_{x \to a} f(x, y) = \varphi(y) ~~~ \lim_{y \to b} \varphi(y) = p_1
  $$
  $$
  \forall x \in \stackrel{\bullet}{O}(a) ~~~
  \lim_{y \to b} f(x, y) = \psi(x) ~~~ \lim_{x \to a} \psi(x) = p_2
  $$
\end{define}

\begin{title}[\Large]
  Непрерывность функции многих переменных. Свойства ФМП непрерывных в точке
\end{title}

\begin{define}[непрерывности ФМП в точке]
  $f(x)$ определена в $O(a)$ если $\lim_{x \to a} f(x) = f(a)$ тогда
  $f(x)$ непрерывна в точке $a$.
\end{define}

\begin{block}[Свойства]
  $f(x), g(x)$ непрерывна в точке $a$ тогда
  $$
  f(a) \pm g(a)
  $$
  $$
  f(a) \cdot g(a)
  $$
  $$
  \frac{f(a)}{g(a)} ~~~ (g(a) \not= 0)
  $$
  также будут непрывны в точке.
\end{block}

\begin{theorem}
  $f(t) = f(t_1, t_2, \ldots, t_m)$ непрерывна в $b = (b_1, b_2, \ldots, b_m)
  ~~ t_k = g_k(x) = g_k(x_1, x_2, \ldots, x_n) ~ k = 1,2, \ldots, m$
  непрерывна в точке $a = (a_1, a_2, \ldots, a_m)$ и $b_k = g_k(a)$ тогда
  $F(x) = f(g_1(x), g_2(x), \ldots, g_m(x))$ непрерывна в точке $a$.
\end{theorem}

\begin{proof}
  $$
  \forall x^{(\varphi)} \to a ~ \Rightarrow ~ g_k(x^{(\varphi)}) \to b_k ~~~
  f(g_1(x^{\varphi}), g_2(x^{\varphi}), \ldots, g_m(x^{\varphi})) \to f(b) ~~~
  k = 1,2, \ldots, m
  $$
  $$
  \Rightarrow F(x^{(\varphi)}) \to f(b) = F(a) ~~~ \text{непрерывна в } ~ a
  $$
\end{proof}

\begin{define}[непрерывности по множеству]
  $f(x)$ определена на $E$ и $a$ предельная точка $E$ тогда непрерывна в точке
  $a$ по множеству $E$ если
  $$
  \lim_{\substack{x \to a \\ x \in E}} f(x) = f(a)
  $$
\end{define}

\begin{title}[\Large]
  Свойства ФМП непрерывных на множествах (теоремы Вейерштрасса, Коши, Кантора)
\end{title}

\begin{define}[непрерывности ФМП на множестве]
  $f(x)$ называется непрерывной на $E$ если она непрерывна в каждой точке
  множества $E$ то eсть
  $$
  \lim_{\substack{x \to a \\ a \in E}} f(x) = f(a)
  $$
\end{define}

\begin{theorem}[Вейерштрасса ФМП]
  $f(x)$ непрерывна на компактном множестве $K$ в метрическом пространстве
  $M$ тогда она ограничена на множестве $K$ то есть
  $$
  \exists L > 0 ~~~ \forall x \in K ~~~ |f(x)| \le L
  $$
\end{theorem}

\begin{proof}
  Доказательство от обратного
  $$
  \forall L > 0 ~~~ \exists x_L \in K ~~~ |f(x_L)| > L
  $$
  $$
  \forall n \in N ~~~ \exists x_n \in K ~~~ |f(x_n)| > n
  $$
  так как $K$ компактное множество, то $x_n$ $x_{k_m} \to a \in K$ тогда
  $|f(x_{n_m})| > n_m ~ \Rightarrow ~ n_m \to \infty ~ |f(a)| \ge +\infty$
  противоречие
\end{proof}

\begin{theorem}
  $f(x)$ непрерывно на компактном множестве $K$ тогда в некоторых точках она
  достигается своего наибольшено и наименьшего значения то есть
  $$
  \exists a \in K ~~~ f(a) = \inf \limits_{x \in K} f(x)
  $$
  $$
  \exists d \in K ~~~ f(d) = \sup \limits_{x \in K} f(x)
  $$
\end{theorem}

\begin{proof}
  $$
  \inf \limits_{x \in K} f(x) = \alpha
  $$
  1) $\forall x \in K ~~~ \alpha \le f(x)$

  2) $\forall \varepsilon > 0 ~~~ \exists x_{\varepsilon} \in K ~~~
  f(x_{\varepsilon}) < \alpha + \varepsilon$
  $$
  \varepsilon = \frac{1}{k} ~~~ k \in N ~~~ \exists x_k \in K ~~~
  \alpha \le f(x_k) < \alpha + \frac{1}{k}
  $$
  так как $K$ компактное множество тогда
  $$
  \exists x_k \in K ~~~ x_{k_{\varphi}} \to a \in K ~~~ \alpha \le
  f(x_{k_{\varphi}}) < \alpha + \frac{1}{k_{\varphi}}
  $$
  $$
  \alpha \le f(a) \le \alpha ~ \Rightarrow ~ f(a) = \alpha = \inf f(x)
  $$
  $$
  \sup \limits_{x \in K} f(x) = \alpha
  $$
  1) $\forall x \in K ~~~ \alpha \ge f(x)$

  2) $\forall \varepsilon > 0 ~~~ \exists x_{\varepsilon} \in K ~~~
  f(x_{\varepsilon}) > \alpha - \varepsilon$
  $$
  \varepsilon = \frac{1}{k} ~~~ k \in N ~~~ \exists x_k \in K ~~~
  \alpha \ge f(x_k) > \alpha - \frac{1}{k}
  $$
  так как $K$ компактное множество тогда
  $$
  \exists x_k \in K ~~~ x_{k_{\varphi}} \to a \in K ~~~ \alpha \ge
  f(x_{k_{\varphi}}) > \alpha - \frac{1}{k_{\varphi}}
  $$
  $$
  \alpha \ge f(a) \ge \alpha ~ \Rightarrow ~ f(a) = \alpha = \sup f(x)
  $$
\end{proof}

\begin{theorem}[Кантора]
  $f(x)$ непрерывная на компактном множестве $K$ является равномерно
  непрерывной на множестве $K$.
\end{theorem}

\begin{proof}
  $f(x)$ является равномерно непрерывной на множестве $K$ если
  $$
  \forall \varepsilon > 0 ~~~ \exists \delta_{\varepsilon} > 0 ~~~
  \forall x', x'' \in K ~~~ \rho(x', x'') < \delta_{\varepsilon} ~~~
  |f(x') - f(x'')| < \varepsilon
  $$
  Допустим это не так
  $$
  \exists \varepsilon_0 > 0 ~~~ \forall \delta > 0 ~~~ \exists x_{\delta}',
  x_{\delta}'' \in K ~~~ \rho(x_{\delta}', x_{\delta}'') < \delta ~~~
  |f(x_{\delta}') - f(x_{\delta}'')| \ge \varepsilon_0
  $$
  $$
  \delta = \frac{1}{k} ~~~ k \in N ~~~ x_k', x_k'' \in K ~~~
  \rho(x_k', x_k'') < \frac{1}{k} ~~~ |f(x_k') - f(x_k'')| \ge \varepsilon_0
  $$
  $$
  x_k' \in K ~~~ x_{k_{\varphi}}' \to a \in K ~~~ x_{k_{\varphi}}'' \to a
  $$
  $$
  |f(x_{k_{\varphi}}') - f(x_{k_{\varphi}}'')| \ge \varepsilon_0
  $$
  $$
  |f(a) - f(a)| \ge \varepsilon_0 > 0
  $$
\end{proof}

\begin{define}[области]
  Областью называют открытое и связанное множество в $R^n$
\end{define}

\begin{theorem}[Коши для открытого множества]
  $f(x)$ непрерывна в области $D \subset R^n ~~~ f(a) = p ~~~ f(b) = q$
  где $a,b \in D$ тогда $\forall s$ между $p$ и $q$
  $\exists c \in D ~~~ f(c) = s$
\end{theorem}

\begin{proof}
  $\gamma \subset D ~~~ x = x(t) = (x_1(t), x_2(t), \ldots, x_n(t)) ~~~
  t \in [\alpha, \beta]$

  Рассмотрим $F(t) = f(x(t))$ которая также непрерывна по теореме Коши о
  функция непрерывных на отрезках $\forall s$ между $p$ и $q$

  $\exists G \in [\alpha, \beta] ~~~ F(G) = s$

  $F(\alpha) = f(a) = p ~~~ x(G) = s$

  $F(\beta) = f(b) = q ~~~ f(c) = F(G) = s$
\end{proof}

\begin{title}[\Large]
  Дифференциируемость ФМП. Необходимое условие дифференцируемости ФМП
\end{title}

\begin{define}[дифференциала в точке]
  $f(x) = f(x_1, \ldots, x_n)$ определена на $O(a_1, \ldots, a_n)$
  $$
  f_{\Delta}(a) = f(x) - f(a) = \sum_{k=1}^n p_k (x_k - a_k) +
  \stackrel{-}{o}(\rho(x, a))
  $$
  $\stackrel{-}{o}(\rho(x, a)) = \alpha(x) \rho(x, a)$ где $\alpha(x) \to 0 ~~
  x \to a$

  тогда $f(x)$ называют дифференцируемой в точке $a$
\end{define}

\begin{block}[Критерий]
  $f(x)$ определена $O(a)$ дифференцируема $\Leftrightarrow$
  $$
  f_{\Delta}(a) = f(x) - f(a) = \sum_{k=1}^n f_k(x)(x_k - a_k)
  $$
\end{block}

\begin{theorem}
  Если функция дифференцируема в точке $a$ тогда она непрерывна в точке $a$
\end{theorem}

\begin{define}[частной производной]
  $f(x) = f(x_1, x_2, \ldots, x_n)$ определена в $O(a_1, \ldots, a_n)$ если
  существует конечный предел
  $$
  \lim_{x_1 \to a_1}
  \frac{f(x_1, a_2, a_3, \ldots, a_n) - f(a_1, a_2, \ldots, a_n)}{x_1 - a_1}
  $$
  то его называют частной производной в точке $a$ и обозначают
  $$
  f_{x_1}'(a) = \frac{\partial f(a)}{\partial x_1}
  $$
\end{define}

\begin{theorem}
  $f(x)$ дифференциируема в точке $a$ тогда существует частная производная
  по $x_k$ в точке $a$ и $p_k = f_{x_k}'(a)$

  Замечание: дает необходимое, но не достаточное условие
\end{theorem}

\begin{proof}
  $$
  \lim_{x_1 \to a_1}
  \frac{f(x_1, a_2, a_3, \ldots, a_n) - f(a_1, a_2, \ldots, a_n)}{x_1 - a_1} =
  $$
  $$
  = \lim_{x_1 \to a_1}
  \frac{p_1(x_1 - a_1) + 0 + \alpha(x) \rho(x, a)}{x_1 - a_1} =
  $$
  $$
  = \lim_{x_1 \to a_1}
  \frac{p_1(x_1 - a_1) + \alpha(x) |x_1 - a_1|}{x_1 - a_1} =
  $$
  $$
  = p_1 \pm \lim_{x_1 \to a_1} \alpha(x) = p_1 \pm 0 = p_1
  $$
\end{proof}

\begin{title}[\Large]
  Достаточные условия дифференцируемости ФМП
\end{title}

\begin{theorem}
  $O(a) \subset R^n ~~ \exists$ все частные производные $f(x)$ которые
  непрерывны в точке $a$ тогда $f(x)$ дифференциируема в точке $a$
\end{theorem}

\begin{proof}
  $f(x,y,z) ~~ \exists (a,b,c) ~~ f'_x (x,y,z) ~~ f'_y (x,y,z) ~~ f'_z (x,y,z)$
  $$
  f_{\Delta}(a,b,c) = f(x,y,z) - f(a,b,c) =
  $$
  $$
  = (f(x,y,z) - f(a,y,z)) +
  (f(a,y,z) - f(a,b,z)) + (f(a,b,z) - f(a,b,c)) =
  $$
  $$
  = f'_x(a + \theta_1(x-a)), y, z)(x-a) + f'_y(a,b + \theta_2(y-b), z)(y-b) +
  f'_z(a,b,c + \theta_3(z-c))(z-c) =
  $$
  $$
  = f_1(x,y,z)(x-a) + f_2(x,y,z)(y-b) + f_3(x,y,z)(z-c)
  $$
  $$
  0 < \theta_1 < 1 ~~ 0 < \theta_2 < 1 ~~ 0 < \theta_3 < 1 ~~ y_{\Delta}
  = f'(c)x_{\Delta}
  $$
\end{proof}

\begin{title}[\Large]
  Дифференцирование сложных функций многих переменных
\end{title}

\begin{theorem}
  $F(x) = f(t_1, t_2, \ldots, t_m)$ дифференцируема в точке
  $b = (b_1, b_2, \ldots, b_m)$ и $t_k = g_k(x) ~~ k = 1,2, \ldots m$
  определена в $O(a)$ $a = (a_1, a_2, \ldots a_n)$ и дифференцируемы в $a$ и
  $b_k = g_k(a)$ тогда $F(x) = f(g_1(x), g_2(x), \ldots, g_m(x))$
  дифференцируема в точке $a$ и
  $$
  F'_{x_i}(a) = \sum_{k=1}^m \frac{\partial f(b)}{\partial t_k}
  \frac{\partial g_k(a)}{\partial x_i}
  $$
\end{theorem}

\begin{proof}
  $f(t) - f(b) = \sum_{k=1}^m f_k(t)(t_k - b_k)$ где $f_k(t)$ определена $O(a)$
  и непрерывная в точке $b$

  $f_k(b) = f'_{t_k}(b)$

  $g_k(x) - g_k(a) =
  \sum_{i=1}^n g_{k_i}(x)(x_i - a_i)$ где $g_{k_i}(x)$ определена $O(a)$ и
  непрерывна в точке $b$

  $g_{k_i}(a) = \frac{\partial g_k(a)}{\partial x_i}$ тогда
  $$
  F(x) - F(a) = f(g_1(x), g_2(x), \ldots, g_m(x)) - f(g_1(a), g_2(a),
  \ldots, g_m(a)) =
  $$
  $$
  = \sum_{k=1}^m f_k(g_1(x), g_2(x), \ldots, g_m(x)) (g_k(x) - g_k(a)) =
  $$
  $$
  = \sum_{k=1}^m f_k(g_1(x), g_2(x), \ldots, g_m(x)) \sum_{i=1}^n g_{k_i}(x)
  (x_i - a_i)=
  $$
  $$
  = \sum_{i=1}^n \overbrace{\left( \sum_{k=1}^m f_k(g_1(x), g_2(x), \ldots,
  g_m(x)) g_{k_i}(x) \right)}^{\text{непр по критерию} ~ F(x) ~
  \text{дифф в точке} ~ a} (x_i - a_i)
  $$
  $$
  F'_{x_i}(a) = \sum_{k=1}^m f_k(b) g_k(a) = \sum_{k=1}^m
  \frac{\partial f(b)}{\partial t_k} \frac{\partial g_k(a)}{\partial x_i}
  $$
\end{proof}

\begin{title}[\Large]
  Формула конечных приращений для ФМП. Производные по направлению. Градиент
\end{title}

\begin{theorem}
  $f(x)$ дифференциируема на выпуклой области $D \subset R^n$ тогда

  $\forall a,b \in D ~~ \exists 0 < \theta < 1$
  $$
  f(b) - f(a) = \sum_{k=1}^n f'_{x_k} (a + \theta(b-a))(b_k-a_k)
  $$
\end{theorem}

\begin{proof}
  $[a,b] = \{x = (x_1, x_2, \ldots, x_n): ~ x_k = a_k + t(b_k - a_k) ~
  t \in [0,1]\}$

  $\varphi(t) = f(a_1 + t(b_1 - a_1), a_2 + t(b_2 - a_2), \ldots,
  a_n + t(b_n-a_n))$

  $\varphi(1) - \varphi(0) = \varphi(\theta) (1 - 0)$
  $$
  f(b) - f(a) = \sum_{k=1}^n f'_{x_k}(a_1 + \theta(b_1 - a_1)),
  a_2 + \theta(b_2 - a_2), \ldots, a_n + \theta(b_k - a_k) (b_k - a_k) =
  $$
  $$
  = \sum_{k=1}^n f'_{x_k}(a + \theta(b-a))(b_k - a_k)
  $$
\end{proof}

\begin{define}[прозводной по направлению]
  $f(x)$ определена в $O(a) \in R^n$ по напрвлению
  $\vec e = (e_1, e_2, \ldots e_n)$ тогда производной по напрвлению в точке
  $f(a)$ называется
  $$
  \lim_{t \to 0 ~ +0} \frac{f(a_1 + te_1, a_2 + te_2, \ldots, a_n + te_n) -
  f(a_1, a_2, \ldots, a_n)}{t} = \frac{\partial f(a)}{\partial \vec e}
  $$
\end{define}

\begin{define}[градиента]
  Градиентом называют вектор составленный из частных производных то есть
  $$
  (f'_{x_1}(a), f'_{x_2}(a), \ldots, f'_{x_n}(a)) = \grad f(a)
  $$
  $\grad f(a)$ показывает направление наибольшего изменения функции.

  $|\grad f(a)|$ показывает скорость изменения.
\end{define}

\begin{theorem}
  $f(x)$ дифференциируема в точке $a \in R^n$ тогда
  $$
  \frac{\partial f(a)}{\partial \vec e} = \sum_{k=1}^n e_k f'_{x_k}(a) =
  (\grad f(a), \vec e) ~ \text{скалярное направление}
  $$
\end{theorem}

\begin{title}[\Large]
  Дифференциал ФМП. Свойство инвариантности формы первого дифференциала.
  Геометрический смысл дифференциала функций двух переменных
\end{title}

\begin{define}[дифференциала фмп]
  $f(x)$ дифференцируема в точке $a \in R^n$ то есть
  $$
  f(x) - f(a) = \sum_{k=1}^n f'_{x_k}(a) (x_k - a_k) + \alpha(x)\rho(x,z)
  $$
  $x_k - a_k = dx_k ~~ df(a) = \sum_{k=1}^n f'_{x_k}(a)dx_k$ дифференциал
  первого порядка
\end{define}

\begin{block}[Свойства]
  $f(x), g(x)$ дифференциируемы в точке $x$ тогда

  1)
  $$
  d(f(x) \pm g(x)) = df(x) \pm dg(x)
  $$
  2)
  $$
  d(f(x) \cdot g(x)) = g(x)df(x) + f(x)dg(x)
  $$
  \begin{proof}
    $$
    d(f(x) \cdot g(x)) = \sum_{k=1}^n (f(x) g(x))'_{x_k} dx_k =
    \sum_{k=1}^n (f'_{x_k}(x) g(x) + f(x) g'_{x_k}(x))dx_k =
    $$
    $$
    = g(x) \sum_{k=1}^n f'_{x_k}(x)dx_k + f(x) \sum_{k=1}^n g'_{x_k}(x)dx_k =
    g(x)df(x) + f(x)dg(x)
    $$
  \end{proof}

  3)
  $$
  d\left(\frac{f(x)}{g(x)}\right) = \frac{g(x)df(x) - f(x)dg(x)}{g^2(x)} ~~~
  g(x) \not= 0
  $$
  \begin{proof}
    $$
    d\left(\frac{f(x)}{g(x)}\right) =
    \sum_{k=1}^n \left(\frac{f(x)}{g(x)}\right)'_{x_k} dx_k =
    \sum_{k=1}^n \left(\frac{f'_{x_k}(x) g(x)- f(x)g'_{x_k}(x)}{g^2(x)}\right)
    dx_k =
    $$
    $$
    = \frac{g(x)\sum_{k=1}^n f'_{x_k}(x) dx_k - f(x)\sum_{k=1}^n
    g'_{x_k}(x) dx_k}{g^2(x)}
    =
    $$
    $$
    =\frac{g(x)df(x) - f(x)dg(x)}{g^2(x)}
    $$
  \end{proof}
\end{block}

\begin{theorem}
  Первый дифференциал обладает свойством инвариантности формы то есть
  форма дифференциала остается неизменной, даже если $x_k$ зависит от
  чего-то другого
  $$
  df(a) = \sum_{k=1}^n f'_{x_k}(a) dx_k
  $$
\end{theorem}

\begin{proof}
  Пусть $x_k = g_k(t) ~~ t \in R^m$

  $F(t) = f(g_1(t), g_2(t), \ldots, g_n(t))$

  $$
  dF(t) = \sum_{i=1}^m F'_{t_i}dt_i = \sum_{i=1}^m \sum_{k=1}^n
  f'_{x_k}(x) (g_k(t))'_{t_i}dt_i =
  $$
  $$
  = \sum_{k=1}^n f'_{x_k}(x) \sum_{i=1}^m (g_k(t))'_{t_i} dt_i =
  \sum_{k=1}^n f'_{x_k}(x)dg_k(t) = \sum_{k=1}^n f'_{x_k}(x) dx_k
  $$
\end{proof}

\begin{block}[Геометрический смысл]
  $z = f(x, y)$ существует дифференциал в $(x_0, y_0)$ и $f(x_0, y_0) = z_0$

  $\rho = \grad f(x, y)$

  возьмем произвольную кривую данной поверхности
  $$
  K = \left\{
  \begin{array}{l}
    x = x(t) \\
    y = y(t) \\
    z = z(t)
  \end{array}
  \right. ~~~ t \in [a,b]
  $$
  все точки кривой связаны уравнением $z(t) = f(x(t), y(t))$
  $$
  dz(t_0) = f'_x(x(t_0), y(t_0)) dx(t_0) + f'_y(x(t_0), y(t_0))dy(t_0)
  $$
  $$
  dz_0 = f'_x (x_0, y_0)dx_0 + f'_y(x_0, y_0)dy_0
  $$
  $(dx_0, dy_0, dz_0)$ касательный вектор кривой $K$
  $$
  dz(t_0) - f'_x(x(t_0), y(t_0)) dx(t_0) + f'_y(x(t_0), y(t_0))dy(t_0) = 0
  $$

  $(-f'_x(x_0, y_0), -f'_y(x_0, y_0), 1) = 0$ ортогональность вектора,
  ортогонален всем кривым то есть это вектор нормали.

  Плоскость проходящая через точку $M$ перпендикулярная нормали это касательная
  плоскость поверхности $\rho$

  $z - f(x_0, y_0) = f'_x(x_0, y_0) + f'_y(x_0, y_0)(y - y_0)$ уравнение
  касательной плоскости.

  $df(x_0,y_0)$ приращение ампликаты касательной
  плоскости к графику функции в определнной точке.
\end{block}

\begin{title}[\Large]
  Производная и дифференциалы высших порядков ФМП
\end{title}

\begin{define}
  $u = f(x)$ дифференциируема в $a \in R^n ~ O(a)$ тогда $x \in O(a) ~~
  \exists f'_{x_k}(x)$ частные производные если
  $$
  \exists (f'_{x_k}(x))'_{x_i} = f''_{x_k x_i}(x) = \frac{\partial^2 f(x)}
  {\partial x_i \partial x_k}
  $$
  $$
  f''_{x_k x_i}(x), f''_{x_i x_k}(x) ~ \text{смешанный производные}
  $$
  $$
  f''_{x_k x_k}(x) = f''_{x_k^2}(x)
  $$
\end{define}

\begin{theorem}
  $\exists O(a,b) ~ \exists f''_{xy}(x,y) ~~ f''_{yx}(x,y)$ и
  непрерывны в точке $(a,b)$ тогда
  $$
  f''_{xy}(a,b) = f''_{yx}(a,b)
  $$
\end{theorem}

\begin{define}
  $f(x)$ дифференцируема в области $D \subset R$
  $$
  d(d(f(x))) = \sum_{k=1}^n dx_k d(f'_{x_k}(x)) = \sum_{k=1}^n dx_k
  \sum_{i=1}^n f''_{x_k x_i}(x) \delta x_i = \sum_{k=1}^n \sum_{i=1}^n
  f''_{x_k x_i} dx_k \delta x_i =
  $$
  $$
  (\delta x_i = d x_k)
  $$
  $$
  = \sum_{k=1}^n \sum_{i=1}^n f''_{x_k x_i} dx_k dx_i ~
  \text{квадратичная форма}
  $$
\end{define}

\begin{title}[\Large]
  Неявные функции. Теорема о существовании непрерывности и дифференцируемости
  неявных функций
\end{title}

\begin{define}
  $F(x, y) = 0$ на $R^n ~~~ \forall x \in P ~~~ \exists! y \in R ~~
  F(x, y) = 0$ тогда $y = f(x)$ задает неявную функцию.

  $\exists P = [a,b] \times [c,d] ~~ \forall x \in [a,b] ~~
  \exists! y \in [c,b] ~~ F(x,y) = 0$ тогда
  $F(x,y) = 0$ в $P$ определяет $y = f(x)$
\end{define}

\begin{theorem}
  1) $F(x, y) = 0$ имеет непрерывные частные производные в точке $(x_0, y_0)$

  2) $F(x_0, y_0) = 0$

  3) $F'_y(x_0, y_0) \not= 0$

  тогда $\exists$ прямоугольник с центром в точке $(x_0, y_0)$ в котором
  $y = y(x)$ задает неявную функцию $F(x,y)=0$ и $y = y(x)$
  непрерывна дифференциируема в точке $(x_0, y_0)$ и
  $$
  f'(x) = - \frac{F'_x(x, y)}{F'_y(x, y)} = -
  \frac{F'_x(x, f(x))}{F'_y(x, f(x))}
  $$
\end{theorem}

\begin{proof}
  1. Пусть для определенности $F'_y(x_0, y_0) > 0$

  существует квадрат внутри круга где производная положительна из свойств
  непрерывной функции

  1) $F(x_0, y)$ - кривая пересекает нашу плоскость (так как $F(x_0, y_0) = 0$)
  на отрезке $[y_0 - \tau_1, y_0 - \tau_1]$ $F_y(x_0, y) > 0$

  $F(x_0, y_0 - \tau_1) < 0$

  $F(x_0, y_0 + \tau_1) > 0$

  2) $F(x, y_0 - \tau_1) ~~~ \exists [x_0 - \tau_2, x_0 + \tau_2]$

  3) $F(x, y_0 + \tau_1) > 0 ~~~ [x_0 - \tau_3, x_0 + \tau_3]$

  $[x_0 - \tau, x_0 + \tau] \times [x_0 - \tau_1, x_0 + \tau_1]$

  $z = m \cdot n(\tau_2, \tau_3)$

  $\forall x^* \in [x_0 - \tau, x_0 + \tau]$

  $F(x^*, y_0 - \tau_1) < 0$

  $F(x^*, y_0 + \tau_2) > 0$

  так как $F(x^*, y) ~ \nearrow ~ F(x^*, y) ~ [y_0 - \tau_1, y_0 + \tau_1]$

  существует непрерывная заданная в неявном виде функция

  2. $F(x, y)$ достигает своего наибольшего значения

  $F'_y(x, y) \ge \alpha = \inf F'_y (x, y) ~ (x, y) \in P$

  $\alpha > 0 ~ \Rightarrow ~ |F'_x (x, y)| \le M$

  Расмотрим $(x, y)$ и $(x + x_{\Delta}, y + y_{\Delta})$ - лежат в $P$ и на
  кривой $F(x, y) = 0 ~~ F(x + x_{\Delta}, y + y_{\Delta}) = 0$

  $F(x + x_{\Delta}, y + y_{\Delta}) - F(x, y) = 0$

  $$
  F'_x (x + \theta x_{\Delta}, y + \theta y_{\Delta})x_{\Delta} +
  F'_y(x + \theta x_{\Delta}, y + \theta y_{\Delta})y_{\Delta} = 0
  $$
  $$
  - \frac{F'_x (x + \theta x_{\Delta}, y + \theta y_{\Delta})}
  {F'_y(x + \theta x_{\Delta}, y + \theta y_{\Delta})x_{\Delta}} = y_{\Delta}
  $$

  $|y_{\Delta}| \le \frac{M}{\alpha} |x_{\Delta}|$ - оценка
  $x_{\Delta} \to 0 ~ y_{\Delta} \to 0$
  $$
  - \frac{F'_x (x + \theta x_{\Delta}, y + \theta y_{\Delta})}
  {F'_y(x + \theta x_{\Delta}, y + \theta y_{\Delta})} =
  \frac{y_{\Delta}}{x_{\Delta}}
  $$
  $- \frac{F'_x (x, y)}{F'_y(x, y)} = f'(x)$
\end{proof}

\begin{define}
  $$
  * =
  \left\{
  \begin{array}{l}
    F_1(x_1, x_2, \ldots, x_n, y_1, y_2, \ldots, y_m) = 0 \\
    F_2(x_1, x_2, \ldots, x_n, y_1, y_2, \ldots, y_m) = 0 \\
    \ldots ~~~~ \ldots ~~~~ \ldots ~~~~ \ldots ~~~~ \ldots ~~~~ \ldots\\
    F_m(x_1, x_2, \ldots, x_n, y_1, y_2, \ldots, y_m) = 0 \\
  \end{array}
  \right.
  $$
  $F_i (x, y) ~ i = 1,2, \ldots, m$ определена на некотором параллелепипеде

  $A \subset R^n =
  [x_{01} - \alpha_1, x_{01} + \alpha_1] \times
  [x_{02} - \alpha_2, x_{02} + \alpha_2] \times \ldots \times
  [x_{0n} - \alpha_n, x_{0n} + \alpha_n]$

  $B \subset R^m =
  [y_{01} - \beta_1, y_{01} + \beta_1] \times
  [y_{02} - \beta_2, y_{02} + \beta_2] \times \ldots \times
  [y_{0m} - \beta_m, y_{0m} + \beta_m]$

  $A \times B \subset R^{n+m}$

  $x = (x_1, x_2, \ldots, x_n)$

  $y = (y_1, y_2, \ldots, y_m)$

  $x_0 = (x_{01}, x_{02}, \ldots, x_{0n})$

  $y_0 = (y_{01}, y_{02}, \ldots, y_{0m})$

  Если $\forall x \in A ~~ \exists! y \in B$ для которого справедливы все
  уравнения системы $*$ тогда в $A \times B$ система $*$ задает неявную функцию

  $y_i = f_i(x) = f_i(x_1, x_2, \ldots, x_n) ~~ i = 1, 2, \ldots, n$
\end{define}

\begin{theorem}
  1) $F_i(x, y) ~~ i = 1,2, \ldots, m$ непрерывна дифференциируема в
  $(x_0, y_0)$ то есть имеют непрерывные частные производные

  2) $F_i(x_0, y_0) = 0 ~~ i = 1,2, \ldots, m$

  3)
  $$
  \left|
  \begin{array}{cccc}
    \frac{\partial F_1(x_0, y_0)}{\partial y_1} &
    \frac{\partial F_1(x_0, y_0)}{\partial y_2} &
    \cdots &
    \frac{\partial F_1(x_0, y_0)}{\partial y_m} \\

    \frac{\partial F_2(x_0, y_0)}{\partial y_1} &
    \frac{\partial F_2(x_0, y_0)}{\partial y_2} &
    \cdots &
    \frac{\partial F_2(x_0, y_0)}{\partial y_m} \\

    \cdots & \cdots &\cdots &\cdots \\

    \frac{\partial F_m(x_0, y_0)}{\partial y_1} &
    \frac{\partial F_m(x_0, y_0)}{\partial y_2} &
    \cdots &
    \frac{\partial F_m(x_0, y_0)}{\partial y_m}
  \end{array}
  \right| \not= 0
  $$
  тогда $\exists A \times B$ с центром в точке $(x_0, y_0)$
  в котором система $*$ задает
  $y_i = f_i(x) = f_i(x_1, x_2, \ldots, x_n)$ которые имеют непрерывные частные
  производные.
\end{theorem}

\begin{title}[\Large]
  Формула Тейлора для ФМП с остаточным членом в форме Лагранжа и в форме Пеано
\end{title}

\begin{theorem}
  $y = f(x)$ определена $O(a) \subset R^n$ и имеет в этой окрестности частные
  производные до $m$ порядка включительно тогда
  $$
  \forall x \in O(a) ~~ \exists \theta \in (0, 1) ~~
  f(x) = f(a) + \sum_{k=1}^{m-1} \frac{d^k f(a)}{k!} + R_m(x)
  $$
  $$
  R_m(x) = \frac{d^m f(a + \theta(x-a))}{m!}
  $$
\end{theorem}

\begin{proof}
  $[a - x_{\Delta}, a + x_{\Delta}] = \{ a + tx_{\Delta}: ~ t \in [-1, 1] \}$

  $\varphi(t) = f(a + tx_{\Delta}) ~~ \varphi(0) = f(a)$
  $$
  \varphi'(t) = \sum_{k=1}^n f'_{x_k}(a + tx_{\Delta})x_{\Delta k} =
  df(a + tx_{\Delta})
  $$
  $$
  \varphi''(t) = \sum_{k=1}^n \sum_{l=1}^n f''_{x_k x_l}
  (a + tx_{\Delta})dx_k dx_l = d^2f(a + tx_{\Delta})
  $$
  $$
  \cdots ~~~~ \cdots ~~~~\cdots ~~~~\cdots ~~~ \cdots ~~~~ \cdots ~~~~
  \cdots ~~~~\cdots
  $$
  $$
  \varphi^m(t) = d^m f(a + tx_{\Delta})
  $$
  $$
  \varphi(1) = \varphi(0) + \sum_{k=1}^{m-1} \frac{\varphi^{(k)}(0)}{k!} +
  \frac{\varphi^{(m)}(\theta)}{m!}
  $$
  $$
  f(x) = f(a) + \sum_{k=1}^{m-1} \frac{d^k f(a)}{k!} +
  \frac{d^m f(a + \theta x_{\Delta})}{m!}
  $$
\end{proof}

\begin{block}[Формула Тейлора с остаточным членом в форме Пеано]
  Если выполнены все условия теоремы выше то
  $$
  f(x) = f(a) + \sum_{k=1}^m \frac{d^k f(a)}{k!} + \overline{o}(|x_{\Delta}|^m)
  ~~~ x_{\Delta} \to 0 ~~~ |x_{\Delta}| = \sqrt{\sum_{k=1}^n (x_{\Delta k})^2}
  $$
  формула Тейлора с остаточным членом в форме Пеано
\end{block}

\begin{title}[\Large]
  Экстремум ФМП. Необходимые условия экстремума ФМП
\end{title}

\begin{define}[$\max ~ \min$ ФМП]
  $f(x)$ определена $O(a) \subset R^n$

  $\forall x \in O(a) ~~~ f(x) \le f(a)$ точка $a$ $\max$

  $\forall x \in \stackrel{\bullet}{O}(a) ~~~ f(x) < f(a)$ точка $a$
  строгова $\max$

  $\forall x \in O(a) ~~~ f(x) \ge f(a)$ точка $a$ $\min$

  $\forall x \in \stackrel{\bullet}{O}(a) ~~~ f(x) > f(a)$ точка $a$
  строгова $\min$
\end{define}

\begin{theorem}
  $f(x)$ имеет в точке $a$ экстремум и $\exists f'_{x_k}(a)$ тогда
  $f'_{x_k}(a) = 0$ (теорема Ферма)
\end{theorem}

\begin{block}[Следствие]
  $f(x)$ имеет в точке $a$ экстремум и $\exists df(a)$ тогда $df(a) = 0$
  стационарная точка
\end{block}

\begin{block}[Лемма]
  Необходимые условия минимума для $f(x)$ одной переменной

  Если $\varphi(t)$ в точке $a=0$ имеет минимум и $\exists \varphi''(0)$
  тогда $\varphi''(0) \ge 0$
\end{block}

\begin{proof}
  $$
  \varphi(t) = \varphi(0) + \frac{\varphi'(0)}{1!}t +
  \frac{\varphi''(0)}{2!}t^2 + \alpha(t)t^2 ~~~ \alpha(t) \to 0 ~~ t \to 0
  $$
  $$
  \varphi(t) - \varphi(0) = \frac{\varphi''(0)}{2!}t^2 + \alpha(t)t^2
  $$
  $$
  0 \le \frac{\varphi(t) - \varphi(0)}{t^2} = \frac{\varphi''(0)}{2!} +
  \alpha(t) ~ \Rightarrow ~ \frac{\varphi''(0)}{2!} \ge 0
  $$
\end{proof}

\begin{theorem}
  $f(x)$ в точке минимума $a$ и $\exists d^2f(a)$ тогда $d^2f(a) \ge 0$
\end{theorem}

\begin{proof}
  $\varphi(t) = f(a + tx_{\Delta}) ~~~ x_{\Delta} = x - a ~~~
  t \in [-\delta, \delta]$ по первой теореме в этом разделе
\end{proof}

\begin{title}[\Large]
  Достаточные условия экстремума ФМП
\end{title}

\begin{define}[квадратичной формы]
  $$
  \Phi(t) = \sum_{k=1}^n \sum_{l=1}^n a_{kl}t_k t_l ~
  \text{квадратичная форма}
  $$
  $$
  A = \left(
  \begin{array}{cccc}
    a_{11} & a_{12} & \ldots & a_{1n} \\
    a_{12} & a_{22} & \ldots & a_{2n} \\
    a_{13} & a_{23} & \ldots & a_{3n} \\
    \ldots & \ldots & \ldots & \ldots \\
    a_{1n} & a_{2n} & \ldots & a_{nn} \\
  \end{array}
  \right)
  $$
  $a_{kl} = a_{lk} ~~~ t = (t_1, t_2, \ldots, t_n) ~~~
  l = (l_1, l_2, \ldots, l_n)$

  $\forall t \not= 0 ~~~ \Phi(t) > 0$ полижтельная определенная квадратичкая
  форма

  $\forall t \not= 0 ~~~ \Phi(t) < 0$ отрицательная определенная квадратичкая
  форма

  $\exists t', t'' \in R^n ~~~ \Phi(t') > 0 ~~~ \Phi(t'') < 0$ неопределенная
  квадратичная форма
\end{define}

\begin{theorem}
  $\Phi(t)$ положительно определенная квадтратичная форма тогда
  $$
  \exists s > 0 ~~~ \forall t \in R^n ~~~ \Phi(t) \ge s|t|^2 ~~~
  |t| = \sqrt{\sum_{k=1}^n t_k^2}
  $$
\end{theorem}

\begin{proof}
  $$
  S_r(a) = \{x ~ : ~ \rho(x, a) \le r\}
  $$
  $\forall t \not= 0$ рассмотрим $\Phi \left( \frac{t}{|t|} \right) ~~~
  \frac{t}{|t|} \in S_1(0)$

  $\Phi \left( \frac{t}{|t|} \right) \ge \Phi \left( \frac{t_0}{|t_0|} \right) =
  \inf \limits_{t \in S_1(0)} \Phi(t) > 0$

  $\Phi \left( \frac{t}{|t|} \right) \ge s > 0$

  $\frac{1}{|t|^2} \Phi(t) \ge s$

  $\Phi(t) \ge s|t|^2$
\end{proof}

\begin{theorem}[достаточное условие $\exists$ экстремума]
  $f(x)$ имеет непрерывные частные производные 2-ого порядка в $a \in R^n$ и
  $df(a) = 0$

  1) Если $d^2 f(a) = \sum_{k=1}^n \sum_{l=1}^n f''_{x_k x_l} dx_k dx_l$
  положительная определенная квадратичная форма относительно
  дифференциалов независимых переменных тогда $a$ точка строгово $\min$.

  2) Если $d^2 f(a) = \sum_{k=1}^n \sum_{l=1}^n f''_{x_k x_l} dx_k dx_l$
  отрицательная определенная квадратичная форма относительно
  дифференциалов независимых переменных тогда $a$ точка строгово $\max$.

  3) Если $d^2 f(a) = \sum_{k=1}^n \sum_{l=1}^n f''_{x_k x_l} dx_k dx_l$
  неопределенная квадратичная формой относительно
  дифференциалов независимых переменных тогда $a$ не точка экстремума.
\end{theorem}

\begin{proof}
  $$
  f(x) - f(a) = \frac{1}{2}d^2f(a) + \alpha(x) \rho^2(x, a) ~~~
  \alpha(x) \to 0 ~~~ x \to a
  $$
  1) Пусть $d^2f(a)$ является положительной определенной квадратичной формой
  тогда на основании первой теоремы в этом разделе
  $$
  f(x) - f(a) = \frac{1}{2} d^2f(a) + \alpha(x) \rho^2(x, a) \ge \frac{1}{2}
  S|x_{\Delta}|^2 + \alpha(x)|x_{\Delta}|^2 =
  $$
  $$
  = |x_{\Delta}|^2 \left( \frac{S}{2} + \alpha(x) \right) \le |x_{\Delta}|^2
  \frac{S}{4} > 0 ~~~~ |\alpha(x)| \le \frac{S}{4}
  $$
  $\forall x_{\Delta} \not= 0 ~~~ f_{\Delta}(a) > 0$ $a$ точка минимума

  2) Если $d^2f(a)$ является отрицательной определенной квадратичной формой то
  заменяем $f(x)$ на $-f(x)$ а дальше принцип чайника

  3) Если $d^2 f(a)$ знакоопределенная квадратичная форма тогда от обратного
\end{proof}

\begin{block}[Критерий Сильвестра]
  1) $\Phi(t) = \sum_{k=l}^n \sum_{l=1}^n a_{kl} t_k t_l$ для того чтобы
  квадратичная форма была положительно определенной необходимо и достаточно
  чтобы главные миноры матрица $A$ были положительны.

  2) Для того чтобы квадратичная форма была отрицательно определенной
  необходимо и достаточно чтобы главные миноры матрица $A$ были
  знакочередующимися $(-1)^k \Delta_k$.
\end{block}

\begin{block}[Критерий знакоопределенности квадратичной формы]
  1) Для того чтобы квадратичная форма была положительно определенной
  необходимо и достаточно чтобы все ее собственные значения были положительны.

  2) Для того чтобы квадратичная форма была положительно определенной
  необходимо и достаточно чтобы все ее собственные значения были отрицательны.
\end{block}

\begin{title}[\Large]
  Условия экстремума ФМП. Прямой метод отыскания точек условного экстремума
\end{title}

\begin{define}[условного экстремума ФМП]
  $u = f(x) ~~~ x \in D \subset R^n$
  $$
  (*) =
  \left\{
  \begin{array}{c}
    \varphi_1(x) = 0 \\
    \varphi_2(x) = 0 \\
    \cdots ~~~ \cdots \\
    \varphi_m(x) = 0 \\
  \end{array}
  \right. ~~ \text{уравнение связи}
  $$
  $\varphi_m(x) \in D ~~~ m < n$

  $E$ множество точек $D$ удовлитворяющих всем уравнениям системы $*$.

  $a \in D \cap E$ называется точкой условного минимума.

  Если $\exists O(a) ~~ \forall x \in O(a) \cap E ~~~ f(x) \ge f(a)$
  тогда точка $a$ условного $\min$.

  Если $\exists \stackrel{\bullet}{O}(a) ~~ \forall x \in O(a) \cap E
  ~~~ f(x) > f(a)$ тогда точка $a$ строгово условного $\min$.

  Если $\exists O(a) ~~ \forall x \in O(a) \cap E ~~~ f(x) \le f(a)$
  тогда точка $a$ условного $\max$.

  Если $\exists \stackrel{\bullet}{O}(a) ~~ \forall x \in O(a) \cap E
  ~~~ f(x) < f(a)$ тогда точка $a$ строгово условного $\max$.
\end{define}

\begin{title}[\Large]
  Методы множетелей Лагранжа. Необходимые и достаточные условия условного
  экстремума
\end{title}

\begin{define}
  $x = (x_1, x_2, \ldots, x_n) ~~~ \lambda = (\lambda_1, \lambda_2, \ldots,
  \lambda_m)$
  $$
  L(x, \lambda) = f(x) + \sum_{i=1}^m \lambda_i \varphi_i(x) ~ \text{
  функция Лагранжа}
  $$
  $(a, \lambda^*)$ называется стационарной точкой функции Лагранжа если
  $$
  \frac{\partial (a, \lambda^*)}{\partial x_k} = 0 ~~~ k = 1,2, \ldots, n ~~~
  \frac{\partial (a, \lambda^*)}{\partial \lambda_i} = 0 ~~~ i = 1,2, \ldots, m
  $$

  $\lambda_i^*$ множетели Лагранжа
\end{define}

\begin{theorem}[Лагранжа необходимое условие условного экстремума]
  1) $a$ точка условного экстремума $f$ при наличии связи системы $*$

  2) $f, \varphi_i$ непрерывно дифференцируема в $O(a)$

  3) Матрица Якоби имеет имеет ранг $m$
  $$
  A(a) =
  \left(
  \begin{array}{cccc}
    \frac{\partial \varphi_1(a)}{\partial x_1} &
    \frac{\partial \varphi_1(a)}{\partial x_2} &
    \cdots &
    \frac{\partial \varphi_1(a)}{\partial x_n} \\

    \frac{\partial \varphi_2(a)}{\partial x_1} &
    \frac{\partial \varphi_2(a)}{\partial x_2} &
    \cdots &
    \frac{\partial \varphi_2(a)}{\partial x_n} \\
    \cdots & \cdots & \cdots & \cdots \\

    \frac{\partial \varphi_m(a)}{\partial x_1} &
    \frac{\partial \varphi_m(a)}{\partial x_2} &
    \cdots &
    \frac{\partial \varphi_m(a)}{\partial x_n} \\
  \end{array}
  \right) ~~ \text{матрица Якоби} ~ \rang A(a) = m
  $$
  тогда $\exists \lambda^*_1, \ldots, \lambda^*_m ~~ (a, \lambda^*)$
  стационарная точка для функции Лагранжа.
\end{theorem}

\begin{proof}
  Из условия 3 следует что
  $$
  \begin{array}{c}
    x_1 = g_1(x_{m+1}, x_{m+2}, \ldots, x_n) \\
    x_2 = g_2(x_{m+1}, x_{m+2}, \ldots, x_n) \\
    \cdots ~~~ \cdots ~~~ \cdots ~~~ \cdots \\
    x_m = g_m(x_{m+1}, x_{m+2}, \ldots, x_n)
  \end{array}
  $$
  $F(x_{m+1}, x_{m+2}, \ldots, x_n) = f(g_1, g_2, \ldots, g_m, x_{m+1},
  \ldots, x_n)$ используя инвариантность

  $\sum_{k=1}^n \frac{\partial f(a)}{\partial x_k} dx_k = 0 ~~~
  \sum_{k=1}^n \frac{\partial \varphi_i(a)}{\partial x_k} dx_k = 0 ~~~
  i = 1, 2, \ldots, m$ если функция равна нулю то и ее дифференциал равен нулю
  $$
  \sum_{k=1}^n \frac{\partial f_i(a)}{\partial x_k} dx_k = 0
  $$
  $$
  \sum_{k=1}^n \frac{\partial \varphi_i(a)}{\partial x_k} dx_k = 0
  $$
  $$
  \sum_{k=1}^n \lambda_i \frac{\partial \varphi_i(a)}{\partial x_k} dx_k = 0
  $$
  $$
  \sum_{k=1}^n \overbrace{\left( \frac{\partial f(a)}{\partial x_k} +
  \sum_{i=1}^m \lambda_i \frac{\partial \varphi_i(a)}{\partial x_k}
  \right)}^{\frac{\partial L(a, \lambda)}{\partial x_k}}dx = 0
  $$
  $\lambda^* = (\lambda_1^*, \lambda_2^*, \ldots, \lambda_m^*)$ единственное
  решение из условия 3 системы
  $$
  \sum_{l=1}^m \lambda_i \frac{\partial \varphi_i(a)}{\partial x_k} dx_k = 0
  ~~~ k = 1,2,\ldots, m ~~~ i = 1,2,\ldots, m
  $$
  $$
  dL(a, \lambda^*) = \sum_{k = m+1}^n \cancelto{= 0}{\frac{\partial
  L(a, \lambda^*)}{\partial x_k}} dx_k + \sum_{l=i}^m \cancelto{= 0}{\frac{
  \partial L(a, \lambda^*)}{\partial x_i}}dx_i = 0
  $$
  все частные производные равны нулю значит точка $(a, \lambda^*)$ стационарна.
\end{proof}

\begin{define}
  $$
  E_T = \{t = (t_1, t_2, \ldots, t_n): ~ \sum_{k=1}^n
  \frac{\partial \varphi_i(a)}{\partial x_k} t_k = 0 ~~ i = 1,2,\ldots,m\}
  $$
  $$
  d_{xx}^2 L(a, \lambda) = \sum_{k=1}^n \sum_{l=1}^n
  \frac{\partial L(a, \lambda)}{\partial x_k \partial x_l} dx_k dx_l
  $$
\end{define}

\begin{block}[Достаточные условия условного экстремума]
  1) $(a, \lambda^*)$ стационарная точка функции Лагранжа

  2) $O(a) ~~ \exists$ все непрерывные частные производные второго порядка
  $f(x) ~~ \varphi_i (x)$

  3) $\rang A(a) = m$

  Если $\forall dx \in E_T ~~~ d_{xx}^2 L(a, \lambda^*) > 0$ относительно
  дифференциалов независимых переменных тогда $a$
  точка условного $\min ~ f$ при наличии связи с системой $*$

  Если $\forall dx \in E_T ~~~ d_{xx}^2 L(a, \lambda^*) < 0$ относительно
  дифференциалов независимых переменных тогда $a$
  точка условного $\max ~ f$ при наличии связи с системой $*$

  Если $\forall dx \in E_T ~~~ d_{xx}^2 L(a, \lambda^*)$ знако
  неопределенная относительно дифференциалов независимых переменных тогда $a$
  не точка условного экстремума $f$ при наличии связи с системой $*$
\end{block}

\begin{title}[\Large]
  Наибольшее и наименьшее значение ФМП на компактных множествах
\end{title}

\begin{block}[Способ нахождения наибольшего и наименьшего значения функции]
  Для нахождения нибольшего или наименьшего занчения ФМП на компактном
  множестве $E \subset R^n$ необходимо

  1) найти стационарные точки и точки в которых $f$ не дефференцируются на $E$

  2) найти возможные условные экстремумы $f$ при условии что $x \in \partial E$

  3) найти значения $f$ в этих точках и выбрать из них наибольшее или наименьшее
  соответсвенно
\end{block}