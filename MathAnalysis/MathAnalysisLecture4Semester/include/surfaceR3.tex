\begin{title}
  Поверхность в пространстве $R^3$
\end{title}

\begin{define}
  $\vec \varphi = (x(u, \upsilon) , y(u, \upsilon), z(u, \upsilon)) ~~~
  (u, \upsilon) \in [D]$ непрерывно дифференцируема
  $$
  \rang \left(
  \begin{array}{ccc}
    x'_u & y'_u & y'_u \\
    x'_{\upsilon} & y'_{\upsilon} & y'_{\upsilon}
  \end{array}
  \right) = 2 ~~~
  \text{тогда называют поверхность гладкой}
  $$
  $[D] \leftrightarrow P \subset R^3$ то образ $\vec \varphi([D]) = P$
  называется простой поверхностью в $R^3$ а отображение
  $\vec \varphi(u, \upsilon)$ еe параметрическим заданием.
\end{define}

\begin{define}[эквивалентности поверхностей]
  $\vec \rho = \vec \rho(p,s) \sim \vec \varphi(u, \upsilon) ~~~
  (\rho, s) \in [\Omega]$ если возможна замена параметра
  $$
  \left\{
  \begin{array}{l}
    u = u(p,s) \\
    \upsilon = \upsilon(p,s)
  \end{array}
  \right. ~~~ [\Omega] \leftrightarrow [D] ~~~
  \left|
  \begin{array}{cc}
    \upsilon'_p & \upsilon'_s \\
    v'_p & v'_s
  \end{array}
  \right| \not = 0
  $$
  тогда $\vec \rho = \vec \rho(p,s) = \vec \varphi (x(u(p,s), \upsilon(p,s)),
  y(u(p,s), \upsilon(p,s)), z(u(p,s), \upsilon(p,s)))$

  $\vec \varphi ([D]) = P = \vec \rho([\Omega])$
\end{define}

\begin{define}[почти простой поверхности]
  $\vec \rho(u, \upsilon)$ гладкое отображение которое отображается
  взаимооднозначно в область $D$
  (может быть не ограничена) на какое-то множество в $R^3$, если $\exists$
  исчерпание $[D_n]$ такое что $P_n = \vec \varphi ([D_n])$
  является простой тогда $\vec P =
  \varphi(D)$ называется почти простой.
\end{define}

\begin{theorem}
  $P$ простая поверхность заданная парасетрически $(u, \upsilon) \in [D]$
  $\vec \varphi (u, \upsilon) = (x(u, \upsilon), y(u, \upsilon),
  z(u, \upsilon))$ тогда $d\vec \varphi = \vec \varphi'_udu +
  \vec \varphi'_{\upsilon} d\upsilon$
\end{theorem}

\begin{proof}
  $$
  [\vec \varphi_u, \vec \varphi_{\upsilon}] \not = \vec 0 =
  \left|
  \begin{array}{ccc}
    \vec i & \vec j & \vec k \\
    x'_u & y'_u & z'_u \\
    x'_{\upsilon} & y'_{\upsilon} & z'_{\upsilon}
  \end{array}
  \right| =
  $$
  $$
  = \vec i \left|
  \begin{array}{cc}
    y'_u & z'_u \\
    y'_{\upsilon} & z'_{\upsilon}
  \end{array}
  \right| +
  \vec j \left|
  \begin{array}{cc}
    z'_u & x'_u \\
    z'_{\upsilon} & x'_{\upsilon}
  \end{array}
  \right| +
  \vec k \left|
  \begin{array}{cc}
    x'_u & y'_u \\
    x'_{\upsilon} & u'_{\upsilon}
  \end{array}
  \right| \not = 0
  $$
  так как $\rang(...) = 2$ и будет хотябы один определитель не нулевой
\end{proof}

\begin{theorem}
  $\vec \varphi(u, \upsilon) \sim \vec \rho(p,s) ~~~ (p, s) \in [\Omega]$ тогда
  $[\vec \rho'_p, \vec \rho'_s]$ либо соноправлены либо противоположенно
  направленны.

  При замене параметризации меняется направления векторов.
\end{theorem}

\begin{proof}
  $\vec \rho(p,s) = \vec \varphi(u(p,s), \upsilon(p,s))$
  $$
  [\vec \rho'_p, \vec \rho'_s] = [\vec \varphi'_u u'_p +
  \vec \varphi'_{\upsilon} \upsilon'_p, \vec \varphi'_u u'_s +
  \vec \varphi'_{\upsilon} \upsilon'_s] =
  $$
  $$
  = [\vec \varphi'_u, \vec \varphi'_{\upsilon}] (u'_p \upsilon'_s -
  u'_s \upsilon'_p) = [\vec \varphi'_u, \vec \varphi'_{\upsilon}]
  \left|
  \begin{array}{cc}
    u'_p & u'_s \\
    \upsilon'_p & \upsilon'_s
  \end{array}
  \right| = 0
  $$
\end{proof}

\begin{define}
  $$
  \frac{d \vec \varphi}{dt} ~~ \text{касательный вектор}
  $$
  $$
  \frac{d \vec \varphi}{dt} \perp [\vec \varphi'_u, \vec \varphi'_{\upsilon}]
  = \vec n
  $$
  Векто нормали это вектор перпендикулярен всем прямым проходящим через точку
  $t$

  Поверхности бывают одностороними и двусторонними.

  Меры обладают свойством конечной адитивности то есть если разделить что-то
  на конечное кол-во частей то сумма всех мер будет равна мере всей части.
\end{define}

\begin{define}[первой квадратичной формы]
  $$
  \vec \varphi' = \vec \varphi_u u' + \vec \varphi_{\upsilon} \upsilon'
  $$
  $$
  |\vec \varphi'|^2 = (\vec \varphi', \vec \varphi') =
  (\vec \varphi_u u' + \vec \varphi_{\upsilon} \upsilon', \vec \varphi_u u' +
  \vec \varphi_{\upsilon} \upsilon') =
  $$
  $$
  = (\vec \varphi_u, \vec \varphi_u)u'^2 + 2(\vec \varphi_u,
  \vec \varphi_{\upsilon}) u' \upsilon' + \upsilon'^2 (\vec \varphi_{\upsilon},
  \vec \varphi_{\upsilon})
  $$
  $E = (\vec \varphi_u, \vec \varphi_u)$

  $F = (\vec \varphi_u, \vec \varphi_{\upsilon})$

  $G = (\vec \varphi_{\upsilon}, \vec \varphi_{\upsilon})$

  $|\vec \varphi'|^2 = E u'^2 + 2Fu'\upsilon' + G\upsilon'^2$

  $ds^2 = Edu^2 + 2Fdud\upsilon + Gd\upsilon^2$
\end{define}

\begin{theorem}
  Первая квадратичная формула положительна определена.
\end{theorem}

\begin{proof}
  $$
  E = (\vec \varphi_u, \vec \varphi_u) = |\vec \varphi_u|^2 > 0
  $$
  $$
  G = (\vec \varphi_{\upsilon}, \vec \varphi_{\upsilon}) =
  |\vec \varphi_{\upsilon}|^2 > 0
  $$
  так как $\vec \varphi_u, \vec \varphi_{\upsilon}$ лнз из за регулярности
  поверхности $\Rightarrow ~ \not= \vec 0$
  $$
  F = (\vec \varphi_u, \vec \varphi_{\upsilon}) = |\vec \varphi_u|
  |\vec \varphi_{\upsilon}| \cos \alpha
  $$
  $$
  \left|
  \begin{array}{cc}
    E & F \\
    F & G
  \end{array}
  \right|
  = EG - F^2 = |\vec \varphi_u|^2|\vec \varphi_{\upsilon}|^2 -
  |\vec \varphi_u|^2 |\vec \varphi_{\upsilon}|^2 \cos^2 \alpha =
  $$
  $$
  = |\vec \varphi_u|^2
  |\vec \varphi_{\upsilon}|^2 \sin^2 \alpha = |[\vec \varphi_u, \vec
  \varphi_{\upsilon}]|^2 > 0
  $$
  так как $\vec \varphi_u, \vec \varphi_{\upsilon}$ лнз из за регулярности
  поверхности $\Rightarrow ~ \not= \vec 0$
\end{proof}
