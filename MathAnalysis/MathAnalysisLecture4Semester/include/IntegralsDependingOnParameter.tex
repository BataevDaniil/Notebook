\begin{title}
  Интегралы зависящие от параметра
\end{title}

\begin{define}
  $$
  F(y) = \int_a^b f(x,y) dx ~~ y \in E
  $$
  Обычная сходимость
  $$
  \forall y \in E
  ~~~
  \forall \varepsilon > 0
  ~~~
  \exists \delta_{\varepsilon; y} \in (a, b)
  ~~~
  \forall c \in [\delta_{\varepsilon; y}, b]
  ~~~
  \left|
    \int_a^b f(x,y) dx
  \right|
  < \varepsilon
  $$
  Равномерная сходимость
  $$
  \forall \varepsilon > 0
  ~~~
  \exists \delta_{\varepsilon} \in (a,b)
  ~~~
  \forall c \in [\delta_{\varepsilon}, b]
  ~~~
  \forall y \in E
  ~~~
  \left|
    \int_c^b f(x,y) dx
  \right|
  < \varepsilon
  $$
  Обозначения
  $$
  \int_a^b f(x,y) dx \stackrel{E}{\to} ~~ \text{обычная сходимость}
  $$
  $$
  \int_a^b f(x,y) dx \stackrel{E}{\rightrightarrows} ~~
  \text{равномерная сходимость}
  $$
  Не равномерная сходимость
  $$
  \exists \varepsilon_0 > 0
  ~~~
  \forall \delta \in (a,b)
  ~~~
  \exists c_{\delta} \in [\delta_{\varepsilon}, b)
  ~~~
  \exists y_{\delta} \in E
  ~~~
  \left|
    \int_{c_{\delta}}^b f(x,y) dx
  \right|
  \ge \varepsilon
  $$
  Интеграл является неравномерно сходящимся если он сходится но сходится плохо.
\end{define}

\begin{title}[\Large]
  Критерий Коши равномерной сходимости определенного интеграла
\end{title}

\begin{block}[Критерий Коши равномерной сходимости определенного интеграла]
  $$
  F(y) = \int_a^b f(x,y) dx ~~ \text{сходится равномерно на} ~~ E ~~
  $$
  $$
  \Leftrightarrow
  \forall \varepsilon > 0
  ~~~
  \exists \delta_{\varepsilon} \in (a,b)
  ~~~
  \forall c', c'' \in [\delta_{\varepsilon}, b)
  ~~~
  \forall y \in E
  ~~~
  \left|
    \int_{c'}^{c''} f(x,y) dx
  \right|
  < \varepsilon
  $$
\end{block}

\begin{proof}
  $\Rightarrow$
  $$
  \forall \varepsilon > 0
  ~~~
  \exists \delta_{\varepsilon} \in (a,b)
  ~~~
  \forall c \in [\delta_{\varepsilon}, b]
  ~~~
  \forall y \in E
  ~~~
  \left|
    \int_c^b f(x,y) dx
  \right|
  < \frac{\varepsilon}{2}
  ~~~
  c = c'
  ~~~
  c = c''
  $$
  $$
  \left|
    \int_{c'}^{c''} f(x,y) dx
  \right|
  \le
  \left|
    \int_{c'}^{b} f(x,y) dx
  \right|
  +
  \left|
    \int_{c''}^{b} f(x,y) dx
  \right|
  <
  \frac{\varepsilon}{2} + \frac{\varepsilon}{2} = \varepsilon
  $$
  $\Leftarrow$
  $$
  \Leftrightarrow
  \forall \varepsilon > 0
  ~~~
  \exists \delta_{\varepsilon} \in (a,b)
  ~~~
  \forall c', c'' \in [\delta_{\varepsilon}, b)
  ~~~
  \forall y \in E
  ~~~
  \left|
    \int_{c'}^{c''} f(x,y) dx
  \right|
  < \frac{\varepsilon}{2}
  $$
  $$
  \lim_{c'' \to b -0}
    \left|
      \int_{c''}^{b} f(x,y) dx
    \right|
  =
  \int_{a}^{b} f(x,y) dx
  < \varepsilon
  $$
\end{proof}

\begin{title}[\Large]
  Признаки равномерной сходимости несобственных интегралов
\end{title}

\begin{block}[Признак Вейерштрасса]
  1)
  $$
  \forall  x \in [a, b)
  ~~~
  \forall y \in E
  ~~~
  |f(x, y)| < g(x)
  $$
  2)
  $$
  \int_a^b g(x) dx ~~ \text{сходится}
  $$
  тогда
  $$
  \int_a^b f(x,y) g(x,y) dx \stackrel{E}{\rightrightarrows}
  $$
\end{block}

\begin{proof}
  $$
  \forall \varepsilon > 0
  ~~~
  \exists \delta_{\varepsilon} \in (a,b)
  ~~~
  \forall c', c'' \in [\delta_{\varepsilon}, b)
  ~~~
  \forall y \in E
  ~~~
  \left|
    \int_{c'}^{c''} f(x,y) dx
  \right|
  < \varepsilon
  $$
  $$
  \left|
    \int_{c'}^{c''} f(x,y) dx
  \right|
  \le
  \int_{c'}^{c''} |f(x,y)| dx
  \le
  \left|
    \int_{c'}^{c''} g(x,y) dx
  \right|
  < \varepsilon
  $$
\end{proof}

\begin{block}[Признак Дирехле]
  $f(x,y) ~~ g_x(x,y) ~~ g(x,y) ~~ [a,b] \times E$

  1)
  $$
  \forall x \in [a, b]
  ~~
  \forall y \in E
  ~~
  | F(x, y)| \le M
  $$
  2) $g'_x (x,y) \le 0$ (или $g'x(x,y) \ge 0$)

  3) $|g(x,y)| \le \alpha(x) ~~~ \lim_{x \to b ~ -0}$

  тогда
  $$
  \int_a^b f(x,y) g(x,y) dx \stackrel{E}{\rightrightarrows}
  $$
\end{block}

\begin{block}[Признак Абеля]
  При $f(x,y) ~~ g(x,y) ~~ g'_x(x,y)$ непрерывны на $[a,b) \times E$

  1)
  $$
  \int_a^b f(x,y) dx ~~ \text{равмноерно сходится на} ~~ E
  $$
  2) $\forall x \in [a,b) ~~ \forall y \in E ~~ g'_x(x,y) \le 0$ (или
  $g'_x(x,y) \ge 0$)

  3) $\exists M > 0 ~~ \forall (x,y) \ in [a,b) \times E ~~ |g(x,y) \le M$

  тогда
  $$
  \int_a^b f(x,y) g(x,y) dx \stackrel{E}{\rightrightarrows}
  $$
\end{block}

\begin{title}[\Large]
  Непрерывность несобственных интегралов зависящих от параметра
\end{title}

\begin{theorem}
  $f(x,y)$ непрерывна на $[a,b) \times [c, d]$
  $$
  \int_a^b f(x,y) dx = F(y) ~~ \text{тогда} ~~ F(y) ~~ \text{непрерывна на} ~~
  [a,b)
  $$
\end{theorem}

\begin{proof}
  $y_0, y_0 + y_{\delta} \in [c, d]$
  $$
  \forall \varepsilon > 0
  ~~~
  \exists \delta_{\varepsilon} \in (a,b)
  ~~~
  \forall l \in [\delta_{\varepsilon}, b)
  ~~~
  \forall y \in [c, d]
  ~~~
  \left|
    \int_l^b f(x,y)dx
  \right|
  < \frac{\varepsilon}{3}
  $$
  $$
  \left|
    \int_l^b f(x,y_0)dx
  \right|
  < \frac{\varepsilon}{3}
  $$
  $$
  \left|
    \int_a^l f(x,y)dx
    -
    \int_a^l f(x,y_0)dx
  \right|
  < \frac{\varepsilon}{3}
  $$
  $|y - y_0| = |y_{\delta}| < \gamma_{\varepsilon}$
  $$
  \left|
    \int_a^b f(x,y)dx
    -
    \int_a^b f(x,y_0)dx
  \right|
  =
  $$
  $$
  =
  \left|
    \int_a^l f(x,y)dx
    -
    \int_a^l f(x,y_0)dx
    +
    \int_l^b f(x,y)dx
    -
    \int_l^b f(x,y_0)dx
  \right|
  \le
  $$
  $$
  \le
  \left|
    \int_a^l f(x,y)dx
    -
    \int_a^l f(x,y_0)dx
  \right|
  +
  \left|
    \int_l^b f(x,y)dx
  \right|
  +
  \left|
    \int_l^b f(x,y_0)dx
  \right|
  <
    \frac{\varepsilon}{3}
    +
    \frac{\varepsilon}{3}
    +
    \frac{\varepsilon}{3}
  =
    \varepsilon
  $$
\end{proof}

\begin{title}[\Large]
  Интегрирование несобственных интегралов зависящих от параметра
\end{title}

\begin{theorem}
  $f(x,y)$ непрерывна $[a,b) \times [c,d]$ $b$ особая точка
  $$
  \int_a^b f(x,y) dx = F(y) ~~~ \text{равномерно сходится на отрезке} [c,d]
  $$
  тогда ее можно проинтегрировать и интеграл равен
  $$
  \int_c^d F(y) dy = \int_a^b dx \int_c^d f(x,y) dy =
  \int_c^d dy \int_a^b f(x,y) dx
  $$
\end{theorem}

\begin{proof}
  $F(y)$ непрерывна по теореме непрерывности несобственных интегралов зависящих
  от параметра
  $$
  \forall \varepsilon > 0
  ~~~
  \exists \delta_{\varepsilon} \in [a, b)
  ~~~
  \forall t \in [\delta_{\varepsilon}, b)
  ~~~
  \forall y \in [c,d]
  ~~~
  \left|
    \int_t^b f(x,y) dx < \frac{\varepsilon}{d-c}
  \right|
  $$
  так как $\int_t^b f(x,y) dx$ удовлитворяет теореме об интегрировании
  собственных интегралов, то
  $$
  \int_a^t dx \int_c^d f(x,y) dy =
  \int_c^d dy \int_a^t f(x,y) dx
  $$
  перейдем к пределу при $t \to b -0$
  $$
  \left|
    \int_c^d dy \int_a^b f(x,y) dx
    -
    \int_c^d dy \int_a^t f(x,y) dx
  \right|
  =
  \left|
    \int_c^d dy \int_t^b f(x,y) dx
  \right|
  \le
  $$
  $$
  \le
  \int_c^d dy
  \left|
    \int_t^b f(x,y) dx
  \right|
  <
  \frac{\varepsilon}{d-c} \int_c^d dy = \varepsilon
  $$
  теперь переходим к пределу
  $$
  \int_c^d dy \int_a^b f(x,y) dx
  =
  \int_a^b dx \int_c^d f(x,y) dy
  $$
\end{proof}
