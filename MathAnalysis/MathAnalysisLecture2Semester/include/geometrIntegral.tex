\begin{title}[\Large]
  Мера Жордана в пространстве $R^n$
\end{title}

  $$
  E \to \mu (E) \ge 0 ~~~
  E = \cup_{k=1}^m ~~~
  \mu(E) = \sum_{k=1}^n \mu (E_k)
  $$

  \kv{умножения множеств} $E \times D = \{ (x,y): x \in E; y \in D \}$

  \kv{параллелипипеда}
  $P \subset R^n ~~~ P = [a_1, b_1] \times [a_2, b_2] \times \ldots \times
  [a_n, b_n]$ или множество которое получится если отбросить всей или части
  границы.

  $m(p)$ мера длины, площади и т.п.

  $$
  m(p) = \prod_{k=1}^n (b_k - a_k)
  $$

  множество $E \subset R^n$ называется \kv{элементарным}
  если его можно представить в виде конечного $E = \sqcup_{k=1}^m P_k$ попарно
  не пересеающихся множеств.

  $\sqcup$ - \kv{попарно непересекающиеся множества}

Мера элементарных множеств $E$ это сумма площадей, из которых она состоит
$$
m(E) = \sum_{k=1}^m m(P_k)
$$

\begin{defin}[измерения по Жордану]
  $E$ называется \kv{измеренным по Жордану} $E \subset R^n$, если
  $$
  \forall \varepsilon > 0 ~~~
  \exists P_{\varepsilon}, Q_{\varepsilon} \subset R^n ~~~
  P_{\varepsilon} \subset E \subset Q_{\varepsilon} ~~~
  0 \le m(Q_{\varepsilon}) - m(P_{\varepsilon}) < \varepsilon
  $$

  Если $E \subset R^n$ то его мерой Жордана называется такое число
  $$
  \forall P,Q \subset R^n ~~~
  P \subset E \subset Q ~~~
  m(P) \le m(E) \le m(Q)
  $$
\end{defin}

\begin{theorem}
  Для любого измеримого по Жорадану множества $E \subset R^n$ существует мера
  Жордана и при том единственна
\end{theorem}

\begin{proof}
  $$
  P \subset E \subset Q ~~~
  \{ m(P): P \subset E \} ~~~
  \{ m(Q): E \subset Q \} ~~~
  m(P) \le m(Q)
  $$
  $$
  \exists c \in R ~~~
  m(P) \le c \le m(Q) ~~~
  c = m(E)
  $$
  Так как $E$ измеримо по Жордану то $E \subset R^n$
  $$
  \forall \varepsilon > 0 ~~~
  \exists P_{\varepsilon},Q_{\varepsilon} \subset R^n ~~~
  P_{\varepsilon} \subset E \subset Q_{\varepsilon} ~~~
  0 \le m(Q_{\varepsilon}) - m(P_\varepsilon) < \varepsilon
  $$
  $$
  m(P_{\varepsilon}) \le m_1 (E) \le m(Q_{\varepsilon})
  $$
  $$
  m(P_{\varepsilon}) \le m_2 (E) \le m(Q_{\varepsilon})
  $$
  $$
  |m_1 (E) - m_2 (E)| \le m(Q_{\varepsilon}) - m(P_{\varepsilon}) < \varepsilon
  $$
  $\Rightarrow m_1(E) = m_2(E)$
\end{proof}

\begin{title}[\Large]
  Вычисление площади плоской фигуры, заданной в декартовой системе координат
\end{title}

\begin{defin}[криволинейной трапеции]
 $$
  D = \{ (x,y): ~~~ x \in [a,b]  ~~~ 0 \le y \le f(x) \}
 $$
\end{defin}

\begin{theorem}
  $f(x)$ непрерывная на $[a,b]$
  $$
  S(D) = \int_a^b f(x) dx
  $$
\end{theorem}

\begin{proof}
  $$
  \int^* (R) = \sum_{k=1}^n M_k \Delta x_k ~~~
  \int_* (R) = \sum_{k=1}^n m_k \Delta x_k
  $$
  $$
  P_n = \sqcup_{k=1}^n P_k ~~~
  Q_n = \sqcup_{k=1}^n Q_k
  $$
  $$
  m(P_n) = \sum_{k=1}^n m(P_k) = \sum_{k=1}^n m_k \Delta x_k = \int_* (R)
  $$
  $$
  m(Q_n) = \sum_{k=1}^n m(Q_k) = \sum_{k=1}^n M_k \Delta x_k = \int^* (R)
  $$
  $f(x)$ непрерывна на $[a,b]$ значит имеет интеграл по критерию интегрируемости
  $$
  \forall \varepsilon > 0 ~~~
  \exists R_{\varepsilon} ([a,b]) ~~~
  \int^*(R_{\varepsilon}) - \int_* (R_{\varepsilon}) < \varepsilon ~~~
  m(Q_n) - m(P_n) < \varepsilon ~~~
  P_n \subset D \subset Q_n
  $$
  Из свойства определенного интеграла
  $$
  \int_* (R) \le \int_a^b f(x)dx \le \int^* (R)
  $$
\end{proof}

\begin{theorem}
  $f(x),g(x)$ непрерывный на $[a,b]$

  $$
  D = \{ (x,y): ~~ x \in [a,b] ~~ f(x) \le y \le g(x) \}
  $$

  $D = D_g \backslash D_f$
  $$
  S(D) = S(D_g) - S(D_f) = \int_a^b g(x)dx - \int_a^b f(x)dx =
  \int_a^b (g(x) - f(x)) dx
  $$
\end{theorem}

\begin{title}[\Large]
  Вычисление площади плоской фигуры заданной в полярных координатах
\end{title}

\begin{theorem}
  $$
  D = \{ (\rho,\varphi): ~~ \alpha \le \varphi \le \beta ~~ 0 \le
  \rho \le \rho(\varphi) \}
  $$
  $$
  S(D) = \frac{1}{2} \int_{\alpha}^{\beta} \rho^2(\varphi) d\varphi
  $$
\end{theorem}

\begin{proof}
  $$
  \int^* (R) = \sum_{k=1}^n M_k \Delta \varphi_k ~~~
  \int_* (R) = \sum_{k=1}^n m_k \Delta \varphi_k
  $$
  $$
  S(P_k) = \frac{m_k^2 \Delta \varphi_k}{2} ~~~
  S(Q_k) = \frac{M_k^2 \Delta \varphi_k}{2}
  $$
  $$
  P_n = \sqcup_{k=1}^n P_k ~~~
  S(P_n) = \sum_{k=1}^n S(P_k) = \frac{1}{2} \sum_{k=1}^n m_k^2
  \Delta \varphi_k = \int_* (R)
  $$
  $$
  Q_n = \sqcup_{k=1}^n Q_k ~~~
  S(Q_n) = \sum_{k=1}^n S(Q_k) = \frac{1}{2} \sum_{k=1}^n M_k^2
  \Delta \varphi_k = \int^* (R)
  $$

  $\frac{\rho^2}{2}$ интегрируема на $[\alpha, \beta]$

  $$
  \forall \varepsilon > 0 ~~~ \exists R_{\varepsilon} ([a,b]) ~~~
  \int^* (R_{\varepsilon}) - \int_* (R_{\varepsilon}) < \varepsilon ~~~
  S(Q_n) - S(P_n) < \varepsilon
  $$
  $$
  \int_* (R_{\varepsilon}) \le
  \frac{1}{2} \int_{\alpha}^{\beta} \rho^2(\varphi) d\varphi \le
  \int^* (R_{\varepsilon})
  $$
\end{proof}

\begin{title}[\Large]
  Вычисление объемов тел вращения с помощью определенного интеграла. Вычисление
  объемов тел с заданными площадями их поперечных сечений
\end{title}

\begin{defin}[цилиндра]
  Множество точек $\vartheta \subset R^3$
  $$
  \vartheta = \{ (x,y,z): ~~~ (x,y) \in D ~~~ 0 < z \le h\}
  $$
  будем называть цилиндрическим телом с основанием $D$ и высотой $h$.
\end{defin}

\begin{theorem}
  $D \subset R^2 ~~~ S(D)$ измеримо по Жордану $\vartheta \subset R^3$ тогда
  $V(\vartheta) = h S(D)$
\end{theorem}

\begin{proof}
  $$
  \forall \varepsilon > 0 ~~~
  \exists P_{\varepsilon}, Q_{\varepsilon} \subset R^2 ~~~
  P_{\varepsilon} \subset D \subset Q_{\varepsilon} ~~~
  S(Q_{\varepsilon}) - S(P_{\varepsilon}) < \frac{\varepsilon}{h}
  $$
  $$
  \vartheta'_{\varepsilon} = P_{\varepsilon} \times [0,h] ~~~
  \vartheta''_{\varepsilon} = Q_{\varepsilon} \times [0,h] ~~~
  \vartheta'_{\varepsilon} \subset \vartheta \subset \vartheta''_{\varepsilon}
  $$
  $$
  V(\vartheta''_{\varepsilon}) - V(\vartheta'_{\varepsilon}) <
  h S(Q_{\varepsilon}) - h S(P_{\varepsilon}) =
  h (S(Q_{\varepsilon}) - S(P_{\varepsilon})) <
  h \frac{\varepsilon}{h} = \varepsilon
  $$
  значит цилиндрическое тело с характеристиками $D \subset R^2$
  $\vartheta \subset R^3$ имеет объем $V(\vartheta) = h S(D)$
  $$
  P_{\varepsilon} \subset D \subset Q_{\varepsilon}
  $$
  $$
  S(P_{\varepsilon}) \le S(D) \le S(Q_{\varepsilon})
  $$
  $$
  hS(P_{\varepsilon}) \le hS(D) \le hS(Q_{\varepsilon})
  $$
  $$
  V(\vartheta'_{\varepsilon}) \le h S(D) \le V(\vartheta''_{\varepsilon})
  $$
\end{proof}

\begin{theorem}
  Тело $T$ получено вращением криволинейной трапеции тогда $T$ измеримо по
  Жордану
  $$
  V(T) = \pi \int_a^b f^2(x) dx
  $$
\end{theorem}

\begin{proof}
  $P_k$ цилиндрическое тело с высотой $\Delta x_k$ в основании круг $m_k$
  $$V(P_k) = \pi m_k^2 \Delta x_k$$

  $Q_k$ цилиндрическое тело с высотой $\Delta x_k$ в основании круг $M_k$
  $$V(Q_k) = \pi M_k^2 \Delta x_k$$
  $$
  P = \sqcup_{k=1}^n P_k ~~~
  Q = \sqcup_{k=1}^n Q_k
  $$
  $$
  V(P) = \sum_{k=1}^n \pi m_k^2 \Delta x_k = \int_* (R) ~~~
  V(Q) = \sum_{k=1}^n \pi M_k^2 \Delta x_k = \int^* (R)
  $$
  так как по условию теоремы $f(x)$ непрерывна, $f^2(x)$ тоже непрерывна значит
  интегрируема
  $$
  \forall \varepsilon > 0 ~~~
  \exists R_{\varepsilon} ([a,b]) ~~~
  0 \le \int^* (R_{\varepsilon}) - \int_* (R_{\varepsilon}) < \varepsilon
  $$
  $$
  \exists P_{\varepsilon}, Q_{\varepsilon} ~~~
  P_{\varepsilon} \subset T \subset Q_{\varepsilon} ~~~
  V(Q_{\varepsilon}) - V(P_{\varepsilon}) < \varepsilon
  $$
  $$
  \int_* (R_{\varepsilon}) \le V(T) \le \int^* (R_{\varepsilon})
  $$
\end{proof}

\begin{theorem}
  Тело $T$ расположено в $R^3$ между плоскостями $x = a ~~ x = b$
  $Dx$ сечение плоскостью $yOz$ и проходящая через точку $x$

  1. $\forall x \in [a,b]$ сечение $Dx$ измеримо по Жордану и имеет площадь
  $S(x)$ - непрерывна

  2. $\forall x', x'' \in [a,b]$ $D_{x'}, D_{x''}$ вложены одна в другую
  проэкции множеств на плоскости $yOz$ тогда тело $T$ измеримо по Жордану
  $$
  V(T) = \int_a^b S(x)dx
  $$
\end{theorem}

\begin{title}[\Large]
  Понятие кривой
\end{title}

\begin{defin}[кривой]
  Кривой $\omega$ называется упорядоченое множество точек $(x,y,z) \in R^3$
  $$
    \left\{
      \begin{array}{l}
        x = x(t) \\
        y = y(t) \\
        z = z(t)
      \end{array}
    \right.
    t \in [a,b]
  $$
  $
  \vec r = \vec r (t) = ( x(t), y(t) , z(t) ) =
  x(t)\vec i + y(t)\vec j + z(t)\vec k
  $

  Кривая непрерывна если все ее функции непрерывны.

  Если $x(t), y(t), z(t)$ дифференцируемы, то кривая дифференцируема.

  $\vec r'(t) = ( x'(t), y'(t) , z'(t) )$ касательный вектор
\end{defin}

\begin{title}[\Large]
  Длина кривой
\end{title}

$L_n$ семейство ломанных вписанных в данную кривую

$\varphi (L_n)$ длина ломанной. $\varphi (\omega) = sup \varphi (L_n)$ конечно
число

\begin{theorem}
  Если $\omega$ непрерывна деференциирумая кривая заданная
  $\vec r = \vec r (t) ~~~ t \in [a,b]$ то она является справляемой
  (то есть у нее есть длина) и ее длина имеет оценку
  $$
  \varphi (\omega) \le (b-a) max |r'(t)| ~~~ t \in [a,b]
  $$
\end{theorem}

\begin{proof}
  $
  R([a,b]) = \{t_k\}
  $
  $$
  \varphi (L_n) = \sum_{k=1}^n |\vec r (t_k) - \vec r (t_{k-1})| \le
  \sum_{k=1}^n (t_k - t_{k-1}) |\vec r' (t_k^*)| \le
  $$
  $t_k^* \in (t_{k-1}, t_k)$
  $$
  \le max_{t \in [a,b]} |\vec r' (t)| \sum_{k=1}^n (t_k - t_{k-1}) \le
  max_{t \in [a,b]} |\vec r' (t)| (b-a)
  $$
  $$
  \varphi (\omega) = sup \varphi (L_n) \le max_{t \in [a,b]} |\vec r' (t)|(b-a)
  $$
\end{proof}

\begin{title}[\Large]
  Вычисление длины кривой с помощью определенного интеграла. Вычисление площади
  поверхности вращения с помощью определенного интеграла
\end{title}

\begin{theorem}
  $\omega$  непрерывна диференциирумая кривая заданная с помощью
  $\vec r = r(t) ~~~ t \in [a,b]$

  $\varphi (t)$ длина части кривой между $\vec r(a), \vec r(b)$ то
  $\forall t \in [a,b] ~~~ \varphi = |\vec r'(t)|$
\end{theorem}

\begin{proof}
  $t, t+\Delta t \in [a,b]$
  $$
  \Delta \varphi (t) = \varphi( t+\Delta t) - \varphi(t)
  $$
  $$
  |\Delta \vec r(t) | = |\vec r(t+\Delta t) - \vec r(t)|
  $$
  $$
  |\Delta \vec r(t)| \le |\Delta \varphi (t)| \le
  |\Delta t| max | \vec r'(\gamma)
  $$
  $$
  \left| \frac{\Delta \vec r(t)}{\Delta t} \right| \le
  \left| \frac{\Delta \varphi (t)}{\Delta t} \right| \le
  max |\vec r'(\gamma)|
  $$
  $$
  \left| \frac{\Delta \vec r(t)}{\Delta t} \right| \le
  \frac{\Delta \varphi (t)}{\Delta t} \le
  max |\vec r'(\gamma)| ~~~ \Delta t \to 0
  $$
  $$
  |\vec r'(t)| \le \varphi'(t) \le |\vec r'(t)|
  $$
\end{proof}

  Если непрерывна $\omega$ дефференциируемая прямая заданная
  $\vec r = \vec r (t) ~~~ t \in [a,b]$ то ее длина
  $$
  \varphi (\omega) = \varphi(b) - \varphi(a) = \int_a^b \varphi' (t) dt =
  \int_a^b | \vec r'(t) | dt
  $$

  Если $\omega$ непрерына деференциирума $y = f(x) ~~~ x \in [a,b]$

$\vec r(x) = (x, f(x))$

$\vec r'(x) = (1, f'(x))$
$$
\varphi(\omega) = \int_a^b |\vec r'(x)|dx = \int_a^b \sqrt{1 + (f'(x))^2}dx
$$

  Вычисление длины в полярной системе координат.
$$
\varphi (\omega) = \int_a^b \sqrt{(x'(t))^2 + (y'(t))^2}dt
$$
$$
x = \rho (\alpha) \cos \alpha ~~~ y = \rho (\alpha) \sin \alpha
$$
$$
x'(\alpha) = \rho'(\alpha) \cos \alpha - \rho(\alpha) \sin \alpha
$$
$$
y'(\alpha) = \rho'(\alpha) \sin \alpha + \rho(\alpha) \cos \alpha
$$
$$
(x'(\alpha))^2 + (y'(\alpha))^2 = \rho^2(\alpha) + (\rho'(\alpha))^2
$$
$$
\varphi (\omega) = \int_{\alpha}^{\beta}
\sqrt{\rho^2 (\alpha) + (\rho'(\alpha))^2} d\alpha
$$

  Если $P$ поверхность вращения $y=f(x) \ge 0 ~~~ x \in [a,b]$
$$
S(P) = 2\pi \int_a^b f(x)dx = 2\pi \int_a^b f(x) \sqrt{1 + (f'(x))^2} dx
$$
$$
d\varphi = \varphi' (x)dx = |\vec r'(x)|dx = \sqrt{1+f'(x)^2}dx =
\sqrt{\rho^2 (\alpha) - (\rho'(\alpha))^2} d\alpha
$$

\begin{title}[\Large]
  Физические приложения определенного интеграла.
\end{title}

%%%%%%%%%%%%%%%%%%%%%%%%%%%%%%%%%%%%%%%%