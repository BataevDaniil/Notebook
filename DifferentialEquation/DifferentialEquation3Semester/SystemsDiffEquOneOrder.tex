\begin{title}[\Large]
  Система дифференциальных уравнений в нормальной форме. Векторная запись.
  Задача Коши
\end{title}

\begin{define}[номальной системы дифференциальных уравнений]
  $$
  \left\{
  \begin{array}{l}
    x_1'(t) = f_1(t, x_1(t), x_2(t), \ldots, x_n(t)) \\
    x_2'(t) = f_2(t, x_1(t), x_2(t), \ldots, x_n(t)) \\
    \ldots ~~~ \ldots ~~~ \ldots ~~~ \ldots ~~~ \ldots ~~~ \ldots \\
    x_n'(t) = f_n(t, x_1(t), x_2(t), \ldots, x_n(t))
  \end{array}
  \right.
  $$
  $$
  x(t) =
  \left(
  \begin{array}{l}
    x_1(t) \\
    x_2(t) \\
    \ldots \\
    x_n(t)
  \end{array}
  \right) ~~~
  f(t) =
  \left(
  \begin{array}{l}
   f_1(t, x_1(t), x_2(t), \ldots, x_n(t)) \\
   f_2(t, x_1(t), x_2(t), \ldots, x_n(t)) \\
    \ldots ~~~ \ldots ~~~ \ldots ~~~ \ldots ~~~ \ldots \\
   f_n(t, x_1(t), x_2(t), \ldots, x_n(t))
  \end{array}
  \right)
  $$
  $x'(t) = f(t, x(t))$ $x(t)$ неизвестный вектор функции
  $$
  x(t):<\alpha, \beta> \to R^n
  $$
  $$
  f(t, n): (D \subset R) \times R^n \to R^n
  $$
\end{define}

\begin{block}[Задача Коши]
  $$
  t_0 \in <\alpha, \beta> ~~~ x_0 \in R^n ~~~ x(t_0) = x_0 ~~~
  \left\{
  \begin{array}{l}
    x_1(t_0) = x_{0_1} \\
    x_2(t_0) = x_{0_2} \\
    \ldots ~~~ \ldots \\
    x_n(t_0) = x_{0_n}
  \end{array}
  \right.
  $$
\end{block}

\begin{block}[Решением системы CДУ называется]
  $\varphi(t): ~ <\varphi, \beta> \to R^n$

  1) непрерывна дифференциируема на $<\alpha, \beta>$

  2) $\forall t \in <\alpha, \beta> ~~~ (t, \varphi(t)) \in D$

  3) $\forall t \in <\alpha, \beta> ~~~ \frac{d\varphi(t)}{dt} \equiv
  f(t, \varphi(t))$
\end{block}

\begin{title}[\Large]
  Системы линейных дифференциальных уравнений в нормальной форме,
  матрично-векторная запись. Эквивалентность комплексной и вещественной системы
\end{title}

\begin{define}[системы линейных дифференциальных урвнений в нормальной форме]
  $$
  \left\{
  \begin{array}{l}
    x'_1(t) = a_{11}(t)x_1(t) + a_{12}(t)x_2(t) + \ldots
    + a_{1n}x_n(t) + g_1(t) \\
    x'_2(t) = a_{21}(t)x_1(t) + a_{22}(t)x_2(t) + \ldots
    + a_{2n}x_n(t) + g_2(t) \\
    \ldots ~~~ \ldots ~~~ \ldots ~~~ \ldots ~~~ \ldots ~~~ \ldots ~~~
    \ldots ~~~ \ldots ~~~ \ldots ~~~ \ldots\\
    x'_n(t) = a_{n1}(t)x_1(t) + a_{n2}(t)x_2(t) + \ldots
    + a_{nn}x_n(t) + g_n(t) \\
  \end{array}
  \right.
  $$
  $$
  x(t) =
  \left(
  \begin{array}{l}
    x_1(t) \\
    x_2(t) \\
    \ldots \\
    x_n(t)
  \end{array}
  \right) ~~~
  A(t) =
  \left(
  \begin{array}{cccc}
    a_{11} & a_{12} & \ldots & a_{1n} \\
    a_{21} & a_{22} & \ldots & a_{2n} \\
    \ldots & \ldots & \ldots & \ldots \\
    a_{n1} & a_{n2} & \ldots & a_{nn}
  \end{array}
  \right) ~~~
  g(t) =
  \left(
  \begin{array}{l}
    g_1(t) \\
    g_2(t) \\
    \ldots \\
    g_n(t) \\
  \end{array}
  \right)
  $$
  $x'(t) = A(t)x(t) + g(t)$ неоднородная система

  $x'(t) = A(t)x(t)$ однородная система
\end{define}

\begin{block}[Эквивалентность комплексной и вещественной системы]
  $a_{ij}(t), g(t)$ непрерывны на $<\alpha, \beta> \in C^n$

  $A(t) = A_1(t) + iA_2(t)$

  $g(t) = g_1(t) + ig_2(t)$

  $x(t) = x_1(t) + ix_2(t)$

  $x'(t) = A(t)x(t) + g(t)$
  $$
  x'_1(t) + ix'_2(t) = (A_1(t) + iA_2(t)) (x_1(t) + ix_2(t)) +
  (g_1(t) + ig_2(t)) =
  $$
  $$
  = A_1(t)x_1(t) + iA_2(t)x_1(t) + iA_1(t)x_2(t) - A_2(t)x_2(t) +
  g_1(t) + ig_2(t)
  $$
  $$
  \left\{
  \begin{array}{l}
    x'_1(t) = A_1(t)x_1(t) - A_2(t)x_2(t) + g_1(t) \\
    x'_2(t) = A_2(t)x_1(t) + A_1(t)x_2(t) + g_2(t)
  \end{array}
  \right.
  $$
  $$
  \left(
  \begin{array}{c}
    x_1'(t) \\
    x_2'(t)
  \end{array}
  \right) =
  \left(
  \begin{array}{cc}
    A_1(t) & -A_2(t) \\
    A_2(t) & A_1(t)
  \end{array}
  \right)
  \left(
  \begin{array}{c}
    x_1(t) \\
    x_2(t)
  \end{array}
  \right) +
  \left(
  \begin{array}{c}
    g_1(t) \\
    g_2(t)
  \end{array}
  \right)
  $$
\end{block}

\begin{title}[\Large]
  Лемма об эквивалентности задачи Коши для линейной системы интегральному
  уравнению
\end{title}

\begin{block}[Лемма об эквивалентности задачи Коши для линейной системы]
  Задача Коши эквивалентна
  $$
  x(t) = x_0 + \int_{t_0}^t A(s)x(s)dx +
  \int_{t_0}^t g(s)ds
  $$
  тоесть любое решение задачи Коши явялется решением этого уравнения и наоборот.
\end{block}

\begin{proof}
  $\Rightarrow$ $\varphi(t)$ решение задачи Коши тогда
  $$
  \varphi'(t) = A(t)\varphi(t) + g(t)
  $$
  $$
  \int_{t_0}^t \varphi'(s)ds = \int_{t_0}^t A(s) \varphi(s)ds +
  \int_{t_0}^t g(s) ds
  $$
  $$
  \int_{t_0}^t \varphi'(s)ds = \varphi(s)|_{t_0}^t = \varphi(t) - x_0
  $$
  $$
  \varphi(t) = x_0 + \int_{t_0}^t A(s) \varphi(s) ds  + \int_{t_0}^t g(s)ds
  $$

  $\Leftarrow$ $\varphi(t)$ решение
  $$
  \varphi(t) = x_0 + \int_{t_0}^t A(s)\varphi(s)dx +
  \int_{t_0}^t g(s)ds
  $$
  тогда $\varphi(t)$ непрерывная $\Rightarrow$ дифференцируема

  $\varphi'(t) = A(t)\varphi(t) + g(t)$

  $\varphi(t_0) = x_0$
\end{proof}

\begin{title}[\Large]
  Теорема о существовании и единственности решения задачи Коши для линейной
  системы (построение и сходимость последовательных приближений,
  единственность решения)
\end{title}

\begin{define}[задачи Коши для линейной системы]
  $$
  \left\{
  \begin{array}{l}
    x'(t) = A(t)x(t) + g(t) ~~~ t \in <\alpha, \beta> \\
    x(t_0) = x_0 ~~~ t_0 \in <\alpha, \beta> ~~~ x_0 \in R^n
  \end{array}
  \right.
  $$
\end{define}

\begin{theorem}
  $A(t), g(t)$ непрерывны на $<\alpha, \beta>$ тогда
  $\forall t_0 \in <\alpha, \beta> ~~ \forall x_0 \in R^n$ задачи Коши
  $\exists !$ решение на $<\alpha, \beta>$
\end{theorem}

\begin{proof}
  1) Построим последовательные приближения
  $$
  \varphi_0(t) ~~ \varphi_0(t_0) = x_0 ~~~ t \in <\alpha, \beta>
  $$
  $$
  \varphi_1(t) = x_0 + \int_{t_0}^t A(s) \varphi_0(s)ds + \int_{t_0}^t g(s) ds
  ~~~ t \in <\alpha, \beta>
  $$
  $$
  \varphi_{k+1}(t) = x_0 + \int_{t_0}^t A(s) \varphi_k(s)ds +
  \int_{t_0}^t g(s)ds ~~~ t \in <\alpha, \beta>
  $$
  $\{\varphi_k(t)\}$

  2) Сходимость последовательных приближений. Покажем что $\{\varphi_k(t)\}$
  сходится на $<\alpha, \beta>$ к $\varphi(t)$ которая является решением задачи
  Коши

  Для этого покажем что $[c,d] \in <\alpha, \beta>$
  $$
  L = \sup_{t \in [c,d]} ||A(t)|| ~~~ M = \sup_{t \in [c,d]} ||\varphi_1(t) -
  \varphi_0(t)||
  $$
  Вместо последовательности расмотрим ряд
  $$
  \varphi_0(t) + \sum_{j=0}^{\infty}(\varphi_{j+1}(t) - \varphi_j(t))
  $$
  $$
  ||\varphi_1(t) - \varphi_0(t)|| \le M
  $$
  $$
  ||\varphi_2(t) - \varphi_1(t)|| =
  $$
  $$
  = \left\Vert x_0 + \int_{t_0}^t A(s)\varphi_1(s)dx +
  \int_{t_0}^t g(s)ds - x_0 - \int_{t_0}^t A(s)\varphi_0(s)dx -
  \int_{t_0}^t g(s)ds\right\Vert =
  $$
  $$
  = \left\Vert \int_{t_0}^t A(s)(\varphi_1(s) - \varphi_0(s))ds\right\Vert \le
  $$
  $$
  \colorbox[rgb]{0.7,0.7,0}{$\left\Vert \int_{t_1}^{t_2} fdx \right\Vert
  \le \left| \int_{t_1}^{t_2} \left\Vert f\right\Vert dx \right|$}
  $$
  $$
  \le \left| \int_{t_0}^t \left\Vert A(s)(\varphi_1(s) - \varphi_0(s))
  \right\Vert ds \right| \le
  $$
  $$
  \le \left| \int_{t_0}^t \left\Vert A(s)\right\Vert
  \left\Vert (\varphi_1(s) - \varphi_0(s)) \right\Vert ds \right| \le
  $$
  $$
  \le ML \left| \int_{t_0}^{t} ds \right| = ML|t-t_0|
  $$
  $$
  ||\varphi_3(t) - \varphi_2(t)|| \le \left| L \int_{t_0}^t||\varphi_2(s) -
  \varphi_1(s)||ds \right| \le L^2 M \left| \int_{t_0}^t |s - t_0| ds \right| =
  $$
  $$
  t > t_0 ~~~ \left| \int_{t_0}^t |s - t_0| ds \right| = \frac{(t- t_0)^2}{2}
  $$
  $$
  t < t_0 ~~~ \left| \int_{t_0}^t |t_0 - s| ds \right| = \frac{(t- t_0)^2}{2}
  $$
  $$
  = L^2 M \frac{(t- t_0)^2}{2}
  $$
  $$
  ||\varphi_{k+1}(t) - \varphi_k(t)|| \le L^k M \frac{|t - t_0|^k}{k!}
  $$
  $$
  ||\varphi_{k+2}(t) - \varphi_{k+1}(t)|| \le \left| \int_{t_0}^t ||A(s)||
  ||\varphi_{k+1}(s) - \varphi_k(s)|| ds \right| \le
  $$
  $$
  \le \frac{L^{k+1}M}{k!} \left| \int_{t_0}^t |s - t_0|^k ds \right| =
  \frac{L^{k+1}M|t-t_0|^{k+1}}{(k+1)!}
  $$
  $$
  \sum_{k=0}^{\infty} \frac{L^kM(d - c)^k}{k!}
  $$
  докажем сходимость по Доламберу
  $$
  \lim_{k \to \infty} \frac{L^{k+1} M(d - c)^{k+1}k!}{(k+1)!L^k M (d - c)^k} =
  \lim_{k \to \infty} \frac{L(d-c)}{k+1} = 0 < 1 ~ \Rightarrow ~
  \text{сходится}
  $$
  $\Rightarrow$ по признаку Вейерштраса
  $$
  \varphi_0(t) + \sum_{j=0}^{\infty}(\varphi_{j+1}(t) - \varphi_j(t))
  $$
  сходится равномерно на $[c,d]$

  $\varphi(t) = \lim_{k \to \infty} \varphi_k(t)$ непрерывна
  $$
  \varphi_{k+1} = x_0 \int_{t_0}^t A(s) \varphi_k(s) ds + \int_{t_0}^t g(s)ds
  ~~~ k \to \infty
  $$
  $$
  \varphi = x_0 + \int_{t_0}^t A(s) \varphi(s) ds + \int_{t_0}^t g(s)ds
  $$
  по Лемме $\varphi(t)$ решение задачи Коши

  3) Единственность решения задачи Коши
  $$
  \varphi = x_0 + \int_{t_0}^t A(s) \varphi(s) ds + \int_{t_0}^t g(s)ds
  $$
  $$
  \psi = x_0 + \int_{t_0}^t A(s) \psi(s) ds + \int_{t_0}^t g(s)ds
  $$
  $$
  N = \sup_{t \in [c,d]} ||\varphi(t) - \psi(t)||
  $$
  $$
  ||\varphi(t) - \psi(t)|| = \left\Vert\int_{t_0}^t A(s)
  (\varphi(s) - \psi(s))ds \right\Vert \le \left| \int_{t_0}^t ||A(s)||
  ||\varphi(s) - \psi(s)||ds \right| \le
  $$
  $$
  \le LN|t-t_0| \le L^2 N \frac{(t - t_0)^2}{2} \le
  \frac{L^k M (t - t_0)^k}{k!} ~~ k \to \infty
  $$
  $\Rightarrow$ $\varphi(t) \equiv \psi(t)$
\end{proof}

\begin{define}
  $\varphi(t)$ решение задачи Коши на $<a,b> \subset R^n$

  $\phi(t)$ решение задачи Коши на $<c,d> \subset R^n$

  если $<c,d> \subset<a,b>$ и $\varphi(t) \equiv \phi(t) ~~~ t \in <c,d>$ тогда

  $\varphi(t)$ это {\bfseries продолжение} $\phi(t)$

  $\phi$ это {\bfseries часть} $\varphi(t)$

  Решение называется {\bfseries непродолжимым} если оно не является частью никакого
  другого решения.
\end{define}

\begin{theorem}
  Всякое решение задачи Коши это часть непродолжимого решения.
\end{theorem}

\begin{title}[\Large]
  Линейные системы дифференциальных уравнений, принцип суперпозиции решений и
  следствия из него
\end{title}

\begin{theorem}[приципа суперпозии решений]
  $\varphi(t)$ решение $x'(t) = A(t)x(t) + g_1(t)$

  $\psi(t)$ решение $x'(t) = A(t)x(t) + g_2(t)$

  тогда $C\varphi(t) + B\psi(t)$ решение $x'(t) = A(t)x(t) + Cg_1(t) + Bg_2(t)$
  $C,B$ числа
\end{theorem}

\begin{proof}
  $$
  C\varphi'(t) + B\psi'(t) = A(t)(C\varphi(t) + B\psi(t)) + Cg_1(t) +
  Bg_2(t)
  $$
  $$
  C\varphi'(t) + B\psi'(t) =
  C \overbrace{(A(t)\varphi(t) + g_1(t))}^{\varphi'(t)}+
  B \overbrace{(A(t)\psi(t) + g_2(t))}^{\psi'(t)}
  $$
\end{proof}

\begin{block}[Следствие 1]
НЕПРАВИЛЬНО
  Разность решений неоднородной системы $x'(t) = A(t)x(t) + g(t)$
  является решением однородной системы $x'(t) = A(t)x(t)$
\end{block}

\begin{proof}
  $x'(t) = A(t)x(t) + g(t) ~~ C = 1 ~~ B = 1$

  $\varphi(t) - \psi(t)$

  $g(t) - g(t) = 0$
\end{proof}

\begin{block}[Слeдствие 2]
  $\varphi_0(t)$ частное решение системы $x'(t) = A(t)x(t) + g(t)$ то
  множество всех решений этой системы совпадает с множеством следующего вида
  $\{\varphi_0(t) + \varphi(t)\}$ где $\varphi(t)$ пробегает множество решений
  однородной системы.
\end{block}

\begin{proof}
  $\varphi_0(t)$ решение $x'(t) = A(t)x(t) + g(t)$

  $\varphi(t)$ решение $x'(t) = A(t)x(t)$

  $\varphi_0(t) + \varphi(t)$ решение $x'(t) = A(t)x(t) + g(t)$

  Пусть $\psi(t)$ решение $x'(t) = A(t)x(t) + g(t)$

  $\psi(t) - \varphi_0(t)$ решение $x'(t) = A(t)x(t)$

  $\Rightarrow ~ \psi(t) = \varphi_0(t) - \varphi(t)$
\end{proof}

\begin{title}[\Large]
  Линейная зависимость и независимость вектор-функций
\end{title}

\begin{define}
  $$
  \varphi_1 (t) =
  \left(
  \begin{array}{c}
    \varphi_{11}(t) \\
    \cdots \\
    \varphi_{1n}(t) \\
  \end{array}
  \right),~~~
  \varphi_2 (t) =
  \left(
  \begin{array}{c}
    \varphi_{21}(t) \\
    \cdots \\
    \varphi_{2n}(t) \\
  \end{array}
  \right), \cdots,
  \varphi_k (t) =
  \left(
  \begin{array}{c}
    \varphi_{k1}(t) \\
    \cdots \\
    \varphi_{kn}(t) \\
  \end{array}
  \right) ~~~ t \in <\alpha, \beta>
  $$
  $\varphi, \cdots, \varphi_k(t)$ называется ЛЗ если $\exists C_1, C_2, \ldots,
  C_k(\sum_{j=1}^k C_j^2 \not= 0)$ то комбинация $C_1\varphi_1(t) +
  C_2\varphi_2(t) + \ldots + C_k\varphi_k(t) = 0 ~ t \in <\alpha, \beta>$ ЛЗ. В
  противном случае система функций ЛНЗ.
\end{define}

\begin{theorem}
  Множество всех решений однородной системы $x'(t) = A(t)x(t)$ образует
  линейное пространство размерность которого совпадает с размерностью системы
\end{theorem}

\begin{proof}
  если $\varphi_1(t), \varphi_2(t)$ - решение $x'(t) = A(t)x(t) ~ \Rightarrow ~
  C_1\varphi_1(t) + C_2\varphi_2(t)$ решение $x'(t) = A(t)x(t)$

  $\varphi'_j(t) = A(t)\varphi_j(t)$

  $C_1 \varphi'_1(t) + C_2 \varphi'_2(t) = C_1(A(t) \varphi_1(t)) +
  C_2A^{(t)}\varphi_2(t) = A(t)(C_1\varphi_1(t) + C_2\varphi_2(t)) ~
  \Rightarrow ~ C_1 \varphi_1(t) + C_2 \varphi_2(t)$ решение
  $x'(t) = A(t)x(t)$\\

  Построим $n$ ЛНЗ решений $t \in <\alpha, \beta>$ возьмем $t_0 \in
  <\alpha, \beta>$
  $$
  e_1 =
  \left(
  \begin{array}{c}
    1 \\
    0 \\
    \cdots \\
    0
  \end{array}
  \right), ~~~
  e_2 =
  \left(
  \begin{array}{c}
    0 \\
    1 \\
    \cdots \\
    0
  \end{array}
  \right), \cdots,
  e_k =
  \left(
  \begin{array}{c}
    0 \\
    0 \\
    \cdots \\
    1
  \end{array}
  \right) ~~ \text{ЛНЗ}
  $$
  $$
  \left\{
  \begin{array}{c}
    x'(t) = A(t)x(t) \\
    x(t_0) = e_k
  \end{array}
  \right. ~~~ k = 1, 2, \ldots, n
  $$
  $\varphi_k(t)$ - единственое решение $\varphi_1(t), \ldots, \varphi_n(t) ~~~
  t \in <\alpha, \beta$

  покажем ЛНЗ $C_1 \varphi_1(t) + \ldots + C_n \varphi_n(t) \equiv 0 ~~~
  t = t_0$

  $C_1 \varphi_1(t_0) + C_2 \varphi_2(t_0) + \ldots + C_n \varphi_n(t_0)
  \equiv 0$
  $$
  C_1
  \left(
  \begin{array}{c}
    1 \\
    0 \\
    \cdots \\
    0
  \end{array}
  \right) +
  C_2
  \left(
  \begin{array}{c}
    0 \\
    1 \\
    \cdots \\
    0
  \end{array}
  \right) + \cdots +
  C_k
  \left(
  \begin{array}{c}
    0 \\
    0 \\
    \cdots \\
    1
  \end{array}
  \right) =
  \left(
  \begin{array}{c}
    0 \\
    0 \\
    \cdots \\
    0
  \end{array}
  \right)
  $$
  $$
  \left(
  \begin{array}{c}
    C_1 \\
    C_2 \\
    \cdots \\
    C_n
  \end{array}
  \right) =
  \left(
  \begin{array}{c}
    0 \\
    0 \\
    \cdots \\
    0
  \end{array}
  \right) ~ \Rightarrow ~ C_1 = C_2 = \ldots = C_n = 0 ~ \Rightarrow ~
  \varphi_1(t), \ldots, \varphi_n(t) ~~~ \text{- ЛНЗ}
  $$\\

  $\psi(t)$ произвольное решение
  $$
  \psi(t_0) =
  \left(
  \begin{array}{c}
    a_1 \\
    a_2 \\
    \cdots \\
    a_n
  \end{array}
  \right)
  $$
  $\varphi = a_1 \varphi_1(t) + \ldots + a_n \varphi_n(t)$

  $\psi(t) \equiv \varphi(t)$

  Заменим:
  $$
  \psi(t) =
  \left\{
  \begin{array}{c}
  x'(t) = A(t)x(t)\\
  x(t_0) =
  \left(
  \begin{array}{c}
    a_1 \\
    a_2 \\
    \cdots \\
    a_n
  \end{array}
  \right)
  \end{array}
  \right. ~~ \text{- решение задачи Коши}
  \varphi(t) =
  \left\{
  \begin{array}{c}
  x'(t) = A(t)x(t)\\
  x(t_0) =
  \left(
  \begin{array}{c}
    a_1 \\
    a_2 \\
    \cdots \\
    a_n
  \end{array}
  \right)
  \end{array}
  \right.
  $$
  $$
  \varphi(t_0) = a_1 \varphi_1(t_0) + \ldots + a_n \varphi_n(t_0) =
  a_1
  \left(
  \begin{array}{c}
    1 \\
    0 \\
    \cdots \\
    0
  \end{array}
  \right) +
  a_2
  \left(
  \begin{array}{c}
    0 \\
    1 \\
    \cdots \\
    0
  \end{array}
  \right) + \ldots +
  a_n
  \left(
  \begin{array}{c}
    0 \\
    0 \\
    \cdots \\
    1
  \end{array}
  \right) =
  \left(
  \begin{array}{c}
    a_1 \\
    a_2 \\
    \cdots \\
    a_n
  \end{array}
  \right)
  $$
  так как они обе удовлетворяеют задачи Коши то $\varphi(t) \equiv \psi(t) ~
  \Rightarrow ~ \psi(t) = \sum_{j=1}^n a_j \varphi_j(t)$
\end{proof}

\begin{define}[ФСР]
  ФСР - называется базис пространства решений тоесть $n$ - ЛНЗ решений данной
  системы
\end{define}

\begin{define}
  $\varphi_1(t), \ldots, \varphi_n(t)$ - вектор функции
  $$
  \varphi_j(t) =
  \left(
  \begin{array}{c}
    \varphi_{j1}(t) \\
    \varphi_{j1}(t) \\
    \cdots \\
    \varphi_{jn}(t)
  \end{array}
  \right)
  $$
  $$
  W[\varphi_1, \ldots, \varphi_n](t) =
  \left|
  \begin{array}{cccc}
    \varphi_{11}(t) & \varphi_{21}(t) & \ldots & \varphi_{n1}(t) \\
    \varphi_{12}(t) & \varphi_{22}(t) & \ldots & \varphi_{n2}(t) \\
    \ldots & \ldots & \ldots & \ldots \\
    \varphi_{1n}(t) & \varphi_{2n}(t) & \ldots & \varphi_{nn}(t)
  \end{array}
  \right|
  $$
\end{define}

\begin{block}[Утверждения]
  1) Если $\varphi_1(t), \ldots, \varphi_n(t)$ ЛЗ $\Rightarrow W(t) \equiv 0
  ~~~ t \in <\alpha, \beta>$

  2) Если $W(t) \not\equiv 0 ~ \Rightarrow ~ \varphi_1(t), \ldots,
  \varphi_n(t)$ ЛНЗ
\end{block}

\begin{block}[Критерий ЛНЗ решений системы]
  Пусть $\varphi_1, \ldots, \varphi_n(t)$
  решение задачи Коши тогда

  1) $W(t) = 0$ хотябы в одной точке
  $t \in <\alpha, \beta>$ то решение $\varphi_1(t), \ldots, \varphi_n(t)$ ЛЗ

  2) $\varphi_1, \ldots, \varphi_n(t)$ ЛНЗ то $W(t) \not0$ ни в одной точке
  $t \in <\alpha, \beta>$
\end{block}

\begin{proof}
  Пусть $W(t_0) = 0 ~~~ t \in <\alpha, \beta> ~ \Rightarrow ~ \varphi_1(t_1),
  \ldots, \varphi_n(t_1)$ ЛЗ $\exists C_1, \ldots, C_n ~~~
  C_1 \varphi_1(t_1) + \ldots + C_n \varphi_n (t_1) = 0$

  $\varphi(t) = C_1\varphi_1(t) + \ldots + C_n \varphi_n(t)$
  $$
  \varphi(t) =
  \left(
  \begin{array}{c}
    x'(t) = A(t)x(t) \\
    x(t_0) = 0
  \end{array}
  \right)
  $$
  $\psi \equiv 0$ другое решение $\Rightarrow ~ \varphi(t) = C_1 \varphi_1(t) +
  \ldots + C_n \varphi_n(t) \equiv 0 ~ \Rightarrow ~ \varphi_1, \ldots,
  \varphi_n$ ЛЗ \\
\end{proof}

\begin{theorem}
  $$
  W(t) =
  \left|
  \begin{array}{cccc}
    \varphi_{11}(t) & \varphi_{21}(t) & \ldots & \varphi_{n1}(t) \\
    \varphi_{12}(t) & \varphi_{22}(t) & \ldots & \varphi_{n2}(t) \\
    \ldots & \ldots & \ldots & \ldots \\
    \varphi_{1n}(t) & \varphi_{2n}(t) & \ldots & \varphi_{nn}(t)
  \end{array}
  \right|
  $$
  пусть $W(t)$ - определитель Вронского решение $\varphi_1(t), \ldots,
  \varphi_n(t)$ системы $x'(t) = A(t)x(t)$ то он является
  $\frac{W(t)}{dt} = (t_2A(t))W(t) ~~~ t \in <\alpha, \beta>$

  $t_2A(t) = a_{11}(t) + a_{22}(t) + \ldots + a_{nn}(t)$
\end{theorem}

\begin{block}[Формула Леовиля]
  $$
  W(t) = W(t_0)e^{\int_{t_0}^t t_2 A(s)ds}
  $$
  если определитель Вронского в некоторой точке $= 0$ то $W(t) \equiv 0$
\end{block}

\begin{title}[\Large]
  Фундаментальная матрица
\end{title}

\begin{define}
  Матрица $\Phi(t) ~ t \in <\alpha, \beta>$ называется решением системы
  $x'(t) = A(t)x(t)$ если столбцы матрицы являются решением системы
\end{define}

\begin{block}[Утверждение]
  $\Phi(t)$ является матрицей решений если удовлетворяет матричной системе
  $X'_{n \times n}(t) = A_{n \times n}(t) X_{n \times n}(t)$ $n$ - уравней
\end{block}

\begin{define}
  Невырожденная матрица решений называется фундаментальной матрицей
  $|\Phi(t)| = W(t)$ столбца ЛНЗ. Фкндаментальная матрица это матрица столбцы
  которой образуют ФСР
\end{define}

\begin{block}[Предложение]
  Если $\Phi(t)$ фундаментальная матрица $x'(t) = A(t)x(t)$
  тогда множество решений системы совпадает со множеством функций вида
  $\varphi(t) = \Phi(t)C$ где $C$ произвольный постоянный вектор из $e^n$
\end{block}

\begin{proof}
  $\Phi(t)$ фундаментальная матрица и $\varphi_1(t), \ldots, \varphi_n(t)$
  это ФСР

  $\Phi(t)C = (\varphi_1(t), \ldots, \varphi_n(t))C = C_1 \varphi_1(t) +
  C_2 \varphi_2(t) + \ldots + C_n \varphi_n(t)$
\end{proof}

\begin{block}[Утверждение]
  $\Phi(t)$ фундаментальная матрица $x'(t) = A(t)x(t)$ тогда множество всех
  фундаментальных матриц системы совпадает с множеством матриц следующего вида
  $\Psi(t) = \Phi(t)B ~~ detB \not= 0$
\end{block}

\begin{proof}
  так как $\Phi(t)$ фундаментальная матрица $det\Phi(t) \not= 0 ~~
  \forall t \in <\alpha, \beta>$ удовлетворяет $\Phi(t) = A(t)\Phi(t)$
  $det\Psi(t) \not= 0$ невырожденная

  $\Phi'(t)B = A(t)(\Phi(t)B)$

  $(\Phi(t)B)' = A(t)(\Phi(t)B)$

  $\Psi'(t) = A(t)\Psi(t)$ $\Psi$ - фундаментальная матрица \\

  Пусть $\Psi(t)$ фиксированная фундаментальная матрица

  $B = \Phi^{-1}(t_0) \Psi(t_0) ~~~ t_0 \in <\alpha, \beta>$

  $\Psi(t) = \Phi(t) \Phi_{-1}(t_0) \Psi(t_0)$

  $t = t_0 ~~~ \Phi(t_0) \Phi^{-1}(t_0) \Psi(t_0)$
  $$
  \left\{
  \begin{array}{c}
    X(t_0) = \Psi(t_0) \\
    X'(t) = A(t)X(t)
  \end{array}
  \right. ~ \Rightarrow ~
  \Psi(t) = \Phi(t) \Phi^{-1}(t_0)\Phi(t_0)
  $$
\end{proof}

Вывод: $x'(t) = A(t)x(t)$

1) Множество ее решений образует линейное пространство степени $n$

2) Для ее решения достаточно найти ФСР
$$
\varphi(t) = \sum_{j=1}^n C_j \varphi_j(t)
$$

\begin{title}[\Large]
  Линейная неоднородная системы
\end{title}

$x'(t) = A(t) x(t) + g(t) ~~~ t \in <\alpha, \beta>$ $A(t), g(t)$ -
непрерывные функции. Метод вариаций производных постоянных.

Расмотрим $x'(t) = A(t)x(t)$ $\Phi(t)$ - фундаментальная матрица
$x(t) = \Phi(t) C$ общее решение $x'(t) = A(t)x(t)$
$$
C =
\left\{
\begin{array}{c}
  C_1 \\
  C_2 \\
  \cdots \\
  C_n
\end{array}
\right) ~~~
C(t) =
\left\{
\begin{array}{c}
  C_1(t) \\
  C_2(t) \\
  \cdots \\
  C_n(t)
\end{array}
\right)
$$
$x(t) = \Phi(t)C(t) ~~~ x'(t) = \Phi'(t)C(t) + \Phi(t)C'(t)$

$\Phi'(t)C(t) + \Phi(t)C'(t) = A(t)\Phi(t)C(t) + g(t)$

$\Phi'(t) = A(t) \Phi(t)$

$\Phi(t)C'(t) = g(t)$

$C'(t) = \Phi^{-1}(t)g(t)$

$C(t) = \int_{t_0}^2 \Phi^{-1}(s)g(s)ds + D$

$x(t) = \Phi(t)D + \int_{t_0}^t\Phi(t) \Phi^{-1}(s)g(s)ds$ формула общего
решения. $x(t_0) = x_0$ задача Коши

$x(t_0) = \Phi(t_0)D ~~~ D = \Phi^{-1}(t_0)x(t_0)$

$x(t) = \Phi(t) \Phi^{-1}(t_0)x_0 + \int_{t_0}^t \Phi(t)\Phi^{-1}(s)g(s)ds$
формула Коши

$$
\left\{
\begin{array}{c}
  x'(t) = A(t)x(t) + g(t) \\
  x(t_0) = x_0
\end{array}
\right. ~~~ \text{задача Коши}
$$

\begin{theorem}
  Пусть $A(t), g(t)$ непрерывная матрица и вектор $t \in <\alpha, \beta>$ тогда
  решение задачи Коши определяется формулой Коши
\end{theorem}

$C(t,s) = \Phi(t)\Phi^{-1}(s)$ матрица Коши

$x(t) = C(t, t_0)x_0 + \int_{t_0}^t C(t,s)g(s)ds$

$C(t,s)$ зависит от выбора фундаментальной матрицы $\Phi(t)$. Пусть $\Psi(t)$
другая фундаментальная матрица $D(t,s) = \Psi(t)\Psi^{-1}(s)$ $\exists B$
невырожденаая матрица которая обеспечивает

$\Psi(t) = \Phi(t)B$

$\Psi^{-1}(s) = B^{-1} \Phi^{-1}(s)$

$\Psi(t) \Psi^{-1}(s) = \Phi(t) B B^{-1} \Phi^{-1}(s)$

$D(t, s) = C(t, s) ~ \Rightarrow$ функция Коши не зависит от выбра
фундаментальной матрица $\Phi(t)$

\begin{block}[Свойства]
  1) При каждом фиксиравоном $s \in <\alpha, \beta>$ $C(t, s)$ решение системы
  $X'(t) = A(t)X(t)$

  2) $C(t,t) = E$ еденичная матрица
\end{block}