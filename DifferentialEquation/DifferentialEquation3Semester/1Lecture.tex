Лектор Виктория Юрьевна Барсукова

ДУ - дефференциальное уравнение.

OДУ - обыкновенное дефференциальное уравнение.

$y(x)$ - ОДУ

$y(t_0, t_1, \ldots, t_n)$ - частные ДУ

\begin{title}[\Large]
  Задачи приводящие к ДУ.
\end{title}

1) Найти закон движение тела с заданной постоянной скоростью $\upsilon$

$s' (t) = \upsilon$ простейшее ДУ

$s (t) = \upsilon t + C$ решение уравнения эта функция

2) Найти закон движение тела при постоянном ускорения

$s'' (t) = a$ - уравнение 2-го порядка

$s'(t) = at + C_1$

$s(t) = \frac{at^2}{2} + C_1 t + C_2$

или если известна начальная скорость и положение, то

$s(t) = \frac{at^2}{2} + \upsilon_0 t + S_0$

3) Задача о распаде радиактивного вещества. Скорость распада радия
прямопрапарциональны его количества $m(t)$. Опредилить закон изменения
$m(t)$ - массы от времени

$m'(t) = k m(t)$

$m(t) = c e^{kt}$

$m(t_0) = c e^{kt_0} = m_0 ~~~ c = m_0 e^{-kt_0}$

$m(t) = m_0e^{k(t-t_0)}$

\begin{define}[обыкновенного дифференциального уравнения]
  ОДУ называется уравнение содержащее неизвестную функцию одной переменной и ее
  производные или дифференциалы т.е. уравнение вида

  $F(x, y(x), y'(x), \ldots, y^{(m)} (x)) = 0$ или


  $F(x, y(x), dx, dy, d^2 x, d^2 y, \ldots d^n x, d^n y) = 0$

  $$
  y'(x) = \frac{dy}{dx}
  $$
\end{define}

\begin{define}[порядка уравнения]
  Порядком уравнения называется максимальный порядок производной или
  дифференциала входящего в уравнение.
\end{define}

\begin{define}[решения ДУ]
  Решение ДУ называется функция определенная на $<a,b>$ дифференциируема столько
  раз каков порядок уравнения и которая при подстановке в уравнение обращает его
  в тождество.
\end{define}

\begin{title}
  Уравнения первого порядка
\end{title}

$F(x, y(x), y'(x)) = 0$ или $F(x, y(x), dy, dx) = 0$

$y'(x) = f(x, y(x))$ - уравнение разрешенное относительно производной. Тоже что
$\frac{dy(x)}{dx} = f(x, y(x))$

$f(x, y)$ - заданая функция с двумя независимыми переменными. Будем считать что
эта функция определена и непрерывна в некоторой односвязной области $D$.

\begin{center}
  Вопросы решаемые ДУ
\end{center}

1) существует ли решение ДУ

2) сколько решений существует ДУ

3) методы нахождения решения ДУ

4) устойчивость к изменениям ДУ

\begin{define}[решения ДУ]
  Пусть функция $y = \varphi (x)$ определена на $<a,b>$. $\varphi (x)$
  называется решением уравнения $y'(x) = f(x, y(x))$ если

  1) $\varphi (x)$ диференциируема на $<a,b>$

  2) $\forall x \in <a,b> ~~~ (x, \varphi(x)) \in D$

  3) $\frac{d\varphi (x)}{dx} \equiv f(x, \varphi(x)) ~~ \forall x \in <a,b>$
\end{define}

\begin{define}[задачи Коши]
  Задачей Коши для уравнения $y'(x) = f(x, y(x))$ называется следущая задача
  найти такое решение уравнения которое при заданом $x$ принимает
  заданное значение $y_0$, тоесть удовлитворяет условию $y(x_0) = y_0$
  - условие Коши или начальное условие
  $$
  \left\{
  \begin{array}{l}
    y'(x) = f(x, y(x)) \\
    y(x_0) = y_0
  \end{array}
  \right.
  $$
\end{define}

\begin{define}[общего решния ДУ]
  Общим решением уравнения $y'(x) = f(x, y(x))$ называется функция $y = y(x,c)$
  такая что $\forall C$ функция является решением уравнения и любое
  решение уравнение входит в это семейство при некоторых $C$.

  Решение при конкретном $C$ называется частным решением.
\end{define}

\begin{title}[\Large]
  Гометрический смысл уравнения первого порядка
\end{title}

\begin{define}[изоклина]
  $y' = f(x,y) ~~~ y'(x) = \tg \alpha ~~~ \tg \alpha = f(x,y) ~~~
  \tg \alpha = A ~ \Rightarrow ~ f(x,y) = A$

  $f(x, y) = A$ - изоклин
\end{define}

\begin{title}[\Large]
  Интегральные типы уравнений
\end{title}

$$
y'(x) = f(x, y(x))
$$
$$
y'(x) = f(x) ~~~ y(x) = \int f(x)dx
$$
$$
\left( \int_a^x f(t) dt \right)' = f(x) ~~~ \int f(x)dx = \int_a^x f(t)dt + C
$$
$$
y(x) = \int_{x_0}^x f(t)dt + C ~ \text{- общее решение}
$$
$$
y(x) = \int_{x_0}^x f(t)dt + y_0 ~ \text{- решение задачи Коши}
$$

\begin{title}[\Large]
  Типы уравнений
\end{title}

\begin{define}[уравнения с разделяющимися переменными]
  Уравнения с разделяющимися переменными (УРП). Тоесть уравнения которые
  могут быть переведены к такому виду
  $$
  y'(x) = f(x) \cdot g(y(x))
  $$
  $$
  \frac{dy(x)}{dx} = f(x) \cdot g(y(x))
  $$
\end{define}

\begin{block}[Общий вид решения УРП]
  $$
  y'(x) = f(x)g(y(x))
  $$
  $$
  \frac{dy(x)}{dx} = f(x) g(y(x))
  $$
  $$
  \frac{dy(x)}{g(y(x))} = \int f(x)dx ~~~ g(y(x)) \not= 0
  $$
  $$
  \int \frac{dy(x)}{g(y(x))} = \int f(x)dx
  $$
\end{block}

\begin{theorem}[о $\exists !$ решения УРП]
  Пусть $f(x), g(u)$ определены и непрерывны на $x \in <a, b> ~ u \in <c, d>$
  $g(u) \not= 0 ~~~ \forall u \in <c, d>$ тогда
  $\forall x_0 \in <a, b> ~~~ \forall y_0 \in <c, d>$
  $$
  \left\{
  \begin{array}{l}
    y'(x) = f(x, y(x)) \\
    y(x_0) = y_0
  \end{array}
  \right.
  $$
  имеет единственное решение

  Замечание: $\exists m \in <c, d> ~ g(m) = 0$ тогда $y(x) \equiv m$
\end{theorem}

\begin{proof}
  Предположим что $\varphi(x)$ - решение уравнения $y'(x) = f(x, y(x))$
  удовлетворяющее $y(x_0) = y_0$ тоесть $\varphi(x_0) = y_0$
  $$
  \frac{d\varphi(x)}{dx} \equiv f(x)g(\varphi(x)) ~~~ x \in <a, b>
  $$
  $$
  \frac{d(\varphi(x))}{g(\varphi(x))} \equiv f(x)dx
  $$
  $$
  \int_{x_0}^x \frac{d(\varphi(t))}{g(\varphi(t))} = \int_{x_0}^x f(t)dt
  $$
  $$
  u = \varphi(t) ~~~ \int_{\varphi(x_0)}^{\varphi(x)} \frac{du}{g(u)} =
  \int_{x_0}^x f(t)dt
  $$
  $$
  \int_{y_0}^{\varphi(x)} \frac{du}{g(u)} = \int_{x_0}^x f(t)dt
  $$
  $$
  G(u) |_{y_0}^{\varphi(x)} = F(x) - F(x_0)
  $$
  $$
  G(\varphi(x)) = G(y_0) + F(x) - F(x_0)
  $$
  $g(u) \not= 0$ $g(u)$ - сохраняет знак $G'(u) = \frac{1}{g(u)}$ - сохраняет
  знак $G'(u) \not= 0$ $G(u)$ - строго монотонна и непрерывна $\Rightarrow$
  существает обратная ей функция
  $$
  \varphi(x) = G^{-1} \left( G(y_0) + \int_{x_0}^x f(t)dt \right)
  $$
  из едеинственности обратной функции $\Rightarrow$ единственность решения \\

  Докажем существование
  $$
  \varphi(x_0) = G^{-1}(G(y_0)) = y_0
  $$
  $$
  (G^{-1}(y))' = \frac{1}{G'(G^{-1}(y))}
  $$
  $$
  \varphi'(x) = \frac{1}{G' \left( G^{-1} \left( G(y_0)
  + \int_{x_0}^x f(t)dt \right) \right)}
  \left( G(y_0) + \int_{x_0}^x f(t)dt \right)' =
  $$
  $$
  = f(x) \cdot g \left( G^{-1} \left( G(y_0) +
  \int_{x_0}^x f(t)dt \right) \right) = f(x) \cdot g(\varphi(x))
  $$
  $\Rightarrow ~ \varphi(x)$ - решение задачи
\end{proof}

\begin{theorem}
  Если $\int_m \frac{du}{g(u)}$ - расходится, то через каждую точку области
  проходит единственное решение

  Если $\int_m \frac{du}{g(u)}$ - сходится, то в точка $y = m$ единтвсенность
  нарушена
\end{theorem}

\begin{title}[\Large]
  Уравнения приводящие к УРП
\end{title}

\begin{block}[Уравнения сводящиеся к линейной заменой]
  $$
  y'(x) = f(\alpha x + \beta y(x) + \gamma) ~~~ \alpha, \beta, \gamma \in R
  $$
  $$
  z(x) = \alpha x + \beta y(x) + \gamma ~ \Rightarrow z'(x) = \alpha +
  \beta y'(x) ~~~ \beta \not= 0
  $$
  $$
  \frac{z'(x) - \alpha}{\beta} = f(z(x))
  $$
  $$
  z'(x) = \beta f(z(x)) + \alpha ~~ \text{тоже что и} ~~
  z'(x) = 1 \cdot g(z(x))
  $$
\end{block}

\begin{block}[Однородные уравнения]
  $$
  y'(x) = \Phi \left( \frac{y(x)}{x} \right) ~ \text{замена} ~ z(x) =
  \frac{y(x)}{x}
  $$
  $$
  y(x) = z(x) \cdot x
  $$
  $$
  y'(x) = z'(x) x + z(x)
  $$
  $$
  x z'(x) + z(x) = \Phi(z(x))
  $$
  $$
  z'(x) = \frac{\Phi(z(x)) - z(x)}{x}
  $$
\end{block}

\begin{title}[\Large]
  Линейное уравнение 1-ого порядка
\end{title}

\begin{define}[линейного однородного и неоднородного уравнения]
  $y'(x) = a(x)y(x) + b(x)$ - линейное неоднородное уравнение

  $y'(x) = a(x)y(x)$ - линейное однородное уравнение
\end{define}

\begin{block}[Общее решение линейного однородного уравнения]
  $$
  y'(x) = a(x)y(x)
  $$
  $$
  \frac{dy(x)}{dx} = a(x)y(x)
  $$
  $$
  \int \frac{dy(x)}{y} = \int a(x)dx ~~~ y \not= 0
  $$
  $$
  \ln y(x) = \in a(x)dx
  $$
  $$
  |y(x)| = e^{\int_{x_0}^x a(t) dt + C}
  $$
  $$
  y(x) = C e^{\int_{x_0}^x a(t) dt}
  $$
\end{block}

\begin{block}[Метод вариации произвольной постоянной]
  $$
  y(x) = C(x) e^{\int_{x_0}^x a(t) dt}
  $$
  найдем $C(x)$ там чтобы $y(x)$ стал решением
  $$
  y'(x) = C'(x) \cdot e^{\int_{x_0}^x a(t) dt} +
  C(x) \cdot \left( e^{\int_{x_0}^x a(t) dt} \right)' =
  $$
  $$
  = C'(x) \cdot e^{\int_{x_0}^x a(t) dt} +
  C(x) \cdot e^{\int_{x_0}^x a(t) dt} a(x)
  $$
  $$
  C'(x) \cdot e^{\int_{x_0}^x a(t) dt} +
  C(x) \cdot e^{\int_{x_0}^x a(t) dt} a(x) =
  a(x) C(x) \cdot e^{\int_{x_0}^x a(t) dt} + b(x)
  $$
  $$
  C'(x) = \frac{b(x)}{e^{\int_{x_0}^x a(t) dt}} =
  b(x) e^{-\int_{x_0}^x a(t) dt}
  $$
  $$
  C(x) = \int_{x_0}^x b(s) e^{\int_{x_0}^s a(t) dt} ds + D
  $$
  $$
  y(x) = \left( \int_{x_0}^{\alpha} e^{-\int_{x_0}^t a(\tau) d\tau}
  b(t)dt + D \right) \cdot e^{\int_{x_0}^x a(\tau) d\tau} =
  $$
  $$
  = D e^{\int_{x_0}^x a(\tau) d\tau} + \int_{x_0}^x
  e^{\int_{x_0}^x a(\tau) d\tau - \int_{x_0}^x a(\tau) b\tau} d(t)dt
  $$
  $$
  y(x) = D e^{\int_{x_0}^x a(\tau) d\tau} + \int_{x_0}^x
  e^{\int_{t}^x a(\tau) d\tau} b(t)dt
  $$
  общее решение неоднородного уравнения
\end{block}

\begin{block}[Формула Коши]
  $$
  \left\{
  \begin{array}{l}
    y'(x) = a(x)y(x) + b(x) \\
    y(x_0) = y_0
  \end{array}
  \right. ~~~ \text{задача Коши}
  $$
  $$
  y(x) = y_0 e^{\int_{x_0}^x a(\tau) d\tau} + \int_{x_0}^x
  e^{\int_{t}^x a(\tau) d\tau} b(t)dt
  $$

  Введем обозначения

  $$
  K(x, t) = e^{\int_t^x a(\tau) d\tau}
  $$
  $$
  y(x) = y_0 K(x, x_0) + \int_{x_0}^x K(x, t) b(t) dt
  $$
  $K(x, t)$ - функция Коши
\end{block}

\begin{block}[Свойства]
  1) $K(x, x) = 1$

  2) при каждом фиксираванном $t \in <\alpha, \beta> ~ K(x, t)$ есть решение
  $y'(x) = a(x)y(x)$
\end{block}

\begin{theorem}
  Пусть $a(x), b(x)$ определена и непрерывна $\forall \in <\alpha, \beta>$
  тогда задача Коши $\forall x_0 \in <\alpha, \beta> ~~~ \forall y_0 \in R$
  $\exists !$ решение которое выражается
  $$
  y(x) = y_0 e^{\int_{x_0}^x a(\tau) d\tau} + \int_{x_0}^x
  e^{\int_{t}^x a(\tau) d\tau} b(t)dt
  $$
\end{theorem}

\begin{proof}
  Предположим что есть другое решение задачи Коши тогда
  $$
  \varphi(x) = u(x) e^{\int_{x_0}^x a(\tau)d\tau}
  $$
  проделов те же действия что и в методе вариаций получим что
  $\varphi(x) = y(x)$ $\Rightarrow$ единственность есть.

  Проверим что формула Коши действительно определяет решение задачи Коши
  $$
  y'(x) = y_0 e^{\int_{x_0}^x a(\tau) d\tau} a(x) + \left( \int_{x_0}^x
  e^{\int_{t}^x a(\tau) d\tau} b(t)dt \cdot
  e^{\int_{x_0}^x a(\tau)d\tau} \right)' =
  $$
  $$
  \left( \int_{\alpha(x)}^{\beta(x)} F(x, t) dt \right)' =
  F(x, \beta(x)) \beta'(x) - F(x, \alpha(x)) \alpha'(x) +
  \int_{\alpha(x)}^{\beta(x)} F_x(x, t)dt
  $$
  $$
  = y_0 e^{\int_{x_0} a(\tau)d\tau} a(x) + b(x) e^{-\int_{x_0}^x a(\tau) d\tau}
  e^{\int_{x_0}^x a(\tau) d\tau} +
  $$
  $$
  + \int_{x_0}^x b(t)
  e^{-\int_{x_0}^t a(\tau) d\tau} e^{\int_{x_0}^x a(\tau) d\tau} a(x) dt =
  $$
  $$
  a(x) \left( y_0 e^{\int_{x_0}^x a(\tau) d\tau} + \int_{x_0}^x b(t)
  e^{\int_t^x a(\tau) d\tau dt} \right) + b(x) =
  $$
  $$
  = a(x)y(x) + b(x)
  $$
\end{proof}

\begin{block}[Уравнение Бернули]
  $$
  y'(x) = a(x)y(x) + b(x)y^m(x) ~~~ m \not= 0 ~~~ m \not= 1
  $$
  $$
  \frac{y'(x)}{y^m(x)} = a(x) y^{1 - m}(x) + b(x)
  $$
  $$
  z(x) = y^{1 - m}(x)
  $$
  $$
  z'(x) = (1 - m) y^{-m}(x) y'(x) = \frac{(1-m)y'(x)}{y^m(x)} ~ \Rightarrow
  $$
  $$
  \frac{y'(x)}{y^m(x)} = \frac{z'(x)}{1 - m}
  $$
  $$
  z'(x) = (1 - m) a(x)z(x) + (1 - m)b(x) ~
  \text{- линейное уравнение}
  $$
\end{block}

\begin{title}[\Large]
  Уравнение в полных дифференциалах
\end{title}

Если частные производные 2-ого порядка существуют и непрерывны, то она равны
$$
\left( \frac{\partial^2 f}{\partial x \partial y},
\frac{\partial^2 f}{\partial y \partial x} \right)
$$

\begin{define}[уравнения в полных дифференциалах]
  Уравнение вида $P(x, y)dx + Q(x, y)dy = 0$ называется, уравнением в полных
  дифференциалах, если левая часть уравнения есть дифференциал некоторой
  функции тоесть выполнено условие
  $$
  \frac{\partial P(x, y)}{\partial y} = \frac{\partial Q(x, y)}{\partial x}
  $$
  $\Rightarrow ~ \exists F(x, y)$ чей дифференциал стоит слева
  $$
  \left\{
  \begin{array}{l}
    \frac{\partial F(x, y)}{\partial x} = P(x, y) \\
    \frac{\partial F(x, y)}{\partial y} = Q(x, y)
  \end{array}
  \right.
  $$
  $dF(x, y) = 0 ~ \Rightarrow ~ F(x, y) \equiv C$ 1-ый интеграл уравнения
\end{define}

\begin{block}[Решение уравнений в полых дифференциалах в общем виде]
  $$
  F(x, y) = \int_{x_0}^x P(t, y)dt + C(y)
  $$
  $$
  \frac{\partial F(x, y)}{\partial y} =
  \left( \int_{x_0}^x P(t, y)dt + C(y) \right)'_y =
  \int_{x_0}^x \frac{\partial P(t, y)}{\partial y} dt + C'(y) =
  $$
  $$
  = \int_{x_0}^x \frac{\partial Q(t, y)}{\partial t} dt + C'(y) =
  Q(t, y)|_{x_0}^x + C'(y) = Q(x, y) - Q(x_0, y) + C'(y) = Q(x, y)
  $$
  $$
  C'(x) = Q(x_0, y)
  $$
  $$
  C(y) = \int_{y_0}^y Q(x_0, s) ds + D
  $$
  $$
  F(x, y) = \int_{x_0}^x P(t, y) \int_{y_0}^y Q(x_0, s) ds + D
  $$
  $$
  \int_{x_0}^x P(t, y)dt + \int_{y_0}^y Q(x_0, s) ds = C ~ \text{- решение}
  $$
\end{block}

\begin{theorem}
  Пусть $P(x, y) ~ Q(x, y)$ непрерывны в некоторой односвязной области $G$
  $Q(x, y) \not= 0 ~~~ (x,y) \in G$ и существуют непрерывные частные
  производные, тогда $\forall (x_0, y_0) \in G_0$ задача Коши
  $\exists !$ решение
\end{theorem}

\begin{title}
  Система дифференциальных уравнений в нормальной форме
\end{title}

\begin{define}
  $$
  \left\{
  \begin{array}{l}
    x_1'(t) = f_1(t, x_1(t), x_2(t), \ldots, x_n(t)) \\
    x_2'(t) = f_2(t, x_1(t), x_2(t), \ldots, x_n(t)) \\
    \ldots ~~~ \ldots ~~~ \ldots ~~~ \ldots ~~~ \ldots ~~~ \ldots \\
    x_n'(t) = f_n(t, x_1(t), x_2(t), \ldots, x_n(t))
  \end{array}
  \right.
  $$
  $$
  x(t) =
  \left(
  \begin{array}{l}
    x_1(t) \\
    x_2(t) \\
    \ldots \\
    x_n(t)
  \end{array}
  \right) ~~~
  f(t) =
  \left(
  \begin{array}{l}
   f_1(t, x_1(t), x_2(t), \ldots, x_n(t)) \\
   f_2(t, x_1(t), x_2(t), \ldots, x_n(t)) \\
    \ldots ~~~ \ldots ~~~ \ldots ~~~ \ldots ~~~ \ldots \\
   f_n(t, x_1(t), x_2(t), \ldots, x_n(t))
  \end{array}
  \right)
  $$
  $x'(t) = f(t, x(t))$ $x(t)$ - неизвестный вектор функции
  $$
  <\alpha, \beta> \in R^n ~~~ x(t_0) = x_0 ~~~ t_0 \in <\alpha, \beta> ~~~
  x_0 \in R^n
  \left\{
  \begin{array}{l}
    x_1(t_0) = x_{0_1} \\
    x_2(t_0) = x_{0_2} \\
    \ldots ~~~ \ldots \\
    x_n(t_0) = x_{0_n}
  \end{array}
  \right.
  $$
\end{define}

\begin{block}[Решением системы ДУ называется]
  $\varphi(t): ~ <\varphi, \beta> \to R^n$ которая

  1) непрерывна дифференциируема на $<\alpha, \beta>$

  2) $\forall t \in <\alpha, \beta> ~~~ (t, \varphi(t)) \in D$

  3) $\frac{d\varphi(t)}{dt} \equiv f(t, \varphi(t)) ~~~
  t \in <\alpha, \beta>$
\end{block}

\begin{title}[\Large]
  Линейные системы дифференциальных уравнений (ЛСДУ)
\end{title}

$$
\left\{
\begin{array}{l}
  x'_1(t) = a_{11}(t)x_1(t) + a_{12}(t)x_2(t) + \ldots
  + a_{1n}x_n(t) + g_1(t) \\
  x'_2(t) = a_{21}(t)x_1(t) + a_{22}(t)x_2(t) + \ldots
  + a_{2n}x_n(t) + g_2(t) \\
  \ldots ~~~ \ldots ~~~ \ldots ~~~ \ldots ~~~ \ldots ~~~ \ldots ~~~
  \ldots ~~~ \ldots ~~~ \ldots ~~~ \ldots\\
  x'_n(t) = a_{n1}(t)x_1(t) + a_{n2}(t)x_2(t) + \ldots
  + a_{nn}x_n(t) + g_n(t) \\
\end{array}
\right.
$$

$$
x(t) =
\left(
\begin{array}{l}
  x_1(t) \\
  x_2(t) \\
  \ldots \\
  x_n(t)
\end{array}
\right) ~~~
A(t) =
\left(
\begin{array}{cccc}
  a_{11} & a_{12} & \ldots & a_{1n} \\
  a_{21} & a_{22} & \ldots & a_{2n} \\
  \ldots & \ldots & \ldots & \ldots \\
  a_{n1} & a_{n2} & \ldots & a_{nn}
\end{array}
\right) ~~~
g(t) =
\left(
\begin{array}{l}
  g_1(t) \\
  g_2(t) \\
  \ldots \\
  g_n(t) \\
\end{array}
\right)
$$

$x'(t) = A(t)x(t) + g(t)$ - неоднородная система

$x'(t) = A(t)x(t)$ - однородная система

\begin{title}[\Large]
  Теорема о существовании единственности для линейных систем. Рещение задачи
  Коши
\end{title}

\begin{theorem}
  Пусть $A(t) g(t)$ непрерывны на $<\alpha, \beta>$ тогда
  $\forall t_0 \in <\alpha, \beta> ~~~ \forall x_0 \in R^n$ задачи Коши
  $\exists !$ решение на $<\alpha, \beta>$
\end{theorem}

\begin{block}[Лемма]
  Задача Коши эквивалентна
  $$
  x(t) = x_0 + \int_{t_0}^t A(s)x(s)dx +
  \int_{t_0}^t g(s)ds
  $$
  тоесть любое решение задачи Коши явялется решением этого уравнения и наоборот
\end{block}

\begin{proof}
  Доказательство в одну сторону
  Пусть $\varphi$ - решение задачи Коши тогда
  $$
  \varphi'(t) = A(t)\varphi(t) + g(t)
  $$
  $$
  \int_{t_0}^t \varphi'(s)ds = \int_{t_0}^t A(s) \varphi(s)ds +
  \int_{t_0}^t g(s) ds
  $$
  $$
  \int_{t_0}^t \varphi'(s)ds = \varphi(s)|_{t_0}^t = \varphi(t) - x_0
  $$
  $$
  \varphi(t) = x_0 + \int_{t_0}^t A(s) \varphi(s) ds  + \int_{t_0}^t g(s)ds
  $$
  Доказательство в другую сторону

  Пусть $\varphi(t)$ - решение
  $$
  x(t) = x_0 + \int_{t_0}^t A(s)x(s)dx +
  \int_{t_0}^t g(s)ds
  $$
  тогда $\varphi(t)$ - непрерывная $\Rightarrow$ дифференцируема

  $\varphi'(t) = A(t)\varphi(t) + g(t)$

  $\varphi(t_0) = x_0$
\end{proof}

\begin{title}[\Large]
  Множество решений линейной системы
\end{title}

\begin{theorem}[прицапа суперпозии решений]
  $\varphi(t)$ решение $x'(t) = A(t)x(t) + g_1(t)$

  $\psi(t)$ решение $x'(t) = A(t)x(t) + g_2(t)$

  тогда $C\varphi(t) + B\psi(t)$ решение $x'(t) = A(t)x(t) + Cg_1(t) + Bg_2(t)$
  $C,B$ - числа
\end{theorem}

\begin{proof}
  $$
  C\varphi'(t) + B\varphi'(t) = A(t)(C\varphi(t) + B\psi(t)) + Cg_1(t) +
  Bg_2(t)
  $$
  $$
  C\varphi'(t) + B\varphi'(t) = C(A(t)\varphi(t) + g_1(t))+
  B(A(t)\psi(t) + g_2(t))
  $$
  $$
  \Rightarrow ~ A(t)\varphi(t) + g_1(t) = \varphi'(t) ~~~
  A(t)\psi(t) + g_2(t) = \psi'(t)
  $$
\end{proof}

\begin{block}[Следствие 1]
  Разность двух решений неоднородной системы $x'(t) = A(t)x(t) + g(t)$
  является решением неоднородной системы $x'(t) = A(t)x(t)$
\end{block}

\begin{proof}
  $x'(t) = A(t)x(t) + g(t) ~~ C = 1 ~~ B = 1$

  $\varphi(t) - \psi(t)$

  $g(t) - g(t) = 0$
\end{proof}

\begin{block}[Слудсвие 2]
  Если $\varphi_0(t)$ некоторое решение сисетмы $x'(t) = A(t)x(t) + g(t)$ то
  множество всех решений этой системы совпадает с множеством следующего вида
  $\{\varphi_0(t) + \varphi(t)\}$ где $\varphi(t)$ пробегает множество решений
  однородной системы
\end{block}

\begin{proof}
  $\varphi_0(t)$ решение $x'(t) = A(t)x(t) + g(t)$

  $\varphi(t)$ решение $x'(t) = A(t)x(t)$

  $\varphi_0(t) + \varphi(t)$ решение $x'(t) = A(t)x(t) + g(t)$

  Пусть $\varphi(t)$ решение $x'(t) = A(t)x(t) + g(t)$

  $\psi(t) - \varphi_0(t)$ решение $x'(t) = A(t)x(t)$

  $\Rightarrow ~ \varphi(t) = \varphi_0(t) - \varphi(t)$
\end{proof}
