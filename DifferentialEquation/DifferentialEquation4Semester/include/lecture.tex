\begin{title}[\Large]
  Линейные дифференциальные уравнения n-ого порядка. Эквивалентность линейной
  системе
\end{title}

\begin{define}[линейных уравнений n-ого порядка]
  $(1) ~~ x^{(n)} + a_{n-1}(t)x^{(n-1)} + \ldots + a_1(t) x'(t) + a_0(t)x(t)
  = g(t)$

  $x(t)$ неизвестная функция

  $a_0(t), a_1(t), \ldots, a_{n-1}(t)$ непрерывные функции

  $g(t)$ непрерывная фукнция (свободный член)

  $t \in <\alpha, \beta>$

  Если $g(t) \equiv 0$ то уравнение наызвается однородным
  $$
  \left\{
  \begin{array}{l}
    x_1(t) = x(t) \\
    x_2(t) = x'(t) \\
    x_3(t) = x''(t) \\
    \ldots ~~~ \ldots \\
    x'_n(t) = g(t) - a_{n-1}(t)x^{(n-1)} - \ldots - a_1(t) x'(t) - a_0(t)x(t)
  \end{array}
  \right.
  $$
  $$
  A(t) =
  \left(
  \begin{array}{ccccc}
    0 & 1 & 0 & \cdots & 0 \\
    0 & 0 & 1 & \cdots & 0 \\
    \cdots & \cdots & \cdots & \cdots & \cdots  \\
    0 & 0 & 0 & \cdots & 1 \\
    -a_0(t) & -a_1(t) & -a_2(t) & \cdots & -a_{n-1}(t)
  \end{array}
  \right)
  ~~~
  b(t) =
  \left(
  \begin{array}{c}
    0 \\
    0 \\
    \vdots \\
    0 \\
    g(t)
  \end{array}
  \right)
  $$
  $(3) ~~ \varphi'(t) = A\varphi(t) + b(t) ~~~ \varphi(t_0) = \varphi_0$
  $$
  \varphi(t) =
  \left(
  \begin{array}{c}
    x_1(t) \\
    x_2(t) \\
    \vdots \\
    x_n(t)
  \end{array}
  \right)
  ~~~
  \varphi_0 =
  \left(
  \begin{array}{c}
    x_0 \\
    x_1 \\
    \vdots \\
    x_{n-1}
  \end{array}
  \right)
  $$
\end{define}

\begin{define}[задачи Коши для линейных уравнений n-ого порядка]
  Задача Коши для линейных уравнений n-ого порядка
  $$
  \left\{
  \begin{array}{l}
    x(t_0) = x_0 \\
    x'(t_0) = x_1 \\
    x''(t_0) = x_2 \\
    \ldots ~~~ \ldots \\
    x^{(n-1)}(t_0) = x_{n-1}
  \end{array}
  \right.
  ~~~
  t_0 \in <\alpha, \beta>
  ~~~
  x_0, x_1, \ldots, x_{n-1} \in C
  $$
\end{define}

\begin{block}[Утверждение]
  Уравнение (1) и система (3) эквивалентны, то есть если $x(t)$ решение
  уравнения (1) то является решением системы (3).

  Обратно, если $\varphi(t)$ решение системы (3) то его первая компонента
  $\varphi_1(t)$ решение уравнения (1)
\end{block}

\begin{proof}
  $\Rightarrow$

  Пусть $\varphi(t)$ решение системы (3)
  $$
  \varphi(t) =
  \left(
  \begin{array}{c}
    \varphi_1(t) \\
    \varphi_2(t) \\
    \vdots \\
    \varphi_n(t) \\
  \end{array}
  \right)
  ~~
  \text{решение}
  ~
  \Rightarrow
  ~
  \left\{
  \begin{array}{l}
    \varphi'_1(t) = \varphi_2(t) \\
    \varphi'_2(t) = \varphi_3(t) \\
    \cdots ~~~ \cdots ~~~ \cdots \\
    \varphi'_{n-1}(t) = \varphi_n(t) \\
    \varphi'_n(t) = g(t) - \sum_{j=0}^{n-1} a_j(t) \varphi_{j+1}(t)
  \end{array}
  \right.
  $$
  $\Rightarrow ~~ \varphi'_n(t) = \varphi_1^{(n)}(t) =
    g(t) - \sum_{j=0}^{n-1} a_j(t) \varphi_1^{(j)}(t)$

  $\varphi_1^{(n)} + a_{n-1}(t) \varphi_1^{(n-1)}(t) + \ldots +
  a_1(t)\varphi'_1(t) + a_0(t) \varphi_1(t) = g(t)$

  $\Leftarrow$
  $$
  \begin{array}{ll}
    x'_1(t) = x_2(t) & (x(t))' = x'(t) \\
    x'_2(t) = x_3(t) & (x'(t))' = x''(t) \\
    \cdots ~~~ \cdots ~~~ \cdots &
    \cdots ~~~ \cdots ~~~ \cdots \\
    x'_{n-1}(t) = x_n(t) & (x^{n-1}(t))' = x^{(n-1)}(t)
  \end{array}
  $$
  Пусть $x(t)$ решение (1)
  $$
  x'_n(t) = g(t) - \sum_{j=0}^{n-1} a_j(t) x_{j+1}(t)
  $$
  $$
  x^n(t) = g(t) - \sum_{j=0}^{n-1} a_j(t) x^{(j)}(t)
  $$
\end{proof}

\begin{block}[Заменачание]
  Если $n$ раз дифференцируемыые функции $\varphi_1(t), \ldots, \varphi_n(t)$
  ЛНЗ тогда
  $$
  \left(
  \begin{array}{c}
    \varphi_1(t) \\
    \varphi'_1(t) \\
    \vdots \\
    \varphi_1^{(n-1)}(t)
  \end{array}
  \right)
  ,
  \left(
  \begin{array}{c}
    \varphi_2(t) \\
    \varphi'_2(t) \\
    \vdots \\
    \varphi_2^{(n-1)}(t)
  \end{array}
  \right)
  , \ldots ,
  \left(
  \begin{array}{c}
    \varphi_n(t) \\
    \varphi'_n(t) \\
    \vdots \\
    \varphi_n^{(n-1)}(t)
  \end{array}
  \right)
  $$
\end{block}

\begin{theorem}[о $\exists !$ задачи Коши (1,2)]
  $a_0(t), a_1(t), \ldots, a_{n-1}(t), g(t)$ непрерывные на $<\alpha, \beta>$
  тогда
  $\forall t_0 \in <\alpha, \beta>
  ~~~
  \forall x_0, x_1, \ldots, x_{n-1} \in C$ задача Коши $(1,2)$ имеет и притом
  единственное решение
\end{theorem}

\begin{title}[\Large]
  Теорема о пространстве решений линейного однородного дифференциального
  уравнения n-ого порядка
\end{title}

\begin{theorem}
  Множество решений однородного уравнения
  $$
  x^{(n)}(t) + \sum_{j=0}^{n-1} a_j(t)x^{(j)}(t) = 0
  $$
  образует линейное пространство размерности $n$
\end{theorem}

\begin{title}[\Large]
  Линейная зависимость и независимость функций. Критерий линейной независимости
  решений однородного дифференциального уравнения n-ого порядка
\end{title}

\begin{block}[Критерий линейной независимости решений]
  Пусть $\varphi_1(t), \ldots, \varphi_n(t)$ решение
  $$
  x^{(n)}(t) + \sum_{j=0}^{n-1} a_j(t)x^{(j)}(t) = 0
  $$
  однородного уравнения

  1) если $W(t) \not= 0$ хотябы в одной точке то решение ЛНЗ

  2) если решение $\varphi_1(t), \ldots, \varphi_n(t)$ ЛНЗ $\Rightarrow$
  $\forall t \in <\alpha, \beta> ~~~ W(t) \not= 0$
  $$
  W(t) = W(t_0) e^{-\int_{t_0}^t a_{n-1}(s)ds}
  ~~~
  \text{формула Леовиля Остраградского}
  $$
\end{block}

\begin{block}[Следствие]
  Вронскиан системы решений $W$ либо ни в одной точке $\not= 0$ либо
  тождественнвый ноль
\end{block}

\begin{title}[\Large]
  Фундаментальная система решений, ее связь с общим решением уравнения
\end{title}

\begin{theorem}
  Множество решений однородного уравнения
  это ФСР.
  $$
  (1_0) ~~ x^{(n)}(t) + \sum_{j=0}^{n-1} a_j(t)x^{(j)}(t) = 0
  $$
  Если $\varphi_1(t), \ldots, \varphi_n(t)$ ФСР то множество решений ($1_0$)
  $x(t) = C_1 \varphi_1(t) + \ldots + C_n \varphi_n(t)$ общее решение ($1_0$)
\end{theorem}

\begin{define}[определителя Вронского]
  Определитель Вронского фукнций $\varphi_1(t), \ldots, \varphi_n(t)$
  $$
  \left|
  \begin{array}{ccc}
    \varphi_1(t) & \cdots & \varphi_n(t) \\
    \varphi'_1(t) & \cdots & \varphi'_n(t) \\
    \cdots &
    \cdots &
    \cdots \\
    \varphi_1^{(n-1)}(t) & \cdots & \varphi_n^{(n-1)}(t)
  \end{array}
  \right|
  $$
\end{define}

\begin{title}[\Large]
  Линейные неоднородное дифференциальное уравнение n-ого порядка. Принцип
  суперпозиции решений и следствия из него
\end{title}

\begin{theorem}[принцип суперпозиции решений]
  Пусть

  $\varphi_1(t)$ решение уравнения $x^{(n)}(t) +
  \sum_{j=0}^{n-1} a_j(t)x^{(j)}(t) = g_1(t)$

  $\varphi_2(t)$ решение уравнения $x^{(n)}(t) +
  \sum_{j=0}^{n-1} a_j(t)x^{(j)}(t) = g_2(t)$

  тогда $\gamma_1\varphi_1(t) + \gamma_2\varphi_2(t)$ решение
\end{theorem}

\begin{block}[Следствия]
  1) Если $\varphi_1(t), \varphi_2(t)$ решения уравнения (1) тогда
  $\varphi_1(t) - \varphi_2(t)$ решение однородного уравнения

  2) Если $\varphi(t)$ решение уравнения (1) a $\varphi_0(t)$ пробегает
  множество решений однородного уравнения тогда множество решений (1) совпадает
  с множеством функций $\{\varphi(t) + \varphi_0(t)\}$
\end{block}

\begin{title}[\Large]
  Метод вариации для линейного неоднородного дифференциального уравнения
  n-ого порядка
\end{title}

\begin{block}[Метод вариации]
  $(1) ~~ x^{(n)}(t) + a_{n-1}(t)x^{(n-1)}(t) + \ldots + a_1(t)x'(t) +
  a_0(t)x(t) = b(t)$

  $\varphi_1(t), \ldots, \varphi_n(t)$ ФСР линейного однородного уравнения

  $x(t) = C_1\varphi_1(t) + \ldots + C_n\varphi_n(t)$ общее решение линейного
  однородного уравнения

  Общее решение (1) имеем в виде $\varphi(t) = u_1(t)\varphi_1(t) + \ldots +
  u_n(t)\varphi_n(t)$
  $$
  \begin{array}{l}
    x'(t) = u_1(t) \varphi'_1(t) + \ldots + u_n(t) \varphi'_n(t) +
    u'_1(t)\varphi_1(t) + \ldots + u'_n(t) \varphi_n(t) \\

    x''(t) = u_1(t) \varphi''_1(t) + \ldots + u_n(t) \varphi''_n(t) +
    u'_1(t)\varphi_1'(t) + \ldots + u'_n(t) \varphi'_n(t) \\

    \ldots ~~~ \ldots ~~~ \ldots ~~~ \ldots ~~~ \ldots ~~~ \ldots ~~~ \ldots
    ~~~ \ldots ~~~ \ldots ~~~ \ldots ~~~ \ldots ~~~ \ldots ~~~ \ldots \\

    x^{(n)}(t) = u_1(t) \varphi^{(n)}_1(t) + \ldots + u_n(t) \varphi^{(n)}_n(t) +
    u'_1(t)\varphi_1^{(n-1)}(t) + \ldots + u'_n(t) \varphi^{(n-1)}_n(t) \\
  \end{array}
  $$
  подставим в уравнение
  $$
  \sum_{j=1}^n u_j(t) \varphi_j^{(n)}(t)
  +
  \sum_{j=1}^n u'_j(t) \varphi_j^{(n-1)}(t)
  +
  a_{n-1}(t)\sum_{j=1}^n u'_j(t) \varphi_j^{(n-1)}(t)
  + \ldots +
  $$
  $$
  +
  a_1(t)\sum_{j=1}^n u_j(t) \varphi'_j(t)
  +
  a_0(t)\sum_{j=1}^n u_j(t) \varphi_j(t)
  =
  b(t)
  $$
  $$
  u_1(t)
  \left(
    \varphi_1^{(n)}(t) + a_{n-1}(t)\varphi_1^{(n-1)} + \ldots + a_1(t)
    \varphi'_1(t) + a_0(t)\varphi_1(t)
  \right)
  + \ldots
  $$
  $$
  \ldots + u_n(t)
  \left(
    \varphi_n^{(n)} + \ldots + a_0(t)\varphi_0(t)
  \right)
  + u_1'(t)\varphi_1^{(n-1)} + \ldots + u_n'(t) \varphi_n^{(n-1)}(t) = b(t)
  $$
  $$
  \left\{
  \begin{array}{l}
    u'_1(t)\varphi(t) + \cdots + u'_n(t)\varphi_n(t) = 0 \\
    u'_1(t)\varphi'(t) + \cdots + u'_n(t)\varphi'_n(t) = 0 \\
    \cdots ~~~ \cdots ~~~ \cdots ~~~ \cdots ~~~ \cdots ~~~ \cdots \\
    u'_1(t)\varphi^{(n-2)}(t) + \cdots + u'_n(t)\varphi^{(n-2)}_n(t) = 0 \\
    u'_1(t)\varphi^{(n-1)}(t) + \cdots + u'_n(t)\varphi^{(n-1)}_n(t) = b(t)
  \end{array}
  \right.
  $$
  Метод Крамера
  $$
  \Delta(t) =
  \left|
  \begin{array}{ccc}
    \varphi_1(t) & \cdots & \varphi_n(t) \\
    \varphi'_1(t) & \cdots & \varphi'_n(t) \\
    \vdots & \cdots & \vdots \\
    \varphi^{(n-1)}_1(t) & \cdots & \varphi^{(n-1)}_n(t) \\
  \end{array}
  \right|
  = W(t) \not= 0
  $$
  $$
  \Delta(t) =
  \left|
  \begin{array}{cccc}
    0 & \varphi_1(t) & \cdots & \varphi_n(t) \\
    0 & \varphi'_2(t) & \cdots & \varphi'_n(t) \\
    \vdots & \vdots & \cdots & \vdots \\
    b(t) & \varphi^{(n-1)}_n(t) & \cdots & \varphi^{(n-1)}_n(t) \\
  \end{array}
  \right|
  = b(t)A_{n1}(t)
  $$
  $\Delta_j(t) = b(t)A_{nj}(t)$
  $$
  u'_j(t) = \frac{b(t)A_{nj}(t)}{w(t)}
  $$
  $$
  u_j(t) = \int_{t_0}^t \frac{b(s)A_{nj}(s)}{w(s)}ds + D_j
  ~~~
  t_0 \in <\alpha, \beta>
  $$
  $$
  x(t) = \sum_{j=1}^n u_j(t)\varphi_j(t) = D_1\varphi_1(t) + \ldots +
  D_n\varphi_n(t) +
  $$
  $$
  + \varphi_1(t) \int_{t_0}^t \frac{b(s)A_{nj}(s)}{w(s)}ds + \ldots +
  \varphi_n(t) \int_{t_0}^t \frac{b(s)A_{nn}(s)}{w(s)}ds =
  $$
  $$
  = \sum_{j=1}^n u_j(t)\varphi_j(t) = D_1\varphi_1(t) + \ldots +
  D_n\varphi_n(t) +
  $$
  $$
  + \int_{t_0}^t \frac{b(s)(\varphi_1(t) A_{nj}(s) + \ldots +
 \varphi_n(t) A_{nn}(s))}{w(s)}ds =
  $$
  $$
  = D_1\varphi_1(t) + \ldots + D_n\varphi_n(t) + \int_{t_0}^t \frac{
    \left|
    \begin{array}{ccc}
      \varphi_1(s) & \cdots & \varphi_n(s) \\
      \varphi'_1(s) & \cdots & \varphi'_n(s) \\
      \vdots & \cdots & \vdots \\
      \varphi^n_1(s) & \cdots & \varphi^n_n(s)
    \end{array}
    \right|
  }{w(s)}b(s)ds
  $$
\end{block}

\begin{title}[\Large]
  Функция Коши. Формула Коши для линейного неоднородного дифференциального
  уравнения n-ого порядка
\end{title}

\begin{block}[Функция Коши]
  $$
  C(t,s) = \frac{
    \left|
    \begin{array}{ccc}
      \varphi_1(s) & 8->\cdots & \varphi_n(s) \\
      \varphi'_1(s) & \cdots & \varphi'_n(s) \\
      \vdots & \cdots & \vdots \\
      \varphi^{(n-2)}_1(s) & \cdots & \varphi^{(n-2)}_n(s) \\
      \varphi^n_1(s) & \cdots & \varphi^n_n(s)
    \end{array}
    \right|
  }{
    \left|
    \begin{array}{ccc}
      \varphi_1(s) & \cdots & \varphi_n(s) \\
      \varphi'_1(s) & \cdots & \varphi'_n(s) \\
      \vdots & \cdots & \vdots \\
      \varphi^{(n-1)}_1(s) & \cdots & \varphi^{(n-1)}_n(s) \\
    \end{array}
    \right|
  }
  $$
  $$
  x(t) = D_1\varphi_1(t) + \ldots D_n\varphi_n(t) + \int_{t_0}^t C(t,s)b(s) ds
  $$
  решение формулы Коши
\end{block}

\begin{block}[Свойства функции Коши]
  1) $\forall$ фиксированного $s$ функция Коши удовлитворяет однородному
  уравнению $(1^0)$

  2) Посчитает функцию Коши в точке $C(s,s) = 0$

  УЗНАТЬ О ТОМ ЧТО СДЕСЬ ДОЛЖНО БЫТЬ
\end{block}

\begin{title}[\Large]
  Теорема о фундаментальной системе решений линейного дифференциального
  уравнения n-ого порядка с постоянными коэффициентами
\end{title}

\begin{define}[уравнений с постоянными коэффициентами]
  $a_j(t) \equiv a_j$

  $x^{(n)} + a_{n-1} x^{(n-1)}(t) + \ldots + a_0x(t) = 0 ~~~ (1_0)$
  $$
  A =
  \left(
  \begin{array}{ccccc}
    0 & 1 & 0 & \cdots & 0 \\
    0 & 0 & 1 & \cdots & 0 \\
    \cdots & \cdots & \cdots & \cdots & \cdots \\
    0 & 0 & 0 & \cdots & 1 \\
    -a_0 & -a_1 & -a_2 & \cdots & -a_{n-1}
  \end{array}
  \right)
  $$
  $$
  |A - \lambda E| =
  \left|
  \begin{array}{ccccc}
    -\lambda & 1 & 0 & \cdots & 0 \\
    0 & -\lambda & 1 & \cdots & 0 \\
    \cdots & \cdots & \cdots & \cdots & \cdots \\
    0 & 0 & 0 & \cdots  &  1\\
    -a_0 & -a_1 & -a_2 & \cdots & -a_{n-1} - \lambda
  \end{array}
  \right|
  = -(\lambda^3 + a_{n-1}\lambda^{n-1} + \ldots + a_1\lambda a_0)(-1)^{n+1}
  $$

  $|A - \lambda E| = 0$

  $\lambda^3 + a_{n-1}\lambda^{n-1} + \ldots + a_1\lambda a_0 = 0$
  характеристическое уравнение для уравнения $(1_0)$
\end{define}

\begin{theorem}
  Пусть $\lambda_1, \lambda_2, \ldots, \lambda_k$ корни характеристического
  уравнения

  $\lambda^n + a_{n-1} \lambda^{n-1} + \ldots + a_n\lambda + a_0 = 0$ кратностей
  $n_1, n_2, \ldots, n_k$ где $n_1, + \ldots + n_k = n$

  $\forall \lambda_j$ соответствует $n_j$ ЛНЗ решений вида $e^{\lambda_j t},
  te^{\lambda_j t}, \ldots, t^{n_j - 1} e^{\lambda_j t}$

  объеденение таких решений по всем $j$ дает ФСР.
\end{theorem}

\begin{proof}
  $(1^0) ~~~ x' = Ax(t)$ по теореме о ФСР

  $\lambda_1, \ldots, \lambda_k$

  $n_1, \ldots, n_k$

  $\lambda_j ~~~ \exists n_j$ решений вида (ЛНЗ) $P_{n_j - 1}(t)e^{\lambda_j t}$

  $\lambda_j$ соответствует $\sum_{m=0}^{n_j - 1} P^{(l)}_m t^m e^{\lambda_j t}
  l = 1, \ldots, n_j$

  $\varphi_l(t)\sum_{m=0}^{n_j - 1} P^{(l)}_m t^m e^{\lambda_j t}$ первые
  компоненты ЛНЗ решений уравнения $(1^0)$ (многочлены образуют базис)

  $1, t, t^2, \ldots, t^{n_j - 1}$ перейдем к новому базису тогда каждое решение
  будет выражаться через ЛНЗ корни вида $e^{\lambda_j t}, te^{\lambda_j t},
  \ldots, t^{n_j - 1} e^{\lambda_j t}$ таким образом решение образует ФСР.
\end{proof}

\begin{block}[Замечание]
  Если все коэффициенты уравнения вещественные то

  $\lambda_1 = \alpha+ i\beta$

  $\lambda_2 = \overline{\lambda_1} = \alpha - i\beta$ корень кратности $n_1$

  в этом случае обязательно переходим к вещественной ФСР

  $e^{(\alpha + i\beta)t}, te^{(\alpha - i\beta)t}, \ldots,
  t^{n_1 - 1}e^{(\alpha - i\beta)t} - \alpha + i\beta$

  $e^{(\alpha + i\beta)t}, te^{(\alpha - i\beta)t}, \ldots,
  t^{n_1 - 1}e^{(\alpha - i\beta)t} - \alpha - i\beta$
  $$
  e^{(\alpha \pm i\beta)t} = e^{\alpha t}(\cos \beta t \pm i \sin \beta t)
  $$
  $e^{\alpha t} \cos \beta t, t e^{\alpha t} \cos \beta t, \ldots,
  t^{n_1 - 1} e^{\alpha t} \cos \beta t$

  $e^{\alpha t} \sin \beta t, t e^{\alpha t} \sin \beta t, \ldots,
  t^{n_1 - 1} e^{\alpha t} \sin \beta t$

  набор вещественных решений
\end{block}

\begin{title}[\Large]
  Нахождение частного решения линейного неоднородного дифференциального
  уравнения n-ого порядка по виду f(x)
\end{title}

\begin{define}[неоднородного уравнения с постоянными коэффицентами]
  $$
  x^{(n)}(t) + \ldots + a_1 x'(t) + a_0 x(t) = f(t)
  $$
\end{define}

\begin{theorem}[определение частного решения во виду правой части]
  $x_{\text{о.р.н}} = x_{\text{о.р.о}} + x_{\text{ч.р.н}}$

  I $f(t) = P_m(t) = b_0 + b_1 t + \ldots b_m t^m$

  1) $\lambda = 0$ не является корнем характеристического уравнения
  $\lambda^n + a_{n-1} \lambda^{n-1} + \ldots + a_0 = 0$

  $x_{\text{ч.н}}(t) = Q_m(t)$ многочлен той же самой степени но с другими
  коэффицентами

  2) $\lambda = 0$ корень характеристического уравнения кратности $l$

  $x_{\text{ч.н}}(t) = t^l Q_m(t)$

  II $f(t) = P_m(t) e^{\alpha t}$

  1) $\lambda = \alpha$ не корень характеристического уравнения

  $\Rightarrow ~ x_{\text{ч.н}}(t) = Q_m(t) e^{\alpha t}$

  2) $\lambda = \alpha$ корень характеристического уаравнения кратности $l$

  $\Rightarrow ~ x_{\text{ч.н}}(t) = t^l Q_m(t) e^{\alpha t}$

  III $f(t) = e^{\alpha t} (P_m(t) \cos \beta t + Q_k(t) \sin \beta t)$

  1) $\lambda = \alpha \pm i\beta$ не корень характеристического уравнения

  $\Rightarrow ~ x_{\text{ч.н}}(t) = e^{\alpha t}(T_2(t) \cos \beta t + S_2(t)
  \sin \beta t) ~~~ r = max\{ k,m \}$

  2) $\alpha \pm i\beta$ корни характеристического уравнения кратности $l$

  $\Rightarrow ~ x_{\text{ч.н}}(t) = t^l e^{\alpha t}(T_2(t) \cos \beta t +
  S_2(t) \sin \beta t)$
\end{theorem}

\begin{define}[устойчивости по Ляпунову]
  $$
  \left\{
  \begin{array}{l}
    x'(t) = f(t, x(t)) \\
    x(t_0) = x_0
  \end{array}
  \right.
  ~~~~
  \left\{
  \begin{array}{l}
    x'(t) = f_1(t, x(t)) \\
    x(t_0) = x_1
  \end{array}
  \right.
  t \in [a,b] ~~~ t_0, t_1 \in (a,b)
  $$
  $$
  \forall \varepsilon > 0 ~~ \exists \delta_{\varepsilon} ~~ ||x_1 - x_0||
  < \delta ~~ |t_0 - t_1| < \delta_1
  $$
  $$
  \sup ||f(t,x) - f_1(t,x)|| < \delta
  ~ \Rightarrow ~ \sup_{t \in [a,b]} ||\varphi(t) - \psi(t)|| < \varepsilon
  $$
\end{define}

\begin{define}
  Решение $\varphi(t)$ системы (1) определенное на $t \in [t_0, \infty]$
  называется устойчивой по Ляпунову если $\forall \varepsilon > 0 ~~ \exists
  \delta$ для любого другого решения (1) $\psi(t)$ удовлетворяющего условию
  $$
  ||\varphi(t_0) - \varphi(t_0)|| < \delta ~ \Rightarrow ~ \sup||\varphi(t) -
  \psi(t)|| < \varepsilon
  $$
\end{define}

\begin{define}[асимтотической устойчивости решения]
  Решение $\varphi$ системы (1) называется ассмтотически устойчивым если

  1) оно $\varphi(t)$ устойчиво

  2) $\exists \Delta > 0 ~~ \forall \psi(t): ~~ || \varphi(t_0) - \psi(t_0)||
  < \Delta$
  $$
  \Rightarrow ~ \lim_{t \to +\infty} ||\varphi(t) - \psi(t)|| = 0
  $$
\end{define}


\begin{define}[неустойчивости решения]
  $\varphi(t)$ решение системы (1) называется неустойчивым если
  $$
  \exists \varepsilon_0 > 0 ~~ \forall \delta > 0 ~~ \exists \text{решение}
  \psi_{\delta}(t) ~~ ||\varphi(t_0) - \psi(t_0)|| < \delta ~ \Rightarrow ~
  \exists t \ge t_0 ~~ || \varphi(t) - \psi(t)|| \ge \varepsilon_0
  $$
\end{define}

\begin{define}[тривиального решения]
  Тривиальное решение системы называется устойчивым если
  $$
  \forall \varepsilon > 0 ~~ \exists \delta_{\varepsilon} ~~ \forall \psi(t) ~~
  ||\psi(t_0)|| < \delta \Rightarrow \sup_{t \ge t_0} ||\psi(t)|| < \varepsilon
  $$
\end{define}

\begin{define}[тривиального решения асимтотической устойчивости]
  $$
  \exists \Delta ~~ \forall \psi(t) ~~ ||\psi(t_0)|| < \Delta ~~
  \lim_{t \to + \infty} ||\psi(t)|| = 0
  $$
\end{define}

\begin{define}[тривиального неустойчивого решения]
  Тривиальное решение неустойчиво если
  $$
  \exists \varepsilon_0 > 0 ~~ \forall \delta ~~ \exists \psi_{\delta}(t) ~~
  ||\psi(t_0)|| < \delta ~~ \exists \overline{t} \ge t_0 ~~
  ||\psi_{\delta}(\overline{t})|| > \varepsilon_0
  $$
\end{define}

\begin{title}[\Large]
  Система приведенная по решению
\end{title}

$x'(t) = f(t, x(t))$

$\varphi(t)$ решение фиксированно

$x(t)$ любое другое решение (1)

$y(t) = x(t) - \varphi(t)$

$$
y'(t) = x'(t) - \varphi'(t) = f(t, x(t)) - f(t, \varphi(t) = \frac{f(t, y(t) +
\varphi(t)) - f(t, \varphi(t))}{F(t, y(t))}
$$

(2) $y'(t) = F(t, y(t)) ~~~ y(t) \equiv 0$ система приведенная по решению
$\varphi(t)$

\begin{block}[Утверждение]
  Решение $\varphi(t)$ системы (1) устойчиво (асомтотически устойчиво) или
  неустойчиво $\Leftrightarrow$ когда устойчиво (ассимтотически устойчиво) или
  неустойчивое тривиальное решение приведенной системы (2)
\end{block}

\begin{proof}
  $$
  \forall \varepsilon > 0 ~~ \exists \delta ~~ \forall \psi(t) ~~
  ||\varphi(t_0) - \psi(t_0)|| < \delta ~ \Leftrightarrow ~ \sup_{t \ge t_0}
  ||\varphi(t) - \psi(t)|| = \sup_{t \ge t_0} y(t) < \varepsilon
  $$
  выпоняется условие устойчивости по тривиальному решению
\end{proof}

\begin{block}[Утверждение]
  Либо все решения всех систем с матрицей $A(t)$ устойчивы либо все неустойчивы

  $\varphi(t)$ решение

  $x(t) = y(t) + \varphi(t)$

  $x'(t) = y'(t) + \varphi'(t)$

  $y'(t) + \varphi'(t) = A(t)(y(t) + \varphi(t)) + b(t)$

  $\varphi'(t) = A(t)\varphi(t) + b(t) ~ \Rightarrow ~ y'(t) = A(t)y(t)$
  система тривиальная по любому решению совпадает с однородной системой
  $\Rightarrow$ $\varphi(t)$ устойчиво $\Leftrightarrow$ тогда устойчиво
  тривиальное решение системы (1). Аналагочино снеустойчивостью.

  Все определяется только матрицей $A(t)$
\end{block}

\begin{define}
  Матрица $A(t)$ и система (1) называется устойчивыми (асимптотически устойчиво)
  или неустойчивыми) если устойчива (ассимтотически устойчиво или неустойчиво)
  тривиальное решение соответствующеей однородной системы.
\end{define}

\begin{block}[Утверждение 1]
  $x'(t) = A(t)x(t) ~~~ (1_0)$
  Система ($1_0$) устойчива $\Leftrightarrow$ когда все ее решения ограничены
\end{block}

\begin{proof}
  Пусть ($1_0$) устойчива $\Rightarrow$ по определению $x \equiv 0$ устойчива
  то есть
  $$
  \forall \varepsilon > 0 ~~ \exists \delta_{\varepsilon} ~~ \forall \varphi(t)
  ~~ ||\varphi(t_0)|| < \delta ~ \Rightarrow ~ \sup_{t \ge t_0} ||\varphi(t)||
  < \varepsilon
  $$
  $x(t)$ ограничена значит $\exists M ~~ \forall \ge t_0 ~~ ||x(t)|| \le M$

  $\varepsilon = 1 ~~ \exists \delta_1 ~~ \forall \varphi(t) ~~ ||\varphi(t_0)||
  < \delta ~ \Rightarrow ~ \forall t \ge t_0 ~~ ||\varphi(t)|| < 1$

  все значения функции с условием $||\varphi(t_0)|| < \delta$ находятся в
  $\varepsilon$ полосе $\varphi(t)$ ограничена

  Возьмите любой решение $\overline{x}(t)$ произведение $\overline{x}(t_0) =
  x_0 \not = 0$ из условия едиснственности решения так как если хотя бы одно
  решение будет нулевым то все решения станут нулевыми так как все решения
  выходят из точки $O$
  $$
  y(t) = \frac{\overline{x}(t)}{||x_0||} \frac{\delta_1}{2} ~ \text{решение} ~
  (1_0)
  $$
  $x'(t) = A(t) x$
  $$
  ||y(t_0)|| = \left|\left| \frac{\overline{x}(t_0)}{||x_0||} \right|\right|
  \frac{\delta_1}{2} = \frac{1}{||x_0||} ||\overline{x}(t_0)||
  \frac{\delta}{2} = \frac{\delta_1}{2} < \delta_1
  $$
  Решение $y$ подходит под определение устойчивости $\Rightarrow ~ \forall
  t \ge t_0 ~~ ||y(t)|| \le 1$
  $$
  \left|\left| \frac{\overline{x}(t_0)}{||x_0||} \frac{\delta_1}{2}
  \right|\right| < 1
  $$
  так как $||x_0||$ и $\frac{\delta_1}{2}$ это константы тогда из под нормы
  можно вынести
  $$
  \frac{||\overline{x}(t_0)||}{||x_0||} \frac{\delta_1}{2} < 1
  $$
  $$
  ||\overline{x}(t)|| < \frac{2||x_0||}{\delta_1}
  $$
  при всех $t$ решение ограничено какойто константой $\Rightarrow$
  $\overline{x}(t)$ ограничена $\forall t \ge 0$

  $\Leftarrow$ все решения ограничены $\Rightarrow$ ограничена $\Phi(t) =
  (\varphi_1(t), \ldots, \varphi_n(t))$

  $\exists M ~~ ||\Phi(t)|| \le M ~~ \forall t \ge t_0 ~~ x(t) = \Phi(t)x(t_0)$

  $\Phi(t_0) = E$ выбрали для удобства
  $$
  \forall \varepsilon > 0 ~~ \exists \delta_{\varepsilon} ~~ \forall x(t) =
  \Phi(t)x(t_0) ~~ ||x(t_0)|| < \delta ~ \Rightarrow ~ \forall t \ge t_0 ~~
  ||x(t)|| < \varepsilon
  $$
  $$
  ||x(t)|| = ||\Phi(t) x(t_0)|| \le ||\Phi(t)|| ||x(t_0)|| < M \delta =
  \varepsilon
  $$
  $\delta = \frac{\varepsilon}{M}$

  $M \not= 0$ так как матрица $\Phi$ не может быть вырожденной $x \equiv$
  устойчиво $\Rightarrow$ система устойчива
\end{proof}

\begin{block}[Утверждение]
  Система ($1_0$) ассимтотически устойчива $\Leftrightarrow$ когда все ее
  решения стремятся у к нулю при $t \to \infty$
\end{block}

\begin{proof}
  Пусть система ассимтотически устойчива

  тр. решение асс. усто.

  1) устойчиво

  2) $\exists \Delta ~~ ||\varphi(t_0)|| < \Delta ~~ \lim_{t \to +\infty}
  \varphi(t) = 0$

  ассимптотически устойчиво $\Rightarrow$ тривиальное решение ассимтотически
  устойчиво $\Rightarrow$ устойчиво
  $$
  \exists \Delta ~~ \forall \varphi(t) ~~ ||\varphi(t_0)|| < \Delta ~~
  \lim_{t \to +\infty} \varphi(t) = 0
  $$
  $\overline{x}(t)$ решение $1_0$

  $y(t) = \frac{\overline{x}(t)}{||\overline{x}(t_0)||} = \frac{\Delta}{2}$
  решение однородной системы ($1_0$)

  $y(t_0) = \left|\left|\frac{\overline{x}(t)}{2||\overline{x}(t_0)||}\right|
  \right| = \frac{\Delta}{2} < \Delta$
  решение однородной системы ($1_0$)
  $$
  \Rightarrow ~ \lim_{t \to +\infty}y(t) = 0
  $$
  $$
  \Rightarrow ~ \lim_{t \to +\infty} \frac{\overline{x}(t)}{
  ||\overline{x}(t_0)||} \frac{\Delta}{2} = 0
  $$
  $\frac{\Delta}{2||\overline{x}(t_0)||} \lim_{t \to +\infty} \overline{x}(t)
  = 0$

  како бы решение мы не взяли все равно будет $0$

  Пусть все решения $x(t) \to 0$ $\Rightarrow$ $\Phi(t) \to 0$ при $t \to
  +\infty$

  Значит $\Phi(t)$ ограничена так как имеет предел $\Rightarrow ~ x(t)\Phi(t)C$
  тоже ограничена то есть система ($1_0$) устойчива
  $$
  \forall x(t) ~~ \lim_{t \to \infty} x(t) = 0 \Rightarrow \Delta \forall
  $$
\end{proof}

\begin{title}[\Large]
  Устойчивость системы с постоянной матрицей
\end{title}

$x'(t) = Ax(t) ~~ A_{n \times n} ~~ t \ge t_0 ~~ t_0 = 0$ с постоянной матрицей
$\det|A - \lambda E| = 0$

$\lambda_1, \ldots, \lambda_m$ кратности $n_1, \ldots, n_m$

$\forall j \to n_j$ решение вида $\varphi(t) = e^{\lambda_j t}(p_0 + p_1 t +
\ldots + p_{n_j - 1} t^{n_j-1}$

\begin{block}[Критерий ассимтотической устойчивости]
  Для того чтоыы система с постоянной матрицей была устойчивой
  $\Leftrightarrow$ чтобы занчения матрицы $A$ имели отрицательные вещественные
  части $\forall j ~~ Re \lambda_j < 0$
\end{block}

\begin{block}[Критерий ассимтотической неустойчивости]
  Если среди корней есть хотя бы один у которого вещественная часть больше нуля
  то система с постоянной матрицей неусточива.
\end{block}

Если среди корней есть корень с нулевой вещественной частью то необходимо
проводить дополнительное иследование в частности если этот корень однократный
то система устойчива но не ассимтотически устойчива

Если корень кратный то завист от корневого пр-ва соответствующего данному корню

Если корневое пространство состоит только из собственных вектором о система
устойчиво если нет то неустойчиво

$\{x \in C^n ~~ (A - \lambda E)^{h_j} x = 0 \}$ корневое пространство

\begin{proof}
  $\Rightarrow$
\end{proof}
