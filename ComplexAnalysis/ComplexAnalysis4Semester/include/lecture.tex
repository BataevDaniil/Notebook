\begin{title}[\Large]
  Комплексная плоскость
\end{title}

\begin{block}[Комплексное число]
  $(a,b)$ комплексное число это два упорядоченных числа.

  Свойства:

  1) $(a,b) + (c,d) = (a+c, b+d)$ ассоциативно и коммутативно

  2) $(a,b)\cdot(c,d) = (ac - bd, ad + bc)$

  3) $(a,0) \sim a \in R$

  4) $(0,1) = i ~~~ i^2 = (0,1)(0,1) = (-1,0)$

  $Re(a,b) = a$ вещественная часть

  $Im(a,b) = b$ комплексная часть

  $a + ib = (a,b) = (a,0) + (0,1)(b,0)$

  $z = a + ib$ тогда $\overline{z} = a - ib$ сопряженная $z$

  $|z| = |a + ib| = \sqrt{a^2 + b^2}$

  $Arg z = \varphi + 2\pi k ~~ k \in Z$

  $arg z = \varphi_0$

  обычно $-\pi < \varphi_0 \le \pi$ или $0 \le \varphi_0 < 2\pi$

  Так как $a = r \cos \varphi ~~~ b = r \sin \varphi$

  $z = a + ib = r(\cos \varphi + i \sin \varphi) = r e^{i\varphi}$

  $a + ib$ алгебраическая форма

  $r(\cos \varphi + i \sin \varphi)$ тригонометрическая форма

  $r e^{i\varphi}$ показательная форма (работают те же свойства что и с
  экспонентой)
  $$
  \varphi_0 = arg(a + ib) =
  \left\{
  \begin{array}{lc}
    \arctg \frac{b}{a} & a > 0 \\
    \arctg \frac{b}{a} + \pi \sign b & a < 0 \\
    \frac{\pi}{2} \sign b & a = 0
  \end{array}
  \right.
  $$
  $$
  re^{i\varphi} = \rho e^{i\psi} ~~ \text{если} ~~
  \left\{
  \begin{array}{l}
    r = \rho \\
    \varphi = \psi + 2\pi k
  \end{array}
  \right.
  $$
  $$
  re^{i\varphi_0} = \rho e^{i\psi_0} ~~ \text{если} ~~
  \left\{
  \begin{array}{l}
    r = \rho \\
    \varphi_0 = \psi_0
  \end{array}
  \right.
  $$

  Окрестность
  $$
  O_{\varepsilon}(a) = \{ z: |z-a| < \varepsilon\}
  $$
  Проколотая окрестность
  $$
  \stackrel{\bullet}{O_{\varepsilon}}(a) = \{ z: |z-a| < \varepsilon\}
  $$
  $$
  \left\{
  \begin{array}{l}
    x = x(t) \\
    y = y(t)
  \end{array}
  \right. ~~~ t \in [a,b]
  $$
  $z = z(t) = x(t) + iy(t)$ кривая

  Формула Муавра
  $(r e^{i\varphi})^n = r^n e^{in\varphi} = r^n(\cos(n\varphi) +
  i\sin(n\varphi))$

  $\sqrt[n]{re^{i\varphi}} = \rho e^{i\psi}$

  $re^{i\varphi} = \rho^n e^{in\psi}$
  $$
  \left\{
  \begin{array}{l}
    r = \rho^n \\
    n\psi = \varphi + 2\pi k ~~ k \in Z
  \end{array}
  \right.
  $$
  $\rho = \sqrt[n]{r} ~~~ \psi = \frac{\varphi + 2\pi k}{n}$

  $k = 0 ~~ z_0 = \sqrt[n]{r} e^{i\frac{\varphi}{n}}$

  $k = 1 ~~ z_1 = \sqrt[n]{r} e^{i\frac{\varphi + 2\pi}{n}}$

  $k = 2 ~~ z_2 = \sqrt[n]{r} e^{i\frac{\varphi + 4\pi}{n}}$

  $\ldots ~~~ \ldots ~~~ \ldots ~~~ \ldots ~~~ \ldots$

  $k = n-1 ~~ z_{n-1} = \sqrt[n]{r} e^{i\frac{\varphi + 2\pi(n-1)}{n}}$

  $k = n ~~ z_n = \sqrt[n]{r} e^{i\frac{\varphi}{n} + 2\pi}$

  $z \cdot \overline{z} = |z|^2$

  $z + \overline{z} = 2x ~~~ x = \frac{z + \overline{z}}{2}$

  $z - \overline{z} = 2iy ~~~ y = \frac{z - \overline{z}}{2i}$
\end{block}

\begin{title}[\Large]
  Последовательности комплексных чисел
\end{title}

\begin{define}[предела комплексной последовательности]
  $k \in N ~~ z_k = x_k + iy_k = r_k e^{i \varphi_k}$ последовательность

  $p \in C ~~~ p = a + ib$ называется пределом последовательности $z_k$
  $p = \lim z_k$
  $$
  \forall \varepsilon > 0 ~~~ \exists n_{\varepsilon} \in N ~~~
  \forall k \le n_{\varepsilon} ~~~ |z_k - p| < \varepsilon ~~ \text{или} ~~
  \sqrt{(x_k - a)^2 + (y_k - b)^2} \varepsilon
  $$
\end{define}

\begin{block}[Критерий предела комплексной последовательности]
  $$
  p = \lim z_k ~ \Leftrightarrow
  \left\{
  \begin{array}{l}
    x_k \to a \\
    y_k \to b
  \end{array}
  \right.
  $$
  $$
  p = \lim z_k ~ \Leftrightarrow
  \left\{
  \begin{array}{l}
    r_k \to r \\
    \varphi_k \to \varphi
  \end{array}
  \right. ~~~ p = re^{i\varphi}
  $$
\end{block}

\begin{block}[Критерий Коши]
  $\forall \varepsilon > 0 ~~~ \exists n_{\varepsilon} \in N ~~~
  \forall k \le n_{\varepsilon} ~~~ \forall m \in N ~~~ |z_{k+m} - z_k| <
  \varepsilon$
\end{block}

\begin{define}[функции комплексного переменного]
  $z \in D \subset C$ по некоторому правилу ставится в соответствии одно или
  несколько значений $w$ то говорят что $w = f(z)$

  1) если одно значение тогда функция однозначна

  2) если много значений тогда функция многозначна

  3) если бесконечно значений тогда функция бесконечнозначна
\end{define}

\begin{block}[Свойства $w = az + b$]
  $w = az + b ~~~ a \not = 0,b \in C$

  1) $w_1 = |a| z$ плоскость либо растянутая либо сжалась относительно нуля

  2) $w_2 = w_1 e^{i \arg a}$

  $a > 0$ поворот против часовой стрелке

  $a < 0$ поворот по часовой стрелке

  3) $w = w_2 + b$ паралельный поворот в направлении вектора $\vec b$ на длину
  вектора $b$
\end{block}

\begin{define}[однолистной функции]
  $f(z)$ называется однолистной в области $D$ если $\forall z_1, z_2 \in D ~~~
  z_1 \not = z_2 ~ \Rightarrow ~ f(z_1) \not = f(z_2)$
\end{define}

\begin{define}[предела комплексной функции]
  По Коши
  $$
  \lim_{z \to z_0} f(z) = p ~~~ \stackrel{\bullet}{O_{\varepsilon}}(z_0) \in D
  $$
  $$
  \forall \varepsilon > 0 ~~~ \exists \delta_{\varepsilon} > 0 ~~~
  \forall z \in D ~~~ 0 < |z - z_0| < \delta_{\varepsilon} ~~~
  |f(z) - p| < \varepsilon
  $$
  По Гейне
  $\stackrel{\bullet}{O_{\varepsilon}}(z_0) \in D ~~~ \forall z_k \to z_0 ~~~
  f(z_k) \to p$
\end{define}

\begin{block}[Критерий предела комплексной функции]
  $f(z) = u(x, y) + i\upsilon(x,y)$
  $$
  \lim_{z \to z_0} f(z) = p ~ \Leftrightarrow ~
  \left\{
  \begin{array}{l}
    u(x,y) \to a \\
    (x,y) \to (x_0, y_0) \\
    \upsilon(x,y) \to b
  \end{array}
  \right.
  $$
\end{block}

\begin{define}[непрерывности комплексной функции в точке]
  $f(z)$ называется непрерывной в точке если
  $$
  \lim_{z \to z_0} f(z) = f(z_0)
  $$
\end{define}

\begin{theorem}[Вейерштрасса]
  Если функция непрерывная на замкнутом множестве то она ограничена.
\end{theorem}

\begin{theorem}[равномерной непрерывности]
  $f(z)$ наызывается равномерно непрерывной на $[a,b]$
  $$
  \forall \varepsilon > 0 ~~~ \exists \delta > 0 ~~~ \forall z', z'' \in [a,b]
  ~~~ |z'' - z'| < \delta ~~~ |f(z'') - f(z') < \varepsilon
  $$
\end{theorem}

\begin{theorem}[Кантора]
  Если функция $f(z)$ определена и непрерывна на $[a,b]$ тогда она равномерно
  непрерывна на $[a,b]$
\end{theorem}

\begin{title}[\Large]
  Дифференцируемость функции комплексной переменной
\end{title}

\begin{define}[производной функции комплексной переменной]
  $w = f(z)$
  $$
  \lim_{z \to z_0} \frac{f(z) - f(z_0)}{z - z_0} =
  \frac{f(z + z_{\Delta}) - f(z)}{z_{\Delta}} =
  \frac{f_{\Delta}(z)}{z_{\Delta}} = f'(z_0)
  $$
  таблица производный и правила производных такие же как и для функций с
  вещественным аргументом.
\end{define}

\begin{theorem}
  $f(z)$ имела производную в точке $z_0$ $\Leftrightarrow$ $u$ и $\upsilon$
  дифференцируемы в этой точке и
  $$
  \left\{
  \begin{array}{l}
    \frac{\partial u}{\partial x} = \frac{\partial \upsilon}{\partial y} \\
    \frac{\partial u}{\partial y} = -\frac{\partial \upsilon}{\partial x}
  \end{array}
  \right. ~~~ \text{условия Коши-Римана (Даламбера - Эйлера)}
  $$
\end{theorem}

\begin{proof}
  1) $\Rightarrow$
  $$
  f'(z) = \lim_{z_{\Delta} \to 0} \frac{f(z + z_{\Delta}) - f(z)}{z_{\Delta}}
  $$
  Пусть $y_{\Delta} = 0$ тогда $z_{\Delta} = x_{\Delta}$
  $$
  f'(z) = \lim_{x_{\Delta} \to 0} \frac{u(x + x_{\Delta}, y) - u(x,y) +
  i(\upsilon(x + x_{\Delta}, y) - \upsilon(x,y))}{x_{\Delta}} = u'_x +
  i\upsilon'_x
  $$
  Пусть $x_{\Delta} = 0$ тогда $z_{\Delta} = iy_{\Delta}$
  $$
  f'(z) = \lim_{y_{\Delta} \to 0} \frac{u(x, y + y_{\Delta}) - u(x,y) +
  i(\upsilon(x, y + y_{\Delta}) - \upsilon(x,y))}{iy_{\Delta}} =
  $$
  $$
  = \lim_{y_{\Delta} \to 0} -i\frac{u(x, y + y_{\Delta}) - u(x,y)}{y_{\Delta}} +
  \frac{\upsilon(x, y + y_{\Delta}) - \upsilon(x,y)}{y_{\Delta}} = -iu'_y +
  i\upsilon'_y
  $$
  $$
  u_{\Delta} = u(x + x_{\Delta}, y + y_{\Delta}) - u(x,y) =
  u'_x x_{\Delta} + u'_y y_{\Delta} + \alpha(x_{\Delta}, y_{\Delta})
  \sqrt{(x_{\Delta})^2 + (y_{\Delta})^2}
  $$
  $$
  \upsilon_{\Delta} = \upsilon'_x x_{\Delta} + \upsilon'_y y_{\Delta} +
  \beta(x_{\Delta}, y_{\Delta})\sqrt{(x_{\Delta})^2 + (y_{\Delta})^2}
  $$
  $$
  f'(z) = \lim_{z_{\Delta} \to 0} \frac{u_{\Delta} +
  i\upsilon_{\Delta}}{x_{\Delta} + iy_{\Delta}} =
  $$
  $$
  = \lim_{z_{\Delta} \to 0} \frac{u'_x x_{\Delta} + u'_y y_{\Delta} +
  i(\upsilon'_x x_{\Delta} + \upsilon'_y y_{\Delta}) + (\alpha + i\beta)
  \sqrt{(x_{\Delta})^2 + (y_{\Delta})^2} }{x_{\Delta} + iy_{\Delta}} =
  $$
  $$
  = \lim_{z_{\Delta} \to 0} \left( \frac{u'_x x_{\Delta} +
  iu'_x y_{\Delta}} {x_{\Delta} + iy_{\Delta}}
  -\frac{\upsilon'_x y_{\Delta} +
  i\upsilon'_x x_{\Delta}}{x_{\Delta} + iy_{\Delta}} + \frac{(\alpha + i\beta)
  \sqrt{(x_{\Delta})^2 + (y_{\Delta})^2} }{x_{\Delta} + iy_{\Delta}}\right) =
  $$
  $$
  = u'_x + i\upsilon'_x = \upsilon'_y - iu'_y
  $$
\end{proof}

\begin{theorem}
  $$
  \left\{
  \begin{array}{l}
    \frac{\partial u}{\partial r} = \frac{1}{r} \frac{\partial \upsilon}
    {\partial \varphi} \\
    \frac{\partial u}{\partial \varphi} = -\frac{1}{r} \frac{\partial \upsilon}
    {\partial r}
  \end{array}
  \right.
  $$
\end{theorem}

\begin{proof}
  $$
  \left\{
  \begin{array}{l}
    x = r \cos \varphi \\
    y = r \sin \varphi
  \end{array}
  \right.
  $$
  $$
  \left\{
  \begin{array}{l}
    \frac{\partial u}{\partial r} = \frac{\partial u}{\partial x} \cos \varphi
    + \frac{\partial u}{\partial y} \sin \varphi \\
    \frac{\partial \upsilon}{\partial \varphi} =
    \frac{\partial \upsilon}{\partial x} (-2 \sin \varphi) +
    \frac{\partial \upsilon}{\partial y} 2\cos \varphi
  \end{array}
  \right.
  $$
\end{proof}

\begin{define}[аналитической функции в точке]
  Функция называется аналитической в точке если она дифференциируема в
  некоторой окрестности этой точки.
\end{define}

\begin{define}[аналитическая функция в области]
  Функция называется аналитической в области если она дифференцируема во всех
  точках области.
\end{define}

\begin{define}[регулярной функции]
  Регулярной функцией (в точке) в области называют аналитическую и однозначную
  (в точке) в области функцию.
\end{define}

\begin{title}[\Large]
  Гармонические функции. Востановление функции по ее вещественной или мнимой
  части
\end{title}

\begin{define}[гармонической функции]
  Гармонической функциией называют дважды непрерывна дифферецируемою и
  $u''_{x^2} + u''_{y^2} = 0$
\end{define}

\begin{theorem}
  Если функция аналитическая в области $D$ $\Rightarrow$ $u$ и $\upsilon$
  является гармонической в области $D$.
\end{theorem}

\begin{proof}
  $$
  \frac{\partial^2 u}{\partial x^2} = \frac{\partial^2
  \upsilon}{\partial x \partial y}
  $$
  $$
  \frac{\partial^2 u}{\partial y^2} = -\frac{\partial^2
  \upsilon}{\partial y \partial x}
  $$
  $\Rightarrow$ $u$ гармоническая
\end{proof}

\begin{define}
  Две гармонические $u$ и $\upsilon$ связаны $(u, \upsilon)$ называются
  сопряженными тогда $f(z) = u(z) + i \upsilon(z)$ аналитическая функция.
\end{define}

$$
u(x,y) = \int_{(x_0, y_0)}^{(x,y)} \upsilon'_x dx + \upsilon'_y dy =
\int_{(x_0, y_0)}^{(x,y)} (-u'_y dx + u'_x dy)
$$

\begin{title}[\Large]
  Геометрический смысл модуля аргумента производной функции в точке
\end{title}

\begin{theorem}
  $\gamma$ кривая которя имеет касательной вектор в $t_0$ и $f'(z_0) \not = 0$
  и $w_0 = f(z_0)$

  образ $\gamma$ в $w$ есть $\Gamma$ и тоже имеет касательный вектор

  $\gamma: ~ z = \vec z(t) ~~~ t \in [a,b]$

  $z_0 = \vec z(t_0) ~~~ t_0 \in [a,b]$

  $z'(t_0) = \alpha$

  $\Gamma: ~ w = f(z(t))$

  $w'(t_0) = f'(z_0)z'(t_0) \not = 0$

  $arg w'(t_0) = \beta$

  $arg w'(t_0) = arg f'(z_0) + arg z'(t_0)$

  $arg f'(z_0) = \beta - \alpha$

  Угол поворота между образом кривой и праобразом
  $$
  |f'(z_0)| = \lim_{z_{\Delta} \to 0} \left| \frac{w_{\Delta}}{z_{\Delta}}
  \right| \not = 0
  $$
  $z_{\Delta} = z - z_0 ~~~ w_{\Delta} = f(z) - f(z_0)$
  $$
  |f'(z_0)| \approx \left| \frac{w_{\Delta}}{z_{\Delta}} \right| +
  \overline{o}(z_{\Delta})
  $$
  $|w_{\Delta}| \approx |z_{\Delta}| |f'(z_0)| ~~~~ |f'(z_0)|$ коэффицент
  линейного искажения характеризует локально что делает данное отображение.

  Геометрический смысл модуля производной состоит в том, что величина
  определяет коэффициент растяжения в точке при отображении. Величину
  $|f'(z_0)|$ при $|f'(z_0)| > 1$ называется коэффициентом растяжения,
  а при $|f'(z_0) < 1$ сжатия.

  $arg f'(z_0)$ это угол, на который надо повернуть касательную к крив $\gamma$
  в точке $t_0$, чтобы получить направление касательной к кривой $\Gamma$ в
  точке $w_0$.
\end{theorem}
