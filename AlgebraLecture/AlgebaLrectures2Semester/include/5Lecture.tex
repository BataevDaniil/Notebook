\begin{title}
	Гомоморфизм колец и Идеалов
\end{title}

Кольцо является группой по сложению и полугруппой по умножению и гомоморфизм
созраняет обе операции. \\
Базовая в кольце операция - это \kv{сложение}. И прежде всего надо изучить
гомоморфизм комутативной группы по сложению.\\

\bd{Структуры гомоморфизма групп}\\
$\varphi:G \to H$\\
Ядром гомоморфизма $\varphi$ -называют такое подмножество группы G, где
$Ker \varphi = {g \in G|\varphi (g) = e_H}$ \kv{Ker - ядро}\\

\begin{defin}[Свойство 1]
	Ядро - подгруппа.
\end{defin}

\begin{proof}
	Докажем по критерию подгруппы\\
	$g_1, g_2, \in Ker\varphi, g^{-1}_1 \in Ker, ~~~ Ker \in 1$\\
	$\varphi(g_1, g_2) = \varphi(g_1) \varphi(g_2) = e_H \cdot e_H = e_H$\\
	По определению обратного элемента $g_1 \cdot g^{-1}_1 = e_G ~~~
	\varphi(e_G) = \varphi(e_G e_G) = \varphi(e_G) \cdot \varphi(e_G)$\\

	Так как \kv{H} - группа, то у элемена $\varphi(e_G)$ - есть обратный. Умножим
	на него обе части равенства $\varphi(e_G) = e_H$\\
	Таким образом единичный элемент переходит в единичный\\
	$e_H = g(e_G) = \varphi(g_1 g^{-1}_1) = \varphi(g_1)\varphi(g^{-1}_1)
	\Rightarrow \varphi(g^{-1}_1) = e_H$
\end{proof}

\begin{defin}
	$G \ge L$ ~~~ $\ge$ - означает, что подмножество имеет согласованную
	структуру\\
	Подгруппа нормальная, если $\forall g \in G ~~~ g^-1 Lg \subseteq L$\\
	$G^{-1}hg - сопряжается элемент h при помощи элемента q$\\
	\kv{Понятие не нормальной подгруппы нет. ее название подгруппа, не являющаяся
	нормальной}
\end{defin}

\begin{defin}[Свойство 2]
	Ядро - нормальная подгруппа.
\end{defin}

\begin{proof}
	Пусть $f \in Ker\varphi ~~~ \varphi(g^{-1} fg) = e_H ~~~ g\in G$\\
	По определению гомоморфизма
	$\varphi(g^{-1} fg) = \varphi(g^{-1}) \varphi(f) \varphi(g) = \varphi(g)^{-1}
	\varphi(f) \varphi(g) = \varphi(g)^{-1} e_H \varphi(g) = \varphi(g)^{-1}
	\varphi(g) = e_H$\\
	Группа называется простой, если нет нетривиальной нормальной подгруппы.
	(Тривиальная - это группа, состоящая из одного элемента)
\end{proof}

\kv{Основной задачей теории групп является описание всех простых групп, а
остальные группы получаются из простых при помощи расширений.}\\
\kv{Бесконечные группы - не описаны.}\\
\kv{В теории конечных групп задача вроде решена (были найдены 17 бесконечных
серий групп и 26 не серийных изолированных групп)}\\

\begin{title}
	Фактор группы Фактор кольца
\end{title}

$G \ge H$ - H нормальная подгруппа. Подмножество вида $gH = {gh | h \in H}$
называется смежными классами.\\
\kv{Не сложно проверить, что разные смежные классы не пересекаются}\\
$h \mapsto gh$ - биекция.\\
$g_1H /cdot g_2H = (g_1 g_2)H$\\

\begin{defin}
	В смежных классах $gH$, элемент п - представитель. Не сложно проверить, что в
	качестве представителя можно взять любой элемент этого смежного класса.
	Необходимо проверить корректность задания операции, так как она определяется
	через представителей убедимся, что заменив представителей, мы получим тот же
	результат.\\
	$g_1H = g'_1H\\
	g'_1 = g_1 h_1 \in H\\
	g_2H = g'_2H\\
	g'_2 h_2 \in H\\
	h \mapsto gh ~~~ (g'_1 H)(g'_2 H) = (g'_1 g'_2)H = (g_1 h_1)(g_2 h_2)H =
	(g_1 g_2)(g'_2 h_1 g_2)h_2 H = g_1 g_2 H$\\
	Так как подгруппа H - нормальная, то $g^{-1}_2 h_1 g_2 \in H = h_3 ~~~
	\forall h \in H ~~~ h \cdot H = H$\\
\end{defin}

\begin{defin}
	Группа элементы, которой являются смежными классами, определенно задана как
	указано выше и называется фактор группы - $G/H$\\
	Нейтральный элемент в этой группе $e \cdot H = H$\\
	Обратный смежный класс $(gH)^{-1} = g^{-1}H$ ассоциотивность следует из
	ассоциотивности умножения.\\
	Когда целове множество воспринимаем как единый объект - оно называется
	фактором множества - стандартная идея при обобщении.\\
	$G/H = L$ Пусть $G$ - группа, $H$ - нормальная подгруппа, $G/H = L$ - фактор
	группы. Говорят что группа $b$ расширение группы $H$ при помощи группы С.\\
	В конце концов можно свести к простым группам, где строить факторы группы не
	получится. Даже зная все простые группы - все группы построить затруднительно
	так как способов расширений - бесконечное число.
\end{defin}

\begin{title}
	Идеалы и фактор кольцa
\end{title}

Пусть $K$ - кольцо, $L$ - подкольцо и подгруппа по сложению. Подкольцо
называется идеалом, если $\forall k \in K ~~~ k \cdot L = {kl | l \in L} =
Lk = L$\\
\bk{Идеал} - обобщение понятия нуля.\\
Смежные класс по идеалу имеют вид $k + L, k \in K$. На множестве смежных классов
введем операцию сложения и умножения\\
$(k_1 + L) + (k_2 + L) = (k_1 + k_2) + L\\
(k_1 + L) \cdot (k_2 + L) = (k_1 \ cdot k_2) + L$\\
Операции задаются при помощи представителей - проверим их корректность.\\
$k'_1 + L = K_1 + L\\
k'_2 + L = K_2 + L ~~~ I \triangle K ~~~ IK \le I ~~~ I + I = I$\\
Если $K$ - комутативное кольцо, то существуют левые идеалы и правые, а также
двусторонние.\\

\begin{defin}
	Идеал называется максимальным, если он не содержится не в каком больше.
	В дальнейшем будем будем считать, что кольцо комутативное $K + I = {K + i
	| i \in I}$\\
	А множество всех счетных классов заданы сложением и умножением и называются
	фактор-кольцом - $K/I$\\

	\kv{Самые полезные идеалы - главые. Главные идеалы - пораждаются одним
	элементом который является главным}
\end{defin}

\begin{theorem}
	Любое Евклидово кольцо - кольцо главных идеалов.
\end{theorem}

\begin{proof}
	Пусть есть произвольный идеал $I = {i_1, i_2, ... i_n}$ Надо выбрать такой
	элемент $I$, через который можно выразить все остальные $i$. Так как кольцо
	Евклидово, то в нем любые 2 элемента имеют Наибольций Общий Делитель.\\
	$d = НОД(i_1; i_2) ~~~ i_1 d \cdot i'_1 ~~~ i_2 = d \cdot i'_2$\\
	Таким образом и $i_1$ и $i_2$ попадают в идеал, пораждая элемент $d$.
	Пораждающий элемент $I$ - НОД всех этих элементов.
\end{proof}

\begin{theorem}[Основная теорема Алгебры]
	Поле комплексных чисел является алгебраически замкнутым, то есть в нем любой
	многочлен с комплексными коэффициентами раскладывается на минимальный
	множитель
\end{theorem}

\begin{title}
	Симметрические многочлены. Теорема Виета
\end{title}

Пусть P - поле. Рассмотрим кольцо многочленов от n элементов\\
$F(x_1, x_4) = \sum \alpha x^{i_1}_1, x^{i_2}_2, x^{i_3}_3, x^{i_4}_4$\\
Каждое слогаемое называется мономом $x^{5}_1, x^{7}_2, x^{16}_3$
Общая степень = 28\\
Когда у нас одна переменная $x$, то мы можем мономы легко сравнивать. Но когда
переменных $x0$ - несколько, то упорядочить их можно многими способами.\\
Способ упорядочения, который принят в словарях - называется
лексико-графическим.\\
Допустим на множестве мономов мы ввели линейное упорядочение:\\
Первое условие - самый большой моном в смысле этого упорядочения - старший
моном.\\
Второе условие - упорядочение должно быть таким, чтобы количество меньших или
старших было конечно.\\
Третье условие - Сначала мономы сравниваются по общей степени, а когда она
совпадает, то сравниваем по лексико-графически.\\

\begin{defin}
	Многочлен от n-переменных называется симметричным, если он не изменяется при
          любой перестановке входящих в него символов.\\

          Например:\\
          $x + y = y + x$\\
          $xy = yx$\\
          $x^2y + xy^2 = yx^2 + y^2x$\\

         Многочлен от $x$ и $y$ называют \kv{симметрическим}, если он не изменяется
         при замене $x$ на $y$,а $y$ на $x$.
\end{defin}

\begin{theorem}
	Любой симметрический многочлен может быть выражен через элементы симметрии.

           Любой симметрический многочлен от $x$ и $y$ можно представить в виде 
           многочлена от $s_1 = x + y$ и $s_2 = x \cdot y$
\end{theorem}

\begin{theorem}[ Виета]
	Пусть $x_1, x_2, ... x_n$ - корни $n$ - ой степени многочлена $f(x)$\\
	$f(x) = (x - x_1)(x - x_2) ... (x - x_n) = \\
           = S_0 \cdot x^n + S_1 \cdot  x^{n - 1} + S_2 \cdot x^{n - 2}+ S_{n-1}\cdot x+ S_n = 0$\\
           $a_0, a_1, ... , a_{n-1}, a_n$\\
	Коэффициент при степени многочлена с точностью до знака - является элементом
	симметрии многочлена от его корней.\\
          Предположим, что $k_1, k_2, ... , k_{n-1}, k_n$ - корни уравнения\\
          Нагляднее видно формулы Виета не в общем виде,\\
           а на примере, допустим $n = 4$\\
          $a_0\cdot x^4 + a_1 \cdot x^3 + a_2\cdot x^2 + a_3\cdot x + a_4 = 0$\\
          \begin{equation*}
          \begin{cases}
            k_1 + k_2 + k_3 + k_4 = - \frac{a_1}{a_0}\\
            k_1\cdot k_2 + k_1\cdot k_3 + k_1\cdot k_4 + k_2\cdot k_3 + k_2\cdot k_4 + k_3\cdot k_4 = \frac{a_2}{a_0}\\
            k_1 \cdot k_2\cdot k_3 + k_1\cdot k_2\cdot k_4 + k_2\cdot k_3\cdot k_4 = - \frac{a_3}{a_0}\\
            k_1\cdot k_2\cdot k_3\cdot k_4 = \frac{a_4}{a_0}\\
          \end{cases}
          \end{equation*}
	Комбинируя теорему Виета и основную теорему о симметричности многочленов, мы
	можем не находить корни многочлена (вычислять некоторые функции от неизвестных
	нам корней).\\
	Так как нам требуются изучать перестановки переменных - надо ввести обозначать
	и терминалогию из теории групп и перестановок.\\
	$S_n = n!$ S - множество всех перестановок.
\end{theorem}

\begin{displaymath}
\left( \begin{array}{lccr}
1 & 2 & ... & n \\
i_1 & i_2 & ... & i_n
\end{array}\right)
\end{displaymath}

\begin{title}
	Математические основы Криптографии
\end{title}

4 математические идеи\\

\bd{Первая идея} - ключ/шифр должен быть выбран случайным образом, но все
имеющиеся алгебраическое основание.\\
\bd{Вторая идея} - однонаправленная функция $f$ - однонаправлена, если
$f(n) = m$ - вычислить легко, но язная $m$ - найти $n$ - трудно.\\
\bd{Третья идея} - Хэш функция - отображает переводящее сообщение произвольной
длины в сообщении фиксированной длины. Разных хэшей - $2^256$. Но один хэш имеет
бесконечно много сообщений. Это свойство не для одного хэша  строго не доказано.
Хэш криптографический, если в процессе его вычисления используется шифрование.\\
\bd{Четвертая идея} - анализ протоколов. Интерактивный алгоритм в котором
примимают участие 2 или больше ПРОПУСТИЛ В ЛЕКЦИИ - называется протоколом.
Если используется шифрование, то протокол - криптографический. Все компьютерные
процессы - осуществляется посредством протокола.