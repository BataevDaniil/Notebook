\begin{title}
  В курсе Алгебры есть 3 супер теоремы.
\end{title}

\bd{III)} основная теорема Арифметики об однозначном разложении на простые
множетели Евклидовых колец.\\
  \kv{1} Определение Евклидова Кольца.\\
  \kv{2} Определение Однозначности разложения.\\
  \kv{3} Лемма о существовании разложения.\\
  \kv{4} Критерий однозначности разложения.\\

\bd{II} Жарданова Форма - в подходящем базисе, называемым Жардановым - Матрица
принимает клеточно-диагональный вид, где каждая клетка - клетка Жардано
(поле алгебраически замкнуто).\\

\begin{displaymath}
\left(\begin{array}{lcr}
J_1 & 0   & 0\\
0   & J_2 & 0\\
0   & 0   & J_3
\end{array}\right)
\end{displaymath}

\begin{displaymath}
J = \left(\begin{array}{lcr}
r & 1 & 0\\
0 & r & 1\\
0 & 0 & r
\end{array}\right)
\end{displaymath}

\bd{I} Основная Теорема Алгеббры о том, что поле компланарных чисел является
алгебраически замкнутым.\\
  \kv{1} Расширение полей (Кольца, Идеалы, Гоморфизмы)\\
  \kv{2} Многочлены с несколькими элементами и с линейным упорядочением мономов.
   При этом упорядочение удовлетворяет условием минимальности (каждая убывающая
   цепочка - конечна)\\
  \kv{3} Симметрия многочленов.\\

\begin{title}
  Группы перестановок. Действие группы на множетели. Определители.
\end{title}

\begin{title}[\Large]
  Детерминанты
\end{title}

Дадим определение детерминанта.\\
Детерминант появился как ответ на вопрос как по координатам системы линейных
уравнений сразу написать ответ.\\
На практике использование правила Крамера - не экономично. Если СЛУ $nxn$, то
придется вычислить $n^2$ определителей и решит методом Гауса $n^2 + 1$ систем.\\
Методом Гауса быстрее всего вычислить определитель, найти многочлены матрицы,
базисы, суммы и пересечения подпространств.\\

Определение по индукции.\\

\begin{displaymath}
\left(\begin{array}{lccr}
a_{11} & a_{12} & \cdots & a_{1n}\\
a_{21} & a_{22} & \cdots & a_{2n}\\
\vdots & \vdots & \ddots & \vdots\\
a_{n1} & a_{n2} & \cdots & a_{nn}
\end{array}\right)
\end{displaymath}

\[n = 1 ~~~ |A| = a_{11} ~~~ |A| = \sum^n_{i = 1} (-1)^{1 + 1} a_{1i} |A_{1i}|\]
$|A_{1i}|$ - минор. Получается вычеркиванием $i$ строки и $j$ столбца.\\
\kv{1) Как вычислять этот определитель?}\\
\kv{2) Сколько в нем слагаемых?}\\
\kv{3) Какова структура слагаемых}\\

При
$n = 1$ - 1 слагаемое\\
$n = 2$ - 2 слагаемых\\
$n = 3$ - 3! слагаемых\\
$(n - 1)$ - (n - 1)! слагаемых\\
$n$ - n(n - 1) = n! слагаемых\\
Вывод. Определитель $n$-ного порядка состоит из $n!$ слагаемых, каждый из которых
является произведением $n$ элементов, половина из них с +, половина с -.\\

\kv{Сложность вычисления.}\\
Примем сложность сложения за 1, а умножение за t. $t >> 1 ~~~ t \approx 10^3$
Найдем сложность вычисления n-ного порядка:
Сложение $= n! \cdot 1$\\
Умножение $= (n - 1) \cdot n! \cdot t$\\
Умножение труднее в $t(n - 1)$ раз.\\

\begin{title}[\Large]
  Техническая часть вычисления детерминанта
\end{title}

Матрица называется треугольной, если имеет вид:
\begin{displaymath}
\left(\begin{array}{lcccr}
a_{11} & a_{12} & a_{13} & \cdots & a_{1n}\\
0      & a_{22} & a_{23 }& \cdots & a_{2n}\\
0      & 0      & a_{33} & \cdots & a_{2n}\\
\vdots & \vdots & \vdots & \ddots & \vdots\\
0      & 0      & 0      & \cdots & a_{nn}
\end{array}\right)
\end{displaymath}

Используя индуктивное определение детерминанта не сложно вычислить, что
определитель этой матрицы равен произведению диагональных элементов.\\
Это наблюдение подсказывает нам способ нахождения детерминанта. А именно при
помощи элементарных преобразований по методу Гауса привести матрицу к
диагональному виду.\\
\kv{Что будет происходить с детерминантом, если применять элементарное
преобразование строк?}\\
\bd{1-е ЭП} умножает строки на $\alpha$ и если эта строка первая, то все
коэффициенты $a_{1i}$ будут умножены на $\alpha$ и значит весь определитель
умножится на $\alpha$.\\
\bd{2-е ЭП} - $a_i + \alpha(a_i)$\\

\bk{Свойства определителей:}\\
\bd{I} при умножении строки/столбца на $\alpha$ - определитель умножается на
$\alpha$\\
\bd{II} при втором ЭП - определитель не изменяется.\\
\bd{III} если 2 строки/столбца поменять местами, то определитель изменит знак.\\
\bd{IV} если в матрице есть 2 одинаковых строки/столбца, то определитель = 0.\\

\begin{proof}
  Поменяв местами строки мы изменим знак определителя и если характеристика поля
  $\not = 2$, то определитель равен 0.\\
  Если характеристика равена 2, то определитель тоже равен 0, но нужно изменить
  способ доказательства (по индукции)\\
\end{proof}

  Матрица называется полураспавшейся, если имеет вид:\\
  Верхняя полураспавшаяся -
  \begin{displaymath}
  A\left(\begin{array}{lr}
  B & 0\\
  D & C
  \end{array}\right)
  \end{displaymath}
  Нижняя полураспавшаяся -
  \begin{displaymath}
  A\left(\begin{array}{lr}
  B & D\\
  0 & C
  \end{array}\right)
  \end{displaymath}
  В и C - квадратные матрицы, D - любая матрица

\bd{V} определитель полураспавшейся матрицы равен $|B|\cdot |C|$\\
\bd{VI} определитель произведения матрицы равен произведению определителя:
$|A \cdot B| = |A| \cdot |B|\\
det M_n(p) \to p^*$\\
Детерминант является гомоморфизмом группы матриц по произведению в
мультипликативную группу поля\\
\bd{VII} определитель - это сумма $n!$ слогаемых, каждое из которых - это
произведение $n$ элементов, взятых по одному из каждой строки/столбца,
а знак определяется четностью перестановки.\\

\[|A| = \sum_{\Pi \in S_n} (-1)^{parity\Pi} a_{1i1} \cdot a_{2i2}
\cdot ... a_{nin}\] parity - четность.
  \begin{displaymath}
  \Pi = \left(\begin{array}{lccr}
  1   & 2   & \cdots & n\\
  i_1 & i_2 & \cdots & i_n
  \end{array}\right)
  \end{displaymath}

\bd{VII} $|A|=|A^t|$ t - транспонирование.\\

\begin{title}
  Группы перестановок. Четные и не четные перестановки
\end{title}

\begin{defin}
  Беспорядком - это когда меньший элемент находится после большего элемента.\\
  Если количество беспорядков четно, то перестановка четная.\\
  Если не четно, то перестановка не четная.
\end{defin}

\begin{defin}
  Перестановка называется транспозицией, если эта перестановка двух элементов.
\end{defin}

  \begin{displaymath}
  \Pi = \left(\begin{array}{lcccr}
  1 & 2 & 3 & 4 & 5\\
  5 & 2 & 2 & 4 & 1
  \end{array}\right) = (15)
  \end{displaymath}

Если транспозиция - это перестановка $i$-го и $j$-го элемента, то записывается
она $(ij)$\\

\begin{title}[\Large]
  Свойства четных и не четных перестановок
\end{title}

\kv{1)} - произведение четных перестановок - четно\\
\kv{2)} - произведение не четных перестановок - четно\\
\kv{3)} - произведение ченой и не четной перестановки - не четно\\

\begin{proof}
  Первый шаг:\\
  Необходимо произвольную перестановку представить как произведение транспозиций
  \begin{displaymath}
  \Pi = \left(\begin{array}{lcccccr}
  1 & 2 & 3 & 4 & 5 & 6 & 7\\
  6 & 3 & 2 & 5 & 1 & 7 & 4
  \end{array}\right) = (16745)(23)
  \end{displaymath}
  $(16745)$ - циклы независимые, если не имеют общих элементов\\

  Второй шаг:\\
  Чтобы разложить всю перестановку, достаточно разложить каждый цикл:\\
  $(12...n) = (12)(13)(14)...(1n)$\\
  Вывод - каждая перестановка произведение транспозиций. И если это цикл длины
  $n$, то транспозиция $n - 1$. Так как каждая транспозиция - это $n - 1$, то
  получим, что произведение не четно.
\end{proof}

\begin{defin}
  Любая биекция множества не себя называется перестановкой.
\end{defin}

Если множество состоит из $n$ эелементов, то группу перестановок обозначают
$S_n = \sum n$, в ней $n!$ элементов.\\
Операцией является супер-позиция отображений.\\
Нейтральная элементарная перестановка:\\
\begin{displaymath}
id = \left(\begin{array}{lccr}
1 & 2 & \cdots & n\\
1 & 2 & \cdots & n
\end{array}\right)
\end{displaymath}

\begin{title}[\Large]
  Форма записи перестановок
\end{title}

\bk{Табличная запись} в виде двух строк: 1-я - сами $n$ элементы, 2-я - их
образы.\\
Данная записьь не экономна и мало информативна.\\
\bk{Запись в виде произведений независимых циклов} (циклы длины 1 не
записываются, а произведение независимых циклов однозначно востанавливает
табличную запись)\\

Если циклы независимы, то они перестановочны. Поэтому легко вычислить порядок
перестановки, то есть ту минимальную степень, в которой она равна
тождественной.\\

Допустим циклы имеют длину $s, e, ... t$, если цикл длины $s$, то его порядок -
$s$. Очевидно, что порядок перестановки будет НОК$(s, e, .. t)$\\

\begin{defin}[Четности и не четности]
  Так как транспозиция не четная перестановка. И умножение на транспозицию
  меняет четность перестановка, то $(12...n) = (12)(13)...(1n)$\\
  Цикл длины $n$ имеет четность $n - 1$. Поэтому, если перестановка
  раскладывается на длины $S_1, S_2, ... S_k$, то четность равна
  $S_1 + S_2 + ... S_{k - 1}$
\end{defin}

\begin{title}[\Large]
  Перемножение перестановок
\end{title}

В алгебре, в отличие от матанализа функция является элементом и при перемножении
$\Pi \cdot \vartheta$ - сначала $\Pi$, затем $\vartheta$.

Обратная перестановка находится сменой элементов с их образами, и последующим
упорядочением.\\

Четные перестановки образуют подгруппу и обозначается она $A_n$ и имеет индекс
равным 2: $|S_n \cdot A_n| = 2$\\
$|A_5| = 60$ - самая маленькая не комутативная простая группа.