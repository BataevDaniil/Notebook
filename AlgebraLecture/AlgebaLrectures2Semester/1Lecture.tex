
\begin{center}
\bd{{\huge Фундаментальная и компьютерная \\
Алгебра}} \\
\LARGE{(Второй семестр)} \\

\end{center}

\begin{title}
	{Теория полей}
\end{title}

\bd{Поле} - множество с двумя алгебраическими операциями (сложение и умножение),
удовлетворяет 10 аксиомам. Фактически выполнимы все 4 арифметические операции.\\

По сравнению с остальными алгебраическими объектами - полей мало. \\

\bd{Поля:}\\
1) {\emph {Конечные}} - Поля Галуа ( построена вся современная криптография)\\
2) {\emph {Бесконечные}} \\

\bd{Характеристика Поля}\\
Определенное минимальное натуральное число \bd{K} - называется {\emph
{характеристикой поля}}, если $\forall \mathit{a} \in \mathit {P,a} \neq 0,
\exists \mathit {k}, \mathit{ka} = 0 $. \\
А если такого {\emph {k}} не существует, то говорят, что характеристика равна
нулю.\\
Характеристика поля - это число, при аддитивном умножении на которое,
любого элемента поля будет получаться ноль.\\

Например, $Z_{2} = \{0,1\}$ имеет характеристику $k$ = 2. 
Так как $\mathit{ka} должен быть = 0$, тогда $a = 1$ (так как 
$a$ не может быть равна 0). И $k = 2$, следовательно
$\mathit{ka} = 2 = 0$, так мы находимся в поле $Z_{2}$.\\

$\mathbb{Q, R, C}$ -имеют нулевую характеристику.\\

\kv {Как связаны конечные и бесконечные характеристики?} \\

\begin{theorem}
	Если поле имеет нулевую характеристику, то оно бесконечно и содержит в
	качестве под-поля, поле рациональных чисел.
\end{theorem}

\begin{proof}
	1) Пусть \bk{e} - нейтральный элемент по умножению (мультипликативная еденица).
	Так как поле имеет нулевую характеристику, то $\mathit {e, e+e, ...}$ или
	$\mathit {(e, 2e, ... ne)}$ - все это не нулевые элементы, более того попарно
	различные.\\
	$\mathit {ne = me, n < m}$ Так как это поле, то у элемента \bk{e} есть
	обратный по сложению $\mathit {(m-n)e=0}$ \\
	Если это равенство умножить на любой , то получим, что $\mt {m-n}$ делится на
	характеристику, а значит характеристика не нулевая.\\
	Таким образом последовательность бесконечна, элементы разные, а значит и поле
	бесконечно.\\
	2) Сконструируем внутри поля \bd{K} поле \bd{Q} \\
	$\mt {e} \mapsto 1 \\
	\mt {ne} \mapsto \mt {n} \\
	\frac{ne}{me} \mapsto \frac{n}{m}$ \\

           Целые числа лежат в поле \bd{K}. Рациональные числа тогда лежат в нём
           как отношения целых к натуральным.\\

	Легко проверить, что для элементов вида \bk{ne и me} выполняются все 10 аксиом
	поля и это поле совпадает с полем \bd{Q}
	\end{proof}

           Что такое конечное поле?\\
           Конечное поле или поле Галуа - это поле, состоящее из конечного чилса элементов.
           Простейшим примером конечного поля является поле вычетов по простому модулю.\\

           Свойства конечных полей:\\
             1. Характеристика является простым числом $p$;\\
             2. Число элементов в конечном поле $F$ является степенью его характеристики: $|F| = p^{q}$\\

\begin{theorem}
	Если поле конечно, то его характеристика обязательно ненулевая.
\end{theorem}

\begin{proof}
	Пусть \bk{e} - нейтральный элемент по умножению. Расмотрим последовательность
	$\mt {\{e,2e...ne\}}$. Так как поле конечно, то в последовательности
	встречаются одинаковые элементы.\\
	$\mt{ne=me}$ \\
	Рассуждая как и выше \\
	$\mathit {(m-n)e=0}$ и характеристика не нулевая.
\end{proof}

\kv {Замечание} \\
Обратное утверждение: из конечной характеристики следует конечное поле -
\bd{неверно} \\

Построим кольцо многочленов. \\
$Z_{2}[x]$ оно бесконечно, но элементов 2 и от них берем поле дробей:
$Z_{2}(x) \frac{f(x)}{g(x)}$ \\

\begin{theorem}[о характеристике поля]
	\kv{Если характеристика не нулевая, то она является простым числом}
\end{theorem}

\begin{proof}
	Пусть \bd{Р} - поле, \bd{k} - его характеристика. $\mt{ke}=0$ \bd {e} -
	нейтральный элемент по умножению. Пусть \bd {k} не простое то есть $k = sr$,
	тогда $(s \cdot r)e = (e+e+e...) \cdot (e+e+e...)$ \\
	Так как в поле нет делителя нуля, то \kv {se = 0 или re = 0} \\
	Допустим \kv {se = 0}. Если $s$ не простое, то продолжим эту же процедуру.\\
	По основной теореме - мы доберемся до простого числа.
\end{proof}

\kv {Пусть \bd{p} - простое число. Существует ли поле с такой
характеристикой?}\\

\begin{theorem}[Галуа]
	Для любого простого числа \bd {p} существует поле с характеристикой \bd {p} \\
	$Z_{p} = \{0,1, p-1\}$
\end{theorem}

\begin{proof}
	Так как оно не коммутативное кольцо с едицей, то для того, чтобы стало полем -
	нужно проверить наличие обратных по умножению. \\
	$0<a<p$ \\
	НОД (a,p) = 1 \\
	и из алгоритма Евклида: \\
	$
	\exists \mt {u,v} \in \mathbb {R} \\
	ua + vp = 1 \\
	ua = 1 - pv
	$ \\
	Остаток деления \kv {ua на p = 1} \\
	$\mt {u \cdot a = 1 ~in~} Z_{p}$ \\
	\kv {u} - обратный по умножению к \kv {a} \\
	Таким образом $Z_p$ - поле или GF(p) - поле Галуа
\end{proof}

\begin{title}
	Расширение полей
\end{title}

Если $P \supset F$, то F - \kv {подполе} поля Р, а Р - \kv {расширение} поля F\\

Прежде чем углубимся в тему расширения полей рассмотрим понятие векторного пространства.

Пусть $F$ - это поле. Веткорное(или линейное) пространство над полем $F$ - это множество $V$ 
(его элементы называются векторами), на котором определена бинарная операция $+$ сложение 
векторов.
Пусть $F$ = подполе $K$. Тогда $K$ можно рассматривать как векторное пространство над
полем $F$ относительно операции сложения элементов поля $K$ и умножения элемента 
поля $K$ на скаляр из $F$, понимаемого как обычное умножение элементов поля $K$.\\
Так, поле $\mathbb{R}$ действительных чисел можно рассматривать, как векторное 
пространство над полем $\mathbb{Q}$ рациональных чисел, а поле $\mathbb{C}$
комплексных чисел - как векторное пространство над полем $\mathbb{R}$.\\

Векторным пространство является множество $F[x]$ всех многочленов от переменной
$x$ над полем $F$ относительно сложения функций и умножения функции на скаляр.\\
Базис векторного пространства $V$ над полем $F$ называется линейно независимый 
набор векторов пространства $V$.\\
Число векторов в базисе векторного пространства $V$ называется размерностью 
пространства.\\

\begin{defin}
	Поле Р - является векторным пространством над полем F
\end{defin}

\begin{proof}
	Пусть $\mt{a,b \in P \quad \alpha,\beta\in F}$ \\
	Нужно 4 аксиомы коммутативности группы, и 4 аксиомы действия, но так как Р -
	поле, то они выполнены. \\
	\kv{Аксиомы действия:} \\
	$\mt{\alpha(a+b)=\alpha a + \alpha b}$ - Дистрибутивность \\
	$\mt{(\alpha \beta)a=\alpha a + \beta a}$ - Дистрибутивность \\
	$\mt{(\alpha \beta)a=\alpha(\beta a)}$ - Ассоциотивность умножения \\
	$1\alpha=a$ - Нейтральный элемент
\end{proof}

\begin{defin}
	Размерность пространства $\mt{dim_{F}P = |P:F|}$ называется степенью
	расширения поля. Если степень \kv{бесконечна}, то размерность
	\kv{бесконечномерна} \\
	$\mt {K \supset P \supset F}$ Если есть цепочка(последовательность)
	расширений, то она называется \kv{башней расширений}
\end{defin}

\begin{theorem}[о башне конечных расширений]
	Пусть $\mt {K \supset P \supset F}$ - башня конечных расширений, тогда
	размерность $\mt {|K:F|=|K:P|\cdot |P:F|}$
\end{theorem}

\begin{proof}
	$\mt {|K:P|=n \quad (a_{1}, a_{2}...a_{n}}$) \\
	$\mt {|P:F|=n \quad (b_{1}, b_{2}...b_{m}}$) \\
	по определению размерности, поле \bd{K} над полем \bd{P} имеет базис из
	$\mt {a_{1}, a_{2}...a_{n}}$ (из n элементов), а поле \bd{P} над полем \bd{F}
	базис $\mt {b_{1}, b_{2}...b_{m}}$ (из m элементов). \\
	Чтобы доказать теорему, необходимо проверить: $\mt{a_i b_j \quad i = 1 \ldots
	n \quad j = 1 \ldots m}$ \\
	являются ли базисом \bd{K} под \bd{F} и проверить: \\
	1) \kv{Они пораждают множество} \\
	2) \kv{Они линейно не зависимы} \\
	по определению базиса: \\
	$\mt {C \in K}$ \\
	$\mt {\alpha_{1} a_{1}+...\alpha_{n}a_{n}}$\\
	$\mt{\alpha_{i} \in P}$ \\
	А элемент из \kv {P} можно выразить через $\mt {b_{1}, b_{2}...b_{m}}$ c
	коэффициентом из \kv {F} \\
	В итоге \kv {C} выразится через $\mt{a_{i}b_{j}}$ с коэффициентом из \kv{F} \\

	\kv{Докажем линейную независимость}\\
	Пусть напротив они линейно зависимы: \\
	\[ \sum_{\substack {ij}} \gamma_{ij} \quad ij \in F \quad a_{i}b_{j} = 0\] \\
	Запишем эту сумму как линейную комбинацию элементов $a_i$ с коэффициентами из
	\kv {b и $\gamma$} \\
	\[ \sum a_{i} \left ( \sum \gamma_{ij} b_{j} \right) = 0 \] \\
	Так как $a_{i}$ - линейно независима, то \\
	\[\sum \gamma_{ij} b_{j} = 0\] \\
	Но так как $b_{i}$ тоже линейно не зависима, то $\gamma_{ij}=0$
\end{proof}

Теперь разберем что же такое расширение полей на самом деле и почему
его сравнивают с векторным пространством.\\
\begin{defin}
Если поле $F$ - подполе поля $P$ и $a$ - некоторый элемент поля $P$.\\
Минимальное поле, которое содержит поле $F$ и элемент $a$,
называют \kv{простым расширением поля} $F$, которое получено
присоединением к полю $F$ элемента $a$.\\
Это расширение будем обозначать через $F(a)$.
\end{defin}

Понятие поля позволяет вводить и использовать большое 
разнообразие колец, элементы которых определяются как многочлены\\
$f(x) = a_{0} + a_{1}x + a_{2}x^{2} + ... + a_{n}x^{n}$\\
с коэффициентами $a_{i}$ из данного поля $F$.\\ 
Такие многочлены называются \kv{многочленами над полем} $F$.\\
\kv{Кольцо многочленов над полем} F образуется всеми многочленами над $F$.
Оно обозначается $F[x]$. Операции сложения и умножения кольца $F[x]$
определяются теми же правилами, по которым складываются или 
перемножаются многочлены над действительным полем.\\
Говорят, что многочлен $g$ делит многочлен $f$ , если существует 
многочлен $h$ такой, что $f = g\ast{h}$, где $f, g, h \subset{F[x]}$.\\

\begin{defin}
Многочлен $f\subset{F[x]}$ называется \kv{неприводимым}, 
если $f = g\ast{h}$ только в том случае, когда либо $g$ либо $h$
является константой.\\
Многочлен $f$ ненулевой степени из кольца $F[x]$ называется
неприводимым в $P[x]$, если он не делится ни на какой многочлен $g\in{P[x]}$.
\end{defin}

В частности всякий многочлен первой степени неприводим.\\
Неприводимые многочлены среди многочленов играют ту же роль,
что и простые числа среди целых чисел.\\ 

А если мы пойдем глубже, то найдем еще одно определение:\\

\begin{defin}
Пусть $F$ - подполе поля $P$ и $f(x)$ - многочлен над полем $F$.
Обозначим через $a$ корень многочлена $f(x)$ в поле $P$
(заметим, что не в $F$). Тогда простое расширение $F(a)$
называют простым алгебраическим расширением, которое
получается путём присоединения к полю $F$ корня $a$
многочлена $f$. 
\end{defin}

Получается, что все элементы поля $P$ являются линейными комбинациями
конечного множества элементов $\quad(a_{0}, a_{1}...a_{n})$ 
с коэффициентами из $F$.\\


Используя неприводимые многочлены, можно строить 
\kv{новые конечные поля - расширения} простых полей $F_{p}$.\\
1. Выбираем простое $p$ и фиксируем поле\\
$F_{p} = [ {\overline{0}, \overline{1}, ... , \overline{p - 1}}, +_{p}, \ast{_p}]$\\
2. Рассматриваем кольцо $F_{p}[x]$ многочленов над ним.\\
3. Выбираем натуральное $n$ и нериводимый многочлен.\\
$P(x) = a_{n}x^{n} + ... + a_{1}x + a_{0} \in F_{p}[x]$ \\
4. Идеал $(P(x))$ порождает фактормножество $F_{p}[x]/(P(x))$,
элементы которого суть совокупность ${R(x)}$ остатков от
делениия многочленов $f\in{F_{p}[x]}$ на $P[x]$.\\
$f(x) = Q(x)\ast{P(x)} + R(x)$.\\

Множество ${R(x)}$ явдяется полем Галуа $GF(p^{n})$.\\

\begin{defin}
	Пусть \kv{P} - некоторое конечное расширение $\mathbb Q$\\
	\[P \supset Q \]
	Число $\alpha$ - называется \kv{алгебраическим}, если оно является корнем
	многочлена с целыми каэффициентами. \\
	Если число не является \kv {алгебраическими}, то оно \bd {трансцендентно}
\end{defin}

\kv {Алгебраических чисел - счетное число}, так как разных многочленов с целыми
коэффициентами - \kv { счетное число}. \\
\kv{Действительных чисел - континуум} \\
Число Эллера {\bk e} = 2.718281828459045 \\
Число ПИ $\pi $ \\
Они трансцендентны.\\

\begin{theorem}
	Если \kv{P} - конечное расширение поля $\mathbb {Q}$, то все его элементы -
	\kv{алгебраические числа}
\end{theorem}

\begin{proof}
	$\mt {a \in P}$ и возводим $\mt {1, a, a^2, a^3...a^n}$ \\
	$\mt {|P:Q| = n}$ \\
	Так как элементов \kv{n+1}, то они \kv{линейно зависимы} \\
	$\alpha_{i} + \alpha_{i}a + ... + \alpha_{i}a$ %в этой формуле я не уверен,
	%да и в следующей тоже. сверь со своим конспектом.
	Находим НОК знаменателя $\alpha_{0} ... \alpha_{n}$ \\
	Умножение на него обе части и выходит многочлен с целыми коэффициентами, а
	элемент \kv{а} - его корень.
\end{proof}