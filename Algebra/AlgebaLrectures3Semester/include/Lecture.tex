\begin{title}[\large]
  Направление информационная безопасность
\end{title}

Используется из математики: вектороное пространоство, конечные поля, теория
кодом (алгебра), основы математической логики, теория чисел, простые чила это
хлеб криптографи, элиптические кривый, протоколы сетевой взаимодействии.

После 2014 года однополярый мир стал рушится и информационная война стала очень
горячей

Организационный правовые методы зашифровывания информации: конституция,
доктриа информационной безопасности, федеральные законы (149) в этом законе
вводится вся терминалогия.

Приодятся не филосовские а юридические термины: защита персанальных данных,
электронная подписть

Программные, аппаратные методы зишифрования информации или компютерная
безопасность: сетевые протоколы, информационная система, антивирус, вирусы,
приптографи, теория кодом, алгебра.

Инженерно технически методы зашифровки информации: системы фунтификации и
дитефикации, система наблюдения, охранные системы, сейфы, ключи, проволка.

1) направление электронные документы, стереографи - скрытие самого факта
информации водяными знаками.

2) Эксперементальная теория числ, жксперементальная алгебра. Эксперементальная
это вычисление нерешенных математических проблем гипотеза для теории математики

длина цепочки

сшучещние просты чисел

$n$ простых числе расположенных на $\max$ коротком отрезке.

простые числа близнецы 17 19, 41 43

в настоящее время саое большое $n$ для которого найдено плотность $n = 21$

в криптографии большинство протоколов алгоритмов начинается с пусть $p$
простое случайно число. Когда $p$ входит в плотную $n$. Плотность $n$
встречается примерно в $2^n$ раза чаще чем должно.

Бинарная алгебраическая операция на множестве $A$ называется любое отображение
$f:A\times A \to A$ (из прямого произведения в себя)

Если $f: A^n \to A$ операция называется $n$-арная и ей занимается полимиальная
алгебра

Например det матрица разложения $n \times n$ можно расмотреть как $n$-ая операция
если разбить матрицу на строки или столюцы и можно расмотреть как $n^2$-орную
если расматривать матрицу по элементно при $n=1$ тоесть когда $f:A \to A$
операция называется уравной (взятие обратного элемента унарное)

Бинарная операция когда двум элемментам множества $A$ ставится в соответсвии
третий элемент множества $A$ $g:(a,b) \mapsto C$ в математическом анализе
$f(a,b) = C$ в алгебре $afb= C$ Вместо длинных слов Бинарная алгебраическая
операция говорят просто умножение или сложение

Операция называется коммутативной если $\forall a,b \in A ~~ f(a,b) = f(b,a)$

В случае если операция коммутативна ее обычно но невсегда называют сложением

$|A| = n$ если множества $A$ содержат $n$ элементов то $n^{n^2}$ алгебраичкских
операций.

Перечень хороших свойств

-коммутативность

-ассоциативность

-нейтральный

-наличие обратно

1) Если операция на множестве $A$ является ассоциативной это называется
полугрупой

2) Если операция ассоциативна и есть нейтральный то это монои. Моноид это
полугруппа с 1

Пример множетво четных чисел без 0

Если на множестве $A$ определена ассоциативность есть нейтральный и обратный
это группа (можно умножать и делить)

На множестве $A$ можно задать несколько алгебраисеских операций

Если операции не будут связаны между собой то это равносильно что у нас есть
каждой множество каждый со своей операциией

Пусть на множестве $A$ задана коммутативная операция сложения и относительно ее
$A$ является группой. И задана вторая операция и на ней не наложено никаких
условий назовем умножение. Но эти операции связаны между собой аксиомой
дистрибутивности

$a(b+c) = ab + ac$ левая

$(b+c)a = ba + ca$ правая

Если умножение коммутативно то обе дистрибутивности эквивалентны

Множества $A$ с коммутативной сложение и с дистрибутивностью называется кольцом

Различается много видов колец

1) если умножение ассоциативно то кольцо называется ассоциативным

Примеры асоциативных колец: кольца целых чисел, кольцо матриц над ассоциативным
кольцом

2) Примеры неассоциативного кольца: трех мерное пространство векторов где
сложение это сложение векторов а умножение это векторное умножение кольца

Так как суперпозиция функции фссоциативна а большинство процесов в природе
и науке это отображение то чаще всего встречаются ассоциативные кольца

3) ассоциативное кольцо с коммутативной умножение называется коммутативным
кольцом

Пример кольцо матрицы размерности > 1 всгда некомутативно

Если в кольце ненулевые элементы по умножению обладают некомутативностью группы
то такое кольцо называется телом

Полю действительных чисел добавис мнимых едениц $R, j,ik$

$i^2 = j^2 = k^2 = -1$

$i-j=k ~~ jk=i ~~ ki = j ~~ ji = -k$

Если в ассоцииативном кольце непустое элементы образуют коммутативную группу то
кольца называются полем

Основные поля $Q<R<C ~~~ GF(P^2)$ поля Галуа

\begin{title}[\large]
  Краткий обзор теории полей
\end{title}

Натуральное число $k$ называется характиристикой поля если оно мнимое число
обладает свойствами адитивности. Если такого $\not \exists$ характеристика поля
$ka = a+ \ldots + a = 0 ~~ \forall a \in p$

Пример кольцо $Z_{14}$ имеет характеристику 14 $Z_{14}[x]$

\begin{theorem}
  Если характеристика поял ненулевая то она обязательно является простым числом
\end{theorem}

Если поле $P$ является подмножеством $P<K$ поля $k$ и на на них заданы одним и
тем же алгебраичесие операции то $P$ называется подполем

Поле называется простым если в нем нет собственных подполей

В теории групп и колец тоже есть операция простой группы и кольца но их описание
чудовищно трудны

\begin{theorem}
  Единственным простым полем нулевой характеристики является поле $Q$
\end{theorem}

\begin{theorem}
  $\forall$ простого числа $p$ единственным простым полем числа $p$ является
  кольцо вычитов $Z_p = \{0,1, \ldots, p-1\} = GF(p)$
\end{theorem}

Расмотрения полей

Если поле $P \le K$ является подполем поля $K$ то поле $K$ называется
расширением поля $P$

Пусть $P$ поле a $A$ коммутативная группа по сложени. $P \times A \to A$
называется действием поля $P$ группы по сложению $A$ если выполнено некоторое
условие

$\alpha, \beta \in P ~~ \forall \in A$

1) $(\alpha + \beta) a = \alpha a \beta a$

2) $(\alpha \beta)a = \alpha(\beta a)$

3) $\alpha(a+b) = \alpha a + \alpha b$

4) $1a = a$

Тут поле $P$ активная часть а множество $A$ пассивная

Если на коммутативной группе $A$ задана действие поля $P$ то эта группа
называется векторным произведением над полем $P$

В кольце заданы две алгебраическии операции которые неравномерно $A$ в
векторном пространсве алгебраическая операция и действие внешнего поля

Элементы векторного пространства называется векторами

Векторая $a_1, \ldots, a_n$ называются линейно не зависимыми если из того
что $\alpha_1 a_1 + \ldots + \alpha_n a_n = \vec 0 ~ \Rightarrow ~
\alpha_1 = \ldots = \alpha_n = 0$

Множество векторов называется пораждющимся если каждый элемент векторного
пространства можно записать как линейную комбинацию векторов

Фундаментальная операция множества векторов называется базисом если оно
порождает ЛНЗ то есть самое экономное порождает множество

Пространство называется конечным мерным если у него есть хотя бы 1 коненый
базис

\begin{theorem}
  Если пространство конечномерно то все его базисы имеют конечное число
  элементов называемое размерностью $\dim_p V$
\end{theorem}

Не все полезные пространства конечномерны например в функциональном анализе
есть бесконечно мерный гилбертово прстранство

Фундаментальное понятие отношение

Бинарное отношение на множестве $A$ называется любое подмножество его декартого
произведения $A \times A$ $|A| = n$ если множества содержит $n$-ов то всего
отношений будет $2^n$

Отношение называется рефлексивным $P$ если $\forall a \in A ~ \Rightarrow ~
(a,a) \in P$

Отношение называется симметричным если $(a,b) \in P ~ \Rightarrow ~
(b,a) \in P$

Отношения называеются антисиметричным $(a,b), (b,a) \in P ~ \Rightarrow ~ a=b$

Отношения называются транзитивным $(a,b), (b,c) \in P ~ \Rightarrow ~ (a,c) \in
P$

Отношения называются нестрогово неравенства если оно рефлексивно, транзитивно
и антисиметрично. И строгово неравенства если оно транзитивно и антисиметрично

Отношения называются отношением эквивалентности если оно рефлексивно
транзитивно и симетрично

\begin{title}[\large]
  Теория полей
\end{title}

Пусть $P < K$ (поле $P$ содержится в поле $K$). На поле $K$ можно смотреть как
на в.п над полем $P$. И его это пространство конечномерно то такое расширение
$K$ назывется конечномерным расширением

Если захотим еще и поле $K$ расширить то $P<K<L$ базис расширений то есть
$\dim_p K = |K : P| ~~ |L : P| = |L:K||K:P|$

Алгебраическое расширение элементовв $g \in K$ называется алгебраическим над
полем $P$ если оно является корнем какого нибудь многочлена из $P$

\begin{theorem}
  Любое конечно расширение является алгебраическим
\end{theorem}

Элемент называется трансцидентым если не существует нужного многочлена корнем
которого он является

Трансцидентым расширение является поле дробей числа $e, \pi$ трансцедентны

-Структура простого алгебраическоо расширения

-1,2,3 теорема о полях Галуа

-Теорема о полях с неприводимым многочленом

-Теорема об описанмм неприводимого многочлена над полем Галуа

Пусть $P$ поле и $\alpha$ аглебраический элемент. По определению аглебраический
элемент существует многочлен $f(x) = x^n + ax^{n-1} + \ldots + a_0$ корнем
которого он является та как таких множеств не единственный а нам надо
определится

Разложим этот многочлен $f(x)$ на неприводимые множетели тогда $\alpha$ будет
корнем одного из неприводимых. На саом деле единственный бует такой многочлен
и его называют мнимым.

Пусть напротив таких многочленов два $g_1(x), g_2(x)$ так как они неприводимые
о их нод$(g_1, g_2) = 1$ По условию из алгоритма Евклида существует такие
многочлена

$h_1, h_2$ что $h_1g_1 + h_2g_2 = 1$

$h_1(\alpha)g_1(\alpha) + h_2(\alpha)g_2(\alpha) = 1$

$0 + 0 = 1$ противоречие

Значит неприводимый многочлен действительно единственный

Можем считать что $f(x)$ мнимый многочлен $\alpha$

Мнимое поле которое содержит и поле $P$ и элемент $\alpha$ называется простым
алгебраическим расширением $P(\alpha)$

Как устроено это поле $P(\alpha)$?

1) Абстрактное строение

Расмотрим идеал порождающиий многочлен $f(x)$ то есть $I = ug(f(x))$ это
главный идеал состоит из всех карытных многочленов $f(x)$
$$
I = uf(f(x)) = \{L(x)| L(x) = f(x)h(x), h(x) \in P[x]\}
$$

По теореме по построению поле разложения у нас получается поле в котором
многочлен $f(x)$ имеет хотя бы один корень мы можем считать что это наш
$\alpha$

2) Символьное описание простого расширение

У нас есть $P$ и символ $\alpha$ который является корнем многочлена то есть
$f(\alpha) = 0$ Расмотрим $\alpha, \alpha^2, \ldots, \alpha^{n-1}$ так как
многочлен $f(x)$ имеет $n$-ую степень то возникает соотношению

$\alpha^n + a\alpha^{n-1}+ \ldots + a_0 = 0$ откуда $\alpha^n$ можем выразить
через элемент меньший то поля $P(\alpha) = \{b_0 + b_1\alpha + \ldots +
b_{n-1}\alpha^{n-1} |b_i \in P\}$ или другими словами является векторным
пространстваом размерности $n$ над полем $P$. Базис $1,\alpha, \ldots,
\alpha^{n-1}$

$\alpha^n = - a_{n-1}\alpha^{n-1} - \ldots - a_0$ вот это соотношение задает
умножение в поле

Степени выше $n$ получаются на умножении на $\alpha$ этого равенства с
последующем исользованием этого же элемента

Формула теоремы (которую доказали выше)

Простая алгебраическое расширение $P(\alpha)$ изоморфно $P[x]/ug(f(x))$ где
$f(x)$ многочлен и $P(\alpha) = a_{n-1}\alpha^{n-1}- \ldots - a$

Обобщение: алгебраическое расширение $P(\alpha_1, \alpha_2, \ldots, \alpha_n)$
строится идуктивно $((P(\alpha_1)(\alpha_2))(\alpha)$

Если поле не производное а поле расширенных чисел то алгебраическое расширения
называется алгебраическим числом.

Формально алгебраические числа это частный случай алгебраического расширения на
само деле все теория поле родилась из алгебраических чисел

\begin{title}[\large]
  Структура полей
\end{title}

Полем Галуф называется любое конечно поле та как поле имеет ненулевую
характеристику то говорят оле Галуа всегда указывает ее характеристическу $P$
простое число

\begin{theorem}
  Единственное простое поле характеристика $p$ это кольцо вычитов
  $Z_p = \{0,1, \ldots, p-1\} = GF(p)$
\end{theorem}

То что $Z_p$ поле было доказано в 1 семестре то что $p$ характеристика это
очевидно

Почему это поле простое то есть не содержит подполей?

Так как любое подполе содержит 1 по умножению то можем эту 1 само с собой
сказывать

\begin{theorem}
  $\forall P$ простое число и $\forall n \in N$ $\exists$ поле $GF(p)$ содержит
  $p^n$ элементов
\end{theorem}

То что существует поле $GF(p)$ следует из теоремы 1

Расмотрим многочлен $x^{p^n} - x$ По теореме о поле разложение существует поле
$F$ в котором этот многочлен разлогается на лмнейные множетели

Докажем что это поле содержит ровно $p^n$ элементов то есть все поле состоит из
корней одного и того же многочлена

Пусть $\alpha$ $\beta$ этого многочлена то есть $\alpha^{p^n} - \alpha = 0$ и
$\beta^{p^{\alpha}} - \beta = 0$ так как $\alpha, \beta \in F$ то по притерию
подполя нужно проверить что

$\alpha + \beta$ корень

$\alpha \cdot \beta$ корень

$-\alpha$ корень

$\alpha^{-1}$ корень

1) Проверим что $\alpha + \beta$ корень

$(\alpha + \beta)^{p^n} - (\alpha + \beta)$ так как характеристика поля равна
$P$ то есть $P\alpha = 0$ то по Биному Ньютона

$(\alpha + \beta)^{p^n} - (\alpha+\beta)$

$\alpha^{p^n} + \beta^{p^n} - (\alpha+\beta) = 0$ корень

2) $(\alpha \beta)^{p^n} - \alpha \beta$

3) $-\alpha$

4) $\alpha^{-1}$

то поле $FG(p) ~ \exists$ и состоит из корней многочлена $x^p - x$

\begin{theorem}
  Пусть $GF(p^n)$ некоторое поле Галуа а $CF(p^n)$ каканибудь другая тогда
  $GF(p^n) \ge GF(P^{\alpha}) ~ \Leftrightarrow ~ m|n$
  то есть структура подполей определяется структурой делимости числа $n$
\end{theorem}

Пусть поле $GF(p^n) \le GF(p^n)$ так как $P \le F$ то F является векторным
пространством над полем $P$

Пусть разложение этого пространства = k тогда $|F| = |P|^k$

$P^n = (p^m)^k$

$n = mk$

в другую сторону. Пусть пол разложения многочлена содержит в поле разложения
$GF(p^n)$ по теореме

\begin{title}[\large]
  Поля Галуа
\end{title}

Неприводимые многочлены в поялх Галуа

Неприводимый многочлен это многочлен не расходится многочленов меньший степени
то есть простой чье кольцо многочленовю

Пример: многочлен 512 состоит из 3-5 слагаемых неприводимых

Пусть $f(x)$ неприводимый многочлен над полем $GF(p)$ Допустим его степень $=n$
$\alpha$ найменьший корень $GF(p^n)$ Тогда $\alpha, \alpha^p, \ldots,
\alpha^{p^{n-1}}$ это все различные корни многочлена $f(x)$

Замечание: Если в поле Галуа мы добавим один корень неприводящего многочлена то
многочлен разлогается на линейные многочлены

Доказательство. Пусть $f(x)$ имеет вид $f(x) = x^n + a_{n-1}x^{n-1} + \ldots +
a_0$ $a_i \in GF(p)$ По условию дано, что $0 = f(\alpha) = \alpha^n +
a_{n-1}\alpha^{n-1} + \ldots + a_0$ Чтобы доказать теорему нужно проверить что
возведение в степень $p$ остается корень корнем. Заметим что $a_i \in GF(p)$
$\forall i ~ a_i^p = a_i (*)$ Заменим $\alpha$ на $\alpha^p$ получим
$(\alpha^p)^n + a_{n-1}(\alpha^p)^{n-1} + \ldots + a_0$ исполняют роль (*) и
$(\alpha^p)^n + a_{n-1}(\alpha^p)^{n-1} + \ldots + a_0 = (\alpha^n)^p +
a^p_{n-1}(\alpha^{n-1})^p + \ldots + a_0^p$ так как характеристика $=p$ то есть
$pa = 0$ то по формуле Бирнуле запиши так $(\alpha^n)^p + a^p_{n-1}
(\alpha^{n-1})^p + \ldots + a_0^p = (\alpha^n + a_{n-1}\alpha^{n-1} + a_0) =
f^p(\alpha) = 0^p = 0$

Почему все корни разные?

Если бы из них пара была одинаковой то тогда $\alpha$ был бы корнем многочлена
степени $<n$ и должен быть бы делит $f(x)$ а что невозможно.

\begin{theorem}
  Пусть $f(x) \in GF(p)[x]$ неприводимый многочлен его степени $f(x)$
\end{theorem}

Утверждение: многочлен $f(x) ~ \Leftrightarrow ~$ делит многочлен $(x^{n} -x)$
когда $m|n$

Вывод из теоремы все неприводимые многочлены степени $m$ если $m|n$ все
находятся как сомножетели $b(x^{p^m}) -x)$

Доказательство: Пусть $f(x)$ $(x^{p^m} - x)$ значит его поле разложения
$GF(p^m)$ сходится внутри поля разложения $GF(p^n)$ то есть
$GF(p^m) \Delta GF(p^n) ~ \Rightarrow ~ m|n$

Обратно: Пусть $m|n$ тогда поля разложения многочлена $GF(p^m) < GF(p^n)$ и
значит все корни многочлена $f$ является корнем больше многочлена значит он его
делит

Пример: Пусть $p=4$ а $n=2$ Расмотрим многочлен $(x^3 - x)$ над $GF(3)$
Пересечением всех неприводимых многочленов 2-ой степени надо полем $GF(3)$
Они имеют вид $x^2 + \alpha x+ \beta ~~ \alpha, \beta \in GF(3)$ Если
многочлен 2 степени неприводим значит у него нет корней

Теорема о примитивном элементе

Порождай элемент мультипликативной группы поле называется примитивным.

Замечание: Практика любой криптографии протокола и алгоритмы называется со
словом. Пусть $p$ большое простое число

$G$ примитивный элемент по $|p|$ Сообщение кодируется по модюлю элемента в
степени все ненудевые элементы записываем как его степени работал с логарифмом
Якоби умножение сведется к сложению.

Теорема: В любом конечном поле $GF(p^n)$ $\exists$ примитивный элемент то есть
мультипликативной группы этого поля умножения

$h = p^n -1$ (порядок мультипликативной группы)

$h = p^n -1 = p_1 - p_s$ разлжение на простые множетели

Для каждого: расмторим многочлен $x^{hp_i}-1$ так как этот многочлен имеет
степень $<h$ то не все ненулевые элементы являеся его корнями

Пусть $a_i$ не корень то есть $a_i^{hp_i} \not= 1 ~~ b_i = a_i^{n}$ По
теореме Логранжа каждый элемент в степени равный порядку группы $=1$
$b_i^{p_i} =1$ но его порядок может быть и меньше однако если
$b_i^{p_i-1} = a_i \not=$ порядок элементов $p_i = p^{\alpha_1 i}$

Элементы $b=b_1b_2\ldots b_s$ и его примитивным элементов так как порядок всех
$b$ взаимно просты мужду собой то их НОД равно их $h = p^n-1 = p_1^{\alpha_1)
\ldots p_s^{\alpha_s}}$ Если по этой теории искать примитивный элемент то нужно
перебрать все элементы в поле

Алгоритм нахождения примитивного элемента

1) Порядок мультипликативной группы $h$ расходится на простые множетели
$h = p_i^{\alpha_1} \ldots p_s^{\alpha_i}$ Если находятся в простом поле
$GF(p)$ то по порядку переберем $g =2,3,5,7,13,17, \ldots$ Тот элементы для
которого эти степени $\not =1$ и будет примитивным

\begin{title}[\large]
  Нахождение примитивных элементов Логарифм Якоби. Решение уравнение в конечных
  полях
\end{title}

Почти любой криптографический алгоритм начинается со слов

Пусть $p$ больщое простое число $g$ примитив элемента поля $GF(p)$

Пример: протокол формирует обмен секретными ключами через открытый канал связи

Если $A$ $B$ $p$простое $g$ примитив

$A: 0 < x_A < p-1$

$B: 0 < x_b < p-1$

$A \to B: g^{x_1} \to g^{x_1 x_2} = k$

$B \to A: g^{x_2} \to g^{x_2 x_1} = k$

$E:g^{x_1}g^{x_2}$

Чтобы вынести какой ключ пересылали то есть чтобы найти $x_1$ $x_2$ нужно
произвести операцию секретного алгоритмизация Его легко обобщить то есть
если у нас не $t$ а $n$ субъектов они придают свои секреты числа
$n_1, n_2 \ldots n_k$


2Пример При создании электронны подписей строится элиптическое кривая над полем
$GF(p)$ потом находит ее аддитивнй группой и ее проъодит элементы

Пусть поля $GF(p)$ тогда кго мультипликативная форма $|GF*(p)| = p-1 =
p_1^{\alpha_1}p_2^{\alpha_2} \ldots p_s^{\alpha_3}$ его разложения на
множетели

\begin{theorem}
  Если поле $p$ содержит $u$ элементов то кол-во различных примитивных
  элементов $\varphi(q-1)$ Зачем нужно знать сколько примитивных элементов?
\end{theorem}

Как вычисляется функция Эйлера?

1) функция Эйлера мультипликативна то есть если $n=mk ~ (m,k) = 1)$
взаимное простое то $\varphi(n) = \varphi(m) \varphi(k)$ поэтому
$n = p_1^{\alpha_1} \ldots p_s^{\alpha_5}$ то $\varphi(n) = \varphi(p_1^{
\alpha_1} \ldots \varphi(p_s^{\alpha_s}))$ Несложно заметить
$\varphi(p^{\alpha})$ что каждое $p$ число делится на $p$ значит общее число

2) $\varphi(p^{\alpha}) = p^{\alpha} - p^{\alpha-1}$

Как строится поле расширения?

Берем неприводимый многочлен (не расподается на многочлены) и добываем
формальный конечнь например $\alpha$

Логарифм Якоби

Пусть поле $p$ $a$ примитивный элемент тогда любой ненулевой элемент этого $p$
может быть представлен $0 < i < |p|$ Операция сложения при комплексной
стерилизации выполнит за 1 2 такта а операция умножения как минимум 2 такта
поэтому всегда стремится умножения заменит сложением и примитивный элемент
идеальное средство $C = a^j ~~ bc a^{i+j}$ При использовании элемента
умножения сводятся к сложению показателся

Если $a$ примитивный элемент $b = a^i$ то $log_a b = i$ возникают проблемы со
сложением $b + c = a^i + d^j = a^i(1+ a^{j-i})$ Проблема чему равняется
$1+a^k = a^l$ $L$ log Якоби

Алгоритма RSA

исходные данный $p,q$ прострые числа

$p = 197$

$q = 199$

$n = p * q$

$n_1 = Ph(n)$

Алгоритм зашифрования

Открытым текстом является число $0<x<n$ зашифрованием $F(x) = x^2 mod n$

Алгоритм расширения

$D(y) = y d mod n$

Норма и сдел элеметов в конечнном поле

ПУсть $\alpha$ корень неприводимого многочлена характреристического $p$ тогда
$\alpha, \alpha^p, \alpha^{p^2}, \ldots \alpha^{p^{n-1}}$ остаются корни

Нормируем след $N(\alpha) = \alpha \alpha^p \ldots \alpha^{p^{n-1}} \in T$

$Tr(\alpha) = \alpha + \alpha^p \ldots \alpha^{p-1} \in T$

Несложно проверить что и след и норма принадлежат исходному полю расмотренная
которая $f(x) \in P[x]$

$N(\alpha \beta) = N(\alpha)N(\beta)$

$Tr(\alpha + \beta) = Tr(\alpha) + Tr(\beta)$

Тогда при помощи элемена $\alpha$ мы можем задать линейное обтображение

$\alpha: P(\alpha) \to P(\alpha)$

$\alpha: x \to \alpha x$

Линейное отображение сумму переводит в сумму

$L_2(x+y) = \alpha(x+y) = \alpha x + \alpha y = L_2(x) L_{\alpha}(y)$

$L_{\alpha}(1) = \alpha 1 = \alpha$

$L_{\alpha} = \alpha \alpha = \alpha^2$

$L_{\alpha} (\alpha^{n+1}) = \alpha^n$

$x^n + a_{n-1}x^{n-1} + \ldots + a_n$

$\alpha^n = - a_{n-1} \alpha^{n-1} + \ldots + a_0$

Если проверить то характеристика многочлена этого множетеля отображение
совпадает с множеством корней которого является $\alpha$

Следом этого многочлена отображения является коэффицент $a_{n-1}$ поэтому что
по Виета и есть сумма всех корней $Tr(L \alpha)= - a_{n-1}$ Норма
$N(L \alpha) = a_0$ так как по виета он является производных всех корней

Элемент теории чисел конечные кольца

Поля Хороши им что в них осуществляется все и арифмитические операции. В
кольцах это не всегда возможно деление

2) В кольцах делители произведение ненулевого элемента может дать $0$

Пример: $Z_n$ имеет разложим на простое многочлен $n = p_1^{\alpha_1}\ldots
p_s^{\alpha_s}$ В конечных полях мультипликативные группы всегда имеет примитив
элементов то есть она ункальная В кольцах вычетов примитив элементов
существует не всегда

\begin{theorem}
  Кольца вычетов имеет примитив элементов $\Leftrightarrow$ когда $n=p$ или
  $n = 2p^{\alpha}$
\end{theorem}

Идея доказательства теоремы. Доказательство следущие по $\alpha$ База
индукции $\alpha = 1$ В этот случае кольца вычитов $Z_p$ является полем в нем
обязанно есть примитивный элемент называется если $g \in Z$

Предположение индукции  пусть для кольца примитив элемента найден то етсь есть
элемент $h$ который $|p_{\alpha-1}|$ уникальной группой

Шаг индукции проверяет что $mh + p$ то $|p^{\alpha}|$ является примитивный. Для
этого возводи в степень $p$ В случае когда орядок кольца это $2x$ в этом
случае. Кольцо является прямым произведение кольца $Z[s]$ поэтому примитвным
элементов будет.

\begin{title}
  {Структура простого алгебраического расширения}
\end{title}

Пусть $P$ - поле, $\alpha$ - алгебраический элемент. По определению
алгебраического элемента,\\
$f(x) = x^{n} + a_{n-1}x^{n-1} +...+a_{n}$\\
корнем которого он является.\\
Так как такой многочлен не единственный, а нам нужна определенность. Разложим
$f(x)$ на неприводимые множители, тогда $\alpha$ - будет корнем одного из них.
На самом деле, единственный будет такой многочлен и его называют минимальным.
Пусть напротив таких многочленов 2: $g_{1}(x) и g_{2}(x)$ т.к они неприводимы,
найдем
$НОД(g_{1}, g_{2}) = 1$. По следствию из алгоритма Евклида
$\exists h_{1}, h_{2}$ так что $h_{1}g_{1} + h_{2}g_{2} = 1\\
h_{1}(\alpha)g_{1}(\alpha) + h_{2}(\alpha)g_{2}(\alpha) = 1 \Rightarrow 0 + 0
= 1$ (противоречие)\\
Минимальное поле, которое содержит $\alpha$ и называется $P(\alpha)$ простым
алгебраическим расширением.\\
{\emph{Как устроено это поле $P(\alpha)$?}}\\
{\bfseries 1. Абстрактное строение:}\\
Рассмотрим идеал, порождаемый многочленом $f(x)$, то есть\\
 $I = ug(f(x)) = {e(x)| e(x) = f(x)h(x), h(x) \in P[x]}$\\
 Рассмотрим фактор кольцо $P[x]/ug(x)$ (из теоремы о построении поля разложения)
многочлен $f(x)$ имеет хотя бы один корень и пусть это будет $\alpha$\\
{\bfseries 2. Символьное описание простого расширения:}\\
$p(\alpha) = {b_{0}+b_{1}\alpha+...+b_{n-1}\alpha^{n-1}| b_{j}\in P}$ - это поле
является векторным пространством размерности $n$ над полем $P$. Базис
$(1, \alpha,...,\alpha^{n-1}), f(\alpha) = 0\\
\alpha^{n} = -a_{n-1}\alpha^{n-1}-...-a_{n}$ - это соотношение задает умножение
в поле.\\

\begin{theorem}
  Простое алгебраическое расширение $P(\alpha)$ изоморфно\\
$P[x]/ug(f(x)) и P(\alpha) = {b_0 + b_1\alpha +...+b_{n-1}\alpha^{n-1}|
b_j \in P}$\\
\end{theorem}

{\bfseries Обобщение}\\
$\alpha_1, \alpha_2,..., \alpha_k\\
P(\alpha_1, \alpha_2,..., \alpha_k)\\
(P(\alpha_1))(\alpha_2))(\alpha_3)$ - алгебраическое расширение
$P(\alpha_1, \alpha_2,..., \alpha_k)$ строится индуктивно.\\
Если поле непроизвольное, а поле рациональных чисел, то алгебраические
расширения
называются алгебраическими числами.\\
Вся теория полей родилась из алгебраических чисел.\\

\begin{title}
  {Структура полей Галуа}
\end{title}

\begin{define}
  Поле Галуа - это любое конечное поле.
\end{define}

Так как конечное поле имеет ненулевую характеристику, то у поля Галуа
характеристика P - простое число.\\

\begin{theorem}
  $\mathbb{Z}_{p} = {0, 1,..., p - 1} = GF(p)$  - единственное простое поле,
  то есть не содержит подполей.\\
\end{theorem}

\begin{proof}
  Так как любое подполе содержит единицу по умножению.\\
  $\{1, 1+1, 1+1+1...\} = \mathbb{Z}_{p}$\\
\end{proof}

\begin{title}
  Теорема о существовании простого поля Галуа
\end{title}

\begin{theorem}
  Для любого простого числа p и натурального числа n существует поле, содержащее
$p^{n}$ элементов.
\end{theorem}

$x^{p^n} - x$ - по теореме о поле разложения, разлагается на множители.\\

\begin{proof}
Пусть $\alpha, \beta \in F$ - корни этого многочлена, то есть\\
$\alpha^{p^n} - \alpha = 0$\\
$\beta^{p^n} - \beta = 0$\\
1. Проверим, что $\alpha + \beta$ - корень\\
${(\alpha + \beta)}^{p^n} - (\alpha + \beta)$ так как характеристика
поля = p, $p\alpha = 0$\\
То по Биному Ньютона:\\
$\alpha^{p^n} + \beta^{p^n} - (\alpha + \beta)$ - действительно корень.\\
$\alpha\beta$ - тоже корень.\\
${(\alpha\beta)}^{p^n} - \alpha\beta$
2. ${\alpha}^{p^n}{\beta}^{p^n} - \alpha\beta$ - корень.\\
3. $-\alpha\\
4. {-\alpha}^{-1}$\\
\end{proof}

Таким образом поле $GF(p^n)\ \exists n$ состоящий из корней многочлена $x^p-x$.

\begin{title}
  Структура подполей поля Галуа
\end{title}

\begin{theorem}[о структуре подполей]
  $GF(p^n) и GF(p^m)$ - поля Галуа.\\
Тогда, $GF(p^n) \ge GF(p^m) \Rightarrow m|n$(делит)
\end{theorem}

То есть структура подполей определяется структурой делителей числа n.\\

\begin{proof}
          Пусть $GF(p^n) \le GF(p^m)$, где $GF(p^n) = P и GF(p^m) = F$\\
Так как P подполе F, то - векторное пространство над полем P. Пусть размерность
этого пространства K.\\
$\mid F \mid = \mid P \mid^k\\
p^n = {(p^m)}^k\\
n = mk$\\
Пусть $m\mid n$, тогда $\frac{x^{p^n} - x}{x^{p^m} - x}$, по формуле геометрической прогрессии.
Значит поле разложение нижнего многочлена содержится в поле разложения верхнего
многочлена. Является подполем, а поле разложением $GF(p^n)$ по 2 - ой теореме.\\
\end{proof}

\begin{title}
  {Неприводимые многочлены в полях Галуа}
\end{title}

\begin{theorem}
  Пусть $f(x)$ неприводимый многочлен над полем $GF(p)$.\\
$\alpha$ - корень $GF(p^n)$(n - степень). Тогда $\alpha, \alpha^p,...,
\alpha^{p^{n-1}}$
- различные корни многочлена $f(x)$. Если в поле Галуа добавить один корень, то
многочлен разлагается на простые множители.\\
\end{theorem}

\begin{proof}
  $0 = f(\alpha) = \alpha^n + a_{n-1}\alpha^{n-1} + ... + a_0 = 0, a \in GF(p)$\\
  По условию дано, что $0 = f(\alpha) = \alpha^n + a_{n-1}\alpha^{n-1} + ...
  + a_0 = 0$.
  Чтобы даказать теорему, нужно проверить, что возведение в степень p ост-ет
  корень корнем.\\
  Заметим, что $a_i \in GF(p) \ \forall\ i\ {a_i}^p = a_i(*)$.\\
  Заметим в многочлене $\alpha$ на $\alpha^p$, получим\\
  ${(\alpha^p)}^n + a_{n-1}{(\alpha^p)}^{n-1} + ... + a_0 \neq$ исп-ся рав(*) и

   ${(\alpha^p)}^n + a_{n-1}{(\alpha^p)}^{n-1} + ... + a_0 =
  {(\alpha^n)}^p + {a_{n-1}}^p{(\alpha^{n-1})}^p + ... + {a_0}^p$,\\
  так как характеристика = p, то есть $p\alpha = 0$, то по фомруле Бинома
  запишем так.\\
  ${(\alpha^n)}^p + {a_{n-1}}^p{(\alpha^{n-1})}^p + ... + {a_0}^p =
  {(\alpha^n + a_{n-1}\alpha^{n-1} + a_0)}^p = {f(\alpha)}^p = 0^p = 0$.\\
\end{proof}

\begin{title}
  Примитивный элемент
\end{title}

\begin{define}
  Пораждающий элемент мультипликативной группы поля называется примитивным.
\end{define}

\begin{theorem}
  В любом конечном поле $GF(p^m)$ существует примитивный элемент, то есть
мультипликативная группа этого поля циклическая.\\
\end{theorem}

\begin{proof}
  $h = p^n - 1 = {p_1}^{\alpha_1} - {p_s}^{\alpha_n}$ - разложение на простые
  множители.\\
  Пусть ${a_i}^{\frac{h}{p_i}} \neq 1$ (не корень)\\
  Введем $b_i = {a_i}^{\frac{n}{{p_i}^{\alpha_i}}}$ по теореме Лагранжа каждый
  элемент в степени
  равный порядку группы равен 1.\\
  ${b_i}^{{p_i}^{\alpha_i}} = 1\quad {a_i}^{\frac{h}{p_i}} \neq 1 \Rightarrow$
  порядок элементов
  $b_i$ равен  ${a}^{{p_i}^{\alpha_i}}$\\
  Элемент $b = b_1, b_2, ...,b_n$ - примитивный элемент, так как порядки всех
  $b$ взаимно просты, то их НОК равно их произведению.
\end{proof}

\begin{title}
  Алгоритм нахождения примитивного элемента
\end{title}

1. Разложить на простые множители.\\
2. Найти все возможные примитивные элементы.\\
3. Предполагаемый примитивный элемент $s$ - раз возводим в степени
$g^{\frac{h}{p_i}} \neq 1
\quad (i = 1, ..., s)$.\\
$\phi(h-1)$ - формула Эйлера.\\
\begin{title}
  Линейные регистры сдвига с обратной связью
\end{title}

Последовательность называется {\bf \emph{рекурентной}}, если ее текущий член
зависит от предыдущих членов.\\
Если член последовательности зависит от одного предыдущего элемента, то данную
последовательность
называют цепью Маркова.\\
\emph{Примеры:}\\
$a_{n+1} = a_n + \alpha$ - Арифметическая прогрессия\\
$a_{n+1} = a_n * q $ - Геометрическая прогрессия\\
$a_{n+2} = a_{n+1} + a_n$ - Ряд Фибоначи\\
$a_{n+k} = b_{k-1} a_{n+k+1} + b_{k-2}a_{n+k-2} + \cdots + b_0 a_n + b$ -
рекурентная последовательность k-го порядка\\
Если b = 0, то последовательность однородная. Если последовательность
рассматривается над конечным полем, то важной
характеристикой является {\bf \emph {период}}.\\

Поиск явного вида n-го члена.\\
Первые k - значений, которые ни от чего не зависят могут быть заданы
произвольно:\\
$\vec{a_0} = (a_1, a_2, \cdots, a_k)$ - вектор инициализации.\\
С каждой рекурентной последовательностью связывается ее характеристический
многочлен.\\
\[x^k = b_{k-1} x^{k-1} + b_{k-2} x^{k-2}+ \cdots + b_0\]
Успех вычислений зависит от корней многочлена. Идеальный случай, когда корни
попарно различны.\\
Пусть $\alpha_1, \alpha_2, \cdots, \alpha_k$ - попарно различные корни
характеристического многочлена рекурентной
последовательности. Тогда n-й член этой последовательности может быть задан
в явном виде:
\[a_n = \beta_1\alpha_1^n + \beta_2\alpha_2^n + \cdots + \beta_n\alpha_n^n\]
Где $\beta$ - решение системы

\begin{equation*}
 \begin{cases}
    \beta_1\alpha_1^0 + \cdots + \beta_k\alpha_k^0 = a_0\\
    \cdots \cdots \cdots\\
    \beta_1\alpha_1^{k-1} + \cdots + \beta_k\alpha_k^{k-1} = a_{k-1}
 \end{cases}
\end{equation*}

Определитель этой системы - это определитель Вандермонда $(W)$ и он не равен 0,
когда все корни попарно различны.\\
Каждая рекурентная последовательность имеет матрицу:

\begin{displaymath}
A = \left(\begin{array}{lccr}
0 & 0 & \cdots & b_0\\
1 & 0 & \cdots & b_1\\
\vdots & \vdots & \ddots & \vdots\\
0 & 0 & \cdots & b_{k-1}
\end{array}\right)
\end{displaymath}

Пусть $P$ - поле, $S_0, S_1, \cdots Sn$ - последовательность.\\
Последовательность называется рекурентной степени $K$: $S_{n+k} = a_{k-1}
S_{n+k-1} + \cdots + a_0 S_n$.\\
Так как первые $K$ - элементов не связаны никакими ограничениями, то $\bar{S_0}
= (S_0, S_1, \cdots S_{k-1})$ - вектор
инициализации может быть задан произвольно. Различных векторов инициализации
может быть $q^k$ штук.\\

\begin{title}
  Длина периода. Матричная запись
\end{title}

Характеристикой последовательности S является ее период.
\[\bar{S_0} = \bar{0} \Rightarrow q^k - 1\]
\begin{displaymath}
A_{(k \times k)} = \left(\begin{array}{lcccr}
0 & 0 & \cdots & 0 & a_0\\
1 & 0 & \cdots & 0 & a_1\\
0 & 1 & \cdots & 0 & a_2\\
\vdots & \vdots & \ddots & \vdots & \vdots\\
0 & 0 & \cdots & 1 & a_{k-1}
\end{array}\right)
\end{displaymath}
\[\bar{S_1} = (S_1, S_2, \cdots S_k)^{q^{k}-1}\]
\[\bar{S_1} = \bar{S_0} \cdot A\]
n-ый вектор состояния $\bar{S_n} = S_0 \cdot A^n$\\
Если n-ая степень матрицы А станет единичной матрицей $\Rightarrow \bar{S_n} =
\bar{S_0}$ и последний начнет
повторяться. В зависимости от коэффициента $a_j$ - единичная матрица может не
получиться. Сначала предпериод, а
потом период.\\
Как только среди векторов состояния появится тот, который был равен,
последовательность начнет повторяться,
поэтому сумма периода не может быть больше чем $q^k - 1$.\\
Лучший вектор инициализации, который гарантированно дает наибольший период -
$x^k = a_{k - 1} x^{k - 1} + \cdots +
a_0$ - характеристический многочлен.\\

\begin{title}
  Решение линейного уравнения в кольце вычетов
\end{title}

Пусть дано: $ax = b(n) \quad d = NOD (a, n)$\\
Теорема:\\
Если $d$ не делит $b$, то уравнение не имеет решение. Если $d$ делит $b$, то
решений будет $d$ штук.\\
Доказательство:\\
Если $d$ не делит $b$, то вычитая из $ax$ у нас всегда будет получаться число,
делящееся на $d$, значит $b$ никогда
не получится.\\
Если $d$ делит $b$, то $a = da_0 \quad b = db_0 \quad n =dn \Rightarrow a_0x = b_0(n_0)$\\
Так как НОД$(a_0, n_0) = 1$, то по следствию из алгоритма Евклида, то у $a_0$
есть обратный по умножению по модулю
$(n_0) \Rightarrow x_0 = a_0^{-1}b_0(n_0)$\\
Непосредственно проверяется, что $x = x_0 + i n_0 \quad 0 < i < d \Rightarrow ax = b(n)$

\begin{title}
  Решение системы линейных уравнений по разным модулям
\end{title}

$x^2 + 1$ - неприводимый многочлен\\
$x^2 + x + 2$ - неприводимый многочлен\\
Допустим наш характеристичесий многочлен над $GF(3)$:
\[f(x) = (x^2 + 1)(x^2 + x + 2)\]
Если бы это было над полем $R$ или  $Q$ характеристика 0, то нужно было бы
добавить корни из этого
характеристического уравнения. По теореме о корнях неприводимых многочленов поля
Галуа добавив корень 1
многочлена, мы автоматически разложим на многочлены все остальные непрерывные
многочлены данной степени:
\[\alpha^2 = -1 \quad \alpha^2 = -\alpha - 2 = 2 \alpha + 1\]
Из соображения удобств вычислений к полю $GF(3)$ мы добавим корень $\alpha$ 1-го многочлена, который будет
удовлетворять соотношению $\alpha^2 = 2$.\\
По теореме о корнях неприводимых многочленов, корням $x^2 + 1$ будет $\alpha,
\alpha^3$:
\[GF(3^2) = \{0, 1, 2, \alpha, 2\alpha, \alpha + 1, 2\alpha + 1, 2\alpha + 2,
\alpha + 2\}\]
Следующий шаг:\\
Теперь среди 9 элементов поле $GF(3^2)$ нужно найти корни 2-го многочлена
(он неприводим).\\
Первый способ:\\
Перебрать все 6 элементов, непринадлежащих полю $GF(3)$. Подставив в многочлен
и проверить кто корень?
2 из 6 - корни.\\
Второй способ:
Найти корни по формуле решения квадратного уравнения.\\

Возвращаемся к нашему полю $GF(2^4)$. Так как $\alpha$ - не является
примитивным, то нам придется найти
примитивный элемент.\\
$|GF(2^4)| = 16 \quad 15 = 5 * 3 \Rightarrow$ примитивный элемент должен
удовлетворрять двум неравенствам:
\[g^3 \neq 1 \quad g^5 \neq 1\]
Функция Эйлера $\varphi(15) = \varphi(3)\varphi(5) = 2 * 4 = 8 \Rightarrow$
примитивных элементов 8 штук.

\begin{title}
  Алгоритм AES - замена столбцов
\end{title}

AES (Advanced Encryption Standard) - основная кодировка c 2002 - 12 раундов, в
каждом раунде из исходного
ключа по хитрому алгоритму выражаются ключ. Каждый раунд состоит из простых
преобразований:\\
1) Сложение с раундом ключом.\\
2) Преобразование строк\\
3) Афинное преобразование на байт смотрят как на 8-мерный вектор над $GF(2)$,
замена
$S \rightarrow As + b$ (A - циркулянт)