  Простое число Вереена $2^p -1 ~~ p \in N$. Необходимое условие простого числа
простота показателя.

  $2^{2^n} + 1$ простое чилос Ферма.

\begin{title}
  Графы
\end{title}
  \kv{Графом} называется фигура состоящая из точек называемые \kv{верщинами}
и отрезаками соеденяющие эти точки называемые \kv{ребрами}.

  Граф называеют \kv{полным графом} если в нем две вершины соеденены ребрами.

  Граф называют \kv{плонарным графом или плоским графом} если он может быть
изображен на плоскости без пересечения ребер.

  Граф называется \kv{двухполосным графом} если множество его веершин
разбивается на две подсистемы, а ребра соеденяются вершины из разных
подмножеств.

  Граф будет неплонарным когда в нем есть граф изоморфный этому.

  Если ребра от $A$ к $B$ отличаются от ребер $B$ к $A$ то граф называется
\kv{орентированным графом}.

  Граф называется \kv{граф с петлями} если есть ребра $A$ к $A$.

  Граф называется \kv{мульти графом} если две вершины могут быть соеденены
несколькими ребрами.

  Мулти граф называют \kv{нагруженым графом} если движение по ребрам имеет
какойто "вес".

\begin{theorem}
  Разложение по $i$ строке: $|A| = \sum_{j=1}^n (-1)^{i+j} a_{ij} |A_{ij}|$
\end{theorem}

\begin{proof}
  $i$ строку меняем с $i-1, i-2 \ldots 1$ при каждой замене определитлеь
  меняет знак $(-1)^{j-1} (-1) = (-1)^i$
\end{proof}

\begin{title}
  Определитель Вандермонда:
\end{title}

\begin{displaymath}
  \left(
  \begin{array}{ccccl}
    x_1   & x_2   & \cdots & x_n \\
    x_1^2 & x_2^2 & \cdots & x_n^n\\
    \cdots & \cdots & \cdots & \cdots\\
    x_1^3 & x_2^3 & \cdots & x_n^{n-1}
  \end{array}
  \right)
  = \prod_{i>j}(x_i - x_j)
\end{displaymath}

  Определитель Вондермонда не равен нулю когда все $x$ попарно различны.
\begin{proof}
  По индукции\\
  1) базис индукции $n = 2 ~~ x_1 - x_2$\\
  2) предположение индекции при $n-1$ утверждение доказано\\
  3) шаг индукции проверим утверждение для n\\
  $(x_2 - x_1)(x_3 - x_1) \ldots (x_n - x_1)$
\end{proof}

  Циркулянтом называется
\begin{displaymath}
  \left( \begin{array}{cccc}
    a_1 & a_2 & \ldots & a_{n} \\
    a_n & a_1 & \ldots & a_{n-1} \\
    \cdots & \cdots & \cdots & \cdots\\
    a_n & a_2 & \ldots & a_1
  \end{array}\right)
  = f(\varepsilon_1) \ldots f(\varepsilon_n)\\
\end{displaymath}

  $f(x) = a_1 +a_2x + \ldots a_n x^{n-1}$

\begin{title}
  Нахождение обратной матрицы по методу Крамера.
\end{title}
  Алгебраическое дополнение развернутое поределение:
$$
|A| = \sum_{}^n (-1)^{n} a_1, a_2, \ldots a_i
$$
Суммы $n!$ слогаемых каждый из которых знак зависит от чентных перестановок.

  Расмотрим этот определитель как $n$ слогаемых которые содержаться в качестве
сомножетеля $(n-1)!$ вынесем $a$ и $j$ за скобки, все что в скобках называется
алгебраическим дополнением.

  Алгебраическое дополнение это многочлен некоторого мида будетлфывоалдоыфв.

  Разложение поределителя по первой строке.
$$
|A| = \sum_{i=1}^n a_{ij} |A|_{ij}
$$

  Связь алгебраического дополнения с минором
$$
|A|_{ij} = (-1)^{i+j} |A_{ij}|
$$

\begin{proof}
  Напишим разложение определителяпо стркое минором и при элменте $a$ и $i$ у нас
  будет часть равенства при разложении по $i$ строке по алгебраическим
  дополнением при это $a$ и $i$ будет левая равенство так как это одно и тоже.
\end{proof}

 \kv{Если матрица обратима, тогда ее определитель обратим}

\begin{displaymath}
A \cdot A^{-1} = \frac{1}{|A|}\left(\begin{array}{lccr}
|A|_{11} & |A|_{12} & \cdots & |A|_{1n}\\
\vdots & \vdots & \ddots & \vdots \\
|A|_{n1} & |A|_{n2} & \cdots & |A|_{nn}
\end{array}\right)^t
\end{displaymath}