\begin{title}
  Детерминант как отображение
\end{title}

\[det: M_n(K) \to K\]
- Отображение над кольцом $K$
\[det(A\cdot B) = det~A \cdot det~B\]
Данное отображение является гомоморфизмом мультипликативных групп.\\

\kv{Отображение кососиметрично}\\

{\bf Абстрактное описание детерминанта.}\\
Если матрицу рассматривать как совокупность строк, то детерминант - это
полилинейное кососиметричное нормарованное отображение кольца $K$, являющееся
гомоморфизмом мультипликативных групп матрицы и кольцы $K$\\

\begin{title}
  Линейные отображения
\end{title}

$V$ и $W$ - векторы пространства надполем $P$. $\varphi: V \to W ~~~ \forall
\alpha, \beta \forall u, v \in V$\\
$\varphi(\alpha u + \beta v) = \alpha\varphi(u) + \beta\varphi(v)$\\
Пусть $e_1 e_2 ... e_n$ - базис $V$ ~~~ $f_1 f_2 ... f_n$ - базис $W$\\
$V = \alpha_1 \varphi(e_1) + ... \alpha_n \varphi(e_n)$\\
Таким образом линейное отображение однозначно определяется своим действием на
базисных векторах. Так как просторанство $W$ имеет размерность $W$, то каждый
образ $\varphi(e_1)$ можно рассмотреть как $[\varphi] = (\varphi(e_1) ...
\varphi(e_n)) \Rightarrow [\varphi(v)] = [\varphi][v]$. Так как $[\varphi]
зависит от базисов, то ее можно записать в виде матрицы$\\
Ненулевой вектор $x$ называется собственным отображением вектора $x$\\
Если существует базис из собственных векторов, отобразим $\varphi ~и~ \lambda_1
... \lambda_n$ соответствующим собственным значениям.
\begin{displaymath}
[\varphi] = \left(\begin{array}{lccr}
\lambda_1 & 0 & \cdots & 0\\
0 & \lambda_2 & \cdots & 0\\
0 & 0 & \ddots & 0\\
0 & 0 & \cdots & \lambda_n
\end{array}\right)
\end{displaymath}

Свойства линейного отображения:\\
{\bf I} При линейном отображении нулевой вектор всегда переходит в нулевой вектор
$\varphi(0) = \varphi(0+0) = \varphi(0) + \varphi(0)$ Так как в $W$ есть обратна
по сложению, то $\varphi(0) = 0$\\
{\bf II} Обратный переходит в обратный.\\
{\bf III} Ядро линейного отображения $\varphi$ обозначается $Ker \varphi =
\{v \in V | \varphi(v) = 0\} ~~~ Ker \varphi \in V$\\

$Im$ - образ линейного отображения\\
$Im \varphi = \{w \in W | \exists v \in V \varphi(v) = w\}$\\

Когда $Ker$ состоит их 1-го нулевого вектора - разные вектора переходят в разные
 и если при этом $Ker = {0}$, то $Im(\varphi) = w$\\

\begin{title}
  {Линейные отображения}
\end{title}
Пусть $V$ - векторное пространство над полем $P$.\\
$e_1 e_2 ... e_n$ - базис.\\
$\varphi : V \to V$ - гомоморфизм векторыных пространств.\\
$V = \alpha_1 e_1 + \cdots + \alpha_n e_n$ - "живой вектор"\\
$[V] = (\alpha_1 e_1 \cdots \alpha_n e_n)$ - "Паспорт"\\
$[V]$' - "вектор тот же самый, а набор координат другой"\\
$[\varphi(v)] = [\varphi] \cdot [V]^{-vect-column}$ - Обозначим формулой №1\\

На смену координат можно смотреть как на тождественные отображения. Первичным
является старый базис, вторичным - новый. Из чего следует - новые базисные
вектора мы выражаем через старые. $V' \to V$\\
Возникающая при этом матрица называется матницей перехода, где ее столбцы - это
образы векторов в старом базисе.\\
\begin{displaymath}
T = \left(\begin{array}{lccr}
e'_1 & e'_2 & \cdots & e'_n\\
\cdots & \cdots & \cdots & \cdots\\
\cdots & \cdots & \cdots & \cdots\\
\end{array}\right)
\end{displaymath}
Формула №1 в этом случае будет выглядеть так:\\
$[V] = T[V']$ - обозначим формулой №2.\\
Мы выражали новые координаты через старые, а получилась формула, в которой
старые выражены через новые. Если на самом деле нам известны только старые
координаты, то\\
$[V'] = T^{-1}[V]$ - формула нахождения новых координат через старые.\\

\[\varphi: V \to V\]
\[e_1 e_2 ... e_n\]
\[[\varphi(V)] = [\varphi][V] ~~~ (1)\]
\[e'_1 e'_2 ... e'_n\]
\[[\varphi(V)]' = [\varphi]'[V]' ~~~ (1)'\]
T - матрица перехода от старого базиса к новому
\[[V] = T[V'] ~~~ (2)\]
\[[\varphi(v)] = T[\varphi (V)]' ~~~ (2)'\]
Используя эти 4 формулы нам нужно найти связь между матрицами через $[\varphi]$
и $[\varphi]'$\\
Подставим формулы (2) и (2)' в формулу (1)\\
\[(1) \Rightarrow T[\varphi(V)]' = [\varphi] \cdot T[V]' \Rightarrow
[\varphi(V)]' = \left(T^{-1} [\varphi] T \right)[V]' \Rightarrow [\varphi]' =
T^{-1}[\varphi]T\]

Функции совпадают тогда и только тогда, когда у них и у левой и у правойй части
областью определения является все пространство $V$. Так как $V$ пробегает все
это пространство и на каждом $V$ имеет место равенство, значет левая и правая
части как функции совпадают.\\

\emph{формула замены матрицы линейного отображения при замене базиса}
\[[\varphi]' = T^{-1} [\varphi]T\]

\begin{title}
{Первиченые линейные отображения}
\end{title}
Матрица - это удобный математический аппарат для их описания.\\

\begin{title}
{Операции над линейными отображениями}
\end{title}
Сумма двух линейных отображений - это обычная поточечная сумма двух функций.\\
\[\varphi: V \to V\]
\[\psi: V \to V\]
\[(\varphi + \psi)V = \varphi(V) + \psi(V)\]
\[[\varphi + \psi][V] = [\varphi][V] + [\psi][V] = ([\varphi] + [\psi])[V]\]
\[\varphi + \psi] = [\varphi] + [\psi]\]

Суперпозиция линейных отображений\\
\[(\varphi \circ \psi)(V) = \psi(\varphi(V))\]
\[[\varphi \circ \psi][V] = [\psi][\varphi(V)] = [\psi] \cdot [\varphi] \cdot [V]\]

\begin{title}
{Собственные и корневые вектора отображения}
\end{title}
В каком базисе матрица линейного отображения будет диагональной?

\begin{displaymath}
[\varphi] = \left(\begin{array}{lccr}
\lambda_1 & 0 & \cdots & 0\\
0 & \lambda_2 & \cdots & 0\\
0 & 0 & \ddots & 0\\
0 & 0 & \cdots & \lambda_n
\end{array}\right)
\end{displaymath}

\[e_1 \cdots e_n\]
\[\varphi(e_1) = \lambda_1 e_1\]
\[\varphi(e_2) = \lambda_2 e_2\]
\[\cdots \cdots \cdots \cdots\]
\[\varphi(e_n) = \lambda_n e_n\]

\begin{defin}
{Определение}
\end{defin}

Не нулевой вектор $x$ - называется собственным вектором линейного отображения
собственного значения $\lambda$, если $\varphi(x) = \lambda x$.\\

Замечание:\\
Собственный вектор обязательно $\ne 0$, а собственное значение может быть $= 0$\\

{\bf Свойство 1}\\
Собственный вектор имеет единственное собственное значение.\\
\begin{proof}
\[\lambda \ne \mu\]
\[\varphi(x) = \lambda x\]
\[\varphi(x) = \mu x = (\lambda = \mu)x = \bar{0}\]
\[\lambda \cdot \mu = 0 \Rightarrow \lambda = \mu\]
\end{proof}

{\bf Свойство 2}\\
Множество векторов, отвечающих данному собственному значению образуют
подпространство.\\
\begin{proof}
Если $\bar{x}$ и $\bar{y}$ - два собственных вектора, отвечающих собственному
значению $\lambda$, то \\
$\forall \alpha, \beta \in P ~~~ \alpha x + \beta y \Rightarrow
\varphi(\alpha x + \beta y) = \alpha \varphi(x) + \beta \varphi(y) =
\alpha \lambda x + \beta \lambda y = \lambda(\alpha x + \beta y)$\\
\end{proof}

{\bf Свойство 3}\\
Собственные вектора, отвечающие разным собственным значениям - линейно
независимы.\\
\begin{proof}
База индукции $n = 1$, так как собственный вектор ненулевой, то он и линейно
независимый.\\
Пусть $\lambda_1 \cdots \lambda_{n-1}, \lambda_n$ - собственные значения\\
$x \cdots x_{n-1}, x_n$\\
Предел индукции - пусть $n - $ векторов линейно независимы.\\
Шаг индукции: Пусть напротив существует\\
$\exists \alpha_1, \cdots \alpha_n: \alpha_1 x_1 + \cdots + \alpha_{n-1} x_{n-1}
+ \alpha_n x_n = \bar{0}$ При этом не все $\alpha_n = 0$\\
Применим к этому вектору отображение $\varphi$\\
\[\bar{0} = \varphi(\bar{0}) = \alpha_1 \lambda_1 x_1 + \cdots + \alpha_{n-1}
\lambda_{n-1} x_{n-1} + \alpha_n \lambda_n x_n = \bar{0} ~~~ (2)\]
Умножим уравнение (1) на $\alpha_n$ и вычтем из второго уравнения\\
\[\alpha_1 (\lambda_1 - \lambda_n)x_1 + \alpha_{n-1} (\lambda_{n-1} -
\lambda_{n-1})x_{n-1} = \bar{0}\]
\[\alpha_1 = \alpha_2 = \cdots \alpha_{n-1} = 0\]
Противоречие\\
\end{proof}

{\bf Обобщение собственных векторов - это корневые вектора}\\
\[\varphi(x) = \lambda x \Rightarrow (\varphi - \lambda \varepsilon)(V) = \bar{0}\]
\[\varepsilon (V) = V\]
\[[\varepsilon] = E\]

\begin{defin}
Вектор называется корневым высоты $h$, если
\[(\varphi - \lambda \varepsilon)^h (V) = \bar{0}\]
\[(\varphi - \lambda \varepsilon)^{h - 1} (V) \ne \bar{0}\]
\end{defin}

В этой терминологии собственный вектор - это корневой вектор высотой $h - 1$\\

\emph{Как найти собственное значение $\lambda$?}\\
\[\varphi (x) = \lambda x\]
\[(\varphi - \lambda \varepsilon)(x) = \bar{0}\]
Перейдя к координатам
\[[\varphi - \lambda \varepsilon][x] = \bar{0}\]

$[\varphi] = A$ - матрица линейного отображения\\
\[(A - \lambda E) x = \bar{0}\]
Чтобы система имела ненулевое решение, а собственный ненулевой вектор,
необходимо и достаточно, чтобы определитель матрицы этой системы был нулевым.\\

\emph{Чтобы найти собственное значение $x$}\\
\[det|A - \lambda E| - ?\]

$F(\lambda) = |A \lambda E|$ - характеристический многочлен линейного оторажения
$\varphi$.\\

\begin{theorem}
Многочлен на самом деле характеристический, так как он не зависит от того, в
каком базисе записана матрица $A$ линейного отображения\\
\end{theorem}
Пусть $T$ - матрица перехода. Тогда в новом базисе матрица отображения:\\
\[|T^{-1} AT - \lambda E| = |T^{-1} AT - \lambda T^{-1} AT| = |T^{-1}
(A - \lambda E)T|\]
Так как определитель произведения матриц равен произведению определителей
\[|T^{-1}| \cdot |A - \lambda E| \cdot |T| = |A \lambda E| = f(\lambda)\]
Определитель не меняется при заменен базиса.\\