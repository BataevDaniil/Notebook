\begin{title}[\Large]
    Основные свойства определенных интегралов.
\end{title}

1) $f(x), g(x)$ интегрируемы на $[a, b] ~~ \alpha , \beta \in R ~~ \alpha \cdot
\beta \not= 0 ~~ \alpha f(x) + \beta g(x)$ интегрируем на $[a, b]$ тогда верно
\[
    \int (\alpha f(x) + \beta g(x))dx = \alpha \int_a^b f(x)dx
    + \beta \int_a^b f(x)dx
\]

\begin{proof}
    Из определения определенного интеграла следует
    \[
        \forall\epsilon>0 ~~ \exists\delta_{\epsilon}>0 ~~ \forall R[a,b] ~~~
        \lambda(R) < \delta_{\epsilon} ~~ \forall\upsilon(R) ~~~
        \left| \sum_{k=1}^{n} f(c_k)\Delta x_k -
        \int_a^b f(x)dx \right| < \frac{\epsilon}{2|\alpha|}
    \]
    \[
        \left| \sum_{k=1}^{n} g(c_k)\Delta x_k -
        \int_a^b g(x)dx \right| < \frac{\epsilon}{2|\beta|}
    \]
    \[
        \left| \sum_{k=1}^{n} (\alpha f(x) + \beta g(x))\Delta x_k \right| -
        \left| \alpha \int_a^b f(x)dx + \beta \int_a^b f(x)dx \right| \le
    \]
    \[
      \le \left| \sum_{k=1}^{n} (\alpha f(c_k) \Delta x_k -
      \alpha \int_a^b f(x)dx \right| + \left| \sum_{k=1}^{n} \beta g(c_k)
      \Delta x_k - \beta \int_a^b g(x)dx \right| \le
    \]
    \[
      \le |\alpha| \left| \sum_{k=1}^{n} (f(c_k) \Delta x_k -
      \int_a^b f(x)dx \right| + |\beta| \left| \sum_{k=1}^{n} g(c_k)
      \Delta x_k - \int_a^b g(x)dx \right| \le
    \]
    \[
        \alpha \frac{\epsilon}{2|\alpha|} + |\beta| \frac{\epsilon}{2|\beta|}
        = \epsilon
    \]
    \[
        \int (\alpha f(x) + \beta g(x))dx
    \]
\end{proof}

2) $f(x), g(x)$ интегрируема на $[a, b]$ то $f(x) \cdot g(x)$ также
интегрируема.\\
3) $f(x)$ интегрируема на $[a, b]$ то $[c, d] \subset [a, b]$ интегрируема.\\
4) $f(x)$ интегрируема на $[a, b]$ $a < c < b$
    \[
        \int_a^b f(x)dx = \int_a^c f(x)dx + \int_c^b f(x)dx
    \]

Договоренность математиков.
\[
    \int_a^a f(x)dx = 0
\]
\[
    \int_a^b f(x)dx = - \int_a^c f(x)dx
\]

5) $f(x)$ интегрируема на $[a, b]$ $c_1, c_2, c_3 \in [a, b]$
\begin{proog}
    \[
    \int_{c_1}^{c_3} f(x)dx = \int_{c_1}^{c_2} f(x)dx + \int_{c_2}{c_3} f(x)dx
    \]
    \[
        \int_{c_3}^{c_2} f(x)dx = \int_{c_3}^{c_1} f(x)dx + \int_{c_1}{c_2} f(x)dx
    \]
    \[
        \int_{c_1}^{c_3} f(x)dx = \int_{c_1}^{c_2} f(x)dx + \int_{c_2}{c_3} f(x)dx
    \]
\end{proog}

\begin{title}[\Large]
    Оценки определенного интеграла.
\end{title}