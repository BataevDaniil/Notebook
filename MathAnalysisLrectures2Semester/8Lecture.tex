\begin{title}[\Large]
    Основные свойства определенных интегралов.
\end{title}

1) $f(x), g(x)$ интегрируемы на $[a, b] ~~ \alpha , \beta \in R ~~ \alpha \cdot
\beta \not= 0 ~~ \alpha f(x) + \beta g(x)$ интегрируем на $[a, b]$ тогда верно
\[
    \int (\alpha f(x) + \beta g(x))dx = \alpha \int_a^b f(x)dx
    + \beta \int_a^b f(x)dx
\]

\begin{proof}
    Из определения определенного интеграла следует
    \[
        \forall\epsilon>0 ~~ \exists\delta_{\epsilon}>0 ~~ \forall R[a,b] ~~~
        \lambda(R) < \delta_{\epsilon} ~~ \forall\upsilon(R) ~~~
        \left| \sum_{k=1}^{n} f(c_k)\Delta x_k -
        \int_a^b f(x)dx \right| < \frac{\epsilon}{2|\alpha|}
    \]
    \[
        \left| \sum_{k=1}^{n} g(c_k)\Delta x_k -
        \int_a^b g(x)dx \right| < \frac{\epsilon}{2|\beta|}
    \]
    \[
        \left| \sum_{k=1}^{n} (\alpha f(x) + \beta g(x))\Delta x_k \right| -
        \left| \alpha \int_a^b f(x)dx + \beta \int_a^b f(x)dx \right| \le
    \]
    \[
      \le \left| \sum_{k=1}^{n} (\alpha f(c_k) \Delta x_k -
      \alpha \int_a^b f(x)dx \right| + \left| \sum_{k=1}^{n} \beta g(c_k)
      \Delta x_k - \beta \int_a^b g(x)dx \right| \le
    \]
    \[
      \le |\alpha| \left| \sum_{k=1}^{n} (f(c_k) \Delta x_k -
      \int_a^b f(x)dx \right| + |\beta| \left| \sum_{k=1}^{n} g(c_k)
      \Delta x_k - \int_a^b g(x)dx \right| \le
    \]
    \[
        \alpha \frac{\epsilon}{2|\alpha|} + |\beta| \frac{\epsilon}{2|\beta|}
        = \epsilon
    \]
    \[
        \int (\alpha f(x) + \beta g(x))dx
    \]
\end{proof}

2) $f(x), g(x)$ интегрируема на $[a, b]$ то $f(x) \cdot g(x)$ также
интегрируема.\\
3) $f(x)$ интегрируема на $[a, b]$ то $[c, d] \subset [a, b]$ интегрируема.\\
4) $f(x)$ интегрируема на $[a, b]$ $a < c < b$
    \[
        \int_a^b f(x)dx = \int_a^c f(x)dx + \int_c^b f(x)dx
    \]

Договоренность математиков.
\[
    \int_a^a f(x)dx = 0
\]
\[
    \int_a^b f(x)dx = - \int_a^c f(x)dx
\]

5) $f(x)$ интегрируема на $[a, b]$ $c_1, c_2, c_3 \in [a, b]$
\begin{proof}
    \[
    \int_{c_1}^{c_3} f(x)dx = \int_{c_1}^{c_2} f(x)dx + \int_{c_2}{c_3} f(x)dx
    \]
    \[
        \int_{c_3}^{c_2} f(x)dx = \int_{c_3}^{c_1}f(x)dx + \int_{c_1}{c_2}f(x)dx
    \]
    \[
        \int_{c_1}^{c_3} f(x)dx = \int_{c_1}^{c_2}f(x)dx + \int_{c_2}{c_3}f(x)dx
    \]
\end{proof}

\begin{title}[\Large]
    Оценки определенного интеграла.
\end{title}
1) $f(x)$ интегрируема на $[a, b]$ $f(x) \ge 0$
\[\int_a^b f(x)dx \ge 0\]
\begin{proof}
    \[\sum_{k=1}^{n} f(c_k)\Delta x_k \ge 0\]
    $f(x), g(x)$ интегрируемы на $[a, b]$ $\forall x \in [a, b] ~~ f(x)\le g(x)$
    \[\int_a^b f(x)dx \le \int_a^b g(x)dx\]
    \[g(x) - f(x) \ge 0 ~~ \int_a^b (g(x) - f(x))dx \ge 0\]
    \[\int_a^b g(x)dx - \int_a^b f(x)dx \ge 0\]
\end{proof}
2) $f(x) \ge 0$ интегрируема на $[a, b]$ $\exists c \in [a, b]$ непрерывна в
$f(c) > 0$ \[\int_a^b f(x)dx > 0\]
\begin{proof}
    3) локальные свойство непрерывности в точке $\exists O_{\delta}c ~~
    \forall x \in O_{\delta}c ~~ f(x) > \frac{f(c)}{2}$
    \[
        \int_a^b f(x)dx = \int_a^{c-\delta} f(x)dx +
        \int_{c-\delta}^{c+\delta} f(x)dx + \int_{c+\delta}^b f(x)dx >
        0 + \int_{c-\delta}^{c+\delta} \frac{f(c)}{2}dx
        = \frac{f(c) 2\delta}{2} > 0
    \]
\end{proof}
3) $f(x)$ интегрируем на $[a, b]$ то $|f(x)|$ также интегрируема
(но не наоборот)
\[\left| \int_a^b f(x)dx \right| \le \int_a^b |f(x)|dx\]
\[||a| - |b|| < |a - b|\]
\[
    \left| \sum_{k=1}^n f(c_k)\Delta x_k \right| \le
    \sum_{k=1}^{n}|f(c_k)|\Delta x_k
\]
4) $f(x)$ интегрируема $[a, b]$ $\forall c_1, c_2 \in [a, b]$
\[
    \left| \int_{c_1}^{c_2} f(x)dx \right| \le
    \left| \int_{c_1}^{c_2} |f(x)|dx \right|
\]
5) $f(x), g(x)$ интегрируемы на $[a, b]$ $\forall x \in [a, b] ~~
m \le f(x) \le M$ $g(x) \ge 0$ или $g(x) \le 0$ $\exists m \le \varphi \le M$
\[\int_a^b f(x)g(x)dx \varphi \int_a^b g(x)dx\]
\begin{proof}
    $g(x) \le 0 ~~ m \le f(x) \le M ~~ mg(x) \ge f(x)g(x) \ge Mg(x)$
    \[\int_a^b mg(x)dx \ge \int_a^b f(x)g(x)dx \ge \int_a^b Mg(x)dx\]
    \[m\int_a^b g(x)dx \ge \int_a^b f(x)g(x)dx \ge M\int_a^b g(x)dx\]
    \[m \le \frac{\int_a^b f(x)g(x)dx}{\int_a^b g(x)dx} \le M\]
\end{proof}

Следствие.\\
\begin{theorem}[о среднем для определенного интеграла]
    $f(x)$ интегрируем на $[a, b]$ $g(x)$ интегрируема и не меняет напрвления на
    $[a, b]$ $\exists c \in [a, b]$
    \[\int_a^b f(x)g(x)dx  = f(c) \int_a^b g(x)dx\]
    $m = inf f(x) ~~ u \in [a, b] ~~ M = sup f(x) ~~ u \in [a, b]$
    $m \le f(x) \le M ~~ f([a, b]) = [m, M]$ на основании свойства 5
    $\forall M \in [m, M] ~~ \exists c \in [a, b] ~~ f(c) = M$
\end{theorem}