\begin{title}
  Математический анализ.\\
  (Второй семестр)
\end{title}

\begin{title}
  Иследование функции с помощью производной.
\end{title}

\begin{title}[\Large]
  Иследование монотонности функции с помощью производной.
\end{title}

\begin{theorem}
  $f(x)$ непрерывана на $<a,b>$ и имеет $f'(x)$ на $(a,b)$ значит
  \\
  Если $\forall x \in (a,b) ~~ f'(x) \ge 0 $ то $f(x) \nearrow ~~ (a,b)$\\
  Если $\forall x \in (a,b) ~~ f'(x) \le 0 $ то $f(x) \searrow ~~ (a,b)$
\end{theorem}

\begin{proof}
  $\forall x_1, x_2 \in <a,b>$ и при этом $x_1 < x_2 $\\
  По теореме \kv{o конечных преращений}.\\
  $f(x_2) - f(x_1) = f'(c)(x_2 - x_1)$ очевидно что $x_2 - x_1 > 0 $\\
  Если $f'(c) \ge 0$ то $f(x_2) \ge f(x_1) \Rightarrow f(x)$ на $(a,b)$
  \kv{возрастает}.\\
  Если $f'(c) \le 0$ то $f(x_2) \le f(x_1) \Rightarrow f(x)$ на $(a,b)$
  \kv{убывает}.
\end{proof}

\begin{theorem}
  $f(x)$ непрерывана на $<a,b>$ и имеет $f'(x)$ на $(a,b)$ значит\\
  Если $\forall x \in (a, b) ~~ f'(x) > 0$ то $f(x)$ \kv{строго}
  $\nearrow (a, b)$\\
  Если $\forall x \in (a, b) ~~ f'(x) < 0$ то $f(x)$ \kv{строго}
  $\searrow (a, b)$
\end{theorem}

\begin{proof}
  $\forall x_1, x_2 \in <a,b>$ и при этом $x_1 < x_2 $\\
  По \kv{теореме о конечных преращений.}\\
  $f(x_2) - f(x_1) = f'(c)(x_2 - x_1)$ очевидно что $x_2 - x_1 > 0$\\
  Если $f'(c) > 0$ то $f(x_1) > f(x_2) \Rightarrow f(x)$ на $(a,b)$
  \kv{строго возрастает}.\\
  Если $f'(c) < 0$ то $f(x_2) < f(x_1) \Rightarrow f(x)$ на $(a,b)$
  \kv{строго убывает}.
\end{proof}

\begin{title}[\Large]
  Исследование функции на экстремум с помощью производной.
\end{title}

\bd{Необходимые и достаточные условия экстремума.}\\
\begin{theorem}
  Если: \\
    1) $f(x)$ дифференцируема в проколотой окрестности точки $a$.\\
    2) Непрерывна в точке $a$.\\
    3) При переходе через точку $a$ $f(x)$ меняется с - на +. То есть\\
      $\exists \delta > 0 ~~ \forall x \in (a - \delta, a) ~~ f'(x) < 0$\\
      $\exists \delta > 0 ~~ \forall x \in (a, a + \delta) ~~ f'(x) > 0$\\
  Тогда $a$ - точка \kv{строгого минимума.}
\end{theorem}

\begin{proof}
  По формуле конечных приращений\\
  $\forall x \in (a - \delta, a) ~~ f(x) - f(a) = f'(c)(x - a)$ очевидно что
  $x - a < 0$ и \\ $f'(c) < 0$ $\Rightarrow f(x) > f(a)$\\
  $\forall x \in (a, a + \delta) ~~ f(x) - f(a) = f'(c)(x - a)$ очевидно что
  $x - a > 0$ и \\ $f'(c) > 0$ $\Rightarrow f(x) > f(a)$\\
  $\forall x \in (a - \delta, a+ \delta) ~~ f(x) > f(a)$\\
  Тогда $a$ - точка \kv{строгого минимума.}
\end{proof}

\begin{theorem}
  Если: \\
    1) $f(x)$ дифференцируема в проколотой окрестности точки $a$.\\
    2) Непрерывна в точке $a$.\\
    3) При переходе через точку $a$ $f(x)$ меняется с + на -. То есть\\
      $\exists \delta > 0 ~~ \forall x \in (a - \delta, a) ~~ f'(x) > 0$\\
      $\exists \delta > 0 ~~ \forall x \in (a, a + \delta) ~~ f'(x) < 0$\\
  Тогда $a$ - точка \kv{строгого максимума.}
\end{theorem}

\begin{proof}
  По формуле конечных приращений\\
  $\forall x \in (a - \delta, a) ~~ f(x) - f(a) = f'(c)(x - a)$ очевидно что
  $x - a < 0$ и \\ $f'(c) > 0$ $\Rightarrow f(x) < f(a)$\\
  $\forall x \in (a, a + \delta) ~~ f(x) - f(a) = f'(c)(x - a)$ очевидно что
  $x - a > 0$ и \\ $f'(c) < 0$ $\Rightarrow f(x) < f(a)$\\
  $\forall x \in (a - \delta, a+ \delta) ~~ f(x) < f(a)$\\
  Тогда $a$ - точка \kv{строгого максимума.}
\end{proof}

\begin{theorem}
  Пусть $f(x)$ в точке $a$\\
  $f^{(n)} (a) \neq 0 ~~ n > 1\\
  f^{(k)} (a) = 0 ~~ k = 1, 2, \ldots, n-1$\\
  Если $n$ - не четный, то в точке $a$ - нет экстремума.\\
  Если $n$ - четный, то $a$ - точка экстремума.\\
  При чем если $f^{(n)} (a) > 0$ то $a$ - точка строгого минимума.\\
  Если  $f^{(n)} (a) < 0$ то $a$ - точка строгого максимума.
\end{theorem}

\begin{proof}
  По формуле Тейлора в форме Пиано:\\
  $f(x) - f(a) = \frac{f^{(n)}(a)}{n!} \cdot (x - a)^n + \alpha (x)
  \cdot (x - a)^n = (x - a)^n \left(\frac{f^{(n)}(a)}{n!} + \alpha(x) \right)\\
  \alpha (x) \to 0 ~~ x \to a$ \\
  Выберем $\delta > 0$ так, чтобы $\forall x \in
  (a - \delta, a + \delta) > |\alpha(x)| < \frac{f^{(n)}(a)}{2n!}\\
  \sign \left(\frac{f^{(n)}(a)}{n!} + \alpha(x)\right) = \sign (f^{(n)}(a))$\\
  Пусть $r$ - не четный. Если $(x - a)^n$ - меняет знак, то и
  f(x) - f(a) - меняет знак и нет экстремумума\\
  Пусть $n$ - нечетное $(x-a)^n$ при переходе через точку $a$ меняет знак.
  Пусть $n$ - четный $f(x) - f(a) > 0$, то $a$ - точка строгого минимума.\\
  Пусть $n$ - четный $f(x) - f(a) < 0$, то $a$ - точка строгого максимума.
\end{proof}

\begin{title}[\Large]
  Алгоритм нахождения наибольшего и наименьшего значения функции.
\end{title}
Дана функция $f(x)$ на $[a,b]$. Необходимо\\
1) Найти все критические точки $x_1, x_2, \ldots, x_n \in [a, b]$\\
2) Вычислить значения критических точек и крайних точек отрезка:\\
$f(a), f(b), f(x_1), f(x_2), \ldots, f(x_k) ~~ k = 1, 2 \ldots n$\\
$max \{f(a), f(b), f(x_k), k = 1, 2, \ldots, n\} = c $ тогда \kv{наибольший}\\
$min \{f(a), f(b), f(x_k), k = 1, 2, \ldots, n\} = c $ тогда \kv{наименьший}\\

\begin{title}[\Large]
  Иследование выпоклости функций с помощью производной.
\end{title}

\begin{defin}[выпуклости вниз]
  Функция \bd{называется выпуклой вниз} если $f(x)$ непрерывна на $<a,b>$ и
  $\forall x_1, x_2 \in (a,b)$
  \[f(\frac{x_1 + x_2}{2}) \le \frac{f(x_1) + f(x_2)}{2}\]
  Функция \bd{называется строго выпуклой вниз} если $f(x)$ непрерывна
  на $<a,b>$ и
  $\forall x_1 \not= x_2 \in (a,b)$
  \[f(\frac{x_1 + x_2}{2}) < \frac{f(x_1) + f(x_2)}{2}\]
\end{defin}

\begin{defin}[выпуклости вверх]
  Функция \bd{называется выпуклой вверх} если $f(x)$ непрерывна на $<a,b>$ и
  $\forall x_1, x_2 \in (a,b)$
  \[f(\frac{x_1 + x_2}{2}) \ge \frac{f(x_1) + f(x_2)}{2}\]
  Функция \bd{называется строго выпуклой вверх} если $f(x)$ непрерывна
  на $<a,b>$ и
  $\forall x_1 \not= x_2 \in (a,b)$
  \[f(\frac{x_1 + x_2}{2}) > \frac{f(x_1) + f(x_2)}{2}\]
\end{defin}

\begin{theorem}
  Пусть $f(x)$ непрерывна на $<a,b>$ и имеет $f'(x)$ на $(a,b)$ то\\
  $\forall x \in (a,b) ~~ f''(x) \ge 0$ функция выпукла вниз на $<a,b>$\\
  $\forall x \in (a,b) ~~ f''(x) \le 0$ функция выпукла вверх на $<a,b>$.\\
  Cправидливой и для строгих выпуклостей.
\end{theorem}

\begin{proof}
  Для определенности $x_1 < x_2 \in (a,b)$ и $2h = x_2 - x_1$\\
  $
  x_0 = \frac{x_2 + x_1}{2}\\
  x_2 = x_0 + h\\
  x_1 = x_0 - h\\
  f(x_2) = f(x_0 + h) = f(x_0) + \frac{f'(x_0)}{1!}h +
    \frac{f''(c_2)}{2!}h^2 ~~ x_0 < c_2 < x_2\\
  f(x_1) = f(x_0 - h) = f(x_0) + \frac{f'(x_0)}{1!}h +
    \frac{f''(c_1)}{2!}h^2 ~~ x_0 > c_1 > x_1\\
  f(x_1) + f(x_2) = 2f(x_0) + 2f'(x_0)h + \frac{h^2}{2}(f''(c_2) + f''(c_1)) ~~
  x_1 < c_1 < x_0 < c_2 < x_2
  $
  $$
  \frac{f(x_1) + f(x_2)}{2} = f(x_0) + f'(x_0)h +
  \frac{h^2}{4}(f''(c_2) + f''(c_1))
  $$
\end{proof}

\begin{title}[\Large]
  Нахождение точек перегиба функции с помощью производной.
\end{title}

\begin{defin}[точки перегиба]
  Точка $a$ называется \kv{точкой перегиба} функции\\
  1) Существует $f'(a)$ конечная или бесконечная определенного знака.\\
  2) При переходе через точку $a$ меняется направление выпуклости.\\
  $(a, f(x))$ точка пергиба графика функции.
\end{defin}

\bd{Необходимые условия точки перегиба:}\\
Если $f(x)$ в некоторой окрестности точки $a$ имеет $f''(x)$ и непрерывна в
точке $a$ то $a$ может быть точкой перегиба если $f''(a) = 0$.\\

\begin{proof}
  Предположим что $f''(a)\not= 0$, то в силу непрерывности $f''(a)$, она имеет
  тот же знак что и знак в некоторой окрестности $a$
  $f''(x) \ge 0 ~ \forall x\in O(a)$ то $f(x)$ выпукла вверх. $a$ не может быть
  точкой перегиба так как направление выпуклости не меняется, противоречие.
\end{proof}

\bd{Достаточные условия:}\\
Пусть в точке $a$ функция имеет конечную или бесконечную 1-ую производную
определенного занка, в проколотой открестности точки $a$ $\exists f''(x)$.
Если $f''(x)$ меняет знак при переходе через точку $a$, то точка $a$ является
точкой перегиба этой функции.\\

\begin{theorem}
  Если $f''(a) = 0$ и $f'''(a) \not= 0$ то $a$ точка перегиба.
\end{theorem}

\begin{proof}
  $f'''(a) > 0$, тогда $f''(a)$ возрастает в точке $a$\\
  $f'''(a) < 0$, тогда $f''(a)$ убывает в точке $a$\\
Тогда $f'''(a) = 0$
\end{proof}

\begin{title}[\Large]
  План иследования функции.
\end{title}
1. D(y), четность, переодичность.\\
2. Найти точки пересечения графика функции с Ox и Oy, и промежутки возрастания
  и убывания.\\
3. Найти точки разрыва и их классификация. Вычисление односторонних приделов
  в точке разрыва и в ограниченных точках области определния.\\
4. Нахождение всех ассимтот графика.\\
5. Иследование функция на монотоность и экстремумы с помощью производной.\\
6. Иследование на выпуклости и перегибы.\\
7. Построение таблицы значений.\\
8. Построение графика.\\