\begin{title}[\Large]
  Несобственный интеграл от неограниченной функции
\end{title}

$f(x)$ определена на $[a, b]$ и интегрируема на $[a, c] \subset [a, b]$. А в
любой окрестной точки $b$ - $f(x)$ не ограничена если существует:
\[
  \lim_{c \to b - 0} \int^{c}_{a} f(x)dx
  \]
то его называют несобственным интегралом $f(x)$ на $[a, b]$ и сходящимся.\\
Если предела нет, то интеграл несобственный и расходящийся.

\begin{title}[\Large]
  Особые точки для функции, неограниченной в их окрестности
\end{title}

$f(x)$ определена на всей прямой, исключая $d_1 < d_2 < ... < d_m$:
\[-\infty < a_1 < d_1 < a_2 < d_2 < ... < a_m < d_m < a_{m + 1} < +\infty\]
По определению несобственных интегралов:
\[
  \int^{+\infty}_{-infty} f(x)dx = \int^{a_1}_{-\infty} f(x)dx +
  \int^{d_1}_{a_1} f(x)dx + \int^{a_2}_{d_1} f(x)dx + \int^{d_2}_{a_2} f(x)dx +
\]
\[
  ... + \int^{d_m}_{a_m} f(x)dx + \int^{a_{m + 1}}_{d_m} f(x)dx +
  \int^{+\infty}_{a_{m + 1} f(x)dx}
\]
Этот интеграл сходящийся, если все интегралы сходящиеся. А расходящийся, если
хотябы один интеграл расходящийся.

\begin{title}[\Large]
  Основные свойства и способы вычисления несобственных интегралов.
\end{title}

\begin{theorem}[1]
  Если:
  \[
    \int^{b}_{a} f(x)dx ~~~ \int^{b}_{a} g(x)dx
  \]
  где $a$ - особая точка и сходящаяся. $b$ либо $+\infty$, либо $-\infty$,
  то $\forall \alpha, \beta
  \in R$
  \[
    \int^{b}_{a} (\alpha f(x) + \beta g(x))dx = \alpha\int^{b}_{a} f(x)dx +
    \beta\int^{b}_{a} g(x)dx
  \]
\end{theorem}

\begin{proof}
  Пусть $c \in (a, b)$
  \[
    \int^{c}_{a} (\alpha f(x) + \beta g(x))dx = \alpha\int^{c}_{a} f(x)dx +
    \beta\int^{c}_{a} g(x)dx
  \]
  \[
    \lim_{c \to b - 0} \int^{c}_{a} (\alpha f(x) + \beta g(x))dx =
    \lim_{c \to b - 0} \alpha\int^{c}_{a} f(x)dx +
    \lim_{c \to b - 0} \beta\int^{c}_{a} g(x)dx =
    \alpha\int^{b}_{a} f(x)dx + \beta\int^{b}_{a} g(x)dx
  \]
\end{proof}

\begin{theorem}[2]
  Пусть интеграл с особой точкой $b$ - сходящейся.
  \[
    \int^{b}_{a} f(x)dx ~~~ \int^{b}_{a} g(x)dx
  \]
  $\forall x \in [a, b] ~~~ f(x) \le g(x)$ Тогда:
  \[
    \int^{b}_{a} f(x)dx \le \int^{b}_{a} g(x)dx
  \]
\end{theorem}

\begin{proof}
  $\forall c \in (a, b)$
  \[
    \int^{c}_{a} f(x)dx \le \int^{c}_{a} g(x)dx
  \]
  \[
    \lim_{c \ to b - 0} \int^{b}_{a} f(x)dx \le \lim_{c \ to b - 0}
    \int^{b}_{a} g(x)dx \Rightarrow \int^{b}_{a} f(x)dx \le \int^{b}_{a} g(x)dx
  \]
\end{proof}

\begin{theorem}[3]
  Пусть интеграл сходящийся и с особой точкой $b$:
  \[
    \int^{b}_{a} f(x)dx
  \]
  $F(x)$ - первообразная $f(x)$ на $[a, b]$. Если:
  \[
    \lim_{x \to  b - 0} F(x) = F(b - 0)
  \]
  то
  \[
    \int^{b}_{a} f(x)dx = \lim_{c \to b - 0} f(c) - F(a) = F(b - 0) - F(a) =
    F(x)|^{b}_{a}
  \]
\end{theorem}

\begin{proof}
  \[
    \int^{b}_{a} f(x)dx = \lim_{c \to b - 0} \int^{c}_{a} f(x)dx =
    \lim_{c \to b - 0} F(c) - F(a) = F(b - 0) - F(a)
  \]
\end{proof}

\begin{theorem}[4]
  Функция $f(x)$ определена на $[a, b)$. $x = \varphi(t) \nearrow ~~~
  [\alpha, \beta)$\\
  $\varphi(\alpha) = a ~~~ \lim_{t \ to \beta - 0} \varphi(t)$ Тогда
  \[
    \int^{b}_{a} f(x)dx = \int^{\beta}_{\alpha} f(\varphi(t)) \varphi'(t)dt
  \]
  При этом несобственный интеграл сходится и расходится одновременно.
\end{theorem}

\begin{proof}
  $c \in (a, b) ~~~ c = \varphi(\gamma) ~~~ \gamma \in (\alpha, \beta)$
  \[
    \int^{c}_{a} f(x)dx = \int^{\gamma}_{\alpha} f(\varphi(t)) \varphi'(t)dt
  \]
  \[
    \lim_{c \to b - 0} \int^{c}_{a} f(x)dx \Leftrightarrow
    \lim_{\gamma \to \beta - 0} \int^{\gamma}_{\alpha} f(\varphi(t))
    \varphi'(t)dt = \int^{b}_{a} f(x)dx = \int^{\beta}_{\alpha} f(\varphi(t))
    \varphi'(t)dt
  \]
\end{proof}

\begin{theorem}[5]
  $u = u(x) ~~~ v = v(x)$ - интегрируемо-дифференцируемые функции на $[a, b)$.
  $b$ - особоя точка.\\
  Если :
  \[
    \lim_{x \to b - 0} u(x) \cdot v(x) = u(b - 0) \cdot v(b - 0)
  \]
  то
  \[
    \int^{b}_{a} udv = u(b - 0) \cdot v(b - 0) - \int^{b}_{a} vdu
  \]
  При этом оба интеграла сходятся и расходятся одновременно.
\end{theorem}

\begin{title}[\Large]
  Признаки сходимости не собственных интегралов от неотрицательных функций
\end{title}

\begin{theorem}[1]
  Если $f(x) \ge 0 ~~~ \forall x \in [a, b)$, $b$ - особая точка и интеграл
  сходящийся
  \[
    \int^{b}_{a} f(x)dx \Leftrightarrow F(x) - \int^{x}_{a} f(t)dt
  \]
  $F(x)$ - ограниченная функция. $\exists M ~~~ |F(x)| \le M$
\end{theorem}

\begin{proof}
  Заметим что $F(x) \nearrow [a, b) ~~~ a \le x_1 < x_2 < b$, то
  \[
    F(x_2) - F(x_1) = \int^{x_2}_{x_1} f(t)dt \ge 0
  \]
  \[
    \lim_{x \to b - 0} F(x) - F(a)
  \]
  Если некая $F$ имеет предел в точке, то она ограничена в окрестности точки $b$
\end{proof}

\begin{theorem}[2]
  Признак сравнения для не собственного интеграла\\
  Если $\forall x \in [a, b) ~~~ 0 \le f(x) \le g(x)$, $b$ - особая точка,
  тогда\\
  \bk{I} Если интеграл $g(x)$ сходящийся, то интеграл от $f(x)$ тоже сходящийся
  \[\int^{b}_{a} g(x)dx \Rightarrow \int^{b}_{a} f(x)dx\]
  \bk{II} Если интеграл $f(x)$ расходящийся, то интеграл от $g(x)$ тоже
  рассходящийся
  \[\int^{b}_{a} f(x)dx \Rightarrow \int^{b}_{a} g(x)dx\]
\end{theorem}

\begin{proof}
  $\forall c \in [a, b)$
  \bk{I}
  \[
    \int^{c}_{a} f(x)dx \le \int^{c}_{a} g(x)dx \le \int^{b}_{a} g(x)dx = M
  \]
  На основании 1й теоремы
  \[
    \lim_{c \to b - 0} \int^{c}_{a} f(x)dx = \int^{b}_{a} f(x)dx
  \]
  \bk{II} От противного (Интеграл от $f(x)$ - расходится, а от $g(x)$ -
  сходится)
  \[
    \int^{b}_{a} f(x)dx \Rightarrow \int^{b}_{a} g(x)dx
  \]
  По теореме 1 - противоречие
\end{proof}