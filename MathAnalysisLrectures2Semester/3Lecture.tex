\begin {center}
\bd{Интегрирование функции одной переменной} 
\end {center}

\bd {Первообразная и неопределенный интеграл. Свойства:} \\
Пусть $F(x); f(x) <a; b>$ F - непрерывна и дифференцируема на (a;b) \\
Если $\forall x \in (a; b) ~~ F`(x) = f(x)$, то \kv {F(x)} - называют первообразной для \kv {f} на (a; b). \\
$f(x) = \frac {1}{x} ~~ (a; b) \\
F(x) = Ln(x)$ \\

\bd {Свойства:}
1) Если F(x) - первообразная для $f(x) ~ \Delta F(x) + C$ также первообразная к f(x) $(F(x) + C)` = F`(x) + 0 = f(x)$ \\
2) $F_{1}(x), F_{2}(x)$ - обе первообразнозные для $f(x) ~ \Delta$, то $F_{1}(x) - F_{2}(x) = C$ \\

$(F_{1}(x) - F_{2}(x))` = F`_{1}(x) - F`_{2}(x) = f(x) - f(x) \equiv 0$ \\
$F_{1}(x) - F_{2}(x) = const$ \\

\bd {Определение} \\
Совокупность всех первообразных для $f(x)$ на промежутке $\Delta$ - называют неопределенным интегралом от $f(x)$ \\
\[\int f(x)dx\] \\
$f(x)$ - называют подынтервальной функцией \\
$f(x)dx$ - Подынтеграл выражения \\
\[\int f(x)dx = F(x) + C\] \\
Где $F(x) + C$ - семейство функций \\

1) Диффиренциал от неопределенного интеграла равен подынтергальному выражению:
\[d \left ( \int f(x)dx \right ) = f(x)dx \] \\
 Доказательство \\
$d \left ( \int f(x)dx \right ) = d(F(x) + C) = dF(x) + 0 = F`(x)dx = f(x)dx$ \\

2)\[\int d(F(x) = F(x) + C\] \\
 Доказательство \\
$\int dF(x) = \int F`(x)dx = F(x) + C$ \\

3) $a, b \in R \\
\int (af(x) + bf(x)) = a \int f(x)dx + b \int g(x)dx$ \\
Доказательство в одну сторону \\
$aF(x) + bG(x) \in \int (af(x) + bg(x))dx \\
aF`(x) + G`(x) = af(x) + bg(x)$ \\

\bd {Замена переменных (подстановка) в неопределенный интеграл} \\
\bk {Теорема} \\
Пусть $F(t)$ - первообразная для $f(x)$ на промежутке $T$ \\
$t = \phi (x) ~ неопределенна ~ и ~ дифференцирована ~ на ~ \delta ~~ \phi
(\delta) \subset T$ \\
$\int f(\phi (x)) \phi`(x)dx = F(\phi (x)) + C \\
(F(\phi(x)) + C)` = F`(\phi (x)) \cdot \phi`(x) + 0 = f(\phi(x)) = \phi`(x)$ \\

