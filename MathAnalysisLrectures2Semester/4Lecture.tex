\begin{title}[\Large]
  Интегрирование по частям в неопределенном интеграле.
\end{title}
$u = u(x)$\\
$v = v(x)$\\
$(u(x)\cdot v(x))' = u'(x)\cdot v(x) + u(x)\cdot v'(x)$\\
$\int(u(x)\cdot v(x))'dx = \int u'(x)\cdot v(x)dx + \int u(x)\cdot v'(x)dx$\\
$u(x)\cdot v(x) = \int v(x)du(x) = \int u(x)dv(x)$\\
\[\int udv = uv - \int vdu\]

\begin{title}[\Large]
  Интегрирование простых рациональных дробей.
\end{title}
Рациональная дробь $\frac{P_n(x)}{Q_m(x)}$ где $P_n(x), Q_m(x)$ многочлены.\\
Если $P_n(x) \ge Q_m(x)$ неправильная дробь.\\
Если $P_n(x) < Q_m(x)$ правильная дробь.\\

Существует 4 типа рациональных дробей.\\
1. $\frac{A}{x-a}$\\
2. $\frac{A}{(x-a)^n} ~~~ n > 1$\\
3. $\frac{Ax + B}{x^2 + px +q} ~~~ D = p^2 - 4q < 0$\\
4. $\frac{Ax + B}{(x^2 + px +q)^n} ~~~ D < 0 ~~ n > 1$\\

1. $\int \frac{A}{x-a}dx = A\int \frac{d(x-a)}{x-a} = A\ln|x-a| + C$\\
2. $\int \frac{Adx}{(x-a)^n} = A\int (x-a)^{-n}d(x-a)
  = \frac{A(x-a)^{1-n}}{1-n} + C$\\
3. Приводим числетель в равенство с продиффиринциоравным знаменателем плюс
  константа. (путем вынесени из числителя числа за интеграл).
  Раскладываем интеграл на суммы (разности) интегралов. Интеграл с константой
  в числителе решаем с помощью выражения в знаменателе полного кдвадрата
  (приведение к табличному интегралу). Интеграл с линейным выражением
  в числителе решаем внеся знаменатель под дефференциал.\\
4. То же что в 3 пункте, только надо брать не весь знаменатель, а то что под
  степенью знаменателя. После применить 2 метод.\\

\begin{title}[\Large]
  Интегрирование рациональных дробей.
\end{title}
Любуюю неправильнульную рациональную дробь можно представить в виде:\\
$\frac{P_n(x)}{Q_m(x)} = S_k(x) + \frac{H_c(x)}{Q_n(x)} ~~ m>n$ можно разложить
на сумму простых дробей.\\