$\varphi (Ln)$ - длина ломанной. $\varphi = sup \varphi (Ln)$ - число

\begin{theorem}
  Если $\omega$ непрерывна деференциирума то
  $$
  \varphi (\omega) \le (b-a) max |r'(t)| ~~~ t \in [a,b]
  $$
\end{theorem}

\begin{proof}
  $
  R([a,b]) = \{t_k\} ~~~ Ln \vec r (t_k)
  $
  $$
  \varphi (Ln) = \sum_{k=1}^n |\vec r (t_k) - \vec r (t_{k-1}) \le
  \sum_{k=1}^n (t_k - t_{k-1}) |\vec r (t_k^*)| \le max |\vec r (t)|
  \sum_{k=1}^n (t_k - t_{k-1}) \le max |\vec r (t)| (b-a)
  $$
  $t_k^* \in (t_{k-1}, t_k) ~~~ t \in [a,b]$
  $$
  \varphi (\omega) = sup \varphi (Ln) \le max |\vec r (t)|(b-a) ~ t \in [a,b]
  $$
\end{proof}

\begin{theorem}
  $\omega$ - непрерывна диференциирума $\vec r = r(t) ~ t \in [a,b]$ если
  $\varphi (t)$ - длина части кривой между $\vec r(a), \vec r(t)$ то
  $\forall t \in [a,b] ~~~ \varphi = |\vec r'(t)|$
\end{theorem}

\begin{proof}
  $t, t+\Delta t \in [a,b]$
  $$
  \Delta \varphi (t) = \varphi( t+\Delta t) - \varphi(t)
  $$
  $$
  |\Delta \vec r(t) | = |\vec r(t+\Delta t) - \vec r(t)|
  $$
  $$
  |\Delta \vec r(t)| \le |\Delta \varphi (t)| \le |\Delta t|
  $$
  $$
  \left| \frac{\Delta \vec r(t)}{\Delta t} \right| \le
  \left| \frac{\Delta \varphi (t)}{\Delta t} \right| \le
  max |\vec r(t)|
  $$
  $$
  \left| \frac{\Delta \vec r(t)}{\Delta t} \right| \le
  \frac{\Delta \varphi (t)}{\Delta t} \le
  max |\vec r(t)| ~~~ \Delta t \to 0
  $$
  $$
  |\vec r'(t)| \le \varphi'(t) \le |\vec r'(t)|
  $$
\end{proof}

  Если $\omega$ непрерывна деференциирумая кривая
$\vec r = \vec r(t) ~~~ t \in [a,b]$
$$
\varphi (\omega) = \varphi(b) - \varphi(a) =
\int_a^b \varphi'(t)dt =
\int_a^b \vec r'(t)dt
$$

  Если $\omega$ непрерына деференциирума $y = f(x) ~~~ x \in [a,b] ~~~
\vec r(x) = (x, f(x)) \vec r'(x) = (1, f'(x))$
$$
\int_a^b |\vec r'(x)|dx = \int_a^b \sqrt{1 + (f'(x))^2}dx
$$

  Вычисление длины в полярной системе координат.
$$
\varphi (\omega) = \int_a^b \sqrt{x'(t)^2 + (y'(t)^2)}dt
$$
$$
x = \rho (\alpha) \cos \alpha ~~~ y = \rho (\alpha) \sin \alpha
$$
$$
x'(\alpha) = \rho'(\alpha) \cos \alpha - \rho(\alpha) \sin \alpha \\
y'(\alpha) = \rho'(\alpha) \sin \alpha + \rho(\alpha) \cos \alpha
$$
$$
\varphi (\omega) = \int_{\alpha}^{\beta}
\sqrt{\rho^2 (\alpha) + (\rho'(\alpha)^2)} d\alpha
$$

  Если $P$ поверхность вращения $y=f(x) > 0 ~~~ x \in [a,b]$
$$
S(P) = 2\pi \int_a^b f(x)dx
$$
$$
d\varphi = \varphi' (x)dx = |\vec r'(x)|dx = \sqrt{1+f'(x)^2}dx
$$

\begin{title}[\Large]
  Физические приложения интеграла.
\end{title}

%%%%%%%%%%%%%%%%%%%%%%%%%%%%%%%%%%%%%%%%