\begin{title}
  Интегрирование тигонометрических выражений.
\end{title}

\[ \int R(\sin x; \cos x)dx \]

\[
  \sin x = \frac{2\sin \frac{x}{2} \cdot \cos \frac{x}{2}}{\sin^2 \frac{x}{2}
  + \cos^2 \frac{x}{2}} = \frac{2\tg \frac{x}{2}}{1 + \tg^2 \frac{x}{2}} =
  | t = \tg \frac{x}{2} | = \frac{2t}{1+t^2}
\]

\[
  \cos x = \frac{\cos^2 \frac{x}{2} - \sin^2 \frac{x}{2}}
  {\cos^2 \frac{x}{2} + \sin^2 \frac{x}{2}} =
  \frac{1 - \tg^2 \frac{x}{2}}{1 + \tg^2 \frac{x}{2}} =
  | t = \tg \frac{x}{2} | =
  \frac{1 - t^2}{1 + t^2}
\]

\[
  \tg x = \frac{\sin x}{\cos x} = \frac{2t}{1+t^2} : \frac{1-t^2}{1+t^2} =
  \frac{2t \cancel{(1+t^2)}}{\cancel{(1+t^2)} (1-t^2)} = \frac{2t}{1-t^2}
\]

\[
  \ctg x = \frac{\cos x}{\sin x} = \frac{1-t^2}{1+t^2} : \frac{2t}{1+t^2} =
  \frac{(1-t^2) \cancel{(1+t^2)}}{\cancel{(1+t^2)} 2t} = \frac{1-t^2}{2t}
\]

Так как мы интегрируем, то $x = 2\arctg t ~~ dx = \frac{2}{1+t^2}dt$
\begin{eqnarray*}
  \int (\sin x)^{\alpha_1} \cdot (\cos x)^{\alpha_2} dx =
  \int (\sin x)^{\alpha_1} \cdot (\sqrt{1 - \sin^2 x})^{\alpha_2} = \\
  | t = \sin x ~ x = \arcsin x ~ dx = \frac{dt}{\sqrt{1-t}} | =
  \int t^{\alpha_1}(1-t^2)^{\frac{\alpha_2 - 1}{2}} dt
\end{eqnarray*}




\begin{title}
  Определение определенного интеграла. Необходимые условия существования.
\end{title}

$y = f(x)$ определена на $[a,b]$\\
$x_k \in [a,b]$ такое что $x_0 = a < x_1 < x_2 \ldots < x_k = b$ называют
разбиением отрезка $[a,b]$ и обозначают $R[a,b] =
\{x_k | k = 1,2,3 \ldots n\}$\\
$\Delta_k = [x_{k-1}, x_k]$ тоже что и $\Delta{x_k} = x_k - x_{k-1}$\\
$\lambda(R) = max\Delta x_k \in 1 \le k \le n$ мелкость разбиения\\
$c_k \in \Delta_k ~ \{c_k\} = \upsilon(R)$ выборка\\
$\sum_{k=1}^{n} f(c_k)\Delta x_k$ интегральная сумма\\

Для $f(x)$ на $[a,b]$ соотвецтвует разбиению R и выборке $\upsilon(R)$\\

\begin{defin}
  Если существует число $I$ такое что для
  \begin{eqnarray*}
    \forall\epsilon>0 ~~ \exists\delta_{\epsilon}>0 ~~ \forall R[a,b]
    \lambda(R)<\delta ~~ \forall\upsilon(R) ~~~
    |\sum_{k=1}^{n} f(c_k)\Delta x_k - I| < \epsilon
  \end{eqnarray*}
  называется опрделенным интегралом от функции $f$ на $[a,b]$ и обознают
  $$\int_{a}^{b} f(x)dx$$
\end{defin}

\begin{theorem}
  Необходимое существование определенного интеграла. Если для $f(x)$ на $[a,b]$
  $\forall \int_{a}{b}$ то $f(x)$ ограничена на $[a,b]$
\end{theorem}

\begin{proof}
  Предположим $f(x)$ ограничена в точке $a$ на первом отрезке любого разбиения
  на $[a,b]$\\
  Пусть $\epsilon = 1$ тогда \[\exists \delta_1 > 0 ~~ \forall R[a,b] ~~
  \upsilon(R) < \delta_1 ~~ \forall \upsilon(R)\]
  \[I-1 < \sum_{k=1}^{n} f(c_k)\Delta x_k < I+\] зафиксируем $c_1, c_2, \ldots c_n$
  тогда \[I-1 < \sum_{k=1}^{n} f(c_k)\Delta x_k < I+1\]
  Если для $f(x)$ на $[a,b]$ $\exists \int_{b}^{a}$ то $f(x)$
  является теоремой по Виету.
\end{proof}