\begin{title}
  Числовые ряды.
\end{title}

\begin{title}[\Large]
  Сходимость чилосвых рядов необходимые и достаточные условия сходимости рядов.
\end{title}

  $a_k$ - числовая последовательность.

  $\sum_{k=1}^{\infty} a_k = a_1 + a_2 + \ldots a_k + \ldots$

  $S_n = \sum_{k=1}^n a_k = a_1 + a_2 + \ldots + a_n$ $n$ частичная (частая)
сумма числового ряда.

  Если существует конечный $\lim_{n \to \infty} S_n = S$ то его называют суммой
числового ряда, который сходится иначе раходящийся рядом.

\begin{theorem}
  Необходимые условия сходимости числового ряда
  $$
  \sum_{k=1}^{\infty} a_k
  $$
  Сходится если
  $$
  \lim_{k \to \infty} a_k = 0
  $$
\end{theorem}

\begin{proof}
  $$
  \exists \lim_{n \to \infty} S_n = S ~~~ a_n = S_n - S_{n-1}
  $$
  $$
  \lim_{n \to \infty} a_n = \lim_{n \to \infty} S_n -
  \lim_{n \to \infty} S_{n-1} = S - S = 0
  $$
\end{proof}

\begin{title}[\Large]
  Свойства сходящегося числового ряда.
\end{title}

\begin{theorem}
  $$
  \sum_{k=1}^{\infty} a_k = A ~~~
  \sum_{k=1}^{\infty} b_k = B ~~~
  \alpha, \beta \in R
  $$
  $$
  \sum_{k=1}^{\infty} (\alpha a_k + \beta b_k) = \alpha A + \beta B
  $$
  $$
  S_n = \sum_{k=1}^n (\alpha a_k + \beta b_k) =
  \alpha \sum_{k=1}^n a_k + \beta \sum_{k=1}^n b_k = \alpha A_n + \beta b_n
  $$
  $$
  \lim_{n \to \infty} S_n = \alpha \lim A_n + \beta \lim B_m
  = \alpha A + \beta B
  $$
\end{theorem}

\begin{theorem}
  $
  \sum_{k=1}^{\infty} a_k = S \sum_{\varphi = 1}^{\infty} b_{\varphi}
  $
  конечная замена некоторым рядом стоящими ряда $a_k$ на одни без перестановке
  $
  \sum_{\varphi}^{\infty} b_{\varphi} = S
  $
  $
  b_1 = a_1 + a_2 + \ldots + a_{k_1}
  b_2 = a_{k_1+1} + a_{k_1+2} + \ldots + a_{k_2}
  b_3 = a_{k_2+1} + a_{k_2+2} + \ldots + a_{k_3}
  $
\end{theorem}

\begin{proof}
  $S_n$ - частная сумма $a_k$. $G_m$ частная сумма $b_k$ $G_n = S_{k_m}$
  последовательность суммы первого ряда
  $$
  \lim_{n \to \infty} G_n = \lim_{n \to \infty} S_n = S
  $$
\end{proof}

  Добавление конечного чилса членов или отбрасывание конечного числа членов к
числовому ряду неизменяет его сходимость но может изменить его сумму.

\begin{theorem}
  $\sum_{k=1}^{\infty}$
\end{theorem}

\begin{proof}
  $$
  b_1 + b_2 + \ldots + b_m + \sum_{k=1}^{\infty} a_k
  $$
  $$
  G_n = b_1 + b_2 + \ldots + b_n + S_{n-m}
  $$
  $$
  G_m = S_n - (a_1 + a_2 + \ldots + a_n)
  $$
\end{proof}

\begin{title}[\Large]
  Критерий Коши сходимости числового ряда.
\end{title}

$\sum_{k+1}^{\infty} a_k$ был сходящимся необходимое и достаточное условие.

$$
\forall \varepsilon > 0 ~~~
\exists n_{\varepsilon} \in N ~~~
\forall n \le n_{\varepsilon} ~~~
\forall p \in N ~~~
\left| \sum_{k=n+1}^{n+1} < \varepsilon \right|
$$

\begin{proof}
  $$
  S_n = \sum_{k=1}^n a_k ~~~ \lim_{n \to \infty} S_n = S
  $$
  $$
  \forall \varepsilon > 0 ~~~
  \exists n_{\varepsilon} \in N ~~~
  \forall \varphi \ge n_{\varepsilon} ~~~
  \forall p \in N ~~~
  \left| S_{\varphi + p} - S_{\varphi} < \varepsilon \right|
  $$
\end{proof}

\begin{title}[\Large]
  Интегральный признак сходимости числового ряда.
\end{title}

$
\forall k \in N ~~~
a_k \ge 0
$
для сходимости $a_k$ необходимо и достаточно чтобы $S_k \le M$

\begin{proof}
  $S_k$ возрастает $S_{n+1} - S_n = a_{n+1} \ge 0$
\end{proof}

\begin{theorem}
  Интегральный признак числового ряда $f(x)$ определена, непрерывна,
  неотрицательна на $[1, +\infty]$
  $$
  \sum_{k=1}^{\infty} f(k) ~~~~~~ \int_1^{\infty} f(x)dx
  $$
  сходится только если оба сходятся
\end{theorem}

\begin{proof}
  $\Delta = [k, k+1] ~~ k \in N$
  $$
  \forall x \in \Delta k ~~~ f(k+1) \le f(x) \le f(k)
  $$
  $$
  f(k+1) \le \int_k^{k+1} f(x)dx \le f(k) ~~~ k = 1, \ldots, n
  $$
  $$
  \sum_{k=1}^n f(k+1) \le \sum_{k=1}^n \int_k^{k+1} f(x)dx \le \sum_{k=1}^n f(k)
  $$
  $$
  S_{n+1} < f(x) \le \int_1^{n+1} f(x)dx \le S_n
  $$
  $$
  \int_1^{+\infty} f(x)dx ~~~ \forall n \in N ~~~
  \int_1^{n+1} f(x)dx \le \int_1^{+\infty} f(x) dx = M
  $$
  $$
  \forall n \in N ~~~ S_{n+1} \le m + f(x)
  $$
  $$
  \sum_{k=1}^{\infty} ~~~
  \forall n \in N ~~~
  S_m \le M ~~~
  \int_1^b f(x)dx \le \int_1^{n+1} f(x)dx \le S_n \le M
  $$
  $\forall b>1 ~~~ \exists n+1 > b$
  $$
  \lim_{b \to +\infty} \int_1^b f(x)dx = \int_1^{\infty} f(x)dx
  $$
\end{proof}

\begin{title}
  Признак сравнения для определения сходимости числового ряда.
\end{title}

\begin{theorem}
  $\forall k \in N ~~~ 0 \le a_k \le b_k$ тогда

  $\sum_{k=1}^{\infty} b_k$ сходится следовательно $\sum_{k=1}^{\infty} a_k$
  сходится

  $\sum_{k=1}^{\infty} a_k$ расходится следовательно $\sum_{k=1}^{\infty} b_k$
  расходится
\end{theorem}

\begin{proof}
  $\sum_{k=1}^{\infty} b_k$ сходится
  $$
  \forall n \in N ~~~ B_n = \sum_{k=1}^n b_k \le M
  $$
  $$
  A_n = \sum_{k=1}^n a_k \le B_n \le M
  $$
  второе утверждение получается и первого методом от противного.
\end{proof}

\begin{theorem}
  $$
  \forall k \in N ~~~ a_k, b_k > 0 ~~~ a_k \sim b_k ~~ k \to \infty
  $$
  $\sum_1^{\infty} ~~~ \sum_1^{\infty} b_k$ сходится
\end{theorem}

\begin{proof}
  $$
  \forall \varepsilon > 0 ~~~ \exists n_{\varepsilon} ~~~
  \forall k \le n_{\varepsilon} \left| \frac{a_k}{b_k} -1 \right| < \varepsilon
  $$
  $$
  1 - \varepsilon < \frac{a_k}{b_k} < 1 + \varepsilon
  $$
  $$
  (1 - \varepsilon)b_k < a_k < (1 + \varepsilon)b_k
  $$
  $$
  \varepsilon = \frac{1}{2} ~~~
  \exists n_{frac{1}{2}} \in N ~~~
  \forall k \ge n_{\frac{1}{2}} ~~~
  \frac{b_k}{2} < a_k < \frac{3}{2} b_k
  $$
  $$
  \sum_{k=1}^{\infty} b_k ~~~ \sum_{k=n_{\frac{1}{2}}} b_k ~~~
  \sum_{k=n_{\frac{1}{2}}} \frac{3}{2} b_k
  $$
\end{proof}

\begin{theorem}
  $$
  \forall k \in N ~~~ a_k > 0 ~~~ \exists \lambda < 1 ~~~
  \frac{a_{k+1}}{a_k} \le \lambda
  $$
  тогда $\sum_{k=1}^{\infty} a_k$ сходится
  $$
  \forall k \in N ~~~ \frac{a_{k+1}}{a_k} \ge 1
  $$
  $\sum_{k=1}^{\infty}$ расходится
\end{theorem}

\begin{proof}
  $$
  a_2 \le \lambda a_1
  $$
  $$
  a_3 \le \lambda a_2 \le \lambda^2 a_1
  $$
  $$
  a_4 \le \lambda a_2 \le \lambda^3 a_1
  $$
  $$
  a_4 \le \lambda a_2 \le \lambda^3 a_1
  $$
  $$
  a_k < a_1 \lambda^{k-1} ~~~ 0 \le \lambda < 1
  $$
  $$
  \sum_{k_1}^{\infty} a_1 \lambda^{k-1} = \frac{a_1}{a-\lambda}
  $$
  $\sum_{k=0}^{\infty} a_k$ сходится
\end{proof}

\begin{theorem}
  $$
  \forall k \in N ~~~ a_k > 0 ~~~
  \lim_{k \to \infty} \frac{a_{n+1}}{a_n} = \lambda
  $$
  если $\lambda < 1 ~~~ \sum_{k=1}^{\infty} a_k$ сходится

  если $\lambda > 1 ~~~ \sum_{k=1}^{\infty}$ расходится

  если $\lambda = 1 ~~~ \sum_{k=1}^{\infty}$ может быть что угодно

  Наиболее эфективно его можно использовать если члены ряда содержат факториал.
\end{theorem}

\begin{proof}
  $$
  \forall \varepsilon > 0 ~~~
  \exists n_{\varepsilon} \in N ~~~
  \forall k \ge n_{\varepsilon} ~~~
  \left| \frac{a_{n+1}}{a_n} \lambda \right| < \varepsilon ~~~
  \lambda - \varepsilon < \frac{a_{n+1}}{a_n} < \lambda + \varepsilon
  $$
  $$
  1) ~~ \lambda < 1 ~~~ \lambda + \varepsilon < 1
  $$
  $\sum_{k = n_{\varepsilon}}^{\infty} a_k$ сходится, значит сходится
$\sum_{k = 1}^{\infty} a_k$

  $$
  2) ~~ \lambda > 1 ~~~ \lambda - \varepsilon > 1
  $$
  $\sum_{k = n_{\varepsilon}}^{\infty} a_k$ расходится, значит расходится
$\sum_{k = 1}^{\infty} a_k$
\end{proof}

\begin{title}[\Large]
  Признак Коши сходимости числового ряда.
\end{title}

\begin{theorem}
  $$
  \forall k \in N ~~~ a_k \ge 0 ~~~ \sqrt[k]{a_n} \le \lambda < 1
  $$
  $\sum_{k=1}^{\infty} a_k$ сходится, тогда
  $\sqrt[k]{a_k} \ge 1 ~~~ \sum_{k=1}^{\infty} a_k$ расходится
\end{theorem}

\begin{proof}
  $a_k \le \lambda^k ~~~ \lambda < 1$

  $\sum_{k=1}^{\infty} \lambda^k$ сходится

  $\sum_{k=1}^{\infty} a_k$ cходится

  $\sqrt[k]{a_k} \ge 1 ~~~ a_k \ge 1$

  $\sum_{k=1}^{\infty} a_k$ расходится. Нарушено необходимое условие сходимости
  ряда.
\end{proof}

\begin{theorem}
  $$
  \forall k \in N ~~~ a_k \ge 0 ~~~ \lim_{k \to \infty} \sqrt[k]{a_k} = \lambda
  $$
  тогда

  $1) ~~ \lambda > 1 ~~~ \sum_{k=1}^{\infty} a_k$ сходится.

  $2) ~~ \lambda < 1 ~~~ \sum_{k=1}^{\infty} a_k$ расходится.

  $3) ~~ \lambda = 1$ может расходится и сходится.
\end{theorem}

\begin{proof}
  надо доказать самостоятельно.
\end{proof}

\begin{title}[\Large]
  Признак Лейбница сходимости знака чередования.
\end{title}

\begin{theorem}
  $\forall k \in N ~~~ a_k > 0 ~~~ a_k \searrow ~~~ a_k \to 0$ тогда

  $\sum_{k=1}^{\infty} (-1)^{k+1} a_k$ сходится.
\end{theorem}

\begin{proof}
  $S_{2n} = a_1 - a_2 + a_3 - a_4 + \ldots + a_{2n-1} - a_{2n}$

  $S_{2n+2} - S_{2n} = -a_{2n+2} + a_{2n+1} > 0$

  $S_{2n+2} > S_{2n}$

  $S_{2n} = a_1 - (a_2 - a_3) - (a_4 - a_5) - \ldots - a_{2n} \le a_1$

  $$
  \lim_{n \to \infty} S_{2n} = S ~~~ \lim_{n \to \infty} S_{2n+1} =
  \lim_{n \to \infty} (S_{2n} + a_{2n+1}) = S + 0 = S
  $$
\end{proof}

Следствие: выполнены все условия Лейбница
$\forall n \in N ~~~ S_{2n} \le S \le S_{2n+1} ~~~ |S_n - S| \le a_{n+1}$

\begin{proof}
  $S_{2n+1} \searrow ~~~ S_{2n+1} - S_{2n-1} = a_{2n+1} - a_{2n} < 0$

  $S_{2n-1} - a_{2n} \le S \le S_{2n} + a_{2n+1}$

  $0 \le S_{2n-1} - S \le a_{2n}$
  $0 \le S - S_{2n} \le a_{2n+1}$
\end{proof}

\begin{title}[\Large]
  Признак дирехле сходимости  числового ряда.
\end{title}

\begin{theorem}
  1) $\sum_{k=1}^{\infty} a_k ~~~ \exists M > 0 ~~~ \forall n \in N
  |\sum_{k=1}^n a_k| \le M$

  2) $b_k \nearrow$ или $\searrow$

  3) $b_k \to 0$

  тогда $\sum_{k=1}^{\infty} a_k b_k$ сходится.
\end{theorem}

\begin{title}[\Large]
  Признак Абеля.
\end{title}

\begin{theorem}
  1) $\sum_{k=1}^{\infty} a_k$ сходится.

  2) $b_k \nearrow$ или $\searrow$

  3) $\exists M > 0 ~~~ \forall k \in N ~~~ |b_k| \le M$

  тогда $\sum_{k=1}^{\infty} a_k b_k$ сходится.
\end{theorem}

\begin{proof}
  $\lim_{k \to \infty} b_k = p ~~~ b_k^* = b_k - p \to 0$

  так как $a_k$ сходится то $| \sum_{k=1}^n a_k \le \Delta$

  $$
  \sum_{k=1}^{\infty} a_k b_k^*  ~~~ \sum_{k=1}^{\infty} a_k (b_k - p) =
  \sum_{k=1}^{\infty} a_k b_k - p \sum_{k=1}^{\infty} a_k
  $$
  $$
  \sum_{k=1}^{\infty} a_k b_k = \sum_{k=1}^{\infty} a_k b_k^* +
  p\sum_{k=1}^{\infty} a_k
  $$
\end{proof}