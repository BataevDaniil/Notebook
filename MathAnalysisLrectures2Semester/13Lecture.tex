\begin{title}
  Абсолютно и условно сходящиеся интегралы.
\end{title}

\begin{defin}
  $\int_a^b f(x)dx$ называется абсолютно сходящимся $f(x)$ абсолютно
интегрируема где $b$ - особая точка, то если сходится $\int_a^b |f(x)|dx$.
\end{defin}

\begin{defin}
  $\int_a^b f(x)dx$ условно сходящимся необсалютно сходящимся если сам интеграл
сходится и интеграл от модуля расходится.
\end{defin}

\begin{theorem}
  Абсолютно сходящийся интеграл является сходящимся
  $$
  \left| \int_a^b f(x)dx \right| \le \int_a^b |f(x)|dx
  $$
\end{theorem}

\begin{proof}
  $$
  \forall \varepsilon > 0 ~~~ \exists \delta_{\varepsilon} \in [a,b) ~~~
  \forall c',c'' \in [d_{\varepsilon}, b) ~~~
  \left| \int_{c'}^{c''} |f(x)|dx \right| < \varepsilon
  $$
  $$
  \left| \int_{c'}^{c''} f(x)dx \right| \le
  \left| \int_{c'}^{c''} |f(x)|dx \right| < \varepsilon
  $$
  $$
  \forall c \in [a,b) ~~~ \left| \int_a^c f(x)dx \right| \le
  \left| \int_a^c |f(x)|dx \right|
  $$
  $$
  \lim_{c \to b-0} \left| \int_a^{c} f(x)dx \right| \le
  \lim_{c \to b-0} \left| \int_a^c |f(x)|dx \right|
  $$
  $$
  \left| \int_a^b f(x)dx \right| \le
  \left| \int_a^b |f(x)|dx \right|
  $$
\end{proof}

\begin{theorem}
  $b$ - может быть особая точка $\int_a^b f(x)dx ~~~ \int_a^b g(x)dx$
абсолютно интегрируемы тогда
  $$
  \int_a^b f(x)dx ~~~ \int_a^b (f(x) + g(x)) dx
  $$
  либо оба сходятся, либо расходятся, либо оба абсолютно сходятся.
\end{theorem}

\begin{proof}
  $$
  \forall \varepsilon < 0 ~~~ \exists \delta_{\varepsilon} \in [a,b)
  \forall c', c'' \in [\delta_{\varepsilon}, b)
  \left| \int_{c'}^{c''} |f(x)|dx \right| < \frac{\varepsilon}{2}
  \left| \int_{c'}^{c''} |f(x)|dx \right| < \frac{\varepsilon}{2}
  $$
  $$
  \left| \int_{c'}^{c''} |(f(x) + g(x))| dx \right| \le
  \left| \int_{c'}^{c''} f(x) \right| \left| \int_{c'}^{c''} g(x) \right|
  < \frac{\varepsilon}{2} + \frac{\varepsilon}{2} = \varepsilon
  $$
  $$
  f(x) = f(x) + g(x) + (-g(x))
  $$
\end{proof}

\begin{title}
  Мера Жордана в пространстве $R^n$
\end{title}

\begin{defin}
  $$
  E \to \mu (E) \ge 0
  $$
  $$
  E = \cup_{k=1}^m ~~~ \mu(E) = \sum_{k=1}^n \mu (E_k)
  $$
\end{defin}

\begin{defin}[умножения множеств]
   $$
   E \times D = \{ (x,y): x \in E; y \in D \}
   $$
\end{defin}

\begin{defin}[параллелипипеда]
  $P \subset R^n ~~~ P = [a_1, b_1] \times [a_2, b_2] \times [a_n, b_n]$ или
множество которое получится если отбросить всей или части границы.

  $m(p) = \prod_{k=1}^n (b_k - a_k)$

  $E \subset R^n$ называется элементарным если его можно представить в виде
конечного $E = \sqcup_{k=1}^m P_k$

  $\sqcup$ - попарно непересекающиеся множества.
\end{defin}

$$
m (E) = \sum_{k=1}^m m(R_k)
$$