\begin{title}
  Интегрировнаие функции одной переменной.
\end{title}

\begin{title}[\Large]
  Первообразная и неопределенный интеграл и их cвойства.
\end{title}

\begin{defin}[первообразной]
  Пусть $F(x); f(x)$ определена на $<a;b>$, F - непрерывна на $<a,b>$ и
  дифференцируема на $(a;b)$ тогда $\forall x \in (a; b) ~~ F'(x) = f(x)$,
  то $F(x)$ называют \kv{первообразной} для $f(x)$ на $(a;b)$.
\end{defin}

\bd{Свойства:}\\
1) Если F(x) - первообразная для $f(x)$ и определена на $\Delta$ то $F(x) + C$
также первообразная к $f(x)$.\\
\begin{proof}
  $(F(x) + C)' = F'(x) + 0 = f(x)$.
\end{proof}

2) Если $F_{1}(x), F_{2}(x)$ обе первообразнозные для $f(x)$ на промежутке
$\Delta$, то $F_{1}(x) - F_{2}(x) = C$.\\
\begin{proof}
  $(F_{1}(x) - F_{2}(x))' = F'_{1}(x) - F'_{2}(x) = f(x) - f(x) \equiv 0$\\
  $F_{1}(x) - F_{2}(x) = const$.
\end{proof}

\begin{defin}[неопредленного интеграла]
  Совокупность всех первообразных для $f(x)$ на промежутке $<\Delta>$ называют
  \kv{неопределенным интегралом} от $f(x)$ и обозначают
  \[\int f(x)dx\]
  $f(x)$ - называют \kv{подинтервальной функцией.}\\
  $f(x)dx$ - \kv{Подинтегральным выражением.}
  \[\int f(x)dx = \cancel{\{} F(x) + C \cancel{\}}\]
  Где $F(x) + C$ - \kv{семейство функций.}
\end{defin}

1) Диффиренциал от неопределенного интеграла равен подинтергальному выражению:
\[d \left ( \int f(x)dx \right ) = f(x)dx \]
\begin{proof}
  \[d \left ( \int f(x)dx \right ) = d(F(x) + C) = d(F(x)) + 0 = F'(x)dx
  = f(x)dx\]
\end{proof}

2) \[\int d(F(x)) = F(x) + C\]
\begin{proof}
  \[\int d(F(x)) = \int F'(x)dx = F(x) + C\]
\end{proof}

3) $a, b \in R$\\
\[\int (af(x) + bf(x)) = a \int f(x)dx + b \int g(x)dx\]
\begin{proof}
  \begin{eqnarray*}
    aF(x) + bG(x) = \int (af(x) + bg(x))dx\\
    aF'(x) + bG'(x) = af(x) + bg(x)
  \end{eqnarray*}
\end{proof}

\begin{title}[\Large]
  Замена переменных (подстановка) в неопределенных интегралах.
\end{title}

\begin{theorem}
  Пусть $F(t)$ - первообразная для $f(t)$ на $<T>$ где $t = \varphi (x)$
  непрерывна и дифференциируема на $\varphi(\Delta) \subset T$ то\\
  \[\int f(\varphi (x)) \varphi'(x)dx = F(\varphi (x)) + C\]
\end{theorem}

\begin{proof}
  \[(F(\varphi(x)) + C)' = F'(\varphi (x)) \varphi'(x) + 0 = f(\varphi(x))
    \varphi'(x)\]\\
\end{proof}

\begin{title}[\Large]
  Интегрирование по частям в неопределенном интеграле.
\end{title}
$u = u(x)$\\
$v = v(x)$\\
$(u(x)\cdot v(x))' = u'(x)\cdot v(x) + u(x)\cdot v'(x)$\\
$\int(u(x)\cdot v(x))'dx = \int u'(x)\cdot v(x)dx + \int u(x)\cdot v'(x)dx$\\
$u(x)\cdot v(x) = \int v(x)du(x) = \int u(x)dv(x)$\\
\[\int udv = uv - \int vdu\]

\begin{title}[\Large]
  Интегрирование простых рациональных дробей.
\end{title}
Рациональная дробь $\frac{P_n(x)}{Q_m(x)}$ где $P_n(x), Q_m(x)$ многочлены.\\
Если $P_n(x) \ge Q_m(x)$ неправильная дробь.\\
Если $P_n(x) < Q_m(x)$ правильная дробь.\\

Существует 4 типа рациональных дробей.\\
1. $\frac{A}{x-a}$\\
2. $\frac{A}{(x-a)^n} ~~~ n > 1$\\
3. $\frac{Ax + B}{x^2 + px +q} ~~~ D = p^2 - 4q < 0$\\
4. $\frac{Ax + B}{(x^2 + px +q)^n} ~~~ D < 0 ~~ n > 1$\\

1. $\int \frac{A}{x-a}dx = A\int \frac{d(x-a)}{x-a} = A\ln|x-a| + C$\\
2. $\int \frac{Adx}{(x-a)^n} = A\int (x-a)^{-n}d(x-a)
  = \frac{A(x-a)^{1-n}}{1-n} + C$\\
3. Приводим числетель в равенство с продиффиринциоравным знаменателем плюс
  константа. (путем вынесени из числителя числа за интеграл).
  Раскладываем интеграл на суммы (разности) интегралов. Интеграл с константой
  в числителе решаем с помощью выражения в знаменателе полного кдвадрата
  (приведение к табличному интегралу). Интеграл с линейным выражением
  в числителе решаем внеся знаменатель под дефференциал.\\
4. То же что в 3 пункте, только надо брать не весь знаменатель, а то что под
  степенью знаменателя. После применить 2 метод.\\

\begin{title}[\Large]
  Интегрирование рациональных дробей.
\end{title}
Любуюю неправильнульную рациональную дробь можно представить в виде:\\
$\frac{P_n(x)}{Q_m(x)} = S_k(x) + \frac{H_c(x)}{Q_n(x)} ~~ m>n$ можно разложить
на сумму простых дробей.\\

\begin{center}
  \bd{Пример Интегрирования рациональных дробей}\\
\end{center}

\[\int \frac{x^5 - 2x^4 + 5x^3 - 12x^2 + 16x}{x^4 + 2x^2 - 8x + 5} dx = \]

\bd{I}\\
Делим уголком числитель на знаменатель, чтобы дробь привести к правильному виду
\[= \int (x - 2)dx + \int \frac{3x^3 - 5x + 10}{x^4 + 2x^2 - 8x + 5} = \]

\bd{II}\\
Раскладываем знаменатель на множетели.
\[= \int (x - 2)dx + \int \frac{3x^3 - 5x + 10}{(x-1)^2 (x^2+2x+5)} dx = \]

\bd{III}\\
Раскладываем дробь на простые дроби с неопрделенными коэффицентами.
\[
  \frac{3x^3 - 5x + 10}{(x-1)^2 (x^2+2x+5)} = \frac{A}{x - 1} +
  \frac{B}{(x - 1)^2} + \frac{C \cdot x + D}{x^2 + 2x + 5}
\]

\bd{IV}\\
Приводим правую часть к общему знаменателю и убираем знаменатель, раскрываем
скобки и записываем в матрицу построчно кэффеценты перед иксами с одинаковой
степенью.\\

\bd{V}\\
Дальше идет стандартное решение интегралов
\begin{eqnarray*}
  \int (x - 2)dx + \int \frac{dx}{(x - 1)^2} + \int \frac{3x + 5}{x^2 + 2x + 5}
    dx = \\
  =\frac{x^2}{2} - 2x - \frac{1}{x-1} + \frac{3}{2} ln(x^2 + 2x + 5) + \arctg
    \frac{x + 1}{2} + C
\end{eqnarray*}

\begin{title}[\Large]
  Интегрирование иррациональных выражений.
\end{title}

$R(u, v, w \ldots)$, где $R$ - Рациональная дробь по отношению к каждому
элементу\\

\bd{I тип}\\
\[
\int R \left( Cx; \left( \frac{ax + b}{cx + d} \right)^{r_1};
\left( \frac{ax + b}{cx + d} \right)^{r_2} \cdots
\left( \frac{ax + b}{cx + d} \right)^{r_m} \right)dx
\]
$ad - bc \neq 0$, $r_1 \ldots r_m\in Q ~~ r_i = \frac{p_i}{n_i}$
$\left( \frac{ax + b}{cx + d} \right) = t^q$ где $q$ наименьший общий делитель
  $n_i$\\

\bd{II тип}\\
\[\int R \left(x; \sqrt{ax^2 + bx + c} \right)dx\]
$a \neq 0$\\
1) $a > 0 ~~~ \sqrt{ax^2 + bx + c} = \pm \sqrt{a}x \pm t$\\
2) $c > 0 ~~~ \sqrt{ax^2 + bx + c} = \pm xt \pm \sqrt{c}$\\
2) $D > 0 ~~~ \sqrt{ax^2 + bx + c} = \pm t (x - x_0)$\\

\bd{III тип}\\
\[\int x^m \cdot (ax^n + b)^p dx\]
$m, n, p \in Q ~~~ a, b \in R$\\
1) $p \in Z ~~~ x = t^q$ где $q$ наименьший общий знаменатель $m$ и $n$.\\
2) $\frac{m+1}{n} \in Z ~~~ ax^n + b = t^q$ где $q$ знаменатель $p$.\\
3) $\frac{m+1}{n} + p \in Z ~~~ a + \frac{b}{x^n} = t^q$ где $q$ знаменатель
  $p$.\\

\begin{title}[\Large]
  Интегрирование тигонометрических выражений.
\end{title}

\[ \int R(\sin x; \cos x)dx \]

\[
  \sin x = \frac{2\sin \frac{x}{2} \cdot \cos \frac{x}{2}}{\sin^2 \frac{x}{2}
  + \cos^2 \frac{x}{2}} = \frac{2\tg \frac{x}{2}}{1 + \tg^2 \frac{x}{2}} =
  \left| t = \tg \frac{x}{2} \right| = \frac{2t}{1+t^2}
\]

\[
  \cos x = \frac{\cos^2 \frac{x}{2} - \sin^2 \frac{x}{2}}
  {\cos^2 \frac{x}{2} + \sin^2 \frac{x}{2}} =
  \frac{1 - \tg^2 \frac{x}{2}}{1 + \tg^2 \frac{x}{2}} =
  \left| t = \tg \frac{x}{2} \right| =
  \frac{1 - t^2}{1 + t^2}
\]

\[
  \tg x = \frac{\sin x}{\cos x} = \frac{2t}{1+t^2} : \frac{1-t^2}{1+t^2} =
  \frac{2t \cancel{(1+t^2)}}{\cancel{(1+t^2)} (1-t^2)} = \frac{2t}{1-t^2}
\]

\[
  \ctg x = \frac{\cos x}{\sin x} = \frac{1-t^2}{1+t^2} : \frac{2t}{1+t^2} =
  \frac{(1-t^2) \cancel{(1+t^2)}}{\cancel{(1+t^2)} 2t} = \frac{1-t^2}{2t}
\]

Так как мы интегрируем, то $t = \tg \frac{x}{2} ~~ x = 2\arctg t ~~
  dx = \frac{2}{1+t^2}dt$
\begin{eqnarray*}
  \int (\sin x)^{\alpha_1} \cdot (\cos x)^{\alpha_2} dx =
  \int (\sin x)^{\alpha_1} \cdot (\sqrt{1 - \sin^2 x})^{\alpha_2} = \\
  \left| t = \sin x ~~ x = \arcsin t ~~ dx = \frac{dt}{\sqrt{1-t^2}} \right| =
  \int t^{\alpha_1}(1-t^2)^{\frac{\alpha_2 - 1}{2}} dt
\end{eqnarray*}