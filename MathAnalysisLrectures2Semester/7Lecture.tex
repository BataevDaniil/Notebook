\begin{title}[\Large]
  Суммы Дарбу и их свойста. Критерий интегрируемости (по Риману)
\end{title}

Пусть $y = f(x)$ на $[a; b] ~~~ R[a; b]$
\[k = 1, 2 ... n\]
\[M_k = sup f(x) ~~~ x \in \Delta_k\]
\[m_k = inf f(x) ~~~ x \in \Delta_k\]
Верхняя сумма Дарбу
\[\sum^{n}_{k = 1}M_k \Delta x_k = \int^* (R)\]
Нижняя сумма Дарбу
\[\sum^{n}_{k = 1}m_k \Delta x_k = \int_* (R)\]

\bd{Верхние и нижние суммы Дарбу обладают следующими свойствами:}\\
\bk{I}\\
Суммы Дарбу существуют для любой выборки разбиения.\\
\begin{proof}
  \[\int_* (R) \le \sum_{k = 1}^{n} f(c_k) \Delta x_k \le \int^* (R)\]
  \[\forall x \in \Delta x_k ~~~ m_k \le f(x) \le M_k\]
  \[
     m_k \Delta x_k \le f(x) \Delta x_k \le
     M_k \Delta x_k
  \]
  \[
     \sum_{k = 1}^{n} m_k \Delta x_k \le \sum_{k = 1}^{n} f(x)
     \Delta x_k \le \sum_{k = 1}^{n} M_k \Delta x_k
  \]
  \[
     \sum_{k = 1}^{n} m_k \Delta x_k \le \sum_{k = 1}^{n} f(c_k)
    \Delta x_k \le \sum_{k = 1}^{n} M_k \Delta x_k
  \]
\end{proof}

\bk{II}\\
\[Sup_{\upsilon (R)} \sum_{k = 1}^{n} f(c_k) \Delta x_k = \int^* (R)\]
\[Inf_{\upsilon (R)} \sum_{k = 1}^{n} f(c_k) \Delta x_k = \int_* (R)\]

\begin{proof}
  Sup
  \[
     \forall \varepsilon > 0 ~~~ \exists \upsilon_\varepsilon (R) ~~~ 0 \le
     M_k - f(c'_k) < \frac{\varepsilon}{b - a}
  \]
  \[
     0 \le M_k \Delta x_k - f(c'_k) \Delta x_k <
     \frac{\varepsilon}{b - a}\Delta x_k
  \]
  \[
     0 \le \sum_{k = 1}^{n} M_k \Delta x_k - \sum_{k = 1}^{n} f(c'_k)
     \Delta x_k < \frac{\varepsilon}{b - a} \sum_{k = 1}^{n} \Delta x_k
  \]
  \[
     0 \le \int^* (R) - \sum_{k = 1}^{n} f(c'_k) \Delta x_k <
     \frac{\varepsilon}{b - a} (b - a) = \varepsilon
  \]

  Inf
  \[
     \forall \varepsilon > 0 ~~~ \exists \upsilon_\varepsilon (R) ~~~ 0 \le
      f(c'_k) - m_k < \frac{\varepsilon}{b - a}
  \]
  \[
     0 \le f(c'_k) \Delta x_k - m_k \Delta x_k <
     \frac{\varepsilon}{b - a}\Delta x_k
  \]
  \[
     0 \le \sum_{k = 1}^{n} f(c'_k) \Delta x_k - \sum_{k = 1}^{n} m_k \Delta x_k
     < \frac{\varepsilon}{b - a} \sum_{k = 1}^{n} \Delta x_k
  \]
  \[
     0 \le \sum_{k = 1}^{n} f(c'_k) \Delta x_k - \int_* (R) <
     \frac{\varepsilon}{b - a} (b - a) = \varepsilon
  \]
\end{proof}

\bk{III}\\
Если разбиение $R_1 [a,b] \subset R_2 [a,b]$, тогда
\[\int_* (R_1) \le \int_* (R_2) \le \int^* (R_2) \le \int^* (R_1)\]

\begin{proof}
  Для доказательства достаточно расссмотреть случай, когда $R_2$ отличается от
  $R_1$ на одну точку.
  \[R_2 = R_1 \cup \{x^*\}\]
  Рассмотрим разность
  \[m'_k = Inf f(x) ~~~ x \in [x_{k - 1}, x^*]\]
  \[m''_k = Sup f(x) ~~~ x \in [x^*, x_k]\]
  \[\int_* (R_2) - \int_* (R_1) = m'_k (x^* - x_k) + m''_k (x_k - x^*) -
    m_k (x_k - x_{x - 1}) = \]
  \[= m'_k (x^* - x_{k - 1} + m''_k (x_k - x^*) - m_k (x_k - x^*) -
    m_k (x - x_{k - 1} =\]
  \[= (m'_k - m_k)(x^* - x_{k - 1}) + (m''_k - m_k)
    (x_k - x^*) \ge 0\]
\end{proof}

\bk{IV}\\
Если $R_1 \ne R_2$ - \underline{произвольные} разбиения, то
\[\int_* (R_1) \le \int^* (R_2) ~~~ R_3 = R_1 \cup R_2\]

\begin{proof}
  На основании третьего свойства
  \[\int_* (R_1) \le \int_* (R_3) \le \int^* (R_3) \le \int^* (R_2)\]
\end{proof}

\bk{V}
\[Inf \int_* (R) = I_*\] - Нижний интеграл Дарбу
\[Inf \int^* (R) = I^*\] - Верхний интеграл Дарбу
\[\forall R \int_* (R) \le I_* \le I^* \le \int^* (R)\]

\begin{proof}
  \[A, B \in \mathbb R ~~~ \exists C \in \mathbb R\]
  \[A < B ~~~ \forall x \in A, \forall y \in B\]
  \[x < C < y\]
  \[x \le Sup A \le Inf B \le y\]
  \[A = {\int_* (R)}\]
  \[B = {\int^* (R)}\]
\end{proof}

\bd{Критерий интегрирования функции (по Риману)}\\

\begin{theorem}
  Для того, чтобы $f(x) \in [a; b]$ была интегрируема на отрезке [a; b] -
  необходимо и достаточно, чтобы она была ограничена и
  \[\forall \varepsilon > 0 ~~~ \exists \delta_\varepsilon > 0 ~~~\forall R [a; b]
    ~~~ \lambda (R) < \delta_\varepsilon ~~~ 0 \le \int^* (R) - \int_* (R) <
    \varepsilon\]

  \kv{Замечание к критерию}\\
  Для того, чтобы функция $f(x) \in [a; b]$ была интегрирована (по Риману)
  на отрезке $[a; b]$ Необходимо и достаточно, чтобы она была ограничена на
  отрезке и
  \[\forall \varepsilon > 0 ~~~ \exists R_\varepsilon [a; b] \int^*
    (R_\varepsilon) - \int_* (R_\varepsilon) < \varepsilon\]
\end{theorem}

\begin{title}
  Интегрируемость непрерывных функций
\end{title}

\begin{theorem}[1]
  Если функция $f(x)$ - непрерывна на отрезке $[a; b]$, то она инегрируема на
  этом отрезке.
\end{theorem}

\begin{proof}
  По теореме Кантора, функция непрерывая на отрезке, является равномерно
  непрерывной на этом отрезке.
  \[\forall \varepsilon > 0 ~~~ \exists \delta_\varepsilon > 0 ~~~ \forall
    x', x'' \in [a; b] ~~~ |x' - x''| < \delta_\varepsilon ~~~
    |f(x') - f(x'')| < \frac{\varepsilon}{b - a}\]
  Рассмотрим разбиение (R) отрезка $[a; b]$
  \[R [a; b] ~~~ \lambda (R) < \delta_\varepsilon\]
  Тогда
  \[0 \le \int^* (R) - \int_* (R) = \sum_{k = 1}^{n} (M_k - m_k)
    \Delta x_k = \sum_{k = 1}^{n} (f(x'_k) - f(x''_k))
    \Delta x_k\]
  $f(x'_k) = M_k ~~~ f(x''_k) = m_k$
  По теореме Вейерштрасса
  \[x'_k, x''_k \in \Delta x_k ~~~ |x'_k - x'_k| < \delta_\varepsilon\]
  \[\sum_{k = 1}^{n} (f(x'_k) - f(x''_k)) \Delta x_k <
    \frac{\varepsilon}{b - a} \cdot \sum_{k = 1}^{n} \Delta x_k <
    \varepsilon\]
\end{proof}

\begin{theorem}[2]
  Если функция $f(x)$ \underline{ограничена на отрезке $[a; b]$} и непрерывна
  на нем, \underline{за исключением конечного числа точек}, то функция
  интегрируема (по Риману) на отрезке $[a; b]$
\end{theorem}

\begin{title}
  Интегрируемость монотонных функций
\end{title}

\begin{theorem}
  Если функция $f(x)$ - монотонна на отрезке $[a; b]$, то она интегрируема по
  Риману на отрезке $[a; b]$
\end{theorem}

\begin{proof}
  Если функция $f(x)$ монотонно возрастает
  \[\forall x \in [a; b] ~~~ f(a) \le f(x) \le f(b)\]
  Рассмотрим \underline{произвольное} разбиение (R)
  \[\int^* (R) - \int_* (R) = \sum_{k = 1}^{n} (M_k - m_k) \Delta x_k
    = \sum_{k = 1}^{n} (f(x_k) - f(x_k)) \Delta x_k\]
  \[\sum_{k = 1}^{n} (f(x_k) - f(x_k)) \Delta x_k \le \lambda(R)
    \sum_{k = 1}^{n} (f(x_k) - (f(x_{k - 1})) = \lambda (R) (f(b) - f(a)\]
  \[\int^* (R) - \int_* (R) \le \lambda (R) (f(b) - f(a)) < \varepsilon\]
  \[\forall \varepsilon > 0 ~~~ \delta_\varepsilon = \frac{\varepsilon}{f(b) -
    f(a)} ~~~ \lambda (R) < \delta_\varepsilon\]
\end{proof}