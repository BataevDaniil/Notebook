\begin{title}
  Определение определенного интеграла. Необходимые условия существования.
\end{title}

$y = f(x)$ определена на $[a,b]$\\
$x_k \in [a,b]$ такое что $x_0 = a < x_1 < x_2 \ldots < x_k = b$ называют
\kv{разбиением} отрезка $[a,b]$ и обозначают $R[a,b] =
\{x_k | k = 1,2,3 \ldots n\}$\\
$\Delta_k = [x_{k-1}, x_k]$ тоже что и $\Delta{x_k} = x_k - x_{k-1}$\\
$\lambda(R) = max\Delta x_k$ где $1 \le k \le n$ \kv{мелкость разбиения}\\
$c_k \in \Delta_k ~~ \{c_k\} = \upsilon(R)$ \kv{выборка}\\
$\sum_{k=1}^{n} f(c_k)\Delta x_k$ \kv{интегральная сумма}\\

Для $f(x)$ на $[a,b]$ соотвецтвует разбиению R и выборке $\upsilon(R)$\\

\begin{defin}[определенного интеграла]
  Если существует число $I$ такое что для
  \begin{eqnarray*}
    \forall\epsilon>0 ~~ \exists\delta_{\epsilon}>0 ~~ \forall R[a,b] ~~~
    \lambda(R)<\delta_{\epsilon} ~~ \forall\upsilon(R) ~~~
    \left| \sum_{k=1}^{n} f(c_k)\Delta x_k - I \right| < \epsilon
  \end{eqnarray*}
  называется \kv{опрделенным интегралом} от функции $f$ на $[a,b]$ и обознают
  $$\int_{a}^{b} f(x)dx$$
\end{defin}

\begin{theorem}
  Необходимое условие существование определенного интеграла. Если для $f(x)$
  на $[a,b]$ $\exists \int_{a}^{b}$ то $f(x)$ ограничена на $[a,b]$.
\end{theorem}

\begin{proof}
  Предположим $f(x)$ ограничена в точке $a$ на первом отрезке любого разбиения
  на $[a,b]$\\
  Пусть $\epsilon = 1$ тогда \[\exists \delta_1 > 0 ~~ \forall R[a,b] ~~
  \upsilon(R) < \delta_1 ~~ \forall \upsilon(R)\]
  \[I-1 < \sum_{k=1}^{n} f(c_k)\Delta x_k < I+1\] зафиксируем $c_1, c_2, \ldots c_n$
  тогда \[I-1 < \sum_{k=1}^{n} f(c_k)\Delta x_k < I+1\]
  Если для $f(x)$ на $[a,b]$ $\exists \int_{b}^{a}$ то $f(x)$
  является интегрируемой по теореме Виета.
\end{proof}

\begin{title}[\Large]
  Суммы Дарбу и их свойста. Критерий интегрируемости (по Риману)
\end{title}

Пусть $y = f(x)$ на $[a; b] ~~~ R[a; b]$
\[k = 1, 2 ... n\]
\[M_k = sup f(x) ~~~ x \in \Delta_k\]
\[m_k = inf f(x) ~~~ x \in \Delta_k\]
Верхняя сумма Дарбу
\[\sum^{n}_{k = 1}M_k \Delta x_k = \int^* (R)\]
Нижняя сумма Дарбу
\[\sum^{n}_{k = 1}m_k \Delta x_k = \int_* (R)\]

\bd{Верхние и нижние суммы Дарбу обладают следующими свойствами:}\\
\bk{I}\\
Суммы Дарбу существуют для любой выборки разбиения.\\
\begin{proof}
  \[\int_* (R) \le \sum_{k = 1}^{n} f(c_k) \Delta x_k \le \int^* (R)\]
  \[\forall x \in \Delta x_k ~~~ m_k \le f(x) \le M_k\]
  \[
     m_k \Delta x_k \le f(x) \Delta x_k \le
     M_k \Delta x_k
  \]
  \[
     \sum_{k = 1}^{n} m_k \Delta x_k \le \sum_{k = 1}^{n} f(x)
     \Delta x_k \le \sum_{k = 1}^{n} M_k \Delta x_k
  \]
  \[
     \sum_{k = 1}^{n} m_k \Delta x_k \le \sum_{k = 1}^{n} f(c_k)
    \Delta x_k \le \sum_{k = 1}^{n} M_k \Delta x_k
  \]
\end{proof}

\bk{II}\\
\[Sup_{\upsilon (R)} \sum_{k = 1}^{n} f(c_k) \Delta x_k = \int^* (R)\]
\[Inf_{\upsilon (R)} \sum_{k = 1}^{n} f(c_k) \Delta x_k = \int_* (R)\]

\begin{proof}
  Sup
  \[
     \forall \varepsilon > 0 ~~~ \exists \upsilon_\varepsilon (R) ~~~ 0 \le
     M_k - f(c'_k) < \frac{\varepsilon}{b - a}
  \]
  \[
     0 \le M_k \Delta x_k - f(c'_k) \Delta x_k <
     \frac{\varepsilon}{b - a}\Delta x_k
  \]
  \[
     0 \le \sum_{k = 1}^{n} M_k \Delta x_k - \sum_{k = 1}^{n} f(c'_k)
     \Delta x_k < \frac{\varepsilon}{b - a} \sum_{k = 1}^{n} \Delta x_k
  \]
  \[
     0 \le \int^* (R) - \sum_{k = 1}^{n} f(c'_k) \Delta x_k <
     \frac{\varepsilon}{b - a} (b - a) = \varepsilon
  \]

  Inf
  \[
     \forall \varepsilon > 0 ~~~ \exists \upsilon_\varepsilon (R) ~~~ 0 \le
      f(c'_k) - m_k < \frac{\varepsilon}{b - a}
  \]
  \[
     0 \le f(c'_k) \Delta x_k - m_k \Delta x_k <
     \frac{\varepsilon}{b - a}\Delta x_k
  \]
  \[
     0 \le \sum_{k = 1}^{n} f(c'_k) \Delta x_k - \sum_{k = 1}^{n} m_k \Delta x_k
     < \frac{\varepsilon}{b - a} \sum_{k = 1}^{n} \Delta x_k
  \]
  \[
     0 \le \sum_{k = 1}^{n} f(c'_k) \Delta x_k - \int_* (R) <
     \frac{\varepsilon}{b - a} (b - a) = \varepsilon
  \]
\end{proof}

\bk{III}\\
Если разбиение $R_1 [a,b] \subset R_2 [a,b]$, тогда
\[\int_* (R_1) \le \int_* (R_2) \le \int^* (R_2) \le \int^* (R_1)\]

\begin{proof}
  Для доказательства достаточно расссмотреть случай, когда $R_2$ отличается от
  $R_1$ на одну точку.
  \[R_2 = R_1 \cup \{x^*\}\]
  \[m'_k = inf f(x) ~~~ x \in [x_{k - 1}, x^*]\]
  \[m''_k = inf f(x) ~~~ x \in [x^*, x_k]\]
  \[
     \int_* (R_2) - \int_* (R_1) = m'_k (x^* - x_{k-1}) + m''_k (x_k - x^*) -
     m_k (x_k - x_{k - 1}) =
  \]
  \[
     = m'_k (x^* - x_{k - 1}) + m''_k (x_k - x^*) - m_k (x_k - x^*) -
     m_k (x^* - x_{k - 1}) =
  \]
  \[
     = (m'_k - m_k)(x^* - x_{k - 1}) + (m''_k - m_k)
     (x_k - x^*)
  \]
     $(m'_k - m_k)(x^* - x_{k - 1}) \ge 0 $\\
     $(m''_k - m_k)(x_k - x^*) \ge 0$\\
\end{proof}

\bk{IV}\\
Если $R_1 \not= R_2$ - произвольные разбиения, то
\[\int_* (R_1) \le \int^* (R_2) ~~~ R_3 = R_1 \cup R_2\]

\begin{proof}
  На основании третьего свойства
  \[\int_* (R_1) \le \int_* (R_3) \le \int^* (R_3) \le \int^* (R_2)\]
\end{proof}

\bk{V}
\[inf \int_* (R) = I_*\] - Нижний интеграл Дарбу
\[sup \int^* (R) = I^*\] - Верхний интеграл Дарбу
\[\forall R ~~~ \int_* (R) \le I_* \le I^* \le \int^* (R)\]

\begin{proof}
  \[A, B \in \mathbb R ~~~ \exists c \in \mathbb R\]
  \[A < B ~~~ \forall x \in A, \forall y \in B\]
  \[x < c < y\]
  \[x \le Sup A \le Inf B \le y\]
  \[A = \left\{ \int_* (R) \right\}\]
  \[B = \left\{ \int^* (R) \right\}\]
\end{proof}

\begin{theorem}[Критерий интегрирования функции (по Риману)]
  Для того, чтобы $f(x)$ на $[a,b]$ была интегрируема на $[a,b]$ необходимо
  и достаточно, чтобы она была ограничена на $[a,b]$ и
  \[
    \forall \varepsilon > 0 ~~~ \exists \delta_\varepsilon > 0 ~~~ \forall R
    [a,b] ~~~ \lambda (R) < \delta_\varepsilon ~~~ 0 \le \int^* (R) -
    \int_* (R) < \varepsilon
  \]

  \kv{Замечание к критерию}\\
  Для того, чтобы функция $f(x)$ на $[a,b]$ была интегрируема (по Риману)
  на $[a,b]$. Необходимо и достаточно, чтобы она была ограничена на $[a,b]$ и
  \[
    \forall \varepsilon > 0 ~~~ \exists R_\varepsilon [a,b] ~~~ 0 \le \int^*
    (R_\varepsilon) - \int_* (R_\varepsilon) < \varepsilon
  \]
\end{theorem}

\begin{title}[\Large]
  Интегрируемость непрерывных функций
\end{title}

\begin{theorem}
  $f(x)$ непрерывна на $[a,b]$, значит инегрируема на $[a,b]$.
\end{theorem}

\begin{proof}
  По теореме Кантора, функция непрерывая на отрезке, является равномерно
  непрерывной на этом отрезке.
  \[
    \forall \varepsilon > 0 ~~~ \exists \delta_\varepsilon > 0 ~~~ \forall
    x', x'' \in [a,b] ~~~ |x' - x''| < \delta_\varepsilon ~~~
    |f(x') - f(x'')| < \frac{\varepsilon}{b - a}
  \]
  \[R [a,b] ~~~ \lambda (R) < \delta_\varepsilon\]
  \[
    0 \le \int^* (R) - \int_* (R) = \sum_{k = 1}^{n} (M_k - m_k)
    \Delta x_k = \sum_{k = 1}^{n} (f(x'_k) - f(x''_k)) \Delta x_k
  \]
  $f(x'_k) = M_k ~~~ f(x''_k) = m_k$
  По теореме Вейерштрасса
  \[x'_k, x''_k \in \Delta x_k ~~~ |x'_k - x'_k| < \delta_\varepsilon\]
  \[
    0 \le \sum_{k = 1}^{n} (f(x'_k) - f(x''_k)) \Delta x_k <
    \frac{\varepsilon}{b - a} \cdot \sum_{k = 1}^{n} \Delta x_k = \varepsilon
  \]
\end{proof}

\begin{theorem}
  $f(x)$ ограничена на $[a,b]$ и непрерывна на $[a,b]$ за исключением конечного
  числа точек, значит $f(x)$ интегрируема (по Риману) на $[a,b]$.
\end{theorem}

\begin{title}[\Large]
  Интегрируемость монотонных функций.
\end{title}

\begin{theorem}
  $f(x)$ монотонна на $[a,b]$, значит $f(x)$ интегрируема (по Риману) на $[a,b]$
\end{theorem}

\begin{proof}
  \[\forall x \in [a,b] ~~~ f(a) \le f(x) \le f(b) ~~~ \forall R[a,b]\]
  \[
    \int^* (R) - \int_* (R) = \sum_{k = 1}^{n} (M_k - m_k) \Delta x_k
    = \sum_{k = 1}^{n} (f(x_k) - f(x_k)) \Delta x_k
  \]
  \[
    \sum_{k = 1}^{n} (f(x_k) - f(x_k)) \Delta x_k \le \lambda(R) \cdot
    \sum_{k = 1}^{n} (f(x_k) - (f(x_{k - 1})) = \lambda (R) (f(b) - f(a))
  \]
  \[
    \forall \varepsilon > 0 ~~~ \delta_\varepsilon = \frac{\varepsilon}{f(b) -
    f(a)} ~~~ \lambda (R) < \delta_\varepsilon
  \]
  \[\int^* (R) - \int_* (R) \le \lambda (R) (f(b) - f(a)) < \varepsilon\]
\end{proof}