\begin{title}
  Определение определенного интеграла. Необходимые условия существования.
\end{title}

$y = f(x)$ определена на $[a,b]$\\
$x_k \in [a,b]$ такое что $x_0 = a < x_1 < x_2 \ldots < x_k = b$ называют
\kv{разбиением} отрезка $[a,b]$ и обозначают $R [a,b] =
\{x_k | k = 1,2,3 \ldots n\}$\\
$\Delta_k = [x_{k-1}, x_k]$ тоже что и $\Delta{x_k} = x_k - x_{k-1}$\\
$\lambda(R) = max\Delta x_k$ где $1 \le k \le n$ \kv{мелкость разбиения}\\
$\forall c_k \in \Delta_k ~~~ \{c_k\} = \upsilon(R)$ \kv{выборка}\\
$\sum_{k=1}^{n} f(c_k)\Delta x_k$ \kv{интегральная сумма}\\

Для $f(x)$ на $[a,b]$ соотвецтвует разбиению R и выборке $\upsilon(R)$\\

\begin{defin}[определенного интеграла]
  Число $I$ называют \kv{опрделенным интегралом} от $f(x)$ на $[a,b]$ если
  улидотворяет следующему условию
  \begin{eqnarray*}
    \forall\varepsilon>0 ~~ \exists\delta_{\varepsilon}>0 ~~ \forall R[a,b] ~~~
    \lambda(R)<\delta_{\varepsilon} ~~ \forall\upsilon(R) ~~~
    \left| \sum_{k=1}^{n} f(c_k)\Delta x_k - I \right| < \varepsilon
  \end{eqnarray*}
   и обозначают
  $$\int_{a}^{b} f(x)dx$$

  Если такой интеграл существует, то его называют \kv{интегрируемым на $[a,b]$
  по Риману}.
\end{defin}

\begin{theorem}
  Необходимое условие существование определенного интеграла. Если для $f(x)$
  на $[a,b]$ $\exists \int_{a}^{b}$ то $\forall x \in [a,b] f(x)$ ограничена.
\end{theorem}

\begin{proof}
  Предположим $f(x)$ неограничена $\forall O(a)$ на первом
  отрезке $\forall R[a,b]$. Так как определенный интеграл существует
  то

  пусть $\varepsilon = 1$ тогда \[\exists \delta_1 > 0 ~~ \forall R[a,b] ~~
  \lambda(R) < \delta_1 ~~ \forall \upsilon(R)\]
  \[I-1 < \sum_{k=1}^{n} f(c_k)\Delta x_k < I+1\] зафиксируем элемент выборки
  $$
  I - 1 - \sum_{k=2}^n f(c_k) \Delta x_k < f(c_1) \Delta x_1 <
  I + 1 - \sum_{k=2}^n f(c_k) \Delta x_k
  $$
  $c_1$ это любая точка первого отрезка $\Delta x_1 \not= 0$
  $$
  \frac{I-1 - \sum_{k=2}^n f(c_k) \Delta x_k}{\Delta x_1} <
  f(c_1) < \frac{I+1 - \sum_{k=2}^n f(c_k) \Delta x_k}{\Delta x_1}
  $$
  $f(c_1)$ ограничена, а это противоречие.
\end{proof}

\begin{title}[\Large]
  Суммы Дарбу и их свойста. Критерий интегрируемости (по Риману)
\end{title}

Пусть $y = f(x)$ на $[a, b] ~~~ R[a, b]$
\[M_k = sup f(x) ~~~ x \in \Delta_k ~~~ k = 1, 2 \ldots n\]
\[m_k = inf f(x) ~~~ x \in \Delta_k ~~~ k = 1, 2 \ldots n\]
\kv{Верхняя сумма Дарбу}
\[\sum^{n}_{k = 1}M_k \Delta x_k = \int^* (R)\]
\kv{Нижняя сумма Дарбу}
\[\sum^{n}_{k = 1}m_k \Delta x_k = \int_* (R)\]

\bd{Свойства:}

\bk{I}\\
$$
\forall \upsilon(R) ~~~\int_* (R) \le \sum_{k = 1}^{n} f(c_k) \Delta x_k \le
\int^* (R)
$$
\begin{proof}
  \[\forall x \in \Delta x_k ~~~ m_k \le f(x) \le M_k\]
  \[
     m_k \Delta x_k \le f(x) \Delta x_k \le
     M_k \Delta x_k
  \]
  \[
     \sum_{k = 1}^{n} m_k \Delta x_k \le \sum_{k = 1}^{n} f(x)
     \Delta x_k \le \sum_{k = 1}^{n} M_k \Delta x_k
  \]
  \[
     \sum_{k = 1}^{n} m_k \Delta x_k \le \sum_{k = 1}^{n} f(c_k)
    \Delta x_k \le \sum_{k = 1}^{n} M_k \Delta x_k
  \]
\end{proof}

\bk{II}\\
\[Sup_{\upsilon (R)} \sum_{k = 1}^{n} f(c_k) \Delta x_k = \int^* (R)\]
\[Inf_{\upsilon (R)} \sum_{k = 1}^{n} f(c_k) \Delta x_k = \int_* (R)\]

\begin{proof}
  Sup
  \[
     \forall \varepsilon > 0 ~~~ \exists \upsilon_\varepsilon (R) ~~~ 0 \le
     M_k - f(c'_k) < \frac{\varepsilon}{b - a}
  \]
  \[
     0 \le M_k \Delta x_k - f(c'_k) \Delta x_k <
     \frac{\varepsilon}{b - a}\Delta x_k
  \]
  \[
     0 \le \sum_{k = 1}^{n} M_k \Delta x_k - \sum_{k = 1}^{n} f(c'_k)
     \Delta x_k < \frac{\varepsilon}{b - a} \sum_{k = 1}^{n} \Delta x_k
  \]
  \[
     0 \le \int^* (R) - \sum_{k = 1}^{n} f(c'_k) \Delta x_k <
     \frac{\varepsilon}{b - a} (b - a) = \varepsilon
  \]

  Inf
  \[
     \forall \varepsilon > 0 ~~~ \exists \upsilon_\varepsilon (R) ~~~ 0 \le
      f(c'_k) - m_k < \frac{\varepsilon}{b - a}
  \]
  \[
     0 \le f(c'_k) \Delta x_k - m_k \Delta x_k <
     \frac{\varepsilon}{b - a}\Delta x_k
  \]
  \[
     0 \le \sum_{k = 1}^{n} f(c'_k) \Delta x_k - \sum_{k = 1}^{n} m_k \Delta x_k
     < \frac{\varepsilon}{b - a} \sum_{k = 1}^{n} \Delta x_k
  \]
  \[
     0 \le \sum_{k = 1}^{n} f(c'_k) \Delta x_k - \int_* (R) <
     \frac{\varepsilon}{b - a} (b - a) = \varepsilon
  \]
\end{proof}

\bk{III}\\
Если разбиение $R_1 [a,b] \subset R_2 [a,b]$, тогда
\[\int_* (R_1) \le \int_* (R_2) \le \int^* (R_2) \le \int^* (R_1)\]

\begin{proof}
  Для доказательства достаточно расссмотреть случай, когда $R_2$ отличается от
  $R_1$ на одну точку.
  \[R_2 = R_1 \cup \{x^*\}\]
  \[m'_k = inf f(x) ~~~ x \in [x_{k - 1}, x^*]\]
  \[m''_k = inf f(x) ~~~ x \in [x^*, x_k]\]
  \[
     \int_* (R_2) - \int_* (R_1) = m'_k (x^* - x_{k-1}) + m''_k (x_k - x^*) -
     m_k (x_k - x_{k - 1}) =
  \]
  \[
     = m'_k (x^* - x_{k - 1}) + m''_k (x_k - x^*) - m_k (x_k - x^*) -
     m_k (x^* - x_{k - 1}) =
  \]
  \[
     = (m'_k - m_k)(x^* - x_{k - 1}) + (m''_k - m_k)
     (x_k - x^*)
  \]
  Так как $m'_k - m_k \ge 0$ то $m'_k \ge m_k$

  Так как $m''_k - m_k \ge 0$ то $m''_k \ge m_k$

  $$
  R_2 = R_1 \cup \{x^*\}
  $$
  $$
  M'_k = sup f(x) ~~~ x \in [x_{k-1}, x^*]
  $$
  $$
  M''_k = sup f(x) ~~~ x \in [x^*, x_k]
  $$
  $$
  \int^* (R_1) - \int^*(R_k) = M_k(x_k - x_{k-1}) - M_k'(x^* - x_{k-1})
  - M''_k(x_k - x^*) =
  $$
  $$
  = M_k(x^* - x{k-1}) + M_k(x_k - x^*) - M'_k(x^* - x{k-1}) - M''_k(x_k - x^*) =
  $$
  $$
  (M_k - M'_k)(x^* - x_{k-1}) + (M_k - M''_k)(x_k - x^*)
  $$
  Так как $M_k - M'_k \ge 0$ то $M \ge M'_k$

  Так как $M_k - M''_k \ge 0$ то $M \ge M''_k$
\end{proof}

\bk{IV}\\
Если $R_1 \not= R_2$ - произвольные разбиения на $[a,b]$, то
\[\int_* (R_1) \le \int^* (R_2) ~~~ \]

\begin{proof}
  На основании третьего свойства, расмотрим $R_3 = R_1 \cup R_2$
  \[\int_* (R_1) \le \int_* (R_3) \le \int^* (R_3) \le \int^* (R_1)\]
  \[\int_* (R_2) \le \int_* (R_3) \le \int^* (R_3) \le \int^* (R_2)\]
  \[\int_* (R_1) \le \int_* (R_3) \le \int^* (R_3) \le \int^* (R_2)\]
\end{proof}

\bk{V}
$$
inf_R \int_* (R) = I_* ~~~ sup_R \int^* (R) = I^*
$$
\[\forall R ~~~ \int_* (R) \le I_* \le I^* \le \int^* (R)\]

\begin{proof}
  \[
  A, B \in \mathbb R ~~~ A \subset B ~~~ \forall x \in A, \forall y \in B
  \]
  \[\exists c \in \mathbb R ~~~ x < c < y\]
  \[x \le Sup A \le Inf B \le y\]
  \[A = \left\{ \int_* (R) \right\} ~~~ B = \left\{ \int^* (R) \right\}\]
\end{proof}

\begin{theorem}[Критерий интегрирования функции (по Риману)]
  Для того, чтобы $f(x)$ на $[a,b]$ была интегрируема на $[a,b]$ необходимо
  и достаточно, чтобы:
  \[
    \forall \varepsilon > 0 ~~~ \exists \delta_\varepsilon > 0 ~~~ \forall R
    [a,b] ~~~ \lambda (R) < \delta_\varepsilon ~~~ 0 \le \int^* (R) -
    \int_* (R) < \varepsilon
  \]
\end{theorem}

\begin{title}[\Large]
  Интегрируемость непрерывных функций
\end{title}

\begin{theorem}
  $f(x)$ непрерывна на $[a,b]$, значит инегрируема на $[a,b]$.
\end{theorem}

\begin{proof}
  По теореме Кантора, функция непрерывая на отрезке, является равномерно
  непрерывной на этом отрезке.
  \[
    \forall \varepsilon > 0 ~~~ \exists \delta_\varepsilon > 0 ~~~ \forall
    x', x'' \in [a,b] ~~~ |x' - x''| < \delta_\varepsilon ~~~
    |f(x') - f(x'')| < \frac{\varepsilon}{b - a}
  \]
  \[R [a,b] ~~~ \lambda (R) < \delta_\varepsilon\]
  \[
    0 \le \int^* (R) - \int_* (R) = \sum_{k = 1}^{n} (M_k - m_k)
    \Delta x_k = \sum_{k = 1}^{n} (f(x'_k) - f(x''_k)) \Delta x_k
  \]
  $f(x'_k) = M_k ~~~ f(x''_k) = m_k$
  По теореме Вейерштрасса
  \[x'_k, x''_k \in \Delta x_k ~~~ |x'_k - x'_k| < \delta_\varepsilon\]
  \[
    0 \le \sum_{k = 1}^{n} (f(x'_k) - f(x''_k)) \Delta x_k <
    \frac{\varepsilon}{b - a} \cdot \sum_{k = 1}^{n} \Delta x_k = \varepsilon
  \]
\end{proof}

\begin{theorem}
  $f(x)$ ограничена на $[a,b]$ и непрерывна на $[a,b]$ за исключением конечного
  числа точек, значит $f(x)$ интегрируема (по Риману) на $[a,b]$.
\end{theorem}

\begin{title}[\Large]
  Интегрируемость монотонных функций.
\end{title}

\begin{theorem}
  $f(x)$ монотонна на $[a,b]$, значит $f(x)$ интегрируема по Риману на $[a,b]$
\end{theorem}

\begin{proof}
  \[\forall x \in [a,b] ~~~ f(a) \le f(x) \le f(b) ~~~ \forall R[a,b]\]
  \[
    \int^* (R) - \int_* (R) = \sum_{k = 1}^{n} (M_k - m_k) \Delta x_k
    = \sum_{k = 1}^{n} (f(x_k) - f(x_{k-1})) \Delta x_k
  \]
  \[
    \sum_{k = 1}^{n} (f(x_k) - f(x_{k-1})) \Delta x_k \le \lambda(R) \cdot
    \sum_{k = 1}^{n} (f(x_k) - (f(x_{k - 1})) = \lambda (R) (f(b) - f(a))
  \]
  \[
    \forall \varepsilon > 0 ~~~ \delta_\varepsilon = \frac{\varepsilon}{f(b) -
    f(a)} ~~~ \lambda (R) < \delta_\varepsilon
  \]
  \[
  \int^* (R) - \int_* (R) \le \lambda (R) (f(b) - f(a)) <
  \frac{f(b)-f(a)}{f(b)-f(a)} \varepsilon = \varepsilon
  \]
\end{proof}
\begin{title}[\Large]
    Основные свойства определенных интегралов.
\end{title}

\bd{I}\\
$f(x), g(x)$ интегрируемы на $[a,b] ~~ \alpha , \beta \in R ~~
\alpha \cdot \beta \not= 0 ~~ \alpha f(x) + \beta g(x)$ интегрируема на $[a,b]$
\[
    \int_a^b (\alpha f(x) + \beta g(x))dx = \alpha \int_a^b f(x)dx
    + \beta \int_a^b f(x)dx
\]

\begin{proof}
    \[
        \forall\varepsilon>0 ~~ \exists\delta_{\varepsilon}>0 ~~ \forall R[a,b]
        ~~~ \lambda(R) < \delta_{\varepsilon} ~~ \forall\upsilon(R) ~~~
    \]
    \[
        \left| \sum_{k=1}^{n} f(c_k)\Delta x_k -
        \int_a^b f(x)dx \right| < \frac{\varepsilon}{2|\alpha|}
    \]
    \[
        \left| \sum_{k=1}^{n} g(c_k)\Delta x_k -
        \int_a^b g(x)dx \right| < \frac{\varepsilon}{2|\beta|}
    \]
    \[
        \left| \sum_{k=1}^{n} (\alpha f(c_k) + \beta g(c_k))\Delta x_k \right| -
        \left| \alpha \int_a^b f(x)dx + \beta \int_a^b g(x)dx \right| \le
    \]
    \[
      \le \left| \sum_{k=1}^{n} (\alpha f(c_k) \Delta x_k) -
      \alpha \int_a^b f(x)dx \right| + \left| \sum_{k=1}^{n} (\beta g(c_k)
      \Delta x_k) - \beta \int_a^b g(x)dx \right| =
    \]
    \[
      = |\alpha| \left| \sum_{k=1}^{n} (f(c_k) \Delta x_k -
      \int_a^b f(x)dx \right| + |\beta| \left| \sum_{k=1}^{n} g(c_k)
      \Delta x_k - \int_a^b g(x)dx \right| <
    \]
    \[
        < |\alpha| \frac{\varepsilon}{2|\alpha|} + |\beta|
        \frac{\varepsilon}{2|\beta|} = \varepsilon
    \]
    \[\alpha \int_a^b f(x)dx + \beta \int_a^b f(x)dx\]
\end{proof}

\bd{II}\\
$f(x),g(x)$ интегрируема на $[a,b]$, значит $f(x) \cdot g(x)$
интегрируема на $[a,b]$.\\
\bd{III}\\
 $f(x)$ интегрируема на $[a,b] ~~~ [c,d] \subset [a,b]$, значит
$f(x)$ интегрируема на $[c,d]$.\\
\bd{IV}\\
$f(x)$ интегрируема на $[a,b]$ $a < c < b$, значит
    \[\int_a^b f(x)dx = \int_a^c f(x)dx + \int_c^b f(x)dx\]

\bd{V}
\[
    \int_a^a f(x)dx = 0
\]
\bd{VI}
\[
    \int_a^b f(x)dx = - \int_b^a f(x)dx
\]

\bd{VII}\\
$f(x)$ интегрируема на $[a, b]$ $\forall c_1, c_2, c_3 \in [a, b]$
    \[
    \int_{c_1}^{c_3} f(x)dx = \int_{c_1}^{c_2} f(x)dx + \int_{c_2}^{c_3} f(x)dx
    \]

\begin{proof}
  $c_3 < c_1 < c_2$
    \[
        \int_{c_3}^{c_2} f(x)dx = \int_{c_3}^{c_1}f(x)dx +
        \int_{c_1}^{c_2}f(x)dx
    \]
    \[
        \int_{c_1}^{c_3} f(x)dx = \int_{c_1}^{c_2}f(x)dx +
        \int_{c_2}^{c_3}f(x)dx
    \]
\end{proof}

\begin{title}[\Large]
    Оценки определенного интеграла.
\end{title}
\bd{I}\\
$f(x)$ интегрируема на $[a,b]$ $f(x) \ge 0$, значит
\[\int_a^b f(x)dx \ge 0\]
\begin{proof}
    \[\sum_{k=1}^{n} f(c_k)\Delta x_k \ge 0\]
\end{proof}
\bd{II}\\
    $f(x), g(x)$ интегрируемы на $[a, b]$ $\forall x \in [a, b] ~~ f(x)\le g(x)$
    \[\int_a^b f(x)dx \le \int_a^b g(x)dx\]
\begin{proof}
    \[g(x) - f(x) \ge 0 ~~ \int_a^b (g(x) - f(x))dx \ge 0\]
    \[\int_a^b g(x)dx - \int_a^b f(x)dx \ge 0\]
    \[\int_a^b g(x)dx \ge \int_a^b f(x)dx \]
\end{proof}

\bd{III}\\
$f(x) \ge 0$ интегрируема на $[a, b]$ $\exists c \in [a, b]$ непрерывна
в $f(c) > 0$ \[\int_a^b f(x)dx > 0\]
\begin{proof}
    Локальные свойство непрерывности в точке $\exists O_{\delta}c ~~
    \forall x \in O_{\delta}c ~~ f(x) > \frac{f(c)}{2}$
    \[
        \int_a^b f(x)dx = \int_a^{c-\delta} f(x)dx +
        \int_{c-\delta}^{c+\delta} f(x)dx + \int_{c+\delta}^b f(x)dx >
        0 + \int_{c-\delta}^{c+\delta} \frac{f(c)}{2}dx
        = \frac{f(c) 2\delta}{2} > 0
    \]
\end{proof}
\bd{IV}\\
$f(x)$ интегрируем на $[a, b]$ то $|f(x)|$ также интегрируема
(но не наоборот)

\begin{proof}
  \[\left| \int_a^b f(x)dx \right| \le \int_a^b |f(x)|dx\]
  \[||a| - |b|| < |a - b|\]
  \[
      \left| \sum_{k=1}^n f(c_k)\Delta x_k \right| \le
      \sum_{k=1}^{n}|f(c_k)|\Delta x_k
  \]
\end{proof}
\bd{V}\\
$f(x)$ интегрируема $[a, b]$ $\forall c_1, c_2 \in [a, b]$
\[
    \left| \int_{c_1}^{c_2} f(x)dx \right| \le
    \left| \int_{c_1}^{c_2} |f(x)|dx \right|
\]
\bd{VI}\\
$f(x), g(x)$ интегрируемы на $[a, b]$ $\forall x \in [a, b] ~~
m \le f(x) \le M$ $g(x) \ge 0$ или $g(x) \le 0$ $\exists m \le \varphi \le M$
\[\int_a^b f(x)g(x)dx = \varphi \int_a^b g(x)dx\]
\begin{proof}
    $g(x) \le 0 ~~ m \le f(x) \le M ~~ mg(x) \ge f(x)g(x) \ge Mg(x)$
    \[\int_a^b mg(x)dx \ge \int_a^b f(x)g(x)dx \ge \int_a^b Mg(x)dx\]
    \[m\int_a^b g(x)dx \ge \int_a^b f(x)g(x)dx \ge M\int_a^b g(x)dx\]
    \[\int_a^b g(x)dx = 0 ~~~ \int_a^b f(x)g(x)dx = 0\]
    \[
    \int_a^b g(x)dx < 0 ~~~
    m \le \frac{\int_a^b f(x)g(x)dx}{\int_a^b g(x)dx} \le M
    \]
\end{proof}

\begin{theorem}[о среднем для определенного интеграла]
  $f(x)$ интегрируем и непрерывна на $[a,b]$

  $g(x)$ интегрируема и не меняет знак тогда $\exists c \in [a,b]$
  \[\int_a^b f(x)g(x)dx  = f(c) \int_a^b g(x)dx\]
\end{theorem}

\begin{proof}
  $m = inf f(x) ~~~ x \in [a,b]$

  $M = sup f(x) ~~~ x \in [a,b]$

  $\forall x [a,b] ~~~ m \le f(x) \le M$

  По свойству функции непрерывной на отрезке $f([a,b]) = [m, M]$

  На основании о среднем для определенного интеграла $\exists m \le \nu \le M$
  $$
  \int_a^b f(x)g(x) dx = \nu \int_a^b g(x) dx
  $$

  По теореме Коши $\forall \nu \in [m,M] ~~~ \exists c \in [a,b] ~~~ f(c) = \nu$
\end{proof}

\begin{title}
  Свойства определенных интегралов как функции верхнего предела.
\end{title}

$f(x)$ определена на $[a, b]$
\[F(x) = \int^x_a f(t)dt\]

\begin{theorem}
  $f(x)$ определена на $[a,b]$ то $F(x)$ непрерывна на $[a,b]$
  \[F(x) = \int^x_a f(t)dt\]
\end{theorem}

\begin{proof}
  \[\exists M > 0 ~~~ \forall x \in [a, b] ~~~ |f(x)| \le M\]
  Пусть $x, x + \Delta x \in [a, b]$
  \[\Delta F(x) < F(x + \Delta x) - F(x) = \int^{x + \Delta x}_a f(t)dt -
    \int^x_a f(t)dt = \int^{x + \Delta x}_x f(t)dt\]
  \[|\Delta F(x)| \le \left |\int^{x + \Delta x}_x |f(t)|dt \right| \le
    M \left |\int^{x + \Delta x}_x |f(t)| dt \right| = M |\Delta x|
    ~~~ \Delta x \to 0 ~~~ |\Delta F(x)| \to 0\]
  Так как $x$ произвольная точка, то $F(x)$ непрерывна на $[a,b]$
\end{proof}

\begin{theorem}
  $f(x)$ интегрируема на $[a,b]$ и непрерывна в точке $x_0 \in [a,b]$
  \[F(x) = \int^x_a f(t)dt\]
  тогда
  \[F'(x_0) = f(x_0)\]
\end{theorem}

\begin{proof}
  Пусть $x_0 + \Delta x \in [a, b]$
  \[\frac{\Delta F(x_0)}{\Delta x} - f(x_0) = \frac{1}{\Delta x}
    \int^{x_0 + \Delta x}_{x_0} f(t)dt - \frac{f(x_0)}{\Delta x}
    \int^{x_0 + \Delta x}_{x_0} dt = \frac{1}{\Delta x}
    \int^{x_0 + \Delta x}_{x_0} (f(t) - f(x_0))dt\]
  \[\left | \frac{\Delta F(x_0)}{\Delta x} - f(x_0)
    \right | \le \frac{1}{|\Delta x|} \left |
    \int^{x_0 + \Delta x}_{x_0} |f(x) - f(x_0)|dt \right |\]
  \[\forall \varepsilon > 0 ~~~ \exists \delta_\varepsilon > 0
    ~~~ \forall \in [a, b] ~~~ |t - x_0| < \delta_\varepsilon \Rightarrow
    |f(t) - f(x_0) < \varepsilon|\]
  Если $|\Delta x| < \delta_\varepsilon$
  \[\left | \frac{\Delta F(x_0)}{\Delta x} - f(x_0) \right | \le
    \frac{1}{|\Delta x|} \left | \int^{x_0 + \Delta x}_{x_0} |f(x) -
    f(x_0)|dt \right | \le \frac{\varepsilon}{|\Delta x|} |\Delta x| =
    \varepsilon\]
  \[\lim_{\Delta x \to 0} \frac{\Delta F(x_0)}{\Delta x} = f(x_0)\]
  \[F'(x_0) = f(x_0)\]

  Следствие\\
  Если $F(x)$ непрерывна на отрезке $[a, b]$, то
  \[F(x) = \int^x_a f(t)dt\] - дифференцирована на всем отрезке и
  \[F'(x) = f(x) ~~~ x \in [a, b]\]
\end{proof}

\begin{title}[\Large]
  Вычисление определенных интегралов. Формула Ньютона Леменца.
\end{title}

\[F(x) = \int^x_a f(t)dt\]
$F'(x) = f(x)$, если $f(x)$ - непрерывна на отрезке $[a, b]$ Любая первообразная
от $f(x) [a, b]$
\[\phi (x) = \int^x_a f(t)dt + C\]
\[x = a ~~~ \phi (a) = \int^a_a f(t)dt + C = C\]
\[x = b ~~~ \phi (b) = \int^b_a f(t)dt + C = \int^b_a f(t)dt + \phi(x)\]
\[\int^b_a f(t)dt = \phi(b) - \phi(a) = \phi(x)|^b_a\]

\begin{title}[\Large]
  Интегрирование по частям в определенных интегралах.
\end{title}

\begin{theorem}
  Пусть $\upsilon = \upsilon (x), \nu = \nu(x)$ на $[a,b]$ непрерывно
  диффиренцируемы на $[a,b]$
  \[
    \int_a^b \upsilon d\nu = \upsilon \nu |_a^b - \int_a^b \nu d\upsilon
  \]
\end{theorem}

\begin{title}[\Large]
  Замена переменных (подстановка) в определенных интегралах.
\end{title}

\begin{theorem}
  Пусть $y = f(x)$ - непрерывна на $\Delta$, $x = \varphi (t)$ - непрерывна и
  диффиренцируема на $T$. При этом $\varphi (T) \subset \Delta$.\\
  Если
  \[a, b \in \Delta ~~~ a = \varphi (\alpha) ~~~ b = \varphi (\beta)\]
  \[\int^b_a f(x)dx = \int^{\beta}_{\alpha} f(\varphi (t)) \varphi' (t)dt\]
\end{theorem}

\begin{proof}
  Пусть $F(x)$ - первообразная $f(x)$ на отрезке $[a, b]$\\
  \[
    (F(\varphi (t)))' = F'(\varphi (t)) \cdot \varphi' (t) = f(\varphi(t))
    \cdot \varphi'(t)
  \]
  Тогда по формуле Ньютона-Лейбница:
  \[\int^b_a f(x)dx = F(b) - F(a)\]
  \[
    \int^{\beta}_{\alpha} f(\varphi (t)) \cdot \varphi' (t)dt =
    F(\varphi (\beta)) - F(\varphi (\alpha)) = F(b) - F(a)
  \]
\end{proof}

\begin{theorem}
  Если $f(x)$ - нечетная на отрезке $[-a, a]$, то
  \[\int^a_{-a} f(x)dx = 0\]
\end{theorem}

\begin{proof}
  Так как функция нечетная, то верно $f(-x) = -f(x)$
  \[\int^a_{-a} f(x)dx = \int^0_{-a} f(x)dx + \int^a_0 f(x)dx =
    |x = -t ~~ dx = -dt| = -\int^0_a f(-t)dt + \int^a_0 f(x)dx =\]
    \[= -\int^a_0 f(t)dt + \int^a_0 f(x)dx = 0\]
\end{proof}

\begin{theorem}
  Если $f(x)$ - четная на отрезке $[-a, a]$, то
  \[\int^a_{-a} f(x)dx = 2\int^a_{0} f(x)dx \]
\end{theorem}

\begin{proof}
  Так как функция четная, то верно $f(-x) = f(x)$
  \[\int^a_{-a} f(x)dx = \int^0_{-a} f(x)dx + \int^a_0 f(x)dx =
    |x = -t ~~ dx = -dt| = -\int^0_a f(-t)dt + \int^a_0 f(x)dx =\]
    \[= \int^a_0 f(t)dt + \int^a_0 f(x)dx = 2\int^a_{0} f(x)dx\]
\end{proof}

\begin{theorem}
  Если $f(x)$ - переодична на отрезке $[-a, a]$, то
  \[\int^a_{-a} f(x)dx = 0\]
\end{theorem}

\begin{proof}
  Так как функция переодична, то верно $T>0$ $f(x) = f(x+T)$
  \[\int^a_{-a} f(x)dx = \int^0_{-a} f(x)dx + \int^a_0 f(x)dx =
    |x = t+T ~~ dx = dt| = \int^0_a f(t+T)dt + \int^a_0 f(x)dx =\]
    \[= -\int^a_0 f(t)dt + \int^a_0 f(x)dx = 0\]
\end{proof}