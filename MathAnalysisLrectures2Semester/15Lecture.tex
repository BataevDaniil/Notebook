\begin{title}[\Large]
  Вычисление объемов с помощью определенного интеграла.
\end{title}

\begin{defin}[цилиндра]
  $$
  \vartheta = \{ (x,y,z) | (x,y) \in D ~~~ 0 < z \le h\}
  $$
\end{defin}

\begin{theorem}
  $D \subset R^2 ~~~ S(D)$ измеримо по Жордану $\vartheta \subset R^3 ~~~
  V(\vartheta) = h S(D)$
\end{theorem}

\begin{proof}
  $$
  \forall \varepsilon > 0 ~~~
  \exists P_{\varepsilon}, Q_{\varepsilon} \subset R^3 ~~~
  P_{\varepsilon}, \subset D \subset Q_{\varepsilon} ~~~
  S(Q_{\varepsilon} - S(P_{\varepsilon})) < \frac{\varepsilon}{h}
  $$
  $$
  \vartheta'_{\varepsilon} = P_{\varepsilon} \times [0,h] ~~~
  \vartheta''_{\varepsilon} = Q_{\varepsilon} \times [0,h] ~~~
  \vartheta'_{\varepsilon} \subset \vartheta \subset \vartheta''_{\varepsilon}
  $$
  $$
  V(\vartheta''_{\varepsilon}) - V(\vartheta'_{\varepsilon}) <
  h S(Q_{\varepsilon}) - h S(P_{\varepsilon}) =
  h (S(Q_{\varepsilon}) - S(P_{\varepsilon}))
  $$
  $$
  P_{\varepsilon} \subset D \subset Q_{\varepsilon} ~~~
  S(P_{\varepsilon}) \le S(D) \le S(Q_{\varepsilon})
  hS(P_{\varepsilon}) \le hS(D) \le hS(Q_{\varepsilon})
  V(\vartheta'_{\varepsilon}) \le h S(D) \le V(\vartheta''_{\varepsilon})
  $$
\end{proof}

\begin{theorem}
  $$
  D = \{ (x,y) ~~~ x \in [a,b] ~~~
  0 \le y \le f(x) \}
  $$
  $$
  V(T) = \pi \int_a^b f^2(x) dx
  $$
\end{theorem}

\begin{proof}
  $$
  P_k ~~ \Delta x_k ~~ m_k ~~~
  V(P_k) = \pi m_k^2 \Delta x_k
  $$
  $$
  Q_k ~~ \Delta x_k ~~ M_k ~~~
  V(Q_k) = \pi M_k^2 \Delta x_k
  $$
  $$
  P = \sqcup_{k=1}^n P_k ~~~
  Q = \sqcup_{k=1}^n Q_k
  $$
  $$
  V(P) = \sum_{k=1}^n \pi m_k^2 \Delta x_k ~~~
  V(Q) = \sum_{k=1}^n \pi M_k^2 \Delta x_k = \int^* (R)
  $$
  $$
  \forall \varepsilon > 0 ~~~
  \exists R_{\varepsilon} ([a,b]) ~~~
  0 \le \int^* (R_{\varepsilon}) - \int_* (R_{\varepsilon}) < \varepsilon
  $$
  $$
  \exists P_{\varepsilon}, Q_{\varepsilon} ~~~
  P_{\varepsilon} \subset T \subset Q_{\varepsilon} ~~~
  V(Q_{\varepsilon} - V(P_{\varepsilon}) < \varepsilon
  $$
  $$
  \int_* (R_{\varepsilon}) \le V(T) \le \int^* (R_{\varepsilon})
  $$
\end{proof}

\begin{theorem}
  Тело $T$ расположено в $R^3$ между плоскостями $x = a ~~ x = b ~~
  \forall x \in [a,b]$ $Ox$ сечение плоскости и $Oz$ и $S(x)$ измеримо по
  Жордану, $f(x)$ непрерывна на $[a,b]$ $\forall x', x'' \in [a,b]$ проэкции
  $Dx', Dx''$ но $Oyz$ вложены одна в другую, тогда $T$ измеримо по Жордану
  $V(T) = \int_a^b S(x) dx$
\end{theorem}

\begin{title}[\Large]
  Вычисление длины кривой с помощью интеграла.
\end{title}

\begin{defin}[кривой]
  $(x,y,z) \in R^3$
  $$
    \left\{
      \begin{array}{l}
        x = x(t) \\
        y = y(t) \\
        z = z(t)
      \end{array}
    \right.
  $$
  $\vec z = \vec z (t) = ( x(t), y(t) , z(t) )$
\end{defin}

