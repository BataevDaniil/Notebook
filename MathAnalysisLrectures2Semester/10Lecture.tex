\begin{theorem}
  Если $f(x)$ - четная на отрезке $[-a; a]$, то
  \[\int^a_{-a} f(x)dx = 2\int^a_{0} f(x)dx \]
\end{theorem}

\begin{proof}
  Так как функция четная, то верно $f(-x) = f(x)$
  \[\int^a_{-a} f(x)dx = \int^0_{-a} f(x)dx + \int^a_0 f(x)dx =
    |x = -t ~~ dx = -dt| = -\int^0_a f(-t)dt + \int^a_0 f(x)dx =\]
    \[= \int^a_0 f(t)dt + \int^a_0 f(x)dx = 2\int^a_{0} f(x)dx\]
\end{proof}

\begin{theorem}
  Если $f(x)$ - переодична на отрезке $[-a; a]$, то
  \[\int^a_{-a} f(x)dx = 0\]
\end{theorem}

\begin{proof}
  Так как функция переодична, то верно $T>0$ $f(x) = f(x+T)$
  \[\int^a_{-a} f(x)dx = \int^0_{-a} f(x)dx + \int^a_0 f(x)dx =
    |x = t+T ~~ dx = dt| = \int^0_a f(t+T)dt + \int^a_0 f(x)dx =\]
    \[= -\int^a_0 f(t)dt + \int^a_0 f(x)dx = 0\]
\end{proof}

\begin{title}[\Large]
    Несобсвенный интеграл по неограниченому промежутку.
\end{title}

\begin{defin}
    Пусть $f(x)$ на $[a,+\infty)$ $\forall c \in (a, +\infty)$ интегрируема
    на $[a,c] \subset [a, +\infty)$
    \[\lim_{c \to +\infty} \int_a^c f(x)dx = \int_a^{+\infty} f(x)dx\]
\end{defin}

\begin{defin}
    Пусть $f(x)$ на $(-\infty, b]$ $\forall c \in (-\infty, b)$ интегрируема
    на $[c,b] \subset (-\infty, b]$
    \[\lim_{c \to -\infty} \int_c^b f(x)dx = \int_{-\infty}^b f(x)dx\]
\end{defin}

Называют \kv{несобственным интегралом}, если он сущетсвует то
\kv{сходящийся}, иначе называют \kv{расходящимся}.

\begin{defin}
    $f(x)$ на $[a,b]$ и интегрируема $[a,c] \subset [a,b)$ неограричена в
    окрестности точки $b$
    \[\lim_{c \to b -0} \int_a^c f(x)dx = \int_a^b f(x)dx\]
\end{defin}

\begin{defin}
    $f(x)$ на $[a,b]$ и интегрируема $[c,b] \subset (a,b]$ неограричена в
    окрестности точки $a$
    \[\lim_{c \to b +0} \int_c^b f(x)dx = \int_a^b f(x)dx\]
\end{defin}

\begin{defin}
    $f(x)$ на $(-\infty, +\infty)$ кроме быть может точек
    $d_1 < d_2 < \ldots < d_m$\\
    $-\infty < a_1 < d_1 < a_2 < d_2 \ldots < a_m < d_m < a_{m+1} < +\infty$
    \[
        \int_{-\infty}^{+\infty} f(x)dx = \int_{-\infty}^{a_1}f(x)dx +
        \int_{a_1}^{d_1}f(x)dx + \int_{d_1}^{a_2}f(x)dx + \ldots +
    \]
    \[
        \int_{a_m}^{d_m}f(x)dx + \int_{d_m}^{a_{m+1}}f(x)dx +
        \int_{a_{m+1}}^{+\infty}f(x)dx
    \]
        Он сущетсвет только когда каждый интеграл из суммы интегралов существует.
\end{defin}