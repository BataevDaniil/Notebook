\begin{defin}[измерения по Жордану]
  $E \subset R^n$ называется измеренным по Жордану
  $$
  \forall \varepsilon > 0 ~~~
  \exists P_{\varepsilon}, Q_{\varepsilon} \subset R^n ~~~
  P_{\varepsilon} \subset E \subset Q_{\varepsilon} ~~~
  0 \le m(Q_{\varepsilon}) - m(P_{\varepsilon}) < \varepsilon
  $$
\end{defin}

  Если $E \subset R_n$ то его мерой Жордана называется такое число $e$
  $$
  \forall P,Q \subset R_n ~~~
  P \subset E \subset Q ~~~
  m(P) \le m(E) \le m(Q)
  $$

\begin{theorem}
  Для любого измеримого по Жорадану множества $E$ пространства $n$ мера Жордана
существует и при том единственна.
\end{theorem}

\begin{proof}
  $$
  P \subset E \subset Q ~~~
  \{ m(P): P \subset E \} ~~~
  \{ m(Q): E \subset Q \} ~~~
  m(P) \le m(Q)
  $$
  $$
  \exists c \in R ~~~
  m(P) \le c \le m(Q) ~~~
  c = m(E)
  $$
  Так как $E$ измеримо по Жордану $E \subset R^n$
  $$
  \forall \varepsilon > 0 ~~~
  \exists P_{\varepsilon},Q_{\varepsilon} \subset R^n ~~~
  P_{\varepsilon} \subset E \subset Q_{\varepsilon} ~~~
  0 \le m(Q_{\varepsilon}) - m(P_\varepsilon) < \varepsilon
  $$
  $$
  m(P_{\varepsilon}) \le m_1 (E) \le m(Q_{\varepsilon})
  $$
  $$
  m(P_{\varepsilon}) \le m_2 (E) \le m(Q_{\varepsilon})
  $$
  $$
  |m_1 (E) - m_2 (E))| \le m(Q_{\varepsilon}) - m(P_{\varepsilon}) < \varepsilon
  $$
\end{proof}

\begin{title}[\Large]
  Вычисление площади плоской фигуры заданной в декартовой системе координат.
\end{title}

\begin{defin}[криво линейной трапеции]
 $$
  D = \{ (x,y): ~~~ x \in [a,b]  ~~~ 0 \le y \le f(x) \}
 $$
\end{defin}

\begin{theorem}
  $$
  S(D) = \int_a^b f(x) dx
  $$
\end{theorem}

\begin{proof}
  $$
  \int^* (R) = \sum_{k=1}^n M_k \Delta x_k ~~~
  \int_* (R) = \sum_{k=1}^n m_k \Delta x_k
  $$
  $$
  P_n = \sqcup_{k=1}^n P_k ~~~
  Q_n = \sqcup_{k=1}^n Q_k
  $$
  $$
  m(P_n) = \sum_{k=1}^n m(P_k) = \sum_{k=1}^n m_k \Delta x_k = \int_* (R)
  $$
  $$
  m(Q_n) = \sum_{k=1}^n m(Q_k) = \sum_{k=1}^n M_k \Delta x_k = \int^* (R)
  $$
  $f(x)$ непрерывна на $[a,b]$ значит имеет интеграл по критерию интегрируемости
  $$
  \forall \varepsilon > 0 ~~~
  \exists R_{\varepsilon} ([a,b]) ~~~
  \int^*(R_{\varepsilon}) - \int_* (R_{\varepsilon}) < \varepsilon ~~~
  m(Q_n) - m(P_n) < \varepsilon ~~~
  P_n \subset D \subset Q_{\varepsilon}
  $$
  Из свойства определенного интеграла
  $$
  \int_* (R) \le \int_a^b f(x)dx \le \int^* (R)
  $$
\end{proof}

\begin{proof}
  $f(x),g(x)$ непрерывный на $[a,b]$

  $$
  D = \{ (x,y): ~~ x \in [a,b] ~~ f(x) \le y \le g(x) \}
  $$

  $c = \inf f(x) ~~~ y' = y + c ~~~ D = D_g \backslash D_f$
  $$
  S(D) = S(D_g) - S(D_f) = \int_a^b g(x)dx - \int_a^b f(x)dx =
  \int_a^b (g(x) - f(x)) dx
  $$
\end{proof}

\begin{title}[\Large]
  Вычисление площади фигуры заданной в полярных координатах.
\end{title}

\begin{theorem}
  $$
  D = \{ (\rho,\alpha): ~~ \alpha \le \phi \le \beta ~~ 0 \le
  \rho \le \beta \}
  $$
  $$
  S(D) = \frac{1}{2} = \int_{\alpha}^{\beta} \phi^2 d\alpha
  $$
\end{theorem}

\begin{proof}
  $$
  \int^* (R) = \sum_{k=1}^n M_k \Delta \phi_k
  $$
  $$
  \int_* (R) = \sum_{k=1}^n m_k \Delta \phi_k
  $$
  $$
  S(P_k) = \frac{m_k \Delta \phi_k}{2} ~~~
  S(Q_k) = \frac{M_k \Delta \phi_k}{2}
  $$
  $$
  P_n = \sqcup_{k=1}^n ~~~
  S(P_n) = \sum_{k=1}^n S(P_k) = \frac{1}{2} \sum_{k=1} m_k \Delta \phi_k =
  \int_* (R)
  $$
  $$
  Q_n = \sqcup_{k=1}^n ~~~
  S(Q_n) = \sum_{k=1}^n S(Q_k) = \frac{1}{2} \sum_{k=1} M_k \Delta \phi_k =
  \int^* (R)
  $$

  $\frac{\rho}{2}$ интегрируема на $[\alpha, \beta]$

  $$
  \int^* (R_{\varepsilon}) - \int_* (R_{\varepsilon}) < \varepsilon ~~~
  S(Q_n) - S(P_n) < \varepsilon
  $$
  $$
  \int_* (R_{\varepsilon}) \le \int_{\alpha}^{\beta} \rho d\phi \le
  \int^* (R_{\varepsilon})
  $$
\end{proof}