\begin{title}[\Large]
    Несобственный интеграл по неограниченому промежутку.
\end{title}

\kv{Несобственный интеграл} называют сходящимя если он сущетсвует,
иначе называют \kv{расходящимся}.

\begin{defin}
    Пусть $f(x)$ на $[a,+\infty)$ $\forall c \in (a, +\infty)$ интегрируема
    на $[a,c] \subset [a, +\infty)$
    \[\lim_{c \to +\infty} \int_a^c f(x)dx = \int_a^{+\infty} f(x)dx\]
\end{defin}

\begin{defin}
    Пусть $f(x)$ на $(-\infty, b]$ $\forall c \in (-\infty, b)$ интегрируема
    на $[c,b] \subset (-\infty, b]$
    \[\lim_{c \to -\infty} \int_c^b f(x)dx = \int_{-\infty}^b f(x)dx\]
\end{defin}

\begin{title}[\Large]
  Несобственный интеграл от неограниченной функции
\end{title}

\begin{defin}
    $f(x)$ на $[a,b]$ и интегрируема $[a,c] \subset [a,b)$ неограричена в
    окрестности точки $b$
    \[\lim_{c \to b -0} \int_a^c f(x)dx = \int_a^b f(x)dx\]
\end{defin}

\begin{defin}
    $f(x)$ на $[a,b]$ и интегрируема $[c,b] \subset (a,b]$ неограричена в
    окрестности точки $a$
    \[\lim_{c \to b +0} \int_c^b f(x)dx = \int_a^b f(x)dx\]
\end{defin}

\begin{title}[\Large]
  Особые точки для функции, неограниченной в их окрестности
\end{title}

\begin{defin}
    $f(x)$ на $(-\infty, +\infty)$ кроме быть может конечного числа особых точек

    $d_1 < d_2 < \ldots < d_m$\\
    $-\infty < a_1 < d_1 < a_2 < d_2 \ldots < a_m < d_m < a_{m+1} < +\infty$
    \[
        \int_{-\infty}^{+\infty} f(x)dx = \int_{-\infty}^{a_1}f(x)dx +
        \int_{a_1}^{d_1}f(x)dx + \int_{d_1}^{a_2}f(x)dx + \ldots +
    \]
    \[
        + \int_{a_m}^{d_m}f(x)dx + \int_{d_m}^{a_{m+1}}f(x)dx +
        \int_{a_{m+1}}^{+\infty}f(x)dx
    \]
        Он сущетсвет только когда каждый интеграл из суммы интегралов существует.
\end{defin}

\begin{title}[\Large]
  Основные свойства и способы вычисления несобственных интегралов.
\end{title}

\begin{theorem}
  \[
    \int^{b}_{a} f(x)dx ~~~ \int^{b}_{a} g(x)dx
  \]
  $b$ особая точка и интегралы сходятся, то $\forall \alpha, \beta
  \in R$
  \[
    \int^{b}_{a} (\alpha f(x) + \beta g(x))dx = \alpha\int^{b}_{a} f(x)dx +
    \beta\int^{b}_{a} g(x)dx
  \]
\end{theorem}

\begin{proof}
  Пусть $c \in (a, b)$
  \[
    \int^{c}_{a} (\alpha f(x) + \beta g(x))dx = \alpha\int^{c}_{a} f(x)dx +
    \beta\int^{c}_{a} g(x)dx
  \]
  \[
    \lim_{c \to b - 0} \int^{c}_{a} (\alpha f(x) + \beta g(x))dx =
    \lim_{c \to b - 0} \alpha\int^{c}_{a} f(x)dx +
    \lim_{c \to b - 0} \beta\int^{c}_{a} g(x)dx =
    \alpha\int^{b}_{a} f(x)dx + \beta\int^{b}_{a} g(x)dx
  \]
\end{proof}

\begin{theorem}
  \[
    \int^{b}_{a} f(x)dx ~~~ \int^{b}_{a} g(x)dx
  \]
  $b$ особая точка и интегралы сходятся
  $\forall x \in [a, b] ~~~ f(x) \le g(x)$ тогда
  \[
    \int^{b}_{a} f(x)dx \le \int^{b}_{a} g(x)dx
  \]
\end{theorem}

\begin{proof}
  $\forall c \in (a, b)$
  \[
    \int^{c}_{a} f(x)dx \le \int^{c}_{a} g(x)dx
  \]
  \[
    \lim_{c \to b - 0} \int^{b}_{a} f(x)dx \le \lim_{c \to b - 0}
    \int^{b}_{a} g(x)dx \Rightarrow \int^{b}_{a} f(x)dx \le \int^{b}_{a} g(x)dx
  \]
\end{proof}

\begin{theorem}
  \[
    \int^{b}_{a} f(x)dx
  \]
  $b$ особая точка и интеграл сходится

  $F(x)$ первообразная $f(x)$ на $[a, b]$ и существует
  \[
    \lim_{x \to  b - 0} F(x) = F(b - 0)
  \]
  тогда
  \[
    \int^{b}_{a} f(x)dx = \lim_{c \to b - 0} f(c) - F(a) = F(b - 0) - F(a) =
    F(x)|^{b}_{a}
  \]
\end{theorem}

\begin{proof}
  \[
    \int^{b}_{a} f(x)dx = \lim_{c \to b - 0} \int^{c}_{a} f(x)dx =
    \lim_{c \to b - 0} F(c) - F(a) = F(b - 0) - F(a)
  \]
\end{proof}

\begin{theorem}
  $f(x)$ на $[a,b)$ $x = \varphi(t)$ строго $\nearrow$ и непрерывна
  дефференциирумой на $[\alpha, \beta)$ тогда

  $$
  \varphi(\alpha) = a ~~~ \lim_{t \to \beta - 0} \varphi(t) = b
  $$
  \[
    \int^{b}_{a} f(x)dx = \int^{\beta}_{\alpha} f(\varphi(t)) \varphi'(t)dt
  \]
  При этом несобственный интеграл сходится и расходится одновременно.
\end{theorem}

\begin{proof}
  $c \in (a, b) ~~~ c = \varphi(\gamma) ~~~ \gamma \in (\alpha, \beta)$
  \[
    \int^{c}_{a} f(x)dx = \int^{\gamma}_{\alpha} f(\varphi(t)) \varphi'(t)dt
  \]
  \[
    \lim_{c \to b - 0} \int^{c}_{a} f(x)dx \Leftrightarrow
    \lim_{\gamma \to \beta - 0} \int^{\gamma}_{\alpha} f(\varphi(t))
    \varphi'(t)dt
  \]
\end{proof}

\begin{theorem}
  $u = u(x) ~~~ v = v(x)$ непрерывно дифференцируемые на $[a, b)$ и
  $b$ особая точка.
  \[
    \lim_{x \to b - 0} u(x) \cdot v(x) = u(b - 0) \cdot v(b - 0)
  \]
  тогда
  \[
    \int^{b}_{a} udv = u(b - 0) \cdot v(b - 0) - \int^{b}_{a} vdu
  \]
  При этом оба интеграла сходятся и расходятся одновременно.
\end{theorem}

\begin{title}[\Large]
  Признаки сходимости не собственных интегралов от неотрицательной функции.
\end{title}

\begin{theorem}
  $\forall x \in [a, b) ~~~ f(x) \ge 0$ $b$ особая точка и интеграл
  сходящийся
  \[
    \int^{b}_{a} f(x)dx \Leftrightarrow F(x) = \int^{x}_{a} f(t)dt
  \]
  $\exists M ~~~ |F(x)| \le M$
\end{theorem}

\begin{proof}
  Заметим что $F(x) \nearrow [a, b) ~~~ a \le x_1 < x_2 < b$, то
  \[
    F(x_2) - F(x_1) = \int^{x_2}_{x_1} f(t)dt \ge 0
  \]
  $F(b)$ ограничена в окрестности $b$

  Если $F(x)$ ограничена, то она имеет $\lim_{x \to b -0} F(x)$ по теореме о
  монотонности функций.
\end{proof}

\begin{block}[Признак сравнения для не собственного интеграла]
  $\forall x \in [a, b) ~~~ 0 \le f(x) \le g(x)$, $b$ особая точка,
  тогда\\
  \bk{I} Если интеграл $g(x)$ сходящийся, то интеграл от $f(x)$ тоже сходящийся
  \[\int^{b}_{a} g(x)dx \Rightarrow \int^{b}_{a} f(x)dx\]
  \bk{II} Если интеграл $f(x)$ расходящийся, то интеграл от $g(x)$ тоже
  рассходящийся
  \[\int^{b}_{a} f(x)dx \Rightarrow \int^{b}_{a} g(x)dx\]
\end{block}

\begin{proof}
  $\forall c \in [a, b)$

  \bk{I}
  \[
    F(c) = \int^{c}_{a} f(x)dx \le \int^{c}_{a} g(x)dx \le
    \int^{b}_{a} g(x)dx = M
  \]
  На основании первой теоремы
  \[
    \lim_{c \to b - 0} \int^{c}_{a} f(x)dx = \int^{b}_{a} f(x)dx
  \]
  \bk{II} От противного (Интеграл от $f(x)$ - расходится, а от $g(x)$ -
  сходится)
  \[
    \int^{b}_{a} f(x)dx \Rightarrow \int^{b}_{a} g(x)dx
  \]
  противоречие с первым утверждением.
\end{proof}

\begin{title}[\Large]
  Признак сравнения в предельной форме.
\end{title}

\begin{theorem}
  $f(x), g(x)$ на $[a, b)$ $b$ особая точка, $\forall x \in [a,b) ~~~
  f(x) > 0 ~~~ g(x) > 0$

  $f(x) \sim g(x) ~~~ x \to b -0$ значит
  $$
  \int_a^b f(x)dx = \int_a^b g(x)dx
  $$
\end{theorem}

\begin{proof}
  $$
  \lim_{x \to b-0} \frac{f(x)}{g(x)} = 1
  $$
  $$
  \forall \varepsilon > 0 ~~~ \exists \delta_{\varepsilon} > 0 ~~~
  b - \delta_{\varepsilon} < x < b ~~~ \left| \frac{f(x)}{g(x)} - 1 \right|
  < \varepsilon
  $$
  $$
  \varepsilon = \frac{1}{2} ~~~ \exists \delta_{\frac{1}{2}} > 0 ~~~
  b - \delta_{\frac{1}{2}} < x < b ~~~
  1 - \frac{1}{2} < \frac{f(x)}{g(x)} < 1 + \frac{1}{2}
  $$
  $$
  \frac{1}{2} < \frac{f(x)}{g(x)} < \frac{3}{2}
  $$
  $$
  \frac{1}{2} g(x) < f(x) < \frac{3}{2} g(x)
  $$
  Применим 2 теорему

  сходится
  $$
  \int_a^b g(x)dx \Rightarrow
  $$

  $$
  \frac{3}{2} \int_{b - \delta_{\frac{1}{2}}}^b g(x)dx \Rightarrow
  \int_{b - \delta_{\frac{1}{2}}}^b f(x)dx \Rightarrow
  \int_a^b f(x)dx
  $$
  сходится.

  В обратную сторону

  сходится
  $$
  \int_a^b f(x)dx \Rightarrow
  $$
  $$
  \int_{b - \delta_{\frac{1}{2}}}^b f(x)dx \Rightarrow
  \frac{1}{2} \int_{b - \delta_{\frac{1}{2}}}^b g(x)dx \Rightarrow
  \int_a^b g(x)dx
  $$
  cходится
\end{proof}

\begin{title}[\Large]
  Критерий Коши сходимости несобственного интеграла.
\end{title}

\begin{block}[Критерий]
  $$
  \forall \varepsilon > 0 ~~~ \exists d_{\varepsilon} \in [a,b) ~~~
  \forall c', c'' \in [d_{\varepsilon}, b) \left| \int_{c'}^{c''} f(x)dx \right| < \varepsilon
  $$
\end{block}

\begin{proof}
  $$
  \int_a^b f(x)dx = \lim_{c \to b-0} \int_a^c f(x)dx = \lim_{c \to b-0} F(c)
  $$
  $$
  F(x) = \int_a^x f(t)dt
  $$
  если первообразная имеет предел то интеграл сходится. Для того чтобы был
  предел
  $$
  \lim_{c \to b-0} F(c) = \int_a^b f(x)dx
  $$
  необходимо и достаточно чтобы
  $$
  \forall \varepsilon > 0 ~~~ \exists d_{\varepsilon} \in [a,b) ~~~
  \forall c', c'' \in [d_{\varepsilon}, b) ~~~
  |F(c'') - F(c')| = \left|\int_{c'}^{c''} f(x) dx \right| < \varepsilon
  $$
\end{proof}

\begin{title}[\Large]
  Признак Дирехле сходоимости несобственных интегралов.
\end{title}

\begin{block}[Признак]
  $f(x), g'(x)$ непрерывны на $[a,b)$ если

  1) $f(x)$ ограничена $F(x)$ тоисть $\forall [a,b] ~~~ |f(x)| \le M$

  2) $\forall x[a,b) ~~~ g'(x) \le 0$ или $g'(x) \ge 0$

  3) $\lim_{x \to b-0} g(x) = 0$

  тогда

  $$\int_a^b f(x)g(x)dx$$ cходится
\end{block}

\begin{proof}
  Расмотрим $c',c'' \in [a,b)$
  $$
  \int_{c'}^{c''} f(x)g(x)dx =
  \left|
    \begin{array}{ll}
      v = g(x)    & dx = g'(x)dx \\
      dv = f(x)dx & v = F(x)
    \end{array}
  \right|
  = g(x)F(x)|_{c'}^{c''} - \int_{c'}^{c''} F(x)g'(x)dx
  $$
  $$
  g(x)F(x)|_{c'}^{c''} \le M |g(c'')| + M |g(c')|
  $$
  $$
  \left| \int_{c'}^{c''} F(x)g'(x)dx \right| \le
  \left| \int_{c'}^{c''} |F(x)||g'(x)|dx \right| \le
  M \left| \int_{c'}^{c''} g'(x)dx \right| =
  $$
  для определенности $\forall x \in [a,b) ~~~ g'(x) \le 0$
  $$
  = M \left| \int_{c'}^{c''} (-g'(x))dx \right| =
  M \left| \int_{c'}^{c''} g'(x)dx \right| =
  M|g(c'') - g(c')| \le
  M|g(c'')| + |g(c')| \le
  $$
  $$
  \left| \int_{c'}^{c''} f(x)g(x) \right| \le 2M(|g(c'')| + |g(c')|)
  $$
  $$
  \forall \varepsilon > 0 ~~~ \exists d_{\varepsilon} \in [a,b) ~~~
  \forall c \in [d_{\varepsilon}, b) ~~~ |g(c)| < \frac{\varepsilon}{4M}
  $$
  $$
  \forall c', c'' \in [d_{\varepsilon}, b) ~~~
  \left| \int_{c'}^{c''} f(x)g(x)dx \right| \le
  2M \left( \frac{\varepsilon}{4M} + \frac{\varepsilon}{4M}\right) = \varepsilon
  $$
  следовательно по критерию Коши
  $$
  \int_a^b f(x)g(x)dx
  $$
  сходится
\end{proof}

\begin{title}[\Large]
  Признаки Абеля сходимости несобственного интеграла.
\end{title}

\begin{block}[Признак]
  $f(x),g'(x)$ непрерывна на $[a,b)$

  1) $$\int_a^b f(x)dx$$ сходится

  2) $\forall x \in [a,b) ~~~ g'(x) \le 0$ или $g'(x) \ge 0$

  3) $\forall x \in [a,b) ~~~ |g(x)| \le \Delta$

  тогда
  $$
  \int_a^b f(x)g(x)dx
  $$
  cходится
\end{block}

\begin{proof}
  Из 2 условия следует что функция монотонна и ограничена $\Rightarrow$ имеет
  предел
  $$
  \lim_{a \to b-0}g(x) = p ~~~ g_1(x) = g(x) - p \to 0 ~~~ x \to b-0
  $$
  $$
  F(x) = \int_a^x f(t)dx \Rightarrow \lim_{x \to b-0} F(x) = \int_a^b f(t)dt
  $$
  $$
  \exists M > 0 ~~~ \forall x \in [a,b) ~~~ |F(x)| \le M
  $$
  по признаку Дирехле
  $$
  \int_a^b f(x)g_1(x)dx
  $$
  сходится
  $$
  \int_a^b f(x)g(x)dx = \int_a^b f(x)g_1(x)dx + p\int_a^b f(x)dx
  $$
\end{proof}

\begin{title}[\Large]
  Абсолютно и условно сходящиеся интегралы.
\end{title}

\begin{defin}
  $\int_a^b f(x)dx$ называется \kv{абсолютно сходящимся}. $f(x)$ \kv{абсолютно
интегрируема}, где $b$ особая точка, если $\int_a^b |f(x)|dx$ сходится.
\end{defin}

\begin{defin}
  $\int_a^b f(x)dx$ называют \kv{условно сходящимся} (необсалютно), если
  $\int_a^b f(x)dx$сходится, а $\int_a^b |f(x)|dx$ расходится.
\end{defin}

\begin{theorem}
  Абсолютно сходящийся интеграл является сходящимся, при этом
  $$
  \left| \int_a^b f(x)dx \right| \le \int_a^b |f(x)|dx
  $$
\end{theorem}

\begin{proof}
  $$
  \forall \varepsilon > 0 ~~~ \exists \delta_{\varepsilon} \in [a,b) ~~~
  \forall c',c'' \in [d_{\varepsilon}, b) ~~~
  \left| \int_{c'}^{c''} |f(x)|dx \right| < \varepsilon
  $$
  $$
  \left| \int_{c'}^{c''} f(x)dx \right| \le
  \left| \int_{c'}^{c''} |f(x)|dx \right| < \varepsilon
  $$
  тогда $\int_a^b f(x) dx$ сходится
  $$
  \forall c \in [a,b) ~~~ \left| \int_a^c f(x)dx \right| \le
  \left| \int_a^c |f(x)|dx \right|
  $$
  $$
  \lim_{c \to b-0} \left| \int_a^{c} f(x)dx \right| \le
  \lim_{c \to b-0} \int_a^c |f(x)|dx
  $$
\end{proof}

\begin{theorem}
  $b$ может быть особая точка $\int_a^b f(x)dx ~~~ \int_a^b g(x)dx$
абсолютно интегрируемы тогда
  $$
  \int_a^b (f(x) + g(x)) dx
  $$
  либо оба сходятся, либо расходятся.
\end{theorem}

\begin{proof}
  $$
  \forall \varepsilon < 0 ~~~ \exists d_{\varepsilon} \in [a,b)
  \forall c', c'' \in [d_{\varepsilon}, b)
  \left| \int_{c'}^{c''} |f(x)|dx \right| < \frac{\varepsilon}{2} ~~~
  \left| \int_{c'}^{c''} |g(x)|dx \right| < \frac{\varepsilon}{2}
  $$
  $$
  \left| \int_{c'}^{c''} |f(x) + g(x)| dx \right| \le
  \left| \int_{c'}^{c''} f(x) \right| + \left| \int_{c'}^{c''} g(x) \right|
  < \frac{\varepsilon}{2} + \frac{\varepsilon}{2} = \varepsilon
  $$
  по критерию Коши интеграл сходится
  $$
  f(x) = f(x) + g(x) + (-g(x))
  $$
  $\int_a^b (f(x) + g(x)) dx$ абсалютно интегрируем

  $\int_a^b (-g(x)) dx$ абсалютно интегрируем

  $\int_a^b f(x) dx$ тоже абсалютно интегрируем\\

  $\int_a^b f(x)dx$ сходится $\Rightarrow$ $\int_a^b (f(x) + g(x)) dx$
  расходящийся

  $\int_a^b g(x) dx$ сходится

  $\int_a^b (f(x) + g(x)) dx \Rightarrow \int_a^b f(x) dx$ расходящийся
\end{proof}