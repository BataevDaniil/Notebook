\begin{title}[\Large]
    Несобственный интеграл по неограниченому промежутку.
\end{title}

\begin{defin}
    Пусть $f(x)$ на $[a,+\infty)$ $\forall c \in (a, +\infty)$ интегрируема
    на $[a,c] \subset [a, +\infty)$
    \[\lim_{c \to +\infty} \int_a^c f(x)dx = \int_a^{+\infty} f(x)dx\]
\end{defin}

\begin{defin}
    Пусть $f(x)$ на $(-\infty, b]$ $\forall c \in (-\infty, b)$ интегрируема
    на $[c,b] \subset (-\infty, b]$
    \[\lim_{c \to -\infty} \int_c^b f(x)dx = \int_{-\infty}^b f(x)dx\]
\end{defin}

Называют \kv{несобственным интегралом}, если он сущетсвует то
\kv{сходящийся}, иначе называют \kv{расходящимся}.

\begin{defin}
    $f(x)$ на $[a,b]$ и интегрируема $[a,c] \subset [a,b)$ неограричена в
    окрестности точки $b$
    \[\lim_{c \to b -0} \int_a^c f(x)dx = \int_a^b f(x)dx\]
\end{defin}

\begin{defin}
    $f(x)$ на $[a,b]$ и интегрируема $[c,b] \subset (a,b]$ неограричена в
    окрестности точки $a$
    \[\lim_{c \to b +0} \int_c^b f(x)dx = \int_a^b f(x)dx\]
\end{defin}

\begin{defin}
    $f(x)$ на $(-\infty, +\infty)$ кроме быть может точек
    $d_1 < d_2 < \ldots < d_m$\\
    $-\infty < a_1 < d_1 < a_2 < d_2 \ldots < a_m < d_m < a_{m+1} < +\infty$
    \[
        \int_{-\infty}^{+\infty} f(x)dx = \int_{-\infty}^{a_1}f(x)dx +
        \int_{a_1}^{d_1}f(x)dx + \int_{d_1}^{a_2}f(x)dx + \ldots +
    \]
    \[
        \int_{a_m}^{d_m}f(x)dx + \int_{d_m}^{a_{m+1}}f(x)dx +
        \int_{a_{m+1}}^{+\infty}f(x)dx
    \]
        Он сущетсвет только когда каждый интеграл из суммы интегралов существует.
\end{defin}
\begin{title}[\Large]
  Несобственный интеграл от неограниченной функции
\end{title}

$f(x)$ определена на $[a, b]$ и интегрируема на $[a, c] \subset [a, b]$. А в
любой окрестной точки $b$ - $f(x)$ не ограничена если существует:
\[
  \lim_{c \to b - 0} \int^{c}_{a} f(x)dx
  \]
то его называют несобственным интегралом $f(x)$ на $[a, b]$ и сходящимся.\\
Если предела нет, то интеграл несобственный и расходящийся.

\begin{title}[\Large]
  Особые точки для функции, неограниченной в их окрестности
\end{title}

$f(x)$ определена на всей прямой, исключая $d_1 < d_2 < ... < d_m$:
\[-\infty < a_1 < d_1 < a_2 < d_2 < ... < a_m < d_m < a_{m + 1} < +\infty\]
По определению несобственных интегралов:
\[
  \int^{+\infty}_{-infty} f(x)dx = \int^{a_1}_{-\infty} f(x)dx +
  \int^{d_1}_{a_1} f(x)dx + \int^{a_2}_{d_1} f(x)dx + \int^{d_2}_{a_2} f(x)dx +
\]
\[
  ... + \int^{d_m}_{a_m} f(x)dx + \int^{a_{m + 1}}_{d_m} f(x)dx +
  \int^{+\infty}_{a_{m + 1} f(x)dx}
\]
Этот интеграл сходящийся, если все интегралы сходящиеся. А расходящийся, если
хотябы один интеграл расходящийся.

\begin{title}[\Large]
  Основные свойства и способы вычисления несобственных интегралов.
\end{title}

\begin{theorem}[1]
  Если:
  \[
    \int^{b}_{a} f(x)dx ~~~ \int^{b}_{a} g(x)dx
  \]
  где $a$ - особая точка и сходящаяся. $b$ либо $+\infty$, либо $-\infty$,
  то $\forall \alpha, \beta
  \in R$
  \[
    \int^{b}_{a} (\alpha f(x) + \beta g(x))dx = \alpha\int^{b}_{a} f(x)dx +
    \beta\int^{b}_{a} g(x)dx
  \]
\end{theorem}

\begin{proof}
  Пусть $c \in (a, b)$
  \[
    \int^{c}_{a} (\alpha f(x) + \beta g(x))dx = \alpha\int^{c}_{a} f(x)dx +
    \beta\int^{c}_{a} g(x)dx
  \]
  \[
    \lim_{c \to b - 0} \int^{c}_{a} (\alpha f(x) + \beta g(x))dx =
    \lim_{c \to b - 0} \alpha\int^{c}_{a} f(x)dx +
    \lim_{c \to b - 0} \beta\int^{c}_{a} g(x)dx =
    \alpha\int^{b}_{a} f(x)dx + \beta\int^{b}_{a} g(x)dx
  \]
\end{proof}

\begin{theorem}[2]
  Пусть интеграл с особой точкой $b$ - сходящейся.
  \[
    \int^{b}_{a} f(x)dx ~~~ \int^{b}_{a} g(x)dx
  \]
  $\forall x \in [a, b] ~~~ f(x) \le g(x)$ Тогда:
  \[
    \int^{b}_{a} f(x)dx \le \int^{b}_{a} g(x)dx
  \]
\end{theorem}

\begin{proof}
  $\forall c \in (a, b)$
  \[
    \int^{c}_{a} f(x)dx \le \int^{c}_{a} g(x)dx
  \]
  \[
    \lim_{c \ to b - 0} \int^{b}_{a} f(x)dx \le \lim_{c \ to b - 0}
    \int^{b}_{a} g(x)dx \Rightarrow \int^{b}_{a} f(x)dx \le \int^{b}_{a} g(x)dx
  \]
\end{proof}

\begin{theorem}[3]
  Пусть интеграл сходящийся и с особой точкой $b$:
  \[
    \int^{b}_{a} f(x)dx
  \]
  $F(x)$ - первообразная $f(x)$ на $[a, b]$. Если:
  \[
    \lim_{x \to  b - 0} F(x) = F(b - 0)
  \]
  то
  \[
    \int^{b}_{a} f(x)dx = \lim_{c \to b - 0} f(c) - F(a) = F(b - 0) - F(a) =
    F(x)|^{b}_{a}
  \]
\end{theorem}

\begin{proof}
  \[
    \int^{b}_{a} f(x)dx = \lim_{c \to b - 0} \int^{c}_{a} f(x)dx =
    \lim_{c \to b - 0} F(c) - F(a) = F(b - 0) - F(a)
  \]
\end{proof}

\begin{theorem}[4]
  Функция $f(x)$ определена на $[a, b)$. $x = \varphi(t) \nearrow ~~~
  [\alpha, \beta)$\\
  $\varphi(\alpha) = a ~~~ \lim_{t \ to \beta - 0} \varphi(t)$ Тогда
  \[
    \int^{b}_{a} f(x)dx = \int^{\beta}_{\alpha} f(\varphi(t)) \varphi'(t)dt
  \]
  При этом несобственный интеграл сходится и расходится одновременно.
\end{theorem}

\begin{proof}
  $c \in (a, b) ~~~ c = \varphi(\gamma) ~~~ \gamma \in (\alpha, \beta)$
  \[
    \int^{c}_{a} f(x)dx = \int^{\gamma}_{\alpha} f(\varphi(t)) \varphi'(t)dt
  \]
  \[
    \lim_{c \to b - 0} \int^{c}_{a} f(x)dx \Leftrightarrow
    \lim_{\gamma \to \beta - 0} \int^{\gamma}_{\alpha} f(\varphi(t))
    \varphi'(t)dt = \int^{b}_{a} f(x)dx = \int^{\beta}_{\alpha} f(\varphi(t))
    \varphi'(t)dt
  \]
\end{proof}

\begin{theorem}[5]
  $u = u(x) ~~~ v = v(x)$ - интегрируемо-дифференцируемые функции на $[a, b)$.
  $b$ - особоя точка.\\
  Если :
  \[
    \lim_{x \to b - 0} u(x) \cdot v(x) = u(b - 0) \cdot v(b - 0)
  \]
  то
  \[
    \int^{b}_{a} udv = u(b - 0) \cdot v(b - 0) - \int^{b}_{a} vdu
  \]
  При этом оба интеграла сходятся и расходятся одновременно.
\end{theorem}

\begin{title}[\Large]
  Признаки сходимости не собственных интегралов от неотрицательных функций
\end{title}

\begin{theorem}[1]
  Если $f(x) \ge 0 ~~~ \forall x \in [a, b)$, $b$ - особая точка и интеграл
  сходящийся
  \[
    \int^{b}_{a} f(x)dx \Leftrightarrow F(x) - \int^{x}_{a} f(t)dt
  \]
  $F(x)$ - ограниченная функция. $\exists M ~~~ |F(x)| \le M$
\end{theorem}

\begin{proof}
  Заметим что $F(x) \nearrow [a, b) ~~~ a \le x_1 < x_2 < b$, то
  \[
    F(x_2) - F(x_1) = \int^{x_2}_{x_1} f(t)dt \ge 0
  \]
  \[
    \lim_{x \to b - 0} F(x) - F(a)
  \]
  Если некая $F$ имеет предел в точке, то она ограничена в окрестности точки $b$
\end{proof}

\begin{theorem}[2]
  Признак сравнения для не собственного интеграла\\
  Если $\forall x \in [a, b) ~~~ 0 \le f(x) \le g(x)$, $b$ - особая точка,
  тогда\\
  \bk{I} Если интеграл $g(x)$ сходящийся, то интеграл от $f(x)$ тоже сходящийся
  \[\int^{b}_{a} g(x)dx \Rightarrow \int^{b}_{a} f(x)dx\]
  \bk{II} Если интеграл $f(x)$ расходящийся, то интеграл от $g(x)$ тоже
  рассходящийся
  \[\int^{b}_{a} f(x)dx \Rightarrow \int^{b}_{a} g(x)dx\]
\end{theorem}

\begin{proof}
  $\forall c \in [a, b)$
  \bk{I}
  \[
    \int^{c}_{a} f(x)dx \le \int^{c}_{a} g(x)dx \le \int^{b}_{a} g(x)dx = M
  \]
  На основании 1й теоремы
  \[
    \lim_{c \to b - 0} \int^{c}_{a} f(x)dx = \int^{b}_{a} f(x)dx
  \]
  \bk{II} От противного (Интеграл от $f(x)$ - расходится, а от $g(x)$ -
  сходится)
  \[
    \int^{b}_{a} f(x)dx \Rightarrow \int^{b}_{a} g(x)dx
  \]
  По теореме 1 - противоречие
\end{proof}

\begin{title}[\Large]
  Признак сравнения в предельной форме.
\end{title}

\begin{theorem}
  $f(x), g(x)$ на $[a, b)$ $b$ особая точка, $f(x) > 0 ~~~ g(x) > 0 ~~~
  f(x) \sim g(x) ~~~ x \to b -0$ значит
  $$
  \int_a^b f(x)dx = \int_a^b g(x)dx
  $$
\end{theorem}

\begin{proof}
  $$
  \lim_{x \to b-0} \frac{f(x)}{g(x)} = 1
  $$
  $$
  \forall \varepsilon > 0 ~~~ \exists \delta_{\varepsilon} > 0 ~~~
  b - \delta_{\varepsilon} < x < b ~~~ \left| \frac{f(x)}{g(x)} - 1 \right|
  < \varepsilon
  $$
  $$
  \varepsilon = \frac{1}{2} ~~~ \exists \delta_{\frac{1}{2}} > 0 ~~~
  b - \delta_{\frac{1}{2}} < x < b ~~~
  1 - \frac{1}{2} < \frac{f(x)}{g(x)} < 1 + \frac{1}{2}
  $$
  $$
  \frac{1}{2} < \frac{f(x)}{g(x)} < \frac{3}{2}
  $$
  $$
  \frac{1}{2} g(x) < f(x) < \frac{3}{2} g(x)
  $$
  $$
  \int_a^b g(x)dx = \frac{3}{2} \int_{b - \delta_{\frac{1}{2}}}^b g(x)dx ~~~
  \int_{b - \delta_{\frac{1}{2}}}^b f(x)dx ~~~
  \int_a^b f(x)dx
  $$
  В обратную сторону
  $$
  \int_a^b f(x)dx ~~~
  \int_{b - \delta_{\frac{1}{2}}}^b f(x)dx ~~~
  \frac{1}{2} \int_{b - \delta_{\frac{1}{2}}}^b g(x)dx ~~~
  \int_a^b g(x)dx
  $$
\end{proof}

\begin{title}[\Large]
  Критерий Коши сходимости несобственного интеграла.
\end{title}
Критерий значить необходимые и достаточные условия.

$$
\forall \varepsilon > 0 ~~~ \exists \delta_{\varepsilon} \in [a,b) ~~~
\forall c', c'' \in [d_a, b) \left| \int_{c'}^{c''} f(x)dx \right| < \varepsilon
$$
$$
\int_a^b f(x)dx = \lim_{c \to b-0} \int_a^c f(x)dx = \lim_{c \to b-0} F(x)
$$
$$
F(x) = \int_a^x f(t)dt
$$
$$
\lim_{c \to b-0} F(c) \int_a^b f(x)dx
$$
$$
\forall \varepsilon > 0 ~~~ \exists \delta_{\varepsilon} \in [a,b) ~~~
\forall c', c'' \in [d_a, b) |F(c'') - F(c')| < \varepsilon
$$

\begin{title}[\Large]
  Признак Дирехле сходоимости несобственных интегралов.
\end{title}

\begin{theorem}
  $f(x), g'(x)$ непрерывны на $[a,b)$ если

  1) $f(x)$ ограничена $F(x)$ тоисть $\forall [a,b] ~~~ |f(x)| \le M$

  2) $\forall x[a,b) ~~~ g'(x) \le 0$ или $g'(x) \ge 0$

  3) $\lim_{x \to b-0} g(x) = 0$ тогда

  $$\int_a^b f(x)g(x)dx$$ cходится
\end{theorem}

\begin{proof}
  $$
  \int_{c'}^{c''} f(x)g(x)dx =
  \left|
    \begin{array}{ll}
      v = g(x)    & dx = g'(x)dx \\
      dv = f(x)dx & v = F(x)
    \end{array}
  \right|
  = g(x)F(x)|_{c'}^{c''} - \int_{c'}^{c''} F(x)g'(x)dx
  $$
  для определенности $g'(x) \le 0$
  $$
  g(x)F(x)|_{c'}{c''} \le M g(c'') + M g(c')
  $$
  $$
  \left| \int_{c'}{c''} F(x)g'(x)dx \right| \le
  \left| \int_{c'}{c''} |F(x)||g'(x)| \right| \le
  M \left| \int_{c'}{c''} g(x)dx \right| = M|g(c'') - g(c')| \le
  $$
  $$
  \int_{c'}^{c''} f(x)g(x) \le 2M||g(c'')| + |g(c')||
  $$
  $$
  \forall \varepsilon > 0 ~~~ \exists d_{\varepsilon} \in [a,b) ~~~
  \forall c \in [d_{\varepsilon}, b) ~~~ |g(c)| < \frac{\varepsilon}{4M}
  $$
  $$
  \forall c', c'' \in [d_{\varepsilon}, b) ~~~ \int_{c'}^{c''} f(x)g(x)dx \le
  (|g(c'') + |g(c')|) <
  2M \left(\frac{\varepsilon + \varepsilon}{4M + 4M}\right) = \varepsilon
  $$
\end{proof}

\begin{title}[\Large]
  Признаки Абеля сходимости несобственного интеграла.
\end{title}

\begin{theorem}
  $f(x),g'(x)$ непрерывна на $[a,b)$

  1) $$\int_a^b f(x)$$ сходится

  2) $\forall x \in [a,b) ~~~ g'(x) \le 0$ или $g'(x) \ge 0$

  3) $\forall x \in [a,b) ~~~ |g(x)| \le \delta$
  $$
  \int_a^b f(x)g(x)dx
  $$
  cходится
\end{theorem}

\begin{proof}
  Из 2 условия следует что функция монотонна возрастает
  $$
  \lim_{a \to b-0}g(x) = p ~~~ g(x) - p \to 0 ~~~ x \to b-0
  $$
  $$
  F(x) = \int_a^x f(t)dx ~~~ \lim_{x \to b-0} F(x) = \int_a^b f(t)dt
  $$
  $$
  \exists M > 0 ~~~ \forall x \in [a,b] ~~~ |F(x)| \le M
  $$
  $$
  \int_a^b f(x)g(x)dx
  $$
  сходится
  $$
  \int_a^b f(x)g(x)dx = \int_a^b f(x)g(x)dx + p\int_a^b f(x)dx
  $$
\end{proof}

\begin{title}[\Large]
  Абсолютно и условно сходящиеся интегралы.
\end{title}

\begin{defin}
  $\int_a^b f(x)dx$ называется абсолютно сходящимся $f(x)$ абсолютно
интегрируема где $b$ - особая точка, то если сходится $\int_a^b |f(x)|dx$.
\end{defin}

\begin{defin}
  $\int_a^b f(x)dx$ условно сходящимся необсалютно сходящимся если сам интеграл
сходится и интеграл от модуля расходится.
\end{defin}

\begin{theorem}
  Абсолютно сходящийся интеграл является сходящимся
  $$
  \left| \int_a^b f(x)dx \right| \le \int_a^b |f(x)|dx
  $$
\end{theorem}

\begin{proof}
  $$
  \forall \varepsilon > 0 ~~~ \exists \delta_{\varepsilon} \in [a,b) ~~~
  \forall c',c'' \in [d_{\varepsilon}, b) ~~~
  \left| \int_{c'}^{c''} |f(x)|dx \right| < \varepsilon
  $$
  $$
  \left| \int_{c'}^{c''} f(x)dx \right| \le
  \left| \int_{c'}^{c''} |f(x)|dx \right| < \varepsilon
  $$
  $$
  \forall c \in [a,b) ~~~ \left| \int_a^c f(x)dx \right| \le
  \left| \int_a^c |f(x)|dx \right|
  $$
  $$
  \lim_{c \to b-0} \left| \int_a^{c} f(x)dx \right| \le
  \lim_{c \to b-0} \left| \int_a^c |f(x)|dx \right|
  $$
  $$
  \left| \int_a^b f(x)dx \right| \le
  \left| \int_a^b |f(x)|dx \right|
  $$
\end{proof}

\begin{theorem}
  $b$ - может быть особая точка $\int_a^b f(x)dx ~~~ \int_a^b g(x)dx$
абсолютно интегрируемы тогда
  $$
  \int_a^b f(x)dx ~~~ \int_a^b (f(x) + g(x)) dx
  $$
  либо оба сходятся, либо расходятся, либо оба абсолютно сходятся.
\end{theorem}

\begin{proof}
  $$
  \forall \varepsilon < 0 ~~~ \exists \delta_{\varepsilon} \in [a,b)
  \forall c', c'' \in [\delta_{\varepsilon}, b)
  \left| \int_{c'}^{c''} |f(x)|dx \right| < \frac{\varepsilon}{2}
  \left| \int_{c'}^{c''} |f(x)|dx \right| < \frac{\varepsilon}{2}
  $$
  $$
  \left| \int_{c'}^{c''} |(f(x) + g(x))| dx \right| \le
  \left| \int_{c'}^{c''} f(x) \right| \left| \int_{c'}^{c''} g(x) \right|
  < \frac{\varepsilon}{2} + \frac{\varepsilon}{2} = \varepsilon
  $$
  $$
  f(x) = f(x) + g(x) + (-g(x))
  $$
\end{proof}