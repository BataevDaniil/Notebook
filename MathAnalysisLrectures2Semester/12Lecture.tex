\begin{title}[\Large]
  Признак сравнения в предельной форме.
\end{title}

\begin{theorem}
  $f(x), g(x)$ на $[a, b)$ $b$ особая точка, $f(x) > 0 ~~~ g(x) > 0 ~~~
  f(x) \sim g(x) ~~~ x \to b -0$ значит
  $$
  \int_a^b f(x)dx = \int_a^b g(x)dx
  $$
\end{theorem}

\begin{proof}
  $$
  \lim_{x \to b-0} \frac{f(x)}{g(x)} = 1
  $$
  $$
  \forall \varepsilon > 0 ~~~ \exists \delta_{\varepsilon} > 0 ~~~
  b - \delta_{\varepsilon} < x < b ~~~ \left| \frac{f(x)}{g(x)} - 1 \right|
  < \varepsilon
  $$
  $$
  \varepsilon = \frac{1}{2} ~~~ \exists \delta_{\frac{1}{2}} > 0 ~~~
  b - \delta_{\frac{1}{2}} < x < b ~~~
  1 - \frac{1}{2} < \frac{f(x)}{g(x)} < 1 + \frac{1}{2}
  $$
  $$
  \frac{1}{2} < \frac{f(x)}{g(x)} < \frac{3}{2}
  $$
  $$
  \frac{1}{2} g(x) < f(x) < \frac{3}{2} g(x)
  $$
  $$
  \int_a^b g(x)dx = \frac{3}{2} \int_{b - \delta_{\frac{1}{2}}}^b g(x)dx ~~~
  \int_{b - \delta_{\frac{1}{2}}}^b f(x)dx ~~~
  \int_a^b f(x)dx
  $$
  В обратную сторону
  $$
  \int_a^b f(x)dx ~~~
  \int_{b - \delta_{\frac{1}{2}}}^b f(x)dx ~~~
  \frac{1}{2} \int_{b - \delta_{\frac{1}{2}}}^b g(x)dx ~~~
  \int_a^b g(x)dx
  $$
\end{proof}

\begin{title}[\Large]
  Критерий Коши сходимости несобственного интеграла.
\end{title}
Критерий значить необходимые и достаточные условия.

$$
\forall \varepsilon > 0 ~~~ \exists \delta_{\varepsilon} \in [a,b) ~~~
\forall c', c'' \in [d_a, b) \left| \int_{c'}^{c''} f(x)dx \right| < \varepsilon
$$
$$
\int_a^b f(x)dx = \lim_{c \to b-0} \int_a^c f(x)dx = \lim_{c \to b-0} F(x)
$$
$$
F(x) = \int_a^x f(t)dt
$$
$$
\lim_{c \to b-0} F(c) \int_a^b f(x)dx
$$
$$
\forall \varepsilon > 0 ~~~ \exists \delta_{\varepsilon} \in [a,b) ~~~
\forall c', c'' \in [d_a, b) |F(c'') - F(c')| < \varepsilon
$$

\begin{title}[\Large]
  Признак Дирехле сходоимости несобственных интегралов.
\end{title}

\begin{theorem}
  $f(x), g'(x)$ непрерывны на $[a,b)$ если

  1) $f(x)$ ограничена $F(x)$ тоисть $\forall [a,b] ~~~ |f(x)| \le M$

  2) $\forall x[a,b) ~~~ g'(x) \le 0$ или $g'(x) \ge 0$

  3) $\lim_{x \to b-0} g(x) = 0$ тогда

  $$\int_a^b f(x)g(x)dx$$ cходится
\end{theorem}

\begin{proof}
  $$
  \int_{c'}^{c''} f(x)g(x)dx =
  \left|
    \begin{array}{ll}
      v = g(x)    & dx = g'(x)dx \\
      dv = f(x)dx & v = F(x)
    \end{array}
  \right|
  = g(x)F(x)|_{c'}^{c''} - \int_{c'}^{c''} F(x)g'(x)dx
  $$
  для определенности $g'(x) \le 0$
  $$
  g(x)F(x)|_{c'}{c''} \le M g(c'') + M g(c')
  $$
  $$
  \left| \int_{c'}{c''} F(x)g'(x)dx \right| \le
  \left| \int_{c'}{c''} |F(x)||g'(x)| \right| \le
  M \left| \int_{c'}{c''} g(x)dx \right| = M|g(c'') - g(c')| \le
  $$
  $$
  \int_{c'}^{c''} f(x)g(x) \le 2M||g(c'')| + |g(c')||
  $$
  $$
  \forall \varepsilon > 0 ~~~ \exists d_{\varepsilon} \in [a,b) ~~~
  \forall c \in [d_{\varepsilon}, b) ~~~ |g(c)| < \frac{\varepsilon}{4M}
  $$
  $$
  \forall c', c'' \in [d_{\varepsilon}, b) ~~~ \int_{c'}^{c''} f(x)g(x)dx \le
  (|g(c'') + |g(c')|) <
  2M \left(\frac{\varepsilon + \varepsilon}{4M + 4M}\right) = \varepsilon
  $$
\end{proof}

\begin{title}[\Large]
  Признаки Абеля сходимости несобственного интеграла.
\end{title}

\begin{theorem}
  $f(x),g'(x)$ непрерывна на $[a,b)$

  1) $$\int_a^b f(x)$$ сходится

  2) $\forall x \in [a,b) ~~~ g'(x) \le 0$ или $g'(x) \ge 0$

  3) $\forall x \in [a,b) ~~~ |g(x)| \le \delta$
  $$
  \int_a^b f(x)g(x)dx
  $$
  cходится
\end{theorem}

\begin{proof}
  Из 2 условия следует что функция монотонна возрастает
  $$
  \lim_{a \to b-0}g(x) = p ~~~ g(x) - p \to 0 ~~~ x \to b-0
  $$
  $$
  F(x) = \int_a^x f(t)dx ~~~ \lim_{x \to b-0} F(x) = \int_a^b f(t)dt
  $$
  $$
  \exists M > 0 ~~~ \forall x \in [a,b] ~~~ |F(x)| \le M
  $$
  $$
  \int_a^b f(x)g(x)dx
  $$
  сходится
  $$
  \int_a^b f(x)g(x)dx = \int_a^b f(x)g(x)dx + p\int_a^b f(x)dx
  $$
\end{proof}