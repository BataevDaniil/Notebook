\begin{title}
  Иследование выпоклости функций с помощью производной.
\end{title}

$f(x)$ непрерывна на $<a,b>$ \bd{называется выпуклой вниз} $\forall x_1, x_2
\in <a,b> f(\frac{x_1 + x_2}{2}) \le \frac{f(x_1) + f(x_2)}{2}$.\\
И \bd{называется строго выпуклой вниз} если  $f(x)$ непрерывна на $<a,b>$
$\forall x_1 \not= x_2 \in <a,b> ~~ f(\frac{x_1 + x_2}{2}) <
\frac{f(x_1) + f(x_2)}{2}$\\

$f(x)$ непрерывна на $<a,b>$ \bd{называется выпуклой вниз} $\forall x_1, x_2
\in <a,b> f(\frac{x_1 + x_2}{2}) \ge \frac{f(x_1) + f(x_2)}{2}$.\\
И \bd{называется строго выпуклой вниз} если  $f(x)$ непрерывна на $<a,b>$
$\forall x_1 \not= x_2 \in <a,b> ~~ f(\frac{x_1 + x_2}{2}) >
\frac{f(x_1) + f(x_2)}{2}$\\

\begin{theorem}
  Пусть $f(x)$ непрерывна на $(a,b)$ и $f'(x) ~~ x \in (a,b) ~~ \forall
  x \in (a,b) ~~ f''(x) \ge 0$ функция выпукла вниз на $(a,b)$, если
  $f''(x) \le 0$ то выпукла вверх.
\end{theorem}

\begin{proof}
  Для определенности $x_1 < x_2$ и $2h = x_2 - x_1$\\
  $
  x_0 = \frac{x_2 + x_1}{2}\\
  x_2 = x_0 + h\\
  x_1 = x_0 - h\\
  f(x_2) = f(x_0 + h) = f(x_0) + \frac{f'(x_0)}{1!}h +
         \frac{f''(c_2)}{2!}h^2 ~ x_0 < c_2 < x_2\\
  f(x_1) = f(x_0 - h) = f(x_0) + \frac{f'(x_0)}{1!}h +
         \frac{f''(c_1)}{2!}h^2 ~ x_1 < c_1 < x_0\\
  f(x_1) + f(x_2) = 2f(x_0) +
                  \frac{h^2}{2}(f''(c_2) + f''(c_1))\\
  $\\
\end{proof}
Теория будет справидливой и для строгих выпуклостей.

\begin{title}
  Нахождение точек перегиба функции с помощью производной.
\end{title}

\bd{Точка $a$ называется точкой перегиба} $y = f(x)$ если в точке
$a ~ \exists f'(x)$ конечная или бесконечная определенного знака и при переходе
через точку $a$ меняется направление выпуклости.\\
$(a, f(x))$ точка пергиба графика функции.\\

\bd{Необходимыйе условия точки перегиба:}\\
$f(x)$ имеет $f''(x)$ в некоторой окрестности точки $a$, $a$ точка перегиба
$f(x)$, тоисть $f''(a) = 0$.

\bd{Докозательство:} Предположим что $f''(a)\not= 0$, то в силу непрерывности
1-ой производной в точке $a$, она имеет тот же знак что и знак в некоторой
окрестности $a$ $f''(x)\ge 0 ~ \forall x\in O(a)$ то $f(x)$ выпукла вверх.
$a$ не может быть точкой перегиба так как направление выпуклости не меняется,
противоречие.\\

\bd{Достаточные условия:}\\
Пусть в точке $a ~ f(x)$ имеет конечную или бесконечную определенного занка
первую производную в проколотой открестности точки $a$ $\exists f''(x)$.
Если $f''$ меняет знак при переходе через точку $a$, то точка $a$ является
точкой перегиба этой функции.\\

\begin{theorem}
  Если $f''(a) = 0$ значит $f'(x)$ - конечная и $f''(a) \not= 0$ то
  $a$ точка перегиба.\\
\end{theorem}

\bd{План иследования функции}\\
1. D(y), четность, переодичность.\\
2. Найти точки пересечения графика функции с Ox и Oy, и
промежутки возрастания и убывания.\\
3. Найти точки разрыва и их классификация. Вычисление
односторонних приделов в точке разрыва и в ограниченных
точках области определния.\\
4. Нахождение всех ассимтот графика.\\
5. Иследование функция на монотоность и экстремумы с
помощью производной.\\
6. Иследование на выпуклости и перегибы.\\
7. Построение таблицы значений.\\
8. Построение графика.\\