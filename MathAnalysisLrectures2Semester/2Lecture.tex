\begin{title}
  Иследование выпоклости функций с помощью производной.
\end{title}

\begin{defin}[выпуклости вниз]
  Функция \bd{называется выпуклой вниз} если $f(x)$ непрерывна на $<a,b>$ и
  $\forall x_1, x_2 \in <a,b>$
  \[f(\frac{x_1 + x_2}{2}) \le \frac{f(x_1) + f(x_2)}{2}\]
  Функция \bd{называется строго выпуклой вниз} если $f(x)$ непрерывна
  на $<a,b>$ и
  $\forall x_1 \not= x_2 \in <a,b>$
  \[f(\frac{x_1 + x_2}{2}) < \frac{f(x_1) + f(x_2)}{2}\]
\end{defin}

\begin{defin}[выпуклости вверх]
  Функция \bd{называется выпуклой вверх} если $f(x)$ непрерывна на $<a,b>$ и
  $\forall x_1, x_2 \in <a,b>$
  \[f(\frac{x_1 + x_2}{2}) \ge \frac{f(x_1) + f(x_2)}{2}\]
  Функция \bd{называется строго выпуклой вверх} если $f(x)$ непрерывна
  на $<a,b>$ и
  $\forall x_1 \not= x_2 \in <a,b>$
  \[f(\frac{x_1 + x_2}{2}) > \frac{f(x_1) + f(x_2)}{2}\]
\end{defin}

\begin{theorem}
  Пусть $f(x)$ непрерывна на $(a,b)$ и имеет $f'(x)$ на $(a,b)$ то\\
  $\forall x \in (a,b) ~~ f''(x) \ge 0$ функция выпукла вниз на $(a,b)$\\
  $\forall x \in (a,b) ~~ f''(x) \le 0$ функция выпукла вверх на $(a,b)$.\\
  Cправидливой и для строгих выпуклостей.
\end{theorem}

\begin{proof}
  Для определенности $x_1 < x_2$ и $2h = x_2 - x_1$\\
  $
  x_0 = \frac{x_2 + x_1}{2}\\
  x_2 = x_0 + h\\
  x_1 = x_0 - h\\
  f(x_2) = f(x_0 + h) = f(x_0) + \frac{f'(x_0)}{1!}h +
    \frac{f''(c_2)}{2!}h^2 ~~ x_0 < c_2 < x_2\\
  f(x_1) = f(x_0 - h) = f(x_0) + \frac{f'(x_0)}{1!}h +
    \frac{f''(c_1)}{2!}h^2 ~~ x_0 > c_1 > x_1\\
  f(x_1) + f(x_2) = 2f(x_0) + 2f'(x_0)h + \frac{h^2}{2}(f''(c_2) + f''(c_1)) ~~
  x_1 < c_1 < x_0 < c_2 > x_2
  $
\end{proof}

\begin{title}
  Нахождение точек перегиба функции с помощью производной.
\end{title}

\begin{defin}[точки перегиба]
  Точка $a$ называется \kv{точкой перегиба} функции если в точке
  $a$ существует конечная или бесконечная $f'(x)$ определенного знака
  и при переходе через точку $a$ меняется направление выпуклости.\\
  $(a, f(x))$ точка пергиба графика функции.
\end{defin}

\bd{Необходимыйе условия точки перегиба:}\\
Если $f(x)$ имеет $f''(x)$ в некоторой окрестности точки $a$ то $a$ может быть
точкой перегиба если $f''(a) = 0$.\\

\begin{proof}
  Предположим что $f''(a)\not= 0$, то в силу непрерывности 1-ой производной
  в точке $a$, она имеет тот же знак что и знак в некоторой
  окрестности $a$ $f''(x)\ge 0 ~ \forall x\in O(a)$ то $f(x)$ выпукла вверх.
  $a$ не может быть точкой перегиба так как направление выпуклости не меняется,
  противоречие.\\
\end{proof}

\bd{Достаточные условия:}\\
Пусть в точке $a$ функция имеет конечную или бесконечную 1-ую производную
определенного занка в проколотой открестности точки $a$ и $\exists f''(x)$.
Если $f''(x)$ меняет знак при переходе через точку $a$, то точка $a$ является
точкой перегиба этой функции.\\

\begin{theorem}
  Если $f''(a) = 0$ значит $f'(x)$ - конечная и $f''(a) \not= 0$ то
  $a$ точка перегиба.\\
\end{theorem}

\bd{План иследования функции}\\
1. D(y), четность, переодичность.\\
2. Найти точки пересечения графика функции с Ox и Oy, и промежутки возрастания
  и убывания.\\
3. Найти точки разрыва и их классификация. Вычисление односторонних приделов
  в точке разрыва и в ограниченных точках области определния.\\
4. Нахождение всех ассимтот графика.\\
5. Иследование функция на монотоность и экстремумы с помощью производной.\\
6. Иследование на выпуклости и перегибы.\\
7. Построение таблицы значений.\\
8. Построение графика.\\